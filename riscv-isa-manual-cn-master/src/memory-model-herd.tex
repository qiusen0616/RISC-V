\section{Herd中的形式公理规范}
% \section{Formal Axiomatic Specification in Herd}
\label{sec:herd}

工具\textsf{herd}把一个内存模型和一个石蕊测试作为输入,并在内存模型顶端模拟测试的执行。内存模型使用领域专用语言\textsc{Cat}写成。
这节提供了两个RVWMO的\textsc{Cat}内存模型。第一个模型,图~\ref{fig:herd2},对于一个Cat模型而言尽可能多地遵循了\emph{全局内存次序}、第~\ref{ch:memorymodel}章、RVWMO的定义。第二个模型,图~\ref{fig:herd3},是一个等价的、更加有效的的、基于部分次序的RVWMO模型。
% The tool \textsf{herd} takes a memory model and a litmus test as input and simulates the execution of the test on top of the memory model. Memory models are written in the domain specific language \textsc{Cat}. This section provides two \textsc{Cat} memory model of RVWMO. The first model, Figure~\ref{fig:herd2}, follows the \emph{global memory order}, Chapter~\ref{ch:memorymodel}, definition of~RVWMO, as much as is possible for a \textsc{Cat} model. The second model, Figure~\ref{fig:herd3}, is an equivalent, more efficient, partial order based RVWMO model.

模拟器~\textsf{herd}是\textsf{DIY}工具套件的一部分——软件和文件见 \url{http://diy.inria.fr} 。模型和更多内容可以从~url{http://diy.inria.fr/cats7/riscv/} 在线获得。
% The simulator~\textsf{herd} is part of the \textsf{diy} tool suite --- see \url{http://diy.inria.fr} for software and documentation. The models and more are available online at~\url{http://diy.inria.fr/cats7/riscv/}.

\begin{figure}[h!]
  {
  \tt\bfseries\centering\footnotesize
  \begin{lstlisting}
(*************)
(* Utilities *)
(*************)

(* All fence relations *)
let fence.r.r = [R];fencerel(Fence.r.r);[R]
let fence.r.w = [R];fencerel(Fence.r.w);[W]
let fence.r.rw = [R];fencerel(Fence.r.rw);[M]
let fence.w.r = [W];fencerel(Fence.w.r);[R]
let fence.w.w = [W];fencerel(Fence.w.w);[W]
let fence.w.rw = [W];fencerel(Fence.w.rw);[M]
let fence.rw.r = [M];fencerel(Fence.rw.r);[R]
let fence.rw.w = [M];fencerel(Fence.rw.w);[W]
let fence.rw.rw = [M];fencerel(Fence.rw.rw);[M]
let fence.tso =
  let f = fencerel(Fence.tso) in
  ([W];f;[W]) | ([R];f;[M])

let fence = 
  fence.r.r | fence.r.w | fence.r.rw |
  fence.w.r | fence.w.w | fence.w.rw |
  fence.rw.r | fence.rw.w | fence.rw.rw |
  fence.tso

(* Same address, no W to the same address in-between *)
let po-loc-no-w = po-loc \ (po-loc?;[W];po-loc)
(* Read same write *)
let rsw = rf^-1;rf
(* Acquire, or stronger  *)
let AQ = Acq|AcqRel
(* Release or stronger *)
and RL = RelAcqRel
(* All RCsc *)
let RCsc = Acq|Rel|AcqRel
(* Amo events are both R and W, relation rmw relates paired lr/sc *)
let AMO = R & W
let StCond = range(rmw)

(*************)
(* ppo rules *)
(*************)

(* Overlapping-Address Orderings *)
let r1 = [M];po-loc;[W]
and r2 = ([R];po-loc-no-w;[R]) \ rsw
and r3 = [AMO|StCond];rfi;[R]
(* Explicit Synchronization *)
and r4 = fence
and r5 = [AQ];po;[M]
and r6 = [M];po;[RL]
and r7 = [RCsc];po;[RCsc]
and r8 = rmw
(* Syntactic Dependencies *)
and r9 = [M];addr;[M]
and r10 = [M];data;[W]
and r11 = [M];ctrl;[W]
(* Pipeline Dependencies *)
and r12 = [R];(addr|data);[W];rfi;[R]
and r13 = [R];addr;[M];po;[W]

let ppo = r1 | r2 | r3 | r4 | r5 | r6 | r7 | r8 | r9 | r10 | r11 | r12 | r13
\end{lstlisting}
  }
  \caption{riscv-defs.cat,一个关于保留程序次序的Herd定义(1/3)
    % {\tt riscv-defs.cat}, a herd definition of preserved program order (1/3)
    }
  \label{fig:herd1}
\end{figure}

\begin{figure}[ht!]
  {
  \tt\bfseries\centering\footnotesize
  \begin{lstlisting}
Total

(* Notice that herd has defined its own rf relation *)

(* Define ppo *)
include "riscv-defs.cat"

(********************************)
(* Generate global memory order *)
(********************************)

let gmo0 = (* precursor: ie build gmo as an total order that include gmo0 *)
  loc & (W\FW) * FW | # Final write after any write to the same location
  ppo |               # ppo compatible
  rfe                 # includes herd external rf (optimization)

(* Walk over all linear extensions of gmo0 *)
with  gmo from linearizations(M\IW,gmo0)

(* Add initial writes upfront -- convenient for computing rfGMO *)
let gmo = gmo | loc & IW * (M\IW)

(**********)
(* Axioms *)
(**********)

(* Compute rf according to the load value axiom, aka rfGMO *)
let WR = loc & ([W];(gmo|po);[R])
let rfGMO = WR \ (loc&([W];gmo);WR)

(* Check equality of herd rf and of rfGMO *)
empty (rf\rfGMO)|(rfGMO\rf) as RfCons

(* Atomicity axiom *)
let infloc = (gmo & loc)^-1
let inflocext = infloc & ext
let winside  = (infloc;rmw;inflocext) & (infloc;rf;rmw;inflocext) & [W]
empty winside as Atomic
\end{lstlisting}
  }
  \caption{riscv.cat,RVWMO内存模型的一个Herd版本(2/3)
    % {\tt riscv.cat}, a herd version of the RVWMO memory model (2/3)
    }
  \label{fig:herd2}
\end{figure}

\begin{figure}[h!]
  {
  \tt\bfseries\centering\footnotesize
  \begin{lstlisting}
Partial

(***************)
(* Definitions *)
(***************)

(* Define ppo *)
include "riscv-defs.cat"

(* Compute coherence relation *)
include "cos-opt.cat"

(**********)
(* Axioms *)
(**********)

(* Sc per location *)
acyclic co|rf|fr|po-loc as Coherence

(* Main model axiom *)
acyclic co|rfe|fr|ppo as Model

(* Atomicity axiom *)
empty rmw & (fre;coe) as Atomic
\end{lstlisting}
  }
  \caption{{\tt riscv.cat},RVWMO内存模型的一种备选的Herd表示(3/3)
    % {\tt riscv.cat}, an alternative herd presentation of the RVWMO memory model (3/3)
    }
  \label{fig:herd3}
\end{figure}

