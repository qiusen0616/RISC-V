%=======================================================================
% riscv-spec.tex
%-----------------------------------------------------------------------


\documentclass[twoside,11pt]{book}

% Fix copy/pasting of ligatures in Acrobat
% \input{glyphtounicode.tex}
% \pdfgentounicode=1 %

% Package includes

\usepackage{graphicx}
\usepackage{geometry}
\usepackage{array}
\usepackage{colortbl}
\usepackage[svgnames]{xcolor}

\usepackage[colorlinks,citecolor=Navy,linkcolor=Navy]{hyperref}
\usepackage{placeins}
\usepackage{longtable}
\usepackage{multirow}
\usepackage{float}
\usepackage{listings}
\usepackage{comment}
\usepackage{enumitem}
\usepackage{verbatimbox}
\usepackage{amsmath}

\usepackage[olditem,oldenum]{paralist}
\usepackage[UTF8,heading = true]{ctex}
\usepackage{indentfirst}
\usepackage{pfnote}

% Setup margins

\setlength{\topmargin}{-0.5in}
\setlength{\textheight}{9in}
\setlength{\oddsidemargin}{0in}
\setlength{\evensidemargin}{0in}
\setlength{\textwidth}{6.5in}

% Useful macros

\newcommand{\note}[1]{{\bf [ NOTE: #1 ]}}
\newcommand{\fixme}[1]{{\bf [ FIXME: #1 ]}}
\newcommand{\todo}[1]{\marginpar{\footnotesize #1}}

\newcommand{\wunits}[2]{\mbox{#1\,#2}}
\newcommand{\um}{\mbox{$\mu$m}}
\newcommand{\xum}[1]{\wunits{#1}{\um}}
\newcommand{\by}[2]{\mbox{#1$\times$#2}}
\newcommand{\byby}[3]{\mbox{#1$\times$#2$\times$#3}}

\newlength\savedwidth
\newcommand\whline[1]{%
  \noalign{%
    \global\savedwidth\arrayrulewidth\global\arrayrulewidth 1.5pt%
  }%
  \cline{#1}%
  \noalign{\vskip\arrayrulewidth}%
  \noalign{\global\arrayrulewidth\savedwidth}%
}

% Custom ctex: chapter and section

\ctexset{
chapter/name = {第,章},
chapter/format = { \huge\bfseries },
section/format = { \Large\bfseries },
}

% Custom list environments

\newlist{tightlist}{itemize}{1}
\setlist[tightlist]{label=\textbullet,nosep}

\newenvironment{titledtightlist}[1]
{\noindent
 ~~\textbf{#1}
 \begin{tightlist}}
{\end{tightlist}}

\newenvironment{commentary}
{	\vspace{-1.5mm}
	\list{}{
		\topsep		0mm
		\partopsep	0mm
		\listparindent	1.5em
		\itemindent	\listparindent
		\rightmargin	\leftmargin
		\parsep		0mm
	}
	\item
	\small\em
	\noindent\nopagebreak\rule{\linewidth}{1pt}\par
	\noindent\ignorespaces
}
{\endlist}

%\newenvironment{discussion}
%{	\vspace{-1.5mm}
%	\list{}{
%		\topsep		0mm
%		\partopsep	0mm
%		\listparindent	1.5em
%		\itemindent	\listparindent
%		\rightmargin	\leftmargin
%		\parsep		0mm
%	}
%	\item
%	\small\em
%	\noindent\nopagebreak\rule{\linewidth}{1pt}\par
%	\noindent\textbf{Discussion:}
%}
%{\endlist}

% Other commands and parameters

\pagestyle{myheadings}
\setlength{\parindent}{0in}
\setlength{\parskip}{10pt}
\sloppy
\raggedbottom
\clubpenalty=10000
\widowpenalty=10000

% Commands for register format figures.

% New column types to use in tabular environment for instruction formats.
% Allocate 0.18in per bit.
\newcolumntype{I}{>{\centering\arraybackslash}p{0.18in}}
% Two-bit centered column.
\newcolumntype{W}{>{\centering\arraybackslash}p{0.36in}}
% Three-bit centered column.
\newcolumntype{F}{>{\centering\arraybackslash}p{0.54in}}
% Four-bit centered column.
\newcolumntype{Y}{>{\centering\arraybackslash}p{0.72in}}
% Five-bit centered column.
\newcolumntype{R}{>{\centering\arraybackslash}p{0.9in}}
% Six-bit centered column.
\newcolumntype{S}{>{\centering\arraybackslash}p{1.08in}}
% Seven-bit centered column.
\newcolumntype{O}{>{\centering\arraybackslash}p{1.26in}}
% Eight-bit centered column.
\newcolumntype{E}{>{\centering\arraybackslash}p{1.44in}}
% Ten-bit centered column.
\newcolumntype{T}{>{\centering\arraybackslash}p{1.8in}}
% Twelve-bit centered column.
\newcolumntype{M}{>{\centering\arraybackslash}p{2.2in}}
% Sixteen-bit centered column.
\newcolumntype{K}{>{\centering\arraybackslash}p{2.88in}}
% Twenty-bit centered column.
\newcolumntype{U}{>{\centering\arraybackslash}p{3.6in}}
% Twenty-bit centered column.
\newcolumntype{L}{>{\centering\arraybackslash}p{3.6in}}
% Twenty-five-bit centered column.
\newcolumntype{J}{>{\centering\arraybackslash}p{4.5in}}

\newcommand{\instbit}[1]{\mbox{\scriptsize #1}}
\newcommand{\instbitrange}[2]{~\instbit{#1} \hfill \instbit{#2}~}
\newcommand{\reglabel}[1]{\hfill {\tt #1}\hfill\ }

\newcommand{\wiri}{\textbf{WIRI}}
\newcommand{\wpri}{\textbf{WPRI}}
\newcommand{\wlrl}{\textbf{WLRL}}
\newcommand{\warl}{\textbf{WARL}}

\newcommand{\unspecified}{\textsc{未指定的}}


\newcommand{\ppost}{$b$是一个存储操作,且$a$和$b$访问了重叠的内存地址
  % $b$ is a store, and $a$ and $b$ access overlapping memory addresses
}
\newcommand{\ppofence}{在$b$之前有一个排序$a$的FENCE指令 
% There is a FENCE instruction that orders $a$ before $b$
}
\newcommand{\ppoacquire}{$a$有一个acquire注释
  % $a$ has an acquire annotation
  }
\newcommand{\pporelease}{$b$有一个release注释
  % $b$ has a release annotation
  }
\newcommand{\pporcsc}{$a$和$b$都有RCsc注释
  % $a$ and $b$ both have RCsc annotations
  }
\newcommand{\ppoamoforward}{$a$是由AMO或SC指令生成的操作,$b$是加载操作,且$b$返回一个由$a$写入的值
  % $a$ is generated by an AMO or SC instruction, $b$ is a load, and $b$ returns a value written by $a$
  }
\newcommand{\ppoaddr}{b有一个关于a的句法地址依赖
  % $b$ has a syntactic address dependency on $a$
  }
\newcommand{\ppodata}{$b$有一个关于$a$的句法数据依赖
  % $b$ has a syntactic data dependency on $a$
  }
\newcommand{\ppoctrl}{$b$是一个存储操作,且$b$有一个关于$a$的句法控制依赖
  % $b$ is a store, and $b$ has a syntactic control dependency on $a$
  }
\newcommand{\ppopair}{$a$与$b$是成对的
  % $a$ is paired with $b$
  }
\newcommand{\ppordw}{$a$和$b$是加载操作,$x$是$a$和$b$都读取的一个字节,以程序次序在$a$和$b$之间没有对$x$的存储操作,并且$a$和$b$返回由不同的内存操作所写入的$x$的值
  % $a$ and $b$ are loads, $x$ is a byte read by both $a$ and $b$, there is no store to $x$ between $a$ and $b$ in program order, and $a$ and $b$ return values for $x$ written by different memory operations
  }
\newcommand{\ppoaddrdatarfi}{
  $b$是一个加载操作,且按程序次序,在$a$和$b$之间存在某些存储操作$m$,使得$m$有一个关于$a$的地址依赖或数据依赖,而$b$返回一个由$m$写入的值
  % $b$ is a load, and there exists some store $m$ between $a$ and $b$ in program order such that $m$ has an address or data dependency on $a$, and $b$ returns a value written by $m$
  }
\newcommand{\ppoaddrpo}{$b$是一个存储操作,且按程序次序,在$a$和$b$之间存在某些指令$m$,使$m$有一个关于$a$的地址依赖
  % $b$ is a store, and there exists some instruction $m$ between $a$ and $b$ in program order such that $m$ has an address dependency on $a$
  }
\newcommand{\loadvalueaxiom}{
  每个加载$i$的各个位所返回的值,由下列存储中在全局内存次序中最近的那个写到该位:
  % Each byte of each load $i$ returns the value written to that byte by the store that is the latest in global memory order among the following stores:
  \begin{enumerate}
    \item 写该位,并且在全局内存次序中先于$i$的存储  % Stores that write that byte and that precede $i$ in the global memory order
    \item 写该位,并且在程序次序中先于$i$的存储  % Stores that write that byte and that precede $i$ in program order
  \end{enumerate}
}

\newcommand{\atomicityaxiom}{
  如果$r$和$w$是由一个硬件线程$h$中对齐的LR和SC指令所生成的配对的加载和存储操作,
  $s$是一个对于字节$x$的存储,而$r$返回$s$所写的值,那么在全局内存次序中,$s$必须先于$w$。
  并且在全局内存次序中,在$s$之后、$w$之前,没有来自同一硬件线程的不同于$h$的存储。
  % If $r$ and $w$ are paired load and store operations generated by aligned LR and SC instructions in a hart $h$, 
  % $s$ is a store to byte $x$, and $r$ returns a value written by $s$, then $s$ must precede $w$ in the global memory order, 
  % and there can be no store from a hart other than $h$ to byte $x$ following $s$ and preceding $w$ in the global memory order.
}

\newcommand{\progressaxiom}{
  在全局内存次序之中,任何内存操作之前都不能有其它内存操作的无限序列。
  % No memory operation may be preceded in the global memory order by an infinite sequence of other memory operations.
  }


\newcommand{\specrev}{\mbox{20191214-{\em draft}}}
\newcommand{\specmonthyear}{\mbox{December 2019}}


\begin{document}

\title{\vspace{-0.7in}\Large {\bf RISC-V指令集手册} \\
  \large {\bf 第I卷: 非特权指令集架构} \\
  文档版本 \specrev
  \vspace{-0.1in}}

\author{编者:安德鲁·沃特曼$^{1}$, 克尔斯泰·阿桑诺维奇$^{1,2}$ \\
  $^{1}$SiFive股份有限公司, \\
  $^{2}$加州伯克利分校,电子工程部,计算机科学与技术系\\
  {\tt waterman@eecs.berkeley.edu, krste@berkeley.edu} \\
  \today
}
\date{} 
\maketitle


\setlength{\parindent}{2em}

本规范的所有版本的贡献者如下,以字母顺序排列(请联系编者以提出更改建议):
阿文,克尔斯泰·阿桑诺维奇,里马斯·阿维齐尼斯,雅各布·巴赫迈耶,克里斯托弗·F·巴顿,艾伦·J·鲍姆,亚历克斯·布拉德伯里,斯科特·比默,普雷斯顿·布里格斯,克里斯托弗·塞利奥,张传华,大卫·奇斯纳尔,保罗·克莱顿,帕默·达贝尔特,肯·多克瑟,罗杰·埃斯帕萨,格雷格·福斯特,谢克德·弗勒,斯特凡·弗洛伊德伯格,马克·高希尔,安迪·格鲁,简·格雷,迈克尔·汉伯格,约翰·豪瑟,戴维·霍纳,布鲁斯·霍尔特,比尔·赫夫曼,亚历山大·琼诺,奥洛夫·约翰森,本·凯勒,大卫·克鲁克迈尔,李云燮,保罗·洛文斯坦,丹尼尔·卢斯蒂格,雅廷·曼尔卡,卢克·马兰杰,玛格丽特·马托诺西,约瑟夫·迈尔斯,维贾亚南德·纳加拉扬,里希尔·尼希尔,乔纳斯·奥伯豪斯,斯特凡·奥雷尔,欧伯特,约翰·奥斯特豪特,大卫·帕特森,克里斯托弗·普尔特,何塞·雷诺,乔希·谢德,科林·施密特,彼得·苏厄尔,萨米特·萨卡尔,迈克尔·泰勒,韦斯利·特普斯特拉,马特·托马斯,汤米·索恩,卡罗琳·特里普,雷·范德瓦尔克,穆拉里达兰·维贾亚拉加万,梅根·瓦克斯,安德鲁·沃特曼,罗伯特·沃森,德里克·威廉姆斯,安德鲁·赖特,雷诺·赞迪克,和张思卓。


本文档在知识共享署名4.0国际许可证(Creative Commons Attribution 4.0 International License)下发布。


本文档是《RISC-V指令集手册,第I卷:用户级指令集架构 2.1版本》的衍生版本,
该手册在\copyright 2010--2017 安德鲁·沃特曼,李云燮,大卫·帕特森,克尔斯泰·阿桑诺维奇,知识共享署名4.0国际许可证下发布。 

% Please cite as: ``The RISC-V Instruction Set
% Manual, Volume I: User-Level ISA, Document Version \specrev'', Editors
% Andrew Waterman and Krste Asanovi\'{c}, RISC-V International, \specmonthyear.

引用请使用:“The RISC-V Instruction Set Manual, Volume I: User-Level ISA, Document Version \specrev”,
编者:安德鲁·沃特曼、克尔斯泰·阿桑诺维奇,RISC-V国际,2019年12月。


\markboth{第I卷:RISC-V 非特权指令集架构 V\specrev} % V\sperev 中 这个V是什么意思?
{第I卷:RISC-V 非特权指令集架构 V\specrev}
\thispagestyle{empty}

\frontmatter

\chapter{前言}

% This document describes the RISC-V unprivileged architecture.

本文描述了RISC-V非特权架构。

% The ISA modules marked Ratified have been ratified at this time.  The modules
% marked {\em Frozen} are not expected to change significantly before being put
% up for ratification.  The modules marked {\em Draft} are expected to change
% before ratification.

当前,标记为“被批准”的指令集架构模块,已经被官方批准了。
在被提交批准之前,标记为“{\em 冻结}”的模块,预计不会有重大改变。
在被批准之前,标记为“{\em 草案}”的模块,预计还会有所改变。

% The document contains the following versions of the RISC-V ISA modules:

本文包含以下版本的RISC-V ISA模块:

{
\begin{table}[htb]
  \centering
  \begin{tabular}{|c|l|c|}
    \hline
    基础模块       & 版本 & 状态      \\
    \hline
    RVWMO          & 2.0 & \bf 被批准   \\
    \bf RV32I      & \bf 2.1 & \bf 被批准 \\
    \bf RV64I      & \bf 2.1 & \bf 被批准 \\
    \em RV32E      & \em 1.9 & \em 草案 \\
    \em RV128I     & \em 1.7 & \em 草案 \\
    \hline
    拓展模块       & 版本 & 状态 \\
    \hline
    \bf M          & \bf 2.0 & \bf 被批准 \\
    \bf A          & \bf 2.1 & \bf 被批准 \\
    \bf F          & \bf 2.2 & \bf 被批准 \\
    \bf D          & \bf 2.2 & \bf 被批准 \\
    \bf Q          & \bf 2.2 & \bf 被批准 \\
    \bf C          & \bf 2.0 & \bf 被批准 \\
    \em Counters   & \em 2.0 & \em 草案 \\
    \em L          & \em 0.0 & \em 草案 \\
    \em B          & \em 0.0 & \em 草案 \\
    \em J          & \em 0.0 & \em 草案 \\
    \em T          & \em 0.0 & \em 草案 \\
    \em P          & \em 0.2 & \em 草案 \\
    \em V          & \em 0.7 & \em 草案 \\
    \bf Zicsr      & \bf 2.0 & \bf 被批准 \\
    \bf Zifencei   & \bf 2.0 & \bf 被批准 \\
    \bf Zihintpause & \bf 2.0 & \bf 被批准 \\
    \em Zihintntl   & \em 0.2 & \em 草案 \\
    \em Zam        & \em 0.1 & \em 草案 \\
    \bf Zfh        & \bf 1.0 & \bf 被批准 \\
    \bf Zfhmin     & \bf 1.0 & \bf 被批准 \\
    \bf Zfinx      & \bf 1.0 & \bf 被批准 \\
    \bf Zdinx      & \bf 1.0 & \bf 被批准 \\
    \bf Zhinx      & \bf 1.0 & \bf 被批准 \\
    \bf Zhinxmin   & \bf 1.0 & \bf 被批准 \\
    \bf Zmmul      & \bf 1.0 & \bf 被批准 \\
    \em Ztso       & \em 0.1 & \em 冻结 \\
    \hline
  \end{tabular}
\end{table}
}

%The changes in this version of the document include:
%\vspace{-0.2in}
%\begin{itemize}
%\parskip 0pt
%\itemsep 1pt
%\end{itemize}

\FloatBarrier

\section*{对基于已批准的20191213版本文档的前言}

% This document describes the RISC-V unprivileged architecture.
本文档描述了RISC-V非特权架构。



% The ISA modules marked Ratified have been ratified at this time.  The modules
% marked {\em Frozen} are not expected to change significantly before being put
% up for ratification.  The modules marked {\em Draft} are expected to change
% before ratification.


当前,标记为“被批准”的指令集架构模块,已经被官方批准了。
在被提交批准之前,标记为“{\em 冻结}”的模块,预计不会有重大改变。
在被批准之前,标记为“{\em 草案}”的模块,预计还会有所改变。

% The document contains the following versions of the RISC-V ISA modules:

本文包含以下版本的RISC-V ISA模块:

{
\begin{table}[hbt]
  \centering
  \begin{tabular}{|c|l|c|}
    \hline
    基础模块        & 版本   & 状态\\
    \hline
    RVWMO          & 2.0 & \bf 被批准   \\
    \bf RV32I      & \bf 2.1 & \bf 被批准 \\
    \bf RV64I      & \bf 2.1 & \bf 被批准 \\
    \em RV32E      & \em 1.9 & \em 草案 \\
    \em RV128I     & \em 1.7 & \em 草案 \\
    \hline
    拓展模块      & 状态 & 状态 \\
    \hline
    \bf M          & \bf 2.0 & \bf 被批准 \\
    \bf A          & \bf 2.1 & \bf 被批准 \\
    \bf F          & \bf 2.2 & \bf 被批准 \\
    \bf D          & \bf 2.2 & \bf 被批准 \\
    \bf Q          & \bf 2.2 & \bf 被批准 \\
    \bf C          & \bf 2.0 & \bf 被批准 \\
    \em Counters   & \em 2.0 & \em 草案 \\
    \em L          & \em 0.0 & \em 草案 \\
    \em B          & \em 0.0 & \em 草案 \\
    \em J          & \em 0.0 & \em 草案 \\
    \em T          & \em 0.0 & \em 草案 \\
    \em P          & \em 0.2 & \em 草案 \\
    \em V          & \em 0.7 & \em 草案 \\
    \bf Zicsr      & \bf 2.0 & \bf 被批准 \\
    \bf Zifencei   & \bf 2.0 & \bf 被批准 \\
    \em Zam        & \em 0.1 & \em 草案 \\
    \em Ztso       & \em 0.1 & \em 冻结 \\
    \hline
  \end{tabular}
\end{table}
}

% The changes in this version of the document include:
此版本文档中的变动包括:
\vspace{-0.2in}
\begin{itemize}
\parskip 0pt
\itemsep 1pt
\item 现在是2.1版本的拓展模块A,已经在2019年12月被理事会批准。
\item 定义了大端序的ISA变体。
\item 把用于用户模式中断的N拓展模块移入到卷II中。
\item 定义了暂停提示指令(PAUSE hint instruction)。
\end{itemize}

\FloatBarrier

\section*{对基于已批准的20190608版本文档的前言}

% This document describes the RISC-V unprivileged architecture.  
本文档描述了RISC-V非特权指令集架构。

% The RVWMO memory model has been ratified at this time.  The ISA
% modules marked Ratified, have been ratified at this time.  The modules
% marked {\em Frozen} are not expected to change significantly before
% being put up for ratification.  The modules marked {\em Draft} are
% expected to change before ratification.

此时,RVWMO内存模型已经被批准了。
当前,标记为“被批准”的指令集架构模块,已经被官方批准了。
在被提交批准之前,标记为“{\em 冻结}”的模块,预计不会有重大改变。
在被批准之前,标记为“{\em 草案}”的模块,预计还会有所改变。

% The document contains the following versions of the RISC-V ISA
% modules:

本文包含以下版本的RISC-V ISA模块:

{
\begin{table}[hbt]
  \centering
  \begin{tabular}{|c|l|c|}
    \hline
    基础模块       & 版本   & 状态\\
    \hline
    RVWMO          & 2.0 & \bf 被批准   \\
    \bf RV32I      & \bf 2.1 & \bf 被批准 \\
    \bf RV64I      & \bf 2.1 & \bf 被批准 \\
    \em RV32E      & \em 1.9 & \em 草案 \\
    \em RV128I     & \em 1.7 & \em 草案 \\
    \hline
    拓展模块       &  版本   &     状态    \\
    \hline
    \bf Zifencei   & \bf 2.0 & \bf 被批准 \\
    \bf Zicsr      & \bf 2.0 & \bf 被批准 \\
    \bf M          & \bf 2.0 & \bf 被批准 \\
    \em A          & \em 2.0 &  冻结 \\
    \bf F          & \bf 2.2 & \bf 被批准 \\
    \bf D          & \bf 2.2 & \bf 被批准 \\
    \bf Q          & \bf 2.2 & \bf 被批准 \\
    \bf C          & \bf 2.0 & \bf 被批准 \\
    \em Ztso       & \em 0.1 & \em 冻结 \\
    \em Counters   & \em 2.0 & \em 草案 \\
    \em L          & \em 0.0 & \em 草案 \\
    \em B          & \em 0.0 & \em 草案 \\
    \em J          & \em 0.0 & \em 草案 \\
    \em T          & \em 0.0 & \em 草案 \\
    \em P          & \em 0.2 & \em 草案 \\
    \em V          & \em 0.7 & \em 草案 \\
    \em N          & \em 1.1 & \em 草案 \\
    \em Zam        & \em 0.1 & \em 草案 \\
    \hline
  \end{tabular}
\end{table}
}

% The changes in this version of the document include:
此版本文档中的变化包括:

\vspace{-0.2in}
\begin{itemize}
\parskip 0pt
\itemsep 1pt
\item 将在2019年初被理事会批准的ISA模块的描述,更正为“{\bf 被批准}”的。

\item 从批准的模块中移除A扩展。

\item 变更文档版本方案,以避免与ISA模块的版本冲突。

\item 把基础整数ISA的版本号增加到2.1,以反映:
  被批准的RVWMO内存模型的出现,和先前基础ISA中的FENCE.I、计数器和CSR指令的去除。

\item 把F扩展和D扩展的版本号增加到2.2,以反映:版本2.1更改了规范的NaN
\footnote{译者注:NaN是Not a Number的缩写,是计算机科学中数值数据类型的一类值,
表示未定义或不可表示的值,常在浮点数运算中使用,参见 IEEE 754-1985 浮点数标准};
而版本2.2定义了NaN装箱(NaN-Boxing,意为NaN的表示方式)方案,并更改了FMIN和FMAX指令的定义。

\item 将文档的名字改为“非特权的”指令,以此作为将ISA规范从平台相关的文件中分离的行动之一。

\item 为执行环境、硬件线程\footnote{译者注:硬件线程hart是RISC-V引入的一个新概念,具体含义请参考 RISC-V 规范的第II卷“特权ISA”}、陷入(traps)和内存访问添加了更清晰和更精确的定义。

\item 定义了指令集的种类: {\em 标准的}, {\em 保留的}, {\em 自定义的}, {\em 非标准的}, and {\em 非符合的}。

\item 移除了隐含着交替字节序操作的相关文本,因为交替字节序操作还没有被RISC-V所定义。

\item 修改了对未对齐的load和store行为的描述。规范现在允许执行环境接口中对未对齐地址陷入进行可见的处理,
而不是仅仅在用户模式中授权对未对齐的加载和存储进行不可见处理。
而且,现在允许报告有关未对齐访问(包括原子访问)的访问异常,而不是仅仅模拟。

\item 把FENCE.I从强制性的基础模块中移出,编入一个独立的扩展,名为Zifencei ISA。
FENCE.I曾经被从Linux用户ABI中去除,它在实现大型非一致性指令和数据缓存时是有问题的。
然而,它仍然是仅有的标准的取指一致性机制。

\item 去除了禁止RV32E和其它扩展一起使用的约束。

\item 去除了平台相关的约束条款,即,在RV32E和RV64I章节中,特定编码会产生非法指令异常

\item 计数器/计时器指令现在不被认为是强制性的基础ISA的一部分,因此CSR指令被移动到独立的章节并被标记为2.0版本,
同时非特权计数器被移动到另一个独立的章节。
计数器由于存在明显的问题(包括计数不精确等),所以还没有准备批准。

\item 添加了CSR有序访问模型。

\item 为2位{\em fmt 域}中的浮点指令明确地定义了16位半精度浮点格式。

\item 定义了FMIN.{\em fmt}和FMAX.{\em fmt}的有符号零行为,并改变了它们遇到NaN信号(Signaling-NaN)输入时的行为,
以符合建议的IEEE 754-201x规范中的minimumNumber和maximumNumber操作规范。

\item 定义了内存一致性模型RVWMO。

\item 定义了“Zam”扩展,它允许未对齐的AMO并指定它们的语义。

\item 定义了“Ztso”扩展,它执行比RVWMO更加严格的内存一致性模型。

\item 改善了描述和注释。

\item 定义了术语IALIGN,作为描述指令地址对齐约束的简写。

\item 去除了P扩展章节的内容,因为它现在已经被活跃的任务组文档所取代。

\item 去除了V扩展章节的内容,因为它现在已经被独立的向量扩展草案文档所代替。

\end{itemize}

\FloatBarrier

\section*{对2.2版本文档的前言}

这是文档的2.2版本,描述了RISC-V的用户级架构。文档包括RISC-V ISA模块的如下版本:
\begin{table}[hbt]
  \centering
  \begin{tabular}{|c|l|c|}
    \hline
    基础模块     & \em 版本 & \em 草案被冻结? \\
    \hline
    RV32I    & 2.0 & 是 \\
    RV32E    & 1.9 & 否 \\
    RV64I    & 2.0 & 是 \\
    RV128I   & 1.7 & 否 \\
    \hline
    拓展模块 & 版本 & 被冻结? \\
    \hline
    M        & 2.0 & 是 \\
    A        & 2.0 & 是 \\
    F        & 2.0 & 是 \\
    D        & 2.0 & 是 \\
    Q        & 2.0 & 是 \\
    L        & 0.0 & 否 \\
    C        & 2.0 & 是 \\
    B        & 0.0 & 否 \\
    J        & 0.0 & 否 \\
    T        & 0.0 & 否 \\
    P        & 0.1 & 否 \\
    V        & 0.7 & 否 \\
    N        & 1.1 & 否 \\
    \hline
  \end{tabular}
\end{table}

到目前为止,此标准还没有任何一部分得到RISC-V基金会的官方批准,
但是上面标记有“被冻结”标签的组件在批准处理期间,除了解决规范中的模糊不清和漏洞以外,预计不会再有变化。

此版本文档的主要变更包括:
\begin{itemize}
\parskip 0pt
\itemsep 1pt
\item 此文档的先前版本是最初的作者在知识共享署名4.0国际许可证下发布的,当前版本和未来的版本将在相同的许可证下发布
\item 重新安排了章节,把所有的扩展按规范次序排列。
\item 改进了描述和注释。
\item 修改了关于JALR的隐式提示的建议,以支持LUI/JALR和AUIPC/JALR配对的更高效的宏操作融合。
\item 澄清了关于加载-保留/存储-条件序列的约束。
\item 一个新的控制和状态寄存器(CSR)映射表。
\item 澄清了{\tt fcsr}高位的作用和行为。
\item 改正了对FNMADD.{\em fmt}和FNMSUB.{\em fmt}指令的描述,它们曾经给出了错误的零结果的符号。
\item 指令FMV.S.X和FMV.X.S的语义没有变化,但是为了和语义更加一致,它们被分别重新命名为FMV.W.X和FMV.X.W。
      旧名字仍将继续被工具支持。
\item 规定了在较宽的{\tt f} 寄存器中,使用NaN装箱模型保存(小于FLEN)的浮点值的行为。
\item 定义了FMA的异常行为($\infty$, 0, qNaN)。
\item 添加注释指出,P扩展可能会为了使用整数寄存器进行定点操作,而被重新写入一个整数SMID(packed-SIMD)方案。
\item 一个V向量指令集扩展的草案。
\item 一个N用户级陷入扩展的早期草案。
\item 扩充了伪指令列表。
\item 移除了调用规约章节,它已经被RISC-V ELF psABI规范~\cite{riscv-elf-psabi}所代替。
\item C扩展已经被冻结,并被重新编号为2.0版本。
\end{itemize}

\FloatBarrier

\section*{对2.1版本文档的前言}

这是文档的2.1版本,描述了RISC-V用户级架构。
注意被冻结的2.0版本的用户级ISA基础和扩展IMAFDQ比起本文档的先前版本~\cite{riscvtr2}还没有发生变化,
但是一些规范漏洞已经被修复,文档也被完善了。一些软件的约定已经发生了改变。
\begin{itemize}
\parskip 0pt
\itemsep 1pt
\item 为评注部分做了大量补充和改进。
\item 分割了各章节的版本号。
\item 修改为大于64位的长指令编码,以避免在非常长的指令格式中移动{\em rd}修饰符。
\item CSR指令现在用基础整数格式来描述,并在此引入了计数寄存器,而不只是稍后在浮点部分(和相应的特权架构手册)中引入。
\item SCALL和SBREAK指令已经被分别重命名为ECALL和EBREAK。它们的编码和功能没有变化。
\item 澄清了浮点NaN的处理,并给出了一个新的规范的NaN值。
\item 澄清了浮点到整数溢出时的返回值。
\item 澄清了LR/SC所允许的成功和必要的失败,包括压缩指令在序列中的使用。
\item 一个新的基础ISA提案RV32E,用于减少整数寄存器的数目,它支持MAC扩展。
\item 一个修正的调用约定。
\item 为软浮点调用约定放松了栈对齐,并描述了RV32E调用约定。
\item 一个1.9版本的C压缩扩展的修正提案。
\end{itemize}

\section*{对2.0版本文档的前言}

这是用户ISA规范的第二次发布,而我们试图让基础用户ISA和通用扩展(例如,IMAFD)在未来的发展中保持固定。
这个ISA规范从1.0版本~\cite{riscvtr}开始,已经有了如下改变:

\vspace{-0.1in}
\begin{itemize}
\parskip 0pt
\itemsep 1pt
\item 将ISA划分为一个整数基础模块和一些标准扩展模块。
\item 重新编排了指令格式,让立即编码更加高效。
\item 基础ISA按小字节序内存体系定义,而把大字节序或双字节序作为非标准的变体。
\item 加载-保留/存储-条件(LR/SC)指令已经加入到原子指令扩展中。
\item AMO和LR/SC可以支持释放一致性(release consistency)模型。
\item FENCE指令提供更细粒度的内存和I/O排序。
\item 为fetch-and-XOR(AMOXOR)添加了一个AMO,并修改了AMOSWAP的编码来为它腾出空间。
\item 用AUIPC指令(它向{\tt pc}加上一个20位的高位立即数)取代了RDNPC指令(它只读取当前的{\tt pc}值)。
这帮助我们显著节省了位置无关的代码。
\item JAL指令现在已经被移动到U-Type格式,它带有明确目的寄存器;
J指令被弃用,由{\em rd}={\tt x0}的JAL代替。这样去掉了仅有的一条目的寄存器不明确的指令,
也把J-Type指令格式从基础ISA中移除。这虽然减少了JAL的适用范围,但是会明显减少基础ISA的复杂性。
\item 关于JALR指令的静态提示已经被丢弃。对于符合标准调用约定的代码,这些提示与{\em rd}和{\em rs1}寄存器的修饰符放在一起都是多余的。
\item 现在,JALR指令在计算出目标地址之后,清除了它的最低位,以此来简化硬件、以及允许把辅助信息存储在函数指针中。
\item MFTX.S和MFTX.D指令已经被分别重命名为FMV.X.S和FMV.X.D。类似地,MXTF.S和MXTF.D指令也已经分别被重命名为FMV.S.X和FMV.D.X。
\item MFFSR和MTFSR指令已经被分别重命名为FRCSR和FSCSR。
添加了FRRM、FSRM、FRFLAGS和FSFLAGS指令来独立地访问{\tt fcsr}的两个子域:舍入模式和异常标志位。
\item FMV.X.S和FMV.X.D指令现在从{\em rs1}获得它们的操作数,而不是{\em rs2}了。这个变化简化了数据通路的设计。
\item 添加了FCLASS.S和FCLASS.D浮点分类指令。
\item 采纳了一种更简单的NaN生成和传播方案。
\item 对于RV32I,系统性能计数器已经被扩展到64位宽,且对于高32位和低32位分开进行读取访问。
\item 定义了规范的NOP和MV编码。
\item 为48位、64位和64以上位指令定义了标准指令长度编码。
\item 添加了128位地址空间的变体——RV128的描述。
\item 32位基础指令格式中的主要操作码已经被分配给了用户自定义的扩展。
\item 改正了一个笔误:建议存储从{\em rd}获得它们的数据,已经更正为从{\em rs2}获取。
\end{itemize}
\vspace{-0.1in}


{\hypersetup{linktoc=all,hidelinks}
\tableofcontents
}

\mainmatter

\chapter{介绍}
RISC-V(发音“risk-five”)是一个新的指令集架构(ISA),它原本是为了支持计算机架构的研究和教育而设计的,
但是我们现在希望它也将成为一种用于工业实现的、标准的、免费和开放的架构。
我们在定义RISC-V方面的目标包括:

\vspace{-0.1in}
\begin{itemize}
\parskip 0pt
\itemsep 1pt
\item 一个完全{\em 开放}的ISA,学术界和工业界可以免费获得它。

\item 一个{\em 真实}的ISA,适用于直接的原生的硬件实现,而不仅仅是进行模拟或二进制翻译。

\item 一个对于特定微架构样式(例如,微编码的、有序的、解耦的、乱序的)
  或者实现技术(例如,全定制的、ASIC、FPGA)而言,避免了“过度的架构设计”(译者注:避免采用大而全的复杂微架构,超出了需求),
  但在这些的任何一个中都能高效实现的ISA。

\item 一个ISA被分成两个部分:1、一个{\em 小型}基础整数ISA,其可以用作定制加速器或教育目的的基础;
  2、可选的标准扩展,用于支持通用目的的软件环境。

\item 支持已修订的2008 IEEE-754浮点标准~\cite{ieee754-2008}。

\item 一个支持广泛ISA扩展和专用变体的ISA。

\item 32位和64位地址空间的变体都可以用于应用程序、操作系统内核、和硬件实现。

\item 一个支持高度并行的多核或众核实现(包括异构多处理器)的ISA。

\item 具有可选的{\em 可变长度指令},可以扩展可用的指令编码空间,以及支持可选的{\em 稠密指令编码},
以提升性能、静态编码尺寸和能效。

\item 一个完全虚拟化的ISA,以便简化监控器(Hypervisor)的开发。

\item 一个使新特权架构上的实验被简化的ISA。
\end{itemize}
\vspace{-0.1in}

\begin{commentary}
  我们设计决定的注释将采用像本段这样的格式。如果读者只对规范本身感兴趣,这种非正规的文本可以跳过。
\end{commentary}
\begin{commentary}
  选用RISC-V来命名,是为了表示UC伯克利设计的第五个主要的RISC ISA
  (前四个是RISC-I~\cite{riscI-isca1981}、RISC-II~\cite{Katevenis:1983}、SOAR~\cite{Ungar:1984}
  和SPUR~\cite{spur-jsscc1989})。
  我们也用罗马字母“V”双关表示“变种(variations)”和“向量(vectors)”,
  因为,支持包括各种数据并行加速器在内的广泛的架构研究,是此ISA设计的一个明确的目标。
\end{commentary}


RISC-V ISA的设计,尽可能地避免了实现的细节(尽管注解包含了一些由实现所驱动的决策);
它应当作为具有许多种实现的软件可见的接口来阅读,而不是作为某一特定硬件的定制品的设计来阅读。
RISC-V手册的结构分为两卷。
这一卷覆盖了基本的{\em 非特权(unprivileged)}指令的设计,
包括可选的非特权ISA扩展。
非特权指令是那些在所有权限架构的所有权限模式中,都能普遍可用的指令,
不过其行为可能随着权限模式和权限架构而变化。
第二卷提供了起初的(“经典的”)特权架构的设计。
手册使用IEC 80000-13:2008约定,每个字节有8位。


\begin{commentary}
  在非特权ISA的设计中,我们尝试去除任何依赖于特定微架构的特征,
  例如高速缓存的行(cache line)大小,或者特权架构的细节,
  例如页转换(page translation)。这既是为了简化,也是为选择各种可能的微架构,或各种可能的特权架构保持最大程度的灵活性。
\end{commentary}


\section{RISC-V硬件平台术语}

一个RISC-V硬件平台可以包含:一个或多个兼容RISC-V的处理核心(译者注:后续简称为RISC-V兼容核心或RISC-V核心)
与其它不兼容RISC-V的核心、固定功能的加速器、各种物理内存结构、I/O设备,和一个允许各组件通信的互联结构。

如果某个组件包含了一个独立的取指单元,那么它被称为一个{\em 核心}。
一个兼容RISC-V兼容的核心可以通过多线程,支持多个RISC-V兼容的{\em 硬件线程}(hardware thread,或称为hart)。

RISC-V核心可以有额外的专用指令集扩展,或者一个附加的{\em 协处理器(coprocessor)}。
我们使用术语“{\em 协处理器(coprocessor)}”来指代被接到RISC-V核心的单元。
其大部分时候顺序执行RISC-V指令流,但其还包含了额外的架构状态和指令集扩展,
并且可能保有与主RISC-V指令流相关的一些有限的自主权(译者注:这里指来自协处理器的、独立于主指令流的指令流)。

我们使用术语“{\em 加速器(accelerator)}”来指代一个不可编程的固定功能单元,
或者一个虽然能自主操作但是专用于特定任务的核心。
在RISC-V系统中,我们希望可编程加速器都基于RISC-V、带有专用指令集扩展和/或定制协处理器的核心。
RISC-V加速器的一个重要类别是I/O加速器,它为主应用核心分担了I/O处理任务的负载。

一个RISC-V硬件平台在系统级别的组织多种多样,范围可以从一个单核心微控制器到一个有数千节点
(其中每个节点又都是一个共享内存的众核服务器)的集群。
甚至小型片上系统都可能具有多层的多计算机和/或多处理器的结构,以使开发工作模块化,或者提供子系统间的安全隔离。

\section{RISC-V软件执行环境和硬件线程}

一个RISC-V程序的行为依赖于它所运行的执行环境。
RISC-V执行环境接口(execution environment interface,EEI)定义了:程序的初始状态、环境中的硬件线程
的数量和类型(包括被硬件线程支持的特权模式)、内存和I/O区域的可访问性和属性、
执行在各硬件线程上的所有合法指令的行为(例如,ISA就是EEI的一个组件),以及在包括环境调用在内的执行期间,任何中断或异常的处理。
EEI的例子包括了Linux应用程序二进制接口(ABI),或者RISC-V管理级(译者注:我们建议把supervisor翻译成监管器,后续的hypervisor翻译成超级监管器)二进制接口(SBI)。
一个RISC-V执行环境的实现可以是纯硬件的、纯软件的、或者是硬件和软件的组合。
例如,操作码陷入和软件模拟可以被用于实现硬件里没有提供的功能。执行环境的实际例子包括:


\begin{itemize}
  \item “裸机”(Bare metal)硬件平台:硬件线程直接通过物理处理器线程实现,
  指令对物理地址空间有完全访问权限。这个硬件平台定义了一个从加电复位开始的执行环境。

  \item RISC-V操作系统:通过将用户级硬件线程多路复用到可用的物理处理器线程上,
  以及通过虚拟内存来控制对内存的访问,提供了多个用户级别的执行环境。

  \item RISC-V 监管器(Hypervisor):为宾客(guest)操作系统提供了多个监管器级别(supervisor-level)的执行环境。 

  \item RISC-V 模拟器(RISC-V emulator):例如Spike、QEMU或rv8,它们在一个底层x86系统上模拟RISC-V硬件线程,
  并提供一个用户级别的或者监管器级别的执行环境。

\end{itemize}

\begin{commentary}
  可以考虑将一个裸的硬件平台定义为一个执行环境接口(EEI),
  它由可访问的硬件线程、内存、和其它设备来构成环境,且初始状态是加电复位时的状态。
  通常,大多数软件被设计为使用更抽象的接口,因为EEI越抽象,它所提供的跨不同硬件平台的可移植性越好。
  EEI经常是一层叠着一层的,一个较高层的EEI使用另一个较低层的EEI。
\end{commentary}

从软件在给定执行环境中运行的观点看,硬件线程是一种资源,它在执行环境中自动地获取和执行RISC-V指令。
在这个方面,硬件线程的行为像是一种硬件线程资源——即使被执行环境时分多路复用到了真实的硬件上。
一些EEI支持额外硬件线程的创建和销毁,例如,通过环境调用来派生新的硬件线程。

执行环境有义务确保它的各个硬件线程的最终向前推进(forward progress)。
当硬件线程在执行要明确等待某个事件的机制(例如本规范第二卷中定义的wait-for-interrupt指令)时,
该责任被挂起;当硬件线程终止时,该责任结束。硬件线程的向前推进是由下列事件构成的:

\vspace{-0.2in}
\begin{itemize}
\parskip 0pt
\itemsep 1pt
\item 一个指令的引退(retirement)。
\item 一个陷入,就像~\ref{sec:trap-defn}节中定义的那样。
\item 由组成向前推进的扩展所定义的任何其它事件。
\end{itemize}

\begin{commentary}

术语“硬件线程(hart)”的引入是在Lithe~\cite{lithe-pan-hotpar09,lithe-pan-pldi10}上的工作中,
是为了提供一个表示一种抽象的执行资源的术语,作为与软件线程编程抽象的对应。

硬件线程(hart)与软件线程上下文之间的重要区别是:
运行在执行环境中的软件不负责引发执行环境的各硬件线程的推进;那是外部执行环境的责任。
因此,从执行环境内部软件的观点看,环境的hart的操作就像硬件的线程一样。

一个执行环境的实现可能将一组宾客硬件线程(guest hart),时间多路复用到由它自己的
执行环境提供的更少的宿主硬件线程(host hart)上,
但是这种做法必须以一种“宾客硬件线程像独立的硬件线程那样操作”的方式进行。
特别地,如果宾客硬件线程比宿主硬件线程更多,那么执行环境必须有能力抢占宾客硬件线程,
而不是必须无限等待宾客硬件线程上的宾客软件来“让步(yield)”对宾客硬件线程的控制。

\end{commentary}

\section{RISC-V ISA 概览}

RISC-V ISA被定义为一个基础的整数ISA(在任何实现中都必须出现)和一些对基础ISA的可选的扩展。
基础整数ISA非常类似于早期的RISC处理器,除了没有分支延迟槽,但支持可选的变长指令编码。
“基础”是被小心地限制在足以为编译器、汇编器、链接器、和操作系统(带有额外特权操作)
提供合理目标的一个最小的指令集合的范围内,
提供了一个便捷的ISA和软件工具链“骨架”,可以围绕它们来构建更多定制的处理器ISA。

尽管使用“RISC-V ISA”这个词汇很方便,但其实RISC-V是一系列相关ISA的ISA族,族中目前有四个基础ISA。
每个基础整数指令集由不同的整数寄存器宽度、对应的地址空间尺寸和整数寄存器数目作为特征。
在第~\ref{rv32}章和第~\ref{rv64}章描述了两个主要的基础整数变体,RV32I和RV64I,
它们分别提供了32位和64位的地址空间。
我们使用术语“XLEN”来指代一个整数寄存器的位宽(32或者64位)。
第~\ref{rv32e}章描述了RV32I基础指令集的子集变体:RV32E,它已经被添加来支持小型微控制器,具有一半数目的整数寄存器。
第~\ref{rv128}章概述了基础整数指令集的一个未来变体RV128I,它将支持扁平的128位地址空间(XLEN=128)。
基础整数指令集使用补码来表示有符号的整数值。

\begin{commentary}

尽管64位地址空间是更大的系统的需求,但我们相信在接下来的数十年里,32位地址空间仍然适合许多嵌入式和客户端设备,
并有望能够降低内存流量和能量消耗。此外,32位地址空间对于教育目的是足够的。
更大的扁平128位地址空间,也许最终会需要,因此我们要确保它能被容纳到RISC-V ISA框架之中。
\end{commentary}

\begin{commentary}

RISC-V中的四个基础ISA被作为不同的基础ISA对待。一个常见的问题是,为什么没有一个单一的ISA?
甚至特别地,为什么RV32I不是RV64I的一个严格的子集?
一些早期的ISA设计(SPARC、MIPS)为了支持已有的32位二进制在新的64位硬件上运行,
在增加地址空间大小的时候就采用了严格的超集策略。

明确地将基础ISA分离的主要优点在于,每个基础ISA可以按照自己的需求而优化,而不需要支持其他基础ISA需要的所有操作。
例如,RV64I可以忽略那些只有RV32I才需要的、处理较窄寄存器的指令和CSR。
RV32I变体则可以使用那些在更宽地址空间变体中需要留给指令的编码空间。

没有作为单一ISA设计的主要缺点是,它使在一个基础ISA上模拟另一个时所需的硬件复杂化(例如,在RV64I上模拟RV32I)。
然而,地址和非法指令陷入方面的不同总体上意味着,在任何时候(即使是完全的超集指令编码),硬件也将需要进行一些模式的切换;
而不同的RISC-V基础ISA是足够相似的,支持多个版本的成本相对较低。
虽然有些人已经提出,严格的超集设计将允许将遗留的32位库链接到64位代码,
但是由于软件调用约定和系统调用接口的不同,即使是兼容编码,这在实践中也是不实际的。

RISC-V权限架构提供了{\tt misa}中的域,用以在各级别控制非特权ISA,来支持在相同的硬件上模拟不同的基础ISA。
我们注意到,较新的SPARC和MIPS ISA修订版已经弃用不经改变就在64位系统上支持运行32位代码了。

一个相关的问题是,为什么32位加法对于RV32I(ADD)和RV64I(ADDW)有不同的编码?
ADDW操作码应当被用于RV32I中的32位加法,而ADDD应当被用于RV64I中的64位加法,而不是像现有设计这样,
将相同的操作码ADD用于RV32I中的32位加法和RV64I中的64位加法、
却将一个不同的操作码ADDW用于RV64I中的32位加法。
这也将与在RV32I和RV64I中对32位加载使用相同的LW操作码的做法保持一致性。
RISC-V ISA的最早版本的确有这种替代的设计,但是在2011年1月,RISC-V的设计变成了如今的选择。
我们的关注点在于在64位ISA中支持32位整数,而不在于提供对32位ISA的兼容性;
并且动机是消除RV32I中,并非所有操作码都有“*W”后缀所引起的不对称性(例如,有ADDW,但是AND没有ANDW)。
事后来看,同时设计两个ISA,而不是先设计一个再于其上追加设计另一个,作为如此做法的结果,这可能是不合适的;
而且,出于我们必须把平台的需求折进ISA规范之中的信条,那意味着在RV64I中将需要所有的RV32I的指令。
虽然现在改变编码已经太晚了,但是由于上述原因,这也几乎没有什么实际后果。

我们也能够将*W变体作为RV32I系统的一个扩展启用,以提供一种跨RV64I和未来RV32变体的常用编码
\end{commentary}

RISC-V已经被设计为支持广泛的定制和专用化。
每个基础整数ISA可以加入一个或多个可选的指令集进行扩展。一个扩展可以被归类为标准的、自定义的,或者不合规的。
出于这个目的,我们把每个RISC-V指令集编码空间(和相关的编码空间,例如CSR)划分为三个不相交的种类:
{\em 标准}、{\em 保留}、和{\em 自定义}。标准扩展和编码由RISC-V国际定义;
任何不由RISC-V国际定义的扩展都是{\em 非标准的}的。每个基础ISA及其标准扩展仅使用标准编码,并且在它们使用这些编码时不能相互冲突。
保留的编码当前还没有被定义,是省下来用于未来的标准扩展的;一旦如此使用,它们将变为标准编码。
自定义编码应当永远不被用于标准扩展,而是可用于特定供应商的非标准扩展。
非标准扩展或者是仅使用自定义编码的自定义扩展,或者是使用了任何标准或保留编码的{\em 非合规}的扩展。
指令集扩展一般是共享的,但是根据基础ISA的不同,也可能提供稍微不同的功能。
第~\ref{extensions}章描述了扩展RISC-V ISA的各种方法。我们也已经为基于RISC-V的指令和指令集开发了一套命名约定,
那将在第~\ref{naming}章进行详细的描述。


为了支持更一般的软件开发,RISC-V定义了一组标准扩展来提供整数乘法/除法、原子操作、和单精度与双精度浮点运算。
基础整数ISA被命名为“I”(根据整数寄存器的宽度配以“RV32”或“RV64”的前缀),
它包括了整数运算指令、整数加载、整数存储、和控制流指令。
标准整数乘法和除法扩展被命名为“M”,并添加了对整数寄存器中的值进行乘法和除法的指令。
标准原子指令扩展(用“A”表示)添加了对内存进行原子读、原子修改、和写内存的指令,用于处理器间的同步。
标准单精度浮点扩展(表示为“F”)添加了浮点寄存器、单精度运算指令,和单精度的加载和存储。
标准双精度浮点扩展(表示为“D”)扩展了浮点寄存器,并添加了双精度运算指令、加载、和存储。
标准“C”压缩指令扩展为通常的指令提供了较窄的16位形式。

在基础整数ISA和这些标准扩展之外,我们相信很少还会有新的指令对所有应用都将提供显著的益处,
尽管它也许对某个特定的领域很有帮助。随着对能效的关注迫使更加的专业化,
我们相信简化一个ISA规范中所必需的部分是很重要的。尽管其它架构通常把它们的ISA视为一个单独的实体,
这些ISA随着时间的推移、指令的添加,而变成一个新的版本;
RISC-V则将努力保持基础和各个标准扩展自始至终的恒定性,
新的指令改为作为未来可选的扩展分层。
例如,不管任何后续的扩展如何,基础整数ISA都将继续作为独立的ISA被完全支持。

\section{内存}

一个RISC-V硬件线程有共计$2^{\text{XLEN}}$字节的单字节可寻址空间,可用于所有的内存访问。
内存的一个“{\em 字(word)}”被定义为\wunits{32}{位}
(\wunits{4}{字节})。对应地,一个“{\em 半字(halfword)}”是\wunits{16}{位}
(\wunits{2}{字节}),一个“{\em 双字(doubleword)}”\wunits{64}{位}
(\wunits{8}{字节}),而一个“{\em 四字(quadword)}”是\wunits{128}{位}
(\wunits{16}{字节})。
内存地址空间是环形的,所以位于地址$2^{\text{XLEN}}-1$的字节与位于地址零的字节是相邻的。
因此,硬件进行内存地址计算时,忽略了溢出,代之以按模$2^{\text{XLEN}}$环绕。

执行环境决定了硬件资源到硬件线程地址空间的映射。
一个硬件线程的地址空间可以有不同地址范围,它可以是(1)空白的,或者(2)包含{\em 主内存},
或者(3)包含一个或多个{\em I/O设备}。
I/O设备的直接读写会造成可见的副作用,但是访问主内存不会。
虽然执行环境有可能把硬件线程地址空间中的所有内容都称作I/O设备,但是通常都会把某些部分指定为主内存。

当一个RISC-V平台有多个硬件线程时,任意两个硬件线程的地址空间可以是完全相同的,或者完全不同的,
或者可以有部分不同但共享资源的一些子集,而这些资源被映射到相同或不同的地址范围。

\begin{commentary}

  对于一个纯粹的“裸机”环境,所有的硬件线程可以看到一个完全相同的地址空间,完全由物理地址进行访问。
  然而,当执行环境包含了带有地址转换的操作系统,通常会给每个硬件线程一个虚拟的地址空间,此空间很大程度上、或者完全就是线程自己的。
\end{commentary}

执行每个RISC-V机器指令涉及了一次或多次内存访问,这进一步可划分为{\em 隐式}和{\em 显式}访问。
对于每个被执行的指令,进行一次{\em 隐式}内存读(指令获取)是为了获得已编码指令进行执行。
许多RISC-V指令在指令获取之外不再进一步地访问内存。在由该指令决定的地址处,有专门的加载指令和存储指令对内存进行{\em 显式}的读或写。
执行环境可能要求指令执行除了非特权ISA所文档化的访问之外的其他{\em 隐式}内存访问(例如进行地址转换)。

执行环境决定了各种内存访问操作可以访问非空地址空间的哪些部分。
例如,可以被取指操作隐式读到的位置集合,可能与那些可以被加载(load)指令操作显式读到的位置集合有交叠;
以及,可以被存储(store)指令操作显式写到的位置集合,可能只是能被读到的位置集合的一个子集。
通常,如果一个指令尝试访问的内存位于一个不可访问的地址处,将因为该指令引发一个异常。
地址空间中的空白位置总是不可访问的。

除非特别说明,否则,不引发异常的隐式读可能会任意提前地、试探地发生,甚至是在机器能够证明的确需要读之前发生。
例如,一种合法实现方式是,可能会尝试第一时间读取所有的主内存,缓存尽可能多的可获取(可执行)字节以供之后的指令获取,
以及避免为了指令获取而再次读主内存(译者注:即通常所说的指令预取)
为了确保某些隐式读只在写入相同内存位置之后是有序的,软件必须执行为此目的而定义的、特定的屏障指令或缓存控制指令
(例如第~\ref{chap:zifencei}章里定义的FENCE.I指令)。

由一个硬件线程发起的内存访问(隐式或显式),在被另一个硬件线程、或者任何其它可访问相同内存的代理线程所感知时,
可能看起来像是以一种不同的顺序发生的。然而,这个被感知到的内存访问重新排序总是受到特定的内存一致性模型的约束。
用于RISC-V的默认的内存一致性模型是RISC-V弱内存排序(RVWMO),定义在第~\ref{ch:memorymodel}章和附录中。
也可以采用更强的模型:全存储排序(Total Store Ordering),定义在第~\ref{sec:ztso}章中。
执行环境也可以添加约束,进一步限制的可感知的内存访问的重排。
由于RVWMO模型是被任何RISC-V实现所允许的最弱的模型,用这个模型写出的软件兼容所有RISC-V实现的实际的内存一致性规则。
与隐式读一样,除非假定的内存一致性模型和执行环境有其他特别需求,否则软件必须执行屏障或缓存控制指令来确保特定顺序的内存访问。

\section{基础指令长度编码}

基础RISC-V ISA有固定长度的32位指令,必须在32位边界上自然地对齐。
然而,标准RISC-V编码策略被设计为支持具有可变长度指令的ISA扩展指令,
每条指令在长度上可以是任意数目的16位指令的{\em 封装包(parcel)},指令封装包在16位边界自然对齐。
第~\ref{compressed}章中描述的标准压缩ISA扩展(译者注:即C扩展)减少了代码尺寸,通过提供压缩的16位指令,以及放松了对齐的限制,
允许所有的指令(16位和32位)在任意16位边界上对齐,而提升了代码的密度。

我们使用术语“IALIGN”(以位为单位)来表示实现层面所采用的指令空间对齐约束。在基础ISA中,IALIGN是32位。
但是在某些ISA扩展中,包括在压缩ISA扩展中,将IALIG是宽松的16位。IALIGN不能取除了16和32以外的任何其它值。

我们使用术语“ILEN”(以位为单位)来表示实现层面所支持的最大指令长度,它总是IALIGN的倍数。
对于只支持一个基础指令集的实现,ILEN是32位。支持更长指令的具体实现架构也就有更大的ILEN值。

图~\ref{instlengthcode}描绘了标准RISC-V指令长度编码约定。
基础ISA中的所有的32位指令都把它们的最低二位设置为“{\tt 11}”。
而可选的压缩16位指令集扩展,它们的最低二位等于“{\tt 00}”、“{\tt 01}”、或“{\tt 10}”。

\subsection*{拓展的指令长度编码}

32位指令编码空间的一部分已经被初步分配给了长度超过32位的指令。
目前这片空间的整体是被保留的,而且下面的关于超过32位编码的提议并没有被认为已被冻结。

带有超过32位编码的标准指令集扩展将额外的若干低序位设置为1(即图~\ref{instlengthcode}中的bbb=111,
关于48位和64位长度的约定如图~\ref{instlengthcode}所示。
指令长度在80位到176位之间的,使用[14:12]中3位(即图~\ref{instlengthcode}中的nnn)来编码,
并给出除最先的5$\times$16位字(80位字)以外的16位的字的数目(即图~\ref{instlengthcode}中的nnn的实际值)。
位[14:12]被设置为“{\tt 111}”的编码被保留,用于未来更长的指令编码。


\begin{figure}[hbt]
{
\begin{center}
\begin{tabular}{ccccl}
\cline{4-4}
& & & \multicolumn{1}{|c|}{\tt xxxxxxxxxxxxxxaa} & 16位 ({\tt aa}
$\neq$ {\tt 11})\\
\cline{4-4}
\\
\cline{3-4}
& & \multicolumn{1}{|c|}{\tt xxxxxxxxxxxxxxxx}
& \multicolumn{1}{c|}{\tt xxxxxxxxxxxbbb11} & 32位 ({\tt bbb}
$\neq$ {\tt 111}) \\
\cline{3-4}
\\
\cline{2-4}
\hspace{0.1in} 
& \multicolumn{1}{c|}{$\cdot\cdot\cdot${\tt xxxx} }
& \multicolumn{1}{c|}{\tt xxxxxxxxxxxxxxxx}
& \multicolumn{1}{c|}{\tt xxxxxxxxxx011111} & 48位 \\
\cline{2-4}
\\
\cline{2-4}
\hspace{0.1in} 
& \multicolumn{1}{c|}{$\cdot\cdot\cdot${\tt xxxx} }
& \multicolumn{1}{c|}{\tt xxxxxxxxxxxxxxxx}
& \multicolumn{1}{c|}{\tt xxxxxxxxx0111111} & 64位 \\
\cline{2-4}
\\
\cline{2-4}
\hspace{0.1in} 
& \multicolumn{1}{c|}{$\cdot\cdot\cdot${\tt xxxx} }
& \multicolumn{1}{c|}{\tt xxxxxxxxxxxxxxxx}
& \multicolumn{1}{c|}{\tt xnnnxxxxx1111111} & (80+16*{\tt nnn})位,
       {\tt nnn}$\neq${\tt 111} \\
\cline{2-4}
\\
\cline{2-4}
\hspace{0.1in} 
& \multicolumn{1}{c|}{$\cdot\cdot\cdot${\tt xxxx} }
& \multicolumn{1}{c|}{\tt xxxxxxxxxxxxxxxx}
& \multicolumn{1}{c|}{\tt x111xxxxx1111111} & 保留用于 $\geq$192位 \\
\cline{2-4}
\\
字节地址: & \multicolumn{1}{r}{base+4} & \multicolumn{1}{r}{base+2} & \multicolumn{1}{r}{base} & \\
 \end{tabular}
\end{center}
}
\caption{RISC-V 指令长度编码。 当前只有16位和32位编码已被冻结。}
\label{instlengthcode}
\end{figure}

\begin{commentary}

  考虑到压缩格式的代码尺寸和节能效果,我们希望在ISA编码策略中构建对压缩格式的支持,
  而不是事后才想起添加它;但是为了允许更简单的实现,我们不想强制规定压缩格式。
  我们也希望允许更长的指令,以支持一些实验和更大的指令集扩展。
  尽管我们这种编码约定要求更严格的核心RISC-V ISA编码,但是这样做收益良多。

  一个标准IMAFD ISA的实现只需要在指令缓存中持有最主要的30位(节省了6.25\%)。在
  指令缓存重新填充时,任何遇到有低位被清除的指令,应当在存进缓存之前,重新编码为非法的30位指令,以保持非法指令异常的行为。

  也许更重要的是,通过把我们的基础ISA凝炼成32位指令字的子集,
  我们为非标准的和自定义的指令集扩展留出了更多可用的空间。
  特别地,基础RV32I ISA在32位指令字中使用少于1/8的编码空间。
  正如第\ref{extensions}章中描述的那样,一个不需要支持标准压缩指令扩展的实现,
  可以将3个额外的不一致的30位指令空间映射到32位固定宽度格式,同时保留对$\geq$32位标准指令集扩展的支持。
  甚至,如果实现层面也不需要长度$>$32位的指令,它可以把这些不一致扩展恢复成另外四种主要的操作码。
\end{commentary}

位[15:0]都是0的编码被定义为非法指令。如果存在任何16位指令集扩展,
则这些指令被认为具有最小的长度:16位,否则是32位。位[ILEN-1:0]都是1的编码也是非法的;这个指令的长度被认为是ILEN位。

\begin{commentary}

我们认为有一个特征是,所有位都是“0”的任意长度的指令都是不合法的,
因为这很快会让陷入处理错误地跳转到零内存区域。类似地,我们也保留了包含所有“1”的指令编码作为非法指令,
以捕获在无编程的非易失性内存设备、断连的内存总线、或者断开的内存设备上通常观测到的出错模式。

在所有的RISC-V实现上,软件可以依靠将一个包含“0”的自然对齐的32位字作为一个非法指令,
以供明确需要非法指令的软件使用。由于可变长度编码,定义一个相应的全是“1”的已知非法值是更加困难的。
软件不能一般地使用ILEN位全是“1”的非法值,因为软件可能不知道最终的目标机器的ILEN
(例如,如果软件被编译为一个用于许多不同机器的标准二进制库)。我们也考虑了定义一个全是“1”的32位字作为非法指令,
因为所有的机器必须支持32位指令尺寸,但是这需要在ILEN$>$32的机器上的指令获取单元报告一个非法指令异常,
而不是在这种指令接近保护边界时报告一个访问故障异常,让可变指令长度的取指和解码变得复杂。

\end{commentary}

RISC-V基础ISA既有小字节序的内存系统,也有大字节序的内存系统,后者需要特权架构进一步定义大字节序的操作。
不论内存系统的字节序如何,指令都作为16位小字节序的封装包的序列被存储在内存中
。构成一个指令的封装包被存储在地址递增的半字地址处,封装包的最低地址持有指令规范中指令的最低若干位。

\begin{commentary}

我们最初为RISC-V内存系统选择小字节序的字节次序,因为小字节序系统当前在商业上占主导
(所有的x86系统;iOS、安卓,和用于ARM的Windows)。
一个小问题是,我们已经发现,小字节序内存系统对于硬件设计者更加自然。
但是,考虑到特定的应用领域(例如IP网络)在大字节序数据结构上的操作,
以及基于大字节序处理器构建的特定遗留代码,所以我们也已经定义了RISC-V的大字节序和双字节序变体。

我们不得不固定指令封装包在内存中存储的顺序,而且不依赖于内存系统的字节序,
来确保长度编码位始终以半字地址顺序首先出现。这允许取指单元通过只检查第一个16位指令包的最初几位,就快速决定可变长度指令的长度。

我们更进一步地把指令封装包本身做成小字节序的,以便从内存系统字节序中把指令编码完全解耦出来。
这个设计对软件工具和双字节序硬件都有好处。否则,例如一个RISC-V汇编器或反汇编器将总是需要预先知道当前运行系统的字节序,
尽管在双字节序系统中,字节序的模式可能在执行期间动态变化。
与之相反,通过给定指令一个固定的字节序,有时可以让编写软件无需感知字节序,甚至是以二进制形式的软件,就像位置无关代码(PIC)一样。

然而,对于编码或解码机器指令的RISC-V软件来说,选择只有小字节序的指令的确会有后果。
例如,大字节序的JIT编译器在向指令内存处执行存储操作的时候,必须交换字节的次序。

一旦我们已经决定了固定为小字节序指令编码,这将自然地导致把长度编码位放置在指令格式的LSB\footnote{译者注:Least Significant Bit,最低有效位}的位置,以避免打断操作码域。
\end{commentary}

\section{异常、陷入和中断}
\label{sec:trap-defn}

我们使用术语“{\em 异常(Exception)}”来指代一种发生在运行时的不正常的状况,
它与当前RISC-V硬件线程中的一条指令相关联。我们使用术语“{\em 中断(Interrupt)}”来指代一种外部的异步事件,
它可能导致一个RISC-V硬件线程经历一次意料之外的控制转移。
我们使用术语“{\em 陷入(Trap)}”来指代由一个异常或中断引发的将控制权转移到陷入处理程序的过程。

下面的章节中的指令描述了在指令执行期间可以引发异常的条件。
大多数RISC-V EEI的通常行为是,当在一个指令上发出异常的信号时,会发生一次到某些处理程序的陷入
(标准浮点扩展中的浮点异常除外,那些并不引起陷入)。硬件线程产生中断、中断路由、和中断启用的具体方式依赖于EEI。


\begin{commentary}
  
  我们使用的“异常”和“陷入”概念与IEEE-754浮点标准中的相兼容。
\end{commentary}

陷入是如何处理的,以及对运行在硬件线程上的软件的可见性如何,依赖于外围的执行环境。
从运行在执行环境内部的软件的视角,在运行时遭遇硬件线程的陷入将有四种不同的影响:
\begin{description}
  \item[被控制的陷入:] 这种陷入对于运行在执行环境中的软件可见,并由软件处理。
  例如,在一个于硬件线程上同时提供监管器模式和用户模式的EEI中,
  用户模式硬件线程的ECALL通常将导致控制被转移到运行在相同硬件线程上的一个监管器模式的处理程序。
  类似地,在相同的环境中,当一个硬件线程被中断,硬件线程上将运行一个监管器模式中的中断处理程序。
  \item[被请求的陷入:] 这种陷入是一个同步的异常,它是对执行环境的一种显式调用,
  请求了一个代表执行环境内部的软件的动作。一个例子便是系统调用。
  在这种情况下,执行环境采取了被请求的动作后,硬件线程上的执行可能继续,也可能不会继续。
  例如,一个系统调用可以移除硬件线程,或者引起整个执行环境的有序终止。
  \item[不可见的陷入:] 这种陷入被执行环境透明地处理了,并且在陷入被处理之后,执行正常继续。
  例子包括模拟缺失的指令、在按需分页的虚拟内存系统中处理非常驻页故障,或者在多程序机器中为不同的事务处理设备中断。
  在这些情况中,运行在执行环境中的软件不会意识到陷入(我们忽略了这些定义中的时间影响)。
  \item[致命的陷入:] 这种陷入代表了一个致命的失败,并引发执行环境终止执行。
  例子包括虚拟内存页保护检查的失败,或者看门狗\footnote{译者注:watchdog,一种用于访问保护的装置}计时器到期。
  每个EEI应当定义执行应如何被终止,以及如何将其汇报给外部环境。
\end{description}

表~\ref{table:trapcharacteristics} 显示了每种陷入的特点:

\begin{table}[hbt]
  \centering
  \begin{tabular}{|l|c|c|c|c|}
      \hline
                & 被控制的 & 被请求的 & 不可见的 & 致命的\\
      \hline
      执行终止   & 否     & 否$_{1}$ & 否  & 是 \\
      软件被遗忘 & 否     & 否       & 是 & 是$_{2}$ \\
      由环境处理 & 否     & 是       & 是 & 是 \\
      \hline
  \end{tabular}
  \caption{陷入的特点。 注:1) 可以被请求终止. 2) 不精确的致命的陷入或许可被软件观测到。}
\label{table:trapcharacteristics}
\end{table}

EEI为每个陷入定义了它是否会被精确处理,尽管通常建议是尽可能地保持精确处理。
被控制的陷入和被请求的陷入可以被执行环境内部的软件观测到是不精确的。
不可见的陷入,根据定义,不能被运行在执行环境内部的软件观测到是否精确。
致命陷入可以被运行在执行环境内部的软件观测到不精确,如果已知错误的指令没有引起直接的终止的话。

因为这篇文档描述了非特权指令,所以陷入是很少被提及的。
处理包含陷入的架构性方法被定义在特权架构手册中,伴有支持更丰富EEI的其它特征。
这里只记录了被单独定义的引发请求陷入的非特权指令。
根据不可见的陷入的性质,其超出了这篇文档的讨论范围。
没有在本文档中定义的指令编码,和没有被一些其它方式定义的指令编码,可以引起致命陷入。

\section{“未指定的”(UNSPECIFIED)行为和值}

架构完全描述了架构必须做的事和任何关于它们可能做的事的约束。
对于那些架构有意不约束实现的情况,会显式地使用术语“\unspecified”。

术语“\unspecified”指代了一种有意不进行约束的行为或值。
这些行为或值对于扩展、平台标准或实现是开放的。
对于基础架构定义为“\unspecified”的情形,扩展、平台标准或实现文档可以提供规范性内容以进一步约束。

像基础架构一样,扩展架构应当完全描述清楚所允许的行为和值,并使用术语“\unspecified”用于有意不做约束的情况。
对于这种情况,就可以被其它的扩展、平台标准或实现来约束或定义。

\chapter{RV32I 基础整数指令集,2.1版本}
\label{rv32}

本章介绍 RV32I 基础整数指令集。

\begin{commentary}

  RV32I的设计要足以能够形成一个编译器的目标码,并足以能够支持现代操作系统环境。
  该ISA也被设计为在最小化的实现中降低对硬件的需求。RV32I包含40条各不相同的指令——尽管在某个简单的实现中,
  可能会用一个总是陷入的SYSTEM硬件指令来覆盖ECALL/EBREAK指令,以及可能会把FENCE指令实现为一个NOP,
  从而把基础指令数目减少到总计38条。RV32I可以模拟几乎任何其它的ISA扩展(除了A扩展,因为它需要原子性的额外硬件支持)

  实际上,一个包含了机器模式(machine-mode)特权架构的硬件实现还将需要6个CSR指令。

  对于教学目的来说,基础整数ISA的子集可能是非常有用的,
  但是“基础”已经定义了,应当尽量不要在一个真实的硬件实现中对基础整数ISA进行子集化——除非想省略掉对非对齐内存访问的支持,
  或者想把所有的SYSTEM指令视为一个单独的陷入。
\end{commentary}

\begin{commentary}

标准RISC-V汇编语言语法的文档在《汇编程序员手册》~\cite{riscv-asm-manual}中。
\end{commentary}

\begin{commentary}

大多数对RV32I的注解也适用于RV64I基础指令集。
\end{commentary}

\section{基础整数 ISA 的编程模型}

表~\ref{gprs}显示了基础整数ISA的非特权状态。
对于RV32I,32个{\tt x}寄存器每个都是32位宽,也就是说,XLEN=32。
寄存器{\tt x0}的所有位都被硬布线为0。通用目的寄存器{\tt x1}-{\tt x31}持有数值,
这些值被各种指令解释为各种布尔值的集合、或者二进制有符号整数,或者无符号整数的二补码。

还有一个额外的非特权寄存器:程序计数器{\tt pc},保存当前指令的地址。

\begin{figure}[H]
{\footnotesize
\begin{center}
\begin{tabular}{p{2in}}
\instbitrange{XLEN-1}{0}                                  \\ \cline{1-1}
\multicolumn{1}{|c|}{\reglabel{\ \ \ \ \ \ x0 / zero}}      \\ \cline{1-1}
\multicolumn{1}{|c|}{\reglabel{\ \ \ \ x1\ \ \ \ \ }}            \\ \cline{1-1}
\multicolumn{1}{|c|}{\reglabel{\ \ \ \ x2\ \ \ \ \ }}       \\ \cline{1-1}
\multicolumn{1}{|c|}{\reglabel{\ \ \ \ x3\ \ \ \ \ }}       \\ \cline{1-1}
\multicolumn{1}{|c|}{\reglabel{\ \ \ \ x4\ \ \ \ \ }}       \\ \cline{1-1}
\multicolumn{1}{|c|}{\reglabel{\ \ \ \ x5\ \ \ \ \ }}       \\ \cline{1-1}
\multicolumn{1}{|c|}{\reglabel{\ \ \ \ x6\ \ \ \ \ }}       \\ \cline{1-1}
\multicolumn{1}{|c|}{\reglabel{\ \ \ \ x7\ \ \ \ \ }}       \\ \cline{1-1}
\multicolumn{1}{|c|}{\reglabel{\ \ \ \ x8\ \ \ \ \ }}       \\ \cline{1-1}
\multicolumn{1}{|c|}{\reglabel{\ \ \ \ x9\ \ \ \ \ }}       \\ \cline{1-1}
\multicolumn{1}{|c|}{\reglabel{\ \ \ x10\ \ \ \ \ }}        \\ \cline{1-1}
\multicolumn{1}{|c|}{\reglabel{\ \ \ x11\ \ \ \ \ }}        \\ \cline{1-1}
\multicolumn{1}{|c|}{\reglabel{\ \ \ x12\ \ \ \ \ }}        \\ \cline{1-1}
\multicolumn{1}{|c|}{\reglabel{\ \ \ x13\ \ \ \ \ }}        \\ \cline{1-1}
\multicolumn{1}{|c|}{\reglabel{\ \ \ x14\ \ \ \ \ }}        \\ \cline{1-1}
\multicolumn{1}{|c|}{\reglabel{\ \ \ x15\ \ \ \ \ }}        \\ \cline{1-1}
\multicolumn{1}{|c|}{\reglabel{\ \ \ x16\ \ \ \ \ }}        \\ \cline{1-1}
\multicolumn{1}{|c|}{\reglabel{\ \ \ x17\ \ \ \ \ }}        \\ \cline{1-1}
\multicolumn{1}{|c|}{\reglabel{\ \ \ x18\ \ \ \ \ }}        \\ \cline{1-1}
\multicolumn{1}{|c|}{\reglabel{\ \ \ x19\ \ \ \ \ }}        \\ \cline{1-1}
\multicolumn{1}{|c|}{\reglabel{\ \ \ x20\ \ \ \ \ }}        \\ \cline{1-1}
\multicolumn{1}{|c|}{\reglabel{\ \ \ x21\ \ \ \ \ }}        \\ \cline{1-1}
\multicolumn{1}{|c|}{\reglabel{\ \ \ x22\ \ \ \ \ }}        \\ \cline{1-1}
\multicolumn{1}{|c|}{\reglabel{\ \ \ x23\ \ \ \ \ }}        \\ \cline{1-1}
\multicolumn{1}{|c|}{\reglabel{\ \ \ x24\ \ \ \ \ }}        \\ \cline{1-1}
\multicolumn{1}{|c|}{\reglabel{\ \ \ x25\ \ \ \ \ }}        \\ \cline{1-1}
\multicolumn{1}{|c|}{\reglabel{\ \ \ x26\ \ \ \ \ }}        \\ \cline{1-1}
\multicolumn{1}{|c|}{\reglabel{\ \ \ x27\ \ \ \ \ }}        \\ \cline{1-1}
\multicolumn{1}{|c|}{\reglabel{\ \ \ x28\ \ \ \ \ }}        \\ \cline{1-1}
\multicolumn{1}{|c|}{\reglabel{\ \ \ x29\ \ \ \ \ }}        \\ \cline{1-1}
\multicolumn{1}{|c|}{\reglabel{\ \ \ x30\ \ \ \ \ }}        \\ \cline{1-1}
\multicolumn{1}{|c|}{\reglabel{\ \ \ x31\ \ \ \ \ }}        \\ \cline{1-1}
\multicolumn{1}{c}{XLEN}                                  \\

\instbitrange{XLEN-1}{0}                                  \\ \cline{1-1}
\multicolumn{1}{|c|}{\reglabel{pc}}                         \\ \cline{1-1}
\multicolumn{1}{c}{XLEN}                                  \\
\end{tabular}
\end{center}
}
\caption{RISC-V基础非特权整数寄存器状态}
\label{gprs}
\end{figure}

\begin{commentary}

  在基础整数ISA中没有专门的栈指针或子程序返回地址链接的寄存器;
  指令编码允许任何的{\tt x}寄存器被用于这些目的。然而,标准软件调用约定使用寄存器{\tt x1}来持有调用的返回地址,
  同时寄存器{\tt x5}可用作备选的链接寄存器。标准调用约定使用寄存器{\tt x2}作为栈指针。

  硬件可能选择加速函数调用并使用{\tt x1}或{\tt x5}返回。见JAL和JALR指令的描述。

  压缩16位指令格式是围绕着{\tt x1}是返回地址寄存器,而{\tt x2}是栈指针的假设设计的。
  使用其它约定(非标准约定)的软件虽然可以正确地执行,但是可能会让导致更大的代码尺寸。

\end{commentary}

\begin{commentary}

  架构寄存器的可用数目对代码尺寸、性能、和能量消耗有很大的影响。
  尽管16个寄存器对于一个整数ISA来说运行已编译代码理应是足够的,
  但是使用3-地址格式,在16位指令中编码一个带有16个寄存器的完整ISA仍然是不可能的。
  尽管2-地址格式是可能的,但是将增加指令数量并降低效率。我们希望避免中间指令尺寸(例如Xtensa的24位指令),
  以简化基础硬件实现。一旦采用了32位指令尺寸,就可以直接支持32个整数寄存器。
  更大数目的整数寄存器也对高性能代码的性能提升有帮助,可以促成循环展开、软件流水线和缓存平铺的广泛使用。

  由于这些原因,我们为RV32I选择了32个整数寄存器作为约定数量。
  动态寄存器使用往往由一些频繁访问的寄存器所控制,而 regfile 的实现可以优化以减少对频繁访问寄存器的访问能耗~\cite{jtseng:sbbci}。
  可选的压缩16位指令格式大多数只访问8个寄存器,并因此可以提供一种稠密的指令编码;
  而额外的指令集扩展,如果愿意,可能支持更大的寄存器空间(或者是扁平的,或者是分层的)。

  对于资源受限的嵌入式应用,我们已经定义了RV32E子集,它只有16个寄存器(第~\ref{rv32e}章)。
\end{commentary}

\section{基础指令格式}

在基础RV32I ISA中,有四个核心指令格式(R/I/S/U),如图~\ref{fig:baseinstformats}所示。
所有这四个格式都是32位固定长度。
基础ISA有IALIGN=32意味着指令必须在内存中对齐到四字节的边界。
如果在执行分支或无条件跳转时,目标地址没有按IALIGN位对齐,将生成一个指令地址未对齐的异常。
这个异常由分支或跳转指令汇报,而不是目标指令。对于还没有被执行的条件分支,不会生成指令地址未对齐异常。

\begin{commentary}

  当加入了16位长度的指令扩展或者其它长度为16位奇数倍的扩展(即,IALIGN=16)时,
  对基础ISA指令的对齐约束被放宽到按双字节边界对齐。

  导致指令未对齐的分支或跳转,将汇报指令地址未对齐异常,以帮助调试,同时有助于简化IALIGN=32的系统硬件设计,
  因为这是唯一可能发生未对齐的情况。
\end{commentary}

解码一个保留指令的行为是“未指定的”(UNSPECIFIED)。

\begin{commentary}
  
  一些平台下,解码为标准使用而保留的操作码会引发一个非法指令异常。其它平台可能允许保留的操作码空间被用于非合规的扩展。
\end{commentary}

\begin{figure}[h]
\begin{center}
\setlength{\tabcolsep}{4pt}
\begin{tabular}{p{1.2in}@{}p{0.8in}@{}p{0.8in}@{}p{0.6in}@{}p{0.8in}@{}p{1in}l}
\\
\instbitrange{31}{25} &
\instbitrange{24}{20} &
\instbitrange{19}{15} &
\instbitrange{14}{12} &
\instbitrange{11}{7} &
\instbitrange{6}{0} \\
\cline{1-6}
\multicolumn{1}{|c|}{funct7} &
\multicolumn{1}{c|}{rs2} &
\multicolumn{1}{c|}{rs1} &
\multicolumn{1}{c|}{funct3} &
\multicolumn{1}{c|}{rd} &
\multicolumn{1}{c|}{opcode} &
R-类型 \\
\cline{1-6}
\\
\cline{1-6}
\multicolumn{2}{|c|}{imm[11:0]} &
\multicolumn{1}{c|}{rs1} &
\multicolumn{1}{c|}{funct3} &
\multicolumn{1}{c|}{rd} &
\multicolumn{1}{c|}{opcode} &
I-类型 \\
\cline{1-6}
\\
\cline{1-6}
\multicolumn{1}{|c|}{imm[11:5]} &
\multicolumn{1}{c|}{rs2} &
\multicolumn{1}{c|}{rs1} &
\multicolumn{1}{c|}{funct3} &
\multicolumn{1}{c|}{imm[4:0]} &
\multicolumn{1}{c|}{opcode} &
S-类型 \\
\cline{1-6}
\\
\cline{1-6}
\multicolumn{4}{|c|}{imm[31:12]} &
\multicolumn{1}{c|}{rd} &
\multicolumn{1}{c|}{opcode} &
U-类型 \\
\cline{1-6}
\end{tabular}
\end{center}
\caption{RISC-V 基础指令格式。 每个立即数子域都用正被产生的立即数值中的位位置(imm[{\em x}])的标签标记,
而不是像通常做的那样,用指令立即数域中的位位置。}
\label{fig:baseinstformats}
\end{figure}

为了简化解码,所有格式中,RISC-V ISA在相同的位置保存源寄存器({\em rs1}和{\em rs2})和目的寄存器({\em rd})。
除了CSR指令(第~\ref{csrinsts}章)中使用的5位立即数,立即数总是符号扩展的,并且通常在指令中被封装在最左端的可用位,
且被提前分配以减少硬件复杂度。特别地,为了加速符号扩展的电路,所有立即数的符号位总是在指令的位31处。

\begin{commentary}

  在实现层面中,解码寄存器标识符通常都是非常关键的路径,因此选择指令格式时,在所有格式中,
  寄存器标识符都保存在相同的位置上;作为代价,不得不跨格式移动立即数位(一个分享自RISC第四版RISC-IV的属性,又称SPUR~\cite{spur-jsscc1989})。

  实际上,大多数立即数或者比较小,或者需要所有的XLEN位。
  我们选择了一种不对称的立即数分割方法(常规指令中的12位加上一个特殊的20位的“加载上位立即数(load-upper-immediate)”指令)
  来为常规指令增加可用的编码空间。

  立即数是符号扩展的,因为对于某些立即数,我们没有观察到使用零扩展的收益(像在MIPS ISA中),并且想保持ISA尽可能地简单。
\end{commentary}

\section{立即数编码变量}

基于对立即数的处理,还有两个指令格式的变体(B/J),如图~\ref{fig:baseinstformatsimm}所示。

\begin{figure}[h]
\begin{small}
\begin{center}
\setlength{\tabcolsep}{4pt}
\begin{tabular}{p{0.3in}@{}p{0.8in}@{}p{0.6in}@{}p{0.18in}@{}p{0.7in}@{}p{0.6in}@{}p{0.6in}@{}p{0.3in}@{}p{0.5in}l}
\\
\multicolumn{1}{c}{\instbit{31}} &
\instbitrange{30}{25} &
\instbitrange{24}{21} &
\multicolumn{1}{c}{\instbit{20}} &
\instbitrange{19}{15} &
\instbitrange{14}{12} &
\instbitrange{11}{8} &
\multicolumn{1}{c}{\instbit{7}} &
\instbitrange{6}{0} \\
\cline{1-9}
\multicolumn{2}{|c|}{funct7} &
\multicolumn{2}{c|}{rs2} &
\multicolumn{1}{c|}{rs1} &
\multicolumn{1}{c|}{funct3} &
\multicolumn{2}{c|}{rd} &
\multicolumn{1}{c|}{opcode} &
R-类型 \\
\cline{1-9}
\\
\cline{1-9}
\multicolumn{4}{|c|}{imm[11:0]} &
\multicolumn{1}{c|}{rs1} &
\multicolumn{1}{c|}{funct3} &
\multicolumn{2}{c|}{rd} &
\multicolumn{1}{c|}{opcode} &
I-类型 \\
\cline{1-9}
\\
\cline{1-9}
\multicolumn{2}{|c|}{imm[11:5]} &
\multicolumn{2}{c|}{rs2} &
\multicolumn{1}{c|}{rs1} &
\multicolumn{1}{c|}{funct3} &
\multicolumn{2}{c|}{imm[4:0]} &
\multicolumn{1}{c|}{opcode} &
S-类型 \\
\cline{1-9}
\\
\cline{1-9}
\multicolumn{1}{|c|}{imm[12]} &
\multicolumn{1}{c|}{imm[10:5]} &
\multicolumn{2}{c|}{rs2} &
\multicolumn{1}{c|}{rs1} &
\multicolumn{1}{c|}{funct3} &
\multicolumn{1}{c|}{imm[4:1]} &
\multicolumn{1}{c|}{imm[11]} &
\multicolumn{1}{c|}{opcode} &
B-类型 \\
\cline{1-9}
\\
\cline{1-9}
\multicolumn{6}{|c|}{imm[31:12]} &
\multicolumn{2}{c|}{rd} &
\multicolumn{1}{c|}{opcode} &
U-类型 \\
\cline{1-9}
\\
\cline{1-9}
\multicolumn{1}{|c|}{imm[20]} &
\multicolumn{2}{c|}{imm[10:1]} &
\multicolumn{1}{c|}{imm[11]} &
\multicolumn{2}{c|}{imm[19:12]} &
\multicolumn{2}{c|}{rd} &
\multicolumn{1}{c|}{opcode} &
J-类型 \\
\cline{1-9}
\end{tabular}
\end{center}
\end{small}
\caption{显式立即数的RISC-V基础指令格式。}
\label{fig:baseinstformatsimm}
\end{figure}

S格式和B格式之间唯一的不同是,在B格式中,12位立即数域被用于以2的倍数对分支的偏移量进行编码。
将中间位(imm[10:1])和符号位放置在固定的位置,同时S格式中的最低位(inst[7])以B格式对高序位进行编码,
而不是像传统的做法那样,在硬件中把编码指令立即数中的所有位直接左移一位。

类似地,U格式和J格式之间唯一的不同是,20位立即数向左移位12位形成U格式立即数,而向左移1位形成J格式立即数。
选择U格式和J格式立即数中的指令位的位置,是为了与其它格式和彼此之间有最大程度的交叠。

图~\ref{fig:immtypes}显示了由每个基础指令格式产生的立即数,并用标记显示了立即数值的各个位是由哪个指令位(inst[{\em y}\,])所产生的。

\begin{figure}[h]
\begin{center}
\setlength{\tabcolsep}{4pt}
\begin{tabular}{p{0.2in}@{}p{1.2in}@{}p{1.0in}@{}p{0.2in}@{}p{0.7in}@{}p{0.7in}@{}p{0.2in}l}
\\
\multicolumn{1}{c}{\instbit{31}} &
\instbitrange{30}{20} &
\instbitrange{19}{12} &
\multicolumn{1}{c}{\instbit{11}} &
\instbitrange{10}{5} &
\instbitrange{4}{1} &
\multicolumn{1}{c}{\instbit{0}} &
\\
\cline{1-7}
\multicolumn{4}{|c|}{--- inst[31] ---} &
\multicolumn{1}{c|}{inst[30:25]} &
\multicolumn{1}{c|}{inst[24:21]} &
\multicolumn{1}{c|}{inst[20]} &
I-立即数 \\
\cline{1-7}
\\
\cline{1-7}
\multicolumn{4}{|c|}{--- inst[31] ---} &
\multicolumn{1}{c|}{inst[30:25]} &
\multicolumn{1}{c|}{inst[11:8]} &
\multicolumn{1}{c|}{inst[7]} &
S-立即数 \\
\cline{1-7}
\\
\cline{1-7}
\multicolumn{3}{|c|}{--- inst[31] ---} &
\multicolumn{1}{c|}{inst[7]} &
\multicolumn{1}{c|}{inst[30:25]} &
\multicolumn{1}{c|}{inst[11:8]} &
\multicolumn{1}{c|}{0} &
B-立即数 \\
\cline{1-7}
\\
\cline{1-7}
\multicolumn{1}{|c|}{inst[31]} &
\multicolumn{1}{c|}{inst[30:20]} &
\multicolumn{1}{c|}{inst[19:12]} &
\multicolumn{4}{c|}{--- 0 ---} &
U-立即数 \\
\cline{1-7}
\\
\cline{1-7}
\multicolumn{2}{|c|}{--- inst[31] ---} &
\multicolumn{1}{c|}{inst[19:12]} &
\multicolumn{1}{c|}{inst[20]} &
\multicolumn{1}{c|}{inst[30:25]} &
\multicolumn{1}{c|}{inst[24:21]} &
\multicolumn{1}{c|}{0} &
J-立即数 \\
\cline{1-7}
\end{tabular}
\end{center}
\caption{由RISC-V指令产生的立即数的类型。 用构造了它们值的指令位对域进行了标记。符号扩展总是使用inst[31]。}
\label{fig:immtypes}
\end{figure}

\begin{commentary}
符号扩展是最关键的立即数操作之一(特别是对XLEN$>$32),
而在RISC-V中,所有立即数的符号位总是保持在指令的位31,以允许符号扩展与指令解码并行处理。

虽然更加复杂的实现可能带有用于分支和跳转计算的独立加法器,
并且,因为在不同指令类型之间,保持立即数位的位置不变并不能从中获得好处,所以我们希望减少最简单实现的硬件开销。
通过旋转由B格式和J格式立即数编码的指令中的位,而不是使用动态的硬件多路复用器(mux),来将立即数扩大2倍,
我们减少了大约一半的指令符号扇出(fanout)和立即数多路复用的开销。加扰(scrambled)立即数编码将对静态编译或事前编译添加微不足道的时间。
为了指令的动态生成,虽然有一些小小的额外的开销,但是最常见的短转向分支却有了直接的立即数编码。
\end{commentary}

\section{整数运算指令}

大多数整数运算指令操作所有XLEN位的值,这些值保存在整数寄存器文件中。
整数运算指令或者被编码为使用I类型格式的寄存器-立即数操作,或者被编码为使用R类型格式的寄存器-寄存器操作。
对于寄存器-立即数指令和寄存器-寄存器指令,目的寄存器都是寄存器{\em rd}。整数运算指令不会引发算术异常。

\begin{commentary}
我们没有在整数指令集中包括对于在整数算术操作时进行溢出检查的特殊指令集的支持,
因为许多溢出检查可以使用RISC-V分支低成本地实现。
对于无符号加法的溢出检查,只需要在加法之后执行一条额外的分支指令:
\verb! add t0, t1, t2; bltu t0, t1, overflow!.

对于有符号加法,如果一个操作数的符号是已知的,溢出检查只需要在加法之后执行一条分支:
\verb! addi t0, t1, +imm; blt t0, t1, overflow!.  
这覆盖了带有一个立即操作数的加法的通常情况。

对于一般的有符号加法,在加法之后需要三条额外的指令,利用了这样一个事实:当且仅当某个操作数是负数时,和应当小于另一个操作数。
\begin{verbatim}
         add t0, t1, t2
         slti t3, t2, 0
         slt t4, t0, t1
         bne t3, t4, overflow
\end{verbatim}
在RV64I中,32位有符号加法的检查可以被进一步优化,通过比较在操作数上进行ADD和ADDW的结果实现。
\end{commentary}

\subsubsection*{整数寄存器 - 立即数指令}
\vspace{-0.4in}
\begin{center}
\begin{tabular}{M@{}R@{}S@{}R@{}O}
\\
\instbitrange{31}{20} &
\instbitrange{19}{15} &
\instbitrange{14}{12} &
\instbitrange{11}{7} &
\instbitrange{6}{0} \\
\hline
\multicolumn{1}{|c|}{imm[11:0]} &
\multicolumn{1}{c|}{rs1} &
\multicolumn{1}{c|}{funct3} &
\multicolumn{1}{c|}{rd} &
\multicolumn{1}{c|}{opcode} \\
\hline
12 & 5 & 3 & 5 & 7 \\
I-立即数[11:0] & src & ADDI/SLTI[U]  & dest & OP-IMM \\
I-立即数[11:0] & src & ANDI/ORI/XORI & dest & OP-IMM \\
\end{tabular}
\end{center}
ADDI将符号扩展的12位立即数加到寄存器{\em rs1}上。
简单地将结果的低XLEN位当作结果,而忽略了算数溢出。ADDI {\em rd, rs1, 0} 被用于实现 MV {\em rd, rs1} 汇编器伪指令。

如果寄存器{\em rs1}小于符号扩展的立即数(当二者都被视为有符号数时),
SLTI(小于立即数时置1)指令把值1放到寄存器{\em rd}中;否则,该指令把0写入{\em rd}中。
SLTIU与之相似,但是将两个值作为无符号数比较(也就是说,前者会把立即数按符号扩展到XLEN位,而后者会将其视为无符号数)。
注意,如果{\em rs1}等于0,那么SLTIU {\em rd, rs1, 1} 会把{\em rd}设置为1,否则会把{\em rd}设置为0(汇编器伪指令SEQZ {\em rd, rs})。

ANDI、ORI、XORI是在寄存器{\em rs1}和符号扩展的12位立即数上执行按位AND、OR和XOR,
并把结果放入{\em rd}的逻辑操作。注意,XORI {\em rd, rs1, -1} 对寄存器{\em rs1}执行按位逻辑反转(汇编器伪指令 NOT {\em rd, rs})。

\vspace{-0.2in}
\begin{center}
\begin{tabular}{S@{}R@{}R@{}S@{}R@{}O}
\\
\instbitrange{31}{25} &
\instbitrange{24}{20} &
\instbitrange{19}{15} &
\instbitrange{14}{12} &
\instbitrange{11}{7} &
\instbitrange{6}{0} \\
\hline
\multicolumn{1}{|c|}{imm[11:5]} &
\multicolumn{1}{c|}{imm[4:0]} &
\multicolumn{1}{c|}{rs1} &
\multicolumn{1}{c|}{funct3} &
\multicolumn{1}{c|}{rd} &
\multicolumn{1}{c|}{opcode} \\
\hline
7 & 5 & 5 & 3 & 5 & 7 \\
0000000 & shamt[4:0]  & src & SLLI & dest & OP-IMM \\
0000000 & shamt[4:0]  & src & SRLI & dest & OP-IMM \\
0100000 & shamt[4:0]  & src & SRAI & dest & OP-IMM \\
\end{tabular}
\end{center}

按常量移位按照I类型格式专门编码。被移位的操作数存放在{\em rs1}中,移位的数目被编码在I立即数域的低5位。
右移类型被编码在位30。SLLI(Shift Logical Right Immediate)是逻辑左移(零被移位到低位);
SRLI(Shift Logical Left Immediate)是逻辑右移(零被移位到高位);
而SRAI(Shift Right Arithmetic Immediate)是算数右移(原来的符号位被复制到空出来的高位)。

\vspace{-0.2in}
\begin{center}
\begin{tabular}{U@{}R@{}O}
\\
\instbitrange{31}{12} &
\instbitrange{11}{7} &
\instbitrange{6}{0} \\
\hline
\multicolumn{1}{|c|}{imm[31:12]} &
\multicolumn{1}{c|}{rd} &
\multicolumn{1}{c|}{opcode} \\
\hline
20 & 5 & 7 \\
U-immediate[31:12] & dest & LUI \\
U-immediate[31:12] & dest & AUIPC
\end{tabular}
\end{center}

LUI(加载高位立即数)被用于构建32位常量,它使用U类型格式。LUI把32位U立即数值放在目的寄存器{\em rd}中,
同时把最低的12位用零填充。

AUIPC(加高位立即数到{\tt pc})被用于构建{\tt pc}相对地址,它使用U类型格式。
AUIPC根据U立即数形成32位偏移量(最低12位填零),把这个偏移量加到AUIPC指令的地址,然后把结果放在寄存器{\em rd}中。

\begin{commentary}

{\tt lui}和{\tt auipc}的汇编语法不代表U立即数的低12位,他们总是零。

AUIPC指令支持双指令序列,以便从{\tt pc}访问任意的偏移量,用于控制流传输和数据访问。
AUIPC与一个JALR中的12位立即数的组合可以把控制传输到任何32位{\tt pc}相对地址,
而AUIPC加上常规加载或存储指令中的12位立即数偏移量可以访问任何32位{\tt pc}相对数据地址。

通过把U立即数设置为0,可以获得当前{\tt pc}。尽管JAL+4指令也可以被用于获得本地{\tt pc}(JAL后续的指令),
但是它可能引起简单微架构中的流水线破坏,或者复杂微架构中的分支目标缓冲区结构污染。

\end{commentary}

\subsubsection*{整数寄存器 - 寄存器操作}

RV32I定义了一些R类型算数操作。所有操作都读取{\em rs1}寄存器和{\em rs2}寄存器作为源操作数,并将结果写入寄存器{\em rd}。
{\em funct7}域和{\em funct3}域制定了操作的类型。

\vspace{-0.2in}
\begin{center}
\begin{tabular}{S@{}R@{}R@{}S@{}R@{}O}
\\
\instbitrange{31}{25} &
\instbitrange{24}{20} &
\instbitrange{19}{15} &
\instbitrange{14}{12} &
\instbitrange{11}{7} &
\instbitrange{6}{0} \\
\hline
\multicolumn{1}{|c|}{funct7} &
\multicolumn{1}{c|}{rs2} &
\multicolumn{1}{c|}{rs1} &
\multicolumn{1}{c|}{funct3} &
\multicolumn{1}{c|}{rd} &
\multicolumn{1}{c|}{opcode} \\
\hline
7 & 5 & 5 & 3 & 5 & 7 \\
0000000 & src2 & src1 & ADD/SLT[U]  & dest & OP    \\
0000000 & src2 & src1 & AND/OR/XOR  & dest & OP    \\
0000000 & src2 & src1 & SLL/SRL     & dest & OP    \\
0100000 & src2 & src1 & SUB/SRA     & dest & OP    \\
\end{tabular}
\end{center}

ADD执行{\em rs1}和{\em rs2}的相加。SUB执行从{\em rs1}中减去{\em rs2}。
忽略结果的溢出,并把结果的低XLEN位写入目的寄存器{\em rd}。SLT和SLTU分别执行有符号和无符号的比较,
如果$\mbox{\em rs1} < \mbox{\em rs2}$,向{\em rd}写入1,否则写入0。
注意,如果{\em rs2}不等于零,SLTU {\em rd}, {\em x0}, {\em rs2}把{\em rd}设置为1,
否则把rd设置为0(汇编器伪指令SNEZ {\em rd, rs})。AND、OR和XOR执行按位逻辑操作。

SLL、SLR和SRA对寄存器{\em rs1}中的值执行逻辑左移、逻辑右移、和算数右移,移位的数目保持在寄存器{\em rs2}的低5位中。

\subsubsection*{NOP 指令}
\vspace{-0.4in}
\begin{center}
\begin{tabular}{M@{}R@{}S@{}R@{}O}
\\
\instbitrange{31}{20} &
\instbitrange{19}{15} &
\instbitrange{14}{12} &
\instbitrange{11}{7} &
\instbitrange{6}{0} \\
\hline
\multicolumn{1}{|c|}{imm[11:0]} &
\multicolumn{1}{c|}{rs1} &
\multicolumn{1}{c|}{funct3} &
\multicolumn{1}{c|}{rd} &
\multicolumn{1}{c|}{opcode} \\
\hline
12 & 5 & 3 & 5 & 7 \\
0 & 0 & ADDI & 0 & OP-IMM \\
\end{tabular}
\end{center}

除了使{\tt pc}前进,以及使任何适用的性能计数器递增以外,NOP指令不改变任何架构上的可见状态。NOP被编码为ADDI {\em x0, x0, 0}。

\begin{commentary}

NOP可以被用于把代码段对齐到微架构上的高位有效地址边界,或者为内联代码的修改留出空间。
尽管有许多可能的方法来编码NOP,我们定义了一个规范的NOP编码,来允许微架构优化,
以及更具可读性的反汇编输出。其它的NOP编码形式可用于HINT指令(第2.9节~\ref{sec:rv32i-hints})。

选用ADDI进行NOP编码是因为,这是在跨多个系统中最可能的、最少资源来执行的方法(如果解码中没有优化的话)。
特别是,指令只会读一个寄存器。并且,ADDI功能单元也更可能用于超标量设计,因为加法是最常见的操作。
特别是,地址生成(address-generation)功能单元可以使用计算基址+偏移量地址计算所需的相同硬件来执行ADDI,
而寄存器-寄存器ADD或者逻辑/移位操作都需要额外的硬件。
\end{commentary}

\section{控制转移指令}

RV32I提供两种类型的控制转移指令:无条件跳转和条件分支。RV32I中的控制转移指令{\em 没有}架构上可见的延迟槽。

如果在一次跳转或发生转移的目标上发生了一个指令访问故障异常或指令缺页故障异常,该异常会报告在目标指令上,而不是报告在跳转或分支指令上。

\subsubsection*{无条件跳转}

\vspace{-0.1in} 跳转和链接(JAL)指令使用J类型格式。J类型指令把J立即数以2字节的倍数编码一个有符号的偏移量。
偏移量是符号扩展的,加到当前跳转指令的地址上以形成跳转目标地址。跳转可以因此到达的目标范围是$\pm$\wunits{1}{MiB}。
JAL把跟在JAL之后的指令的地址({\tt pc}+4)存储到寄存器{\em rd}中。
标准软件调用约定使用{\tt x1}作为返回地址寄存器,使用{\tt x5} 作为备选的链接寄存器。

\begin{commentary}

备选的链接寄存器支持调用millicode例程(例如,那些在压缩代码中的保存和恢复寄存器的例程),
同时保留常规的返回地址寄存器。寄存器{\tt x5}被选为备用链接寄存器,映射到标准调用约定中的一个临时值,其编码与常规链接寄存器相比只有一位不同。
\end{commentary}

一般的无条件跳转(汇编器伪指令J)被编码为{\em rd}={\tt x0}的JAL。

\vspace{-0.2in}
\begin{center}
\begin{tabular}{W@{}E@{}W@{}R@{}R@{}O}
\\
\multicolumn{1}{c}{\instbit{31}} &
\instbitrange{30}{21} &
\multicolumn{1}{c}{\instbit{20}} &
\instbitrange{19}{12} &
\instbitrange{11}{7} &
\instbitrange{6}{0} \\
\hline
\multicolumn{1}{|c|}{imm[20]} &
\multicolumn{1}{c|}{imm[10:1]} &
\multicolumn{1}{c|}{imm[11]} &
\multicolumn{1}{c|}{imm[19:12]} &
\multicolumn{1}{c|}{rd} &
\multicolumn{1}{c|}{opcode} \\
\hline
1 & 10 & \multicolumn{1}{c}{1} & 8 & 5 & 7 \\
\multicolumn{4}{c}{offset[20:1]} & dest & JAL \\
\end{tabular}
\end{center}

间接跳转指令JALR(跳转和链接寄存器)使用I类型编码。
通过把符号扩展的12位I立即数加到寄存器{\em rs1}来获得目标地址,然后把结果的最低有效位设置为零。
紧接着跳转的指令的地址({\tt pc}+4)被写入寄存器{\em rd}。如果不需要结果,寄存器{\tt x0}也可以被用作目的寄存器。
\vspace{-0.4in}
\begin{center}
\begin{tabular}{M@{}R@{}F@{}R@{}O}
\\
\instbitrange{31}{20} &
\instbitrange{19}{15} &
\instbitrange{14}{12} &
\instbitrange{11}{7} &
\instbitrange{6}{0} \\
\hline
\multicolumn{1}{|c|}{imm[11:0]} &
\multicolumn{1}{c|}{rs1} &
\multicolumn{1}{c|}{funct3} &
\multicolumn{1}{c|}{rd} &
\multicolumn{1}{c|}{opcode} \\
\hline
12 & 5 & 3 & 5 & 7 \\
offset[11:0] & base & 0 & dest & JALR \\
\end{tabular}
\end{center}

\begin{commentary}

无条件跳转指令都使用{\tt pc}相对地址,以此支持位置无关代码。
JALR指令被定义为能够使用双指令序列跳转到32位绝对地址空间范围内的任何地方。
一条LUI指令可以首先把目标地址的高20位加载到{\em rs1},然后JALR指令可以加上低位。
类似地,先用AUIPC再用JALR可以跳转到32位{\tt pc}相对地址范围中的任何地方。

注意JALR指令不会像条件分支指令那样,把12位立即数当作2字节的倍数对待。这回避了硬件中出现的另一种立即数格式。
实际上,大多数JALR的使用,要么有一个零立即数,要么是与LUI或AUIPC搭配成对,所以有一点范围减少是无关紧要的。

在计算JALR目标地址时清理最低有效的位,既稍微简化了硬件,又允许函数指针的低位被用于存储辅助信息。
尽管这种情况中,会有一些潜在的错误检查的轻微丢失,但是实际上,跳转到一个不正确的指令地址通常将很快引发一个异常。

当以{\em rs1}$=${\tt x0}作为基地址使用时,JALR可以被用于实现地址空间中从任何地方到最低\wunits{2}{KiB}
或最高\wunits{2}{KiB}地址区域的单一指令子例程调用,这可以被用于实现对小型运行时库的快速调用。
或者,ABI可以专用于通用目的寄存器,以指向地址空间中任何其它地方的一个库。
\end{commentary}

如果目标地址没有对齐到IALIGN位边界,JAL和JALR指令将产生一个指令地址未对齐( instruction-address-misaligned)异常。

\begin{commentary}

指令地址未对齐异常不会发生在IALIGN=16的机器上,例如那些支持压缩指令集扩展(C)的机器。
\end{commentary}

返回地址预测栈是高性能取指单元的一个常见特征,但是需要精确地探测用于过程调用和有效返回、即将生效的指令。
对于RISC-V,有关指令用途的提示,是通过使用的寄存器号码被隐式地编码的。
只有当{\em rd} = {\tt x1}/{\tt x5}时,JAL指令才应当把返回地址推入到返回地址栈(RAS)上。
JALR指令应当压入/弹出一个RAS的所有情形如表~\ref{rashints}所示。

\begin{table}[hbt]
\centering
\begin{tabular}{|c|c|c|l|}
  \hline
  \textit{rd} is \texttt{x1}/\texttt{x5}
      & \textit{rs1} is \texttt{x1}/\texttt{x5}
            & \textit{rd}$=$\textit{rs1} & RAS action \\
  \hline
  否  & 否  & --  & 无 \\
  否  & 是 & --  & 弹出 \\
  是 & 否  & --  & 压入 \\
  是 & 是 & 否  & 弹出,然后压入 \\
  是 & 是 & 是 & 压入 \\
   \hline
\end{tabular}
\caption{在JALR指令的寄存器操作数中编码的返回地址栈预测提示。}
\label{rashints}
\end{table}

\begin{commentary}
  
一些其它的ISA把显式的提示位添加到了它们的间接跳转指令上,来指导返回地址栈的操作。
我们使用绑定寄存器号码的隐式提示和调用约定,以减少用于这些提示的编码空间。

当两个不同的链接寄存器({\tt x1}和{\tt x5})被给定为{\em rs1}和{\em rd}时,
接下来RAS会被同时弹出和推入,以支持协程。
如果{\em rs1}和{\em rd}是相同的链接寄存器({\tt x1} 或者 {\tt x5}),
RAS只是为使能序列中宏操作融合(macro-op fusion)而被推入: \\
% \linebreak
{\tt lui ra, imm20; jalr ra, imm12(ra)} \ 和 \ 
{\tt auipc ra, imm20; jalr ra, imm12(ra)}
\end{commentary}

\subsubsection*{条件分支}

所有的分支指令使用B类型指令格式。12位B立即数以2字节的倍数编码符号偏移量。
偏移量是符号扩展的,加到分支指令的地址上以给出目标地址。条件分支的范围是$\pm$\wunits{4}{KiB}。

\vspace{-0.2in}
\begin{center}
\begin{tabular}{W@{}R@{}F@{}F@{}R@{}R@{}F@{}S}
\\
\multicolumn{1}{c}{\instbit{31}} &
\instbitrange{30}{25} &
\instbitrange{24}{20} &
\instbitrange{19}{15} &
\instbitrange{14}{12} &
\instbitrange{11}{8} &
\multicolumn{1}{c}{\instbit{7}} &
\instbitrange{6}{0} \\
\hline
\multicolumn{1}{|c|}{imm[12]} &
\multicolumn{1}{c|}{imm[10:5]} &
\multicolumn{1}{c|}{rs2} &
\multicolumn{1}{c|}{rs1} &
\multicolumn{1}{c|}{funct3} &
\multicolumn{1}{c|}{imm[4:1]} &
\multicolumn{1}{c|}{imm[11]} &
\multicolumn{1}{c|}{opcode} \\
\hline
1 & 6 & 5 & 5 & 3 & 4 & 1 & 7 \\
\multicolumn{2}{c}{offset[12$\vert$10:5]} & src2 & src1 & BEQ/BNE & \multicolumn{2}{c}{offset[11$\vert$4:1]} & BRANCH \\
\multicolumn{2}{c}{offset[12$\vert$10:5]} & src2 & src1 & BLT[U] & \multicolumn{2}{c}{offset[11$\vert$4:1]} & BRANCH \\
\multicolumn{2}{c}{offset[12$\vert$10:5]} & src2 & src1 & BGE[U]  & \multicolumn{2}{c}{offset[11$\vert$4:1]} & BRANCH \\
\end{tabular}
\end{center}

分支指令对两个寄存器进行比较。BEQ和BNE分别在寄存器{\em rs1}和{\em rs2}相等或不等时采取分支。
BLT和BLTU分别使用有符号和无符号的比较,如果{\em rs1}小于{\em rs2}则采取分支。
BGE和BGEU分别使用有符号和无符号的比较,如果rs1大于或等于rs2则采取分支。
注意,BGT、BGTU、BLE和BLEU可以分别通过反转BLT、BLTU、BGE和BGEU的操作数来合成。

\begin{commentary}

可以用一条BLTU指令检查有符号的数组边界,因为任意负数索引都将比任意非负数边界要大。
\end{commentary}

软件应当按这样的原则来优化编写:按顺序执行的代码路径是占大部分的常见路径,
而顺序外的代码路径被采取的频率较低。软件也应当假定,向后的分支将被预测采取,而向前的分支被预测不采取,
至少在它们第一次被遇到时如此。动态预测应当快速地学习任何可预测的分支行为。

不像其它的一些架构,对于无条件分支,应当总是使用跳转指令({\em rd}={\tt x0}的JAL),
而不是使用一个条件总是真的条件分支指令。
RISC-V的跳转也是{\tt pc}相对的,并支持比分支更宽的偏移量范围,而且将不会污染条件分支预测表。

\begin{commentary} 

条件分支被设计为包含两个寄存器之间的算数比较操作(PA-RISC、Xtensa和MIPS R6中也是这样做的),
而不是使用条件代码(x86、ARM、SPARC、PowerPC),或者只用一个寄存器和零比较(Alpha、MIPS),
又或是只比较两个寄存器是否相等(MIPS)。
这个设计的动机是观察到:比较与分支的组合指令适合于常规流水线,避免了额外的条件代码状态或者临时寄存器的使用,
并减少了静态代码的尺寸和动态指令获取的流量。
另一个考虑点是,与零比较需要非平凡的电路延迟(特别是在高级处理中运行流程到达静态逻辑后),
并因此与算数量级的比较几乎同样代价高昂。融合的比较与分支指令的另一个优势是,
分支可以在前端指令流中被更早地观察到,并因此能够被更早地预测。
在基于相同的条件代码可以采取多个分支的情况中,使用条件代码的设计或许有优势,但是我们相信这种情况是相对稀少的。

我们考虑过,但是没有在指令编码中包含静态分支提示。这些虽然可以减少动态预测器的压力,
但是需要更多指令编码空间和软件画像来达到最佳结果,并且如果产品的运行没有匹配画像(profiling)运行的话,会导致性能变差。

我们考虑过,但是没有包含条件移动或谓词指令,它们可以有效地替换不可预测的短向前分支(short forward branches)。
条件移动是二者中较简单的,但是难以和条件代码一起使用,因为那会引起异常(内存访问和浮点操作)。
谓词会给系统添加额外的标志,添加额外的指令来设置和清除标志,以及在每个指令上增加额外的编码负担。
条件移动和谓词指令都会增加乱序微架构的复杂度,因为如果谓词为假,则需要把目的架构寄存器的原始值复制到重命名后的目的物理寄存器,
因此会添加隐含的第三个源操作数。此外,静态编译时间决定使用谓词而不是分支,可以导致没有包含在编译器训练集中的输入的性能降低,
尤其是考虑到不可预测的分支是稀少的,而且随着分支预测技术的改进会而变得更加稀少。

我们注意到,现存的各种微架构技术会把不可预测的短向前分支转化为内部谓词代码,
以避免分支误预测时冲刷流水线的开销~\cite{heil-tr1996,Klauser-1998,Kim-micro2005},
并且已经在商业处理器中被实现~\cite{ibmpower7}。最简单的技术只是通过只冲刷分支阴影(branch shadown)中的指令,
而不是整个取指流水线,或者通过使用宽指令取指或空闲指令取指槽从两端获取指令,从而减少了从误预测短向前分支恢复的代价。
用于乱序核心中的更加复杂的技术是在分支阴影中的指令上添加内部谓词,内部谓词的值由分支指令写入,
这允许分支和随后的指令相比于其他代码被推测执行和乱序执行,而与其它代码的执行顺序不一致~\cite{ibmpower7}。
\end{commentary}

如果目标地址没有对齐到IALIGN位边界,并且分支条件评估为真,那么条件分支指令将生成一个指令地址未对齐异常。
如果分支条件评估为假,那么指令地址未对齐异常将不会产生。

\begin{commentary}

指令地址未对齐异常不会发生在支持16位对齐指令扩展(例如,压缩指令集扩展C)的机器上。
\end{commentary}

\section{加载和存储指令}
\label{sec:rv32:ldst}

RV32I是一个“加载-存储”架构,只有加载和存储指令访问内存,而算数指令只操作CPU寄存器。
RV32I提供一个32位的地址空间,按字节编址。
EEI将定义该地址空间的哪一部分是哪个指令可以合法访问的(例如,一些地址可能是只读的,或者只支持按字访问)。
以{\tt x0}为目的寄存器的加载操作必定会引发某些异常,并引起其它的副作用,即使所加载的值被丢弃。

EEI将定义内存系统是否是小字节序或大字节序的。在RISC-V中,字节序是按字节编址的不变量。

\begin{commentary}
  
  在字节序是按字节编址不变量的系统中,有如下的属性:如果一个字节以某个字节序被存储到内存的某个地址,
  那么从那个地址开始以任何字节序的一个字节大小的加载操作都将返回被存储的值。

  在一个小字节序的配置中,对于多字节的存储,在最低内存字节地址处写入寄存器字节的最低有效位,
  然后按有效位的升序写入其它的寄存器字节。加载类似,把较小有效位的内存字节地址的内容传输到较低有效位的寄存器字节。

  在一个大字节序的配置中,对于多字节的存储,在最低内存字节地址处写入寄存器字节的最高有效位,然后按有效位的降序写入其它的寄存器字节。
  加载类似,把较大有效位的内存字节地址的内容传输到较低有效位的寄存器字节。

\end{commentary}

\vspace{-0.4in}
\begin{center}
\begin{tabular}{M@{}R@{}F@{}R@{}O}
\\
\instbitrange{31}{20} &
\instbitrange{19}{15} &
\instbitrange{14}{12} &
\instbitrange{11}{7} &
\instbitrange{6}{0} \\
\hline
\multicolumn{1}{|c|}{imm[11:0]} &
\multicolumn{1}{c|}{rs1} &
\multicolumn{1}{c|}{funct3} &
\multicolumn{1}{c|}{rd} &
\multicolumn{1}{c|}{opcode} \\
\hline
12 & 5 & 3 & 5 & 7 \\
offset[11:0] & base & width & dest & LOAD \\
\end{tabular}
\end{center}

\vspace{-0.2in}
\begin{center}
\begin{tabular}{O@{}R@{}R@{}F@{}R@{}O}
\\
\instbitrange{31}{25} &
\instbitrange{24}{20} &
\instbitrange{19}{15} &
\instbitrange{14}{12} &
\instbitrange{11}{7} &
\instbitrange{6}{0} \\
\hline
\multicolumn{1}{|c|}{imm[11:5]} &
\multicolumn{1}{c|}{rs2} &
\multicolumn{1}{c|}{rs1} &
\multicolumn{1}{c|}{funct3} &
\multicolumn{1}{c|}{imm[4:0]} &
\multicolumn{1}{c|}{opcode} \\
\hline
7 & 5 & 5 & 3 & 5 & 7 \\
offset[11:5] & src & base & width & offset[4:0] & STORE \\
\end{tabular}
\end{center}

加载和存储指令在寄存器和内存之间传输一个值。加载指令被编码为I类型格式,存储指令则是S类型。
通过把寄存器{\em rs1}加到符号扩展的12位偏移量,可以获得有效地址。加载指令从内存复制一个值到寄存器{\em rd}。
存储指令把寄存器{\em rs2}中的值复制到内存。

LW指令从内存加载一个32位的值到{\em rd}。LH先从内存加载一个16位的值,然后在存储到{\em rd}中之前,把它符号扩展到32位。
LHU先从内存加载一个16位的值,然后,在存储到{\em rd}中之前,把它用零扩展到32位。
LB和LBU被类似地定义于8位的值。SW、SH和SB指令从寄存器{\em rs2}的低位将32位、16位和8位的值存储到内存。

不管EEI如何,有效位地址自然对齐的加载和存储不应当引发地址未对齐的异常。对于有效位地址没有自然对齐到被引用的数据类型的情况
(即,有效地址不能被以字节为单位的访问大小整除),其行为取决于EEI。

EEI也可以完全支持未对齐的加载和存储,使得运行在执行环境内部的软件将永不会经历包含的或者致命的
(译者注:见1.6节提到的陷入类型)地址未对齐陷入。在这种情况中,未对齐的加载和存储可以在硬件中被处理,
或者通过一个不可见的陷入进入执行环境,或者根据具体地址,可能是硬件和不可见陷入的组合。

EEI可以不保证未对齐的加载和存储被不可见地处理掉。在这种情况中,没有自然对齐的加载和存储或者可以成功地完成执行,或者可以引发一个异常。所引发的异常可以是一个地址未对齐异常,也可以是一个访问故障异常。对于除了未对齐外都能够完成的内存访问,如果不能模拟未对齐访问(例如,如果对内存区域的访问有副作用),那么可以引发一个访问故障异常而不是一个地址未对齐异常。当EEI不保证隐式地处理未对齐的加载和存储时,EEI必须定义由地址未对齐引起的异常是否导致被包含的陷入(允许运行在执行环境中的软件来处理该陷入),或者导致致命的陷入(终止执行)。


\begin{commentary}

  当移植遗留代码时,偶尔需要未对齐的访问;在使用某些形式的packed-SIMD扩展(译者注:打包的SIMD扩展指令,即P扩展)、或者处理外部打包的数据结构时,这些未对齐的访问对应用程序的性能会有帮助。对于是否允许EEI通过常规的加载和存储指令来选择支持未对齐的访问,我们的基本原则是,是否能够简化因额外处理未对齐而引入的硬件电路设计的复杂性。一个选择是,在基础ISA中将不允许未对齐的访问,然后为未对齐访问提供一些单独的ISA支持:或者是用一些特殊指令来帮助软件处理未对齐访问,或者是用未对齐访问的新的硬件编址模式。特殊指令是难以使用的,会让ISA复杂化,并通常增加了新的处理器状态(例如,SPARC VIS对齐地址偏移量寄存器),或是让现有处理器状态的访问复杂化(例如,MIPS LWL/LWR部分寄存器写操作)。此外,对于面向循环的packed-SIMD代码,当操作数未对齐时,会迫使软件根据操作数的对齐方式提供多种形式的循环,这使代码生成复杂化,并增加了循环启动的负担。新的未对齐硬件编址模式或者会占据相当多的指令编码空间,或者需要非常简化的编址模式(例如,只有寄存器间接寻址模式)。
\end{commentary}

即使是当未对齐的加载和存储成功完成时,根据实现不同,这些访问也可能运行得极度缓慢(例如,当通过一个不可见的陷入实现时)。此外,自然对齐的加载和存储会被保证原子执行,但未对齐的加载和存储却可能不会,并因此需要额外的同步来保证原子性。

% Even when misaligned loads and stores complete successfully, these
% accesses might run extremely slowly depending on the implementation
% (e.g., when implemented via an invisible trap).  Furthermore, whereas
% naturally aligned loads and stores are guaranteed to execute
% atomically, misaligned loads and stores might not, and hence
% require additional synchronization to ensure atomicity.

\begin{commentary}

  我们没有强制要求未对齐访问的原子性,所以执行环境可以使用一种不可见的机器陷入和一个软件处理程序来处理部分或所有的未对齐访问。
  如果提供了未对齐硬件支持,软件利用它简单地使用常规加载和存储指令就可以。然后,硬件可以根据运行时地址是否对齐自动优化访问。
\end{commentary}

\pagebreak

\section{内存排序指令}
\label{sec:fence}

\vspace{-0.2in}
\begin{center}
\begin{tabular}{F@{}IIIIIIIIF@{}F@{}F@{}S}
\\
\instbitrange{31}{28} &
\multicolumn{1}{c}{\instbit{27}} &
\multicolumn{1}{c}{\instbit{26}} &
\multicolumn{1}{c}{\instbit{25}} &
\multicolumn{1}{c}{\instbit{24}} &
\multicolumn{1}{c}{\instbit{23}} &
\multicolumn{1}{c}{\instbit{22}} &
\multicolumn{1}{c}{\instbit{21}} &
\multicolumn{1}{c}{\instbit{20}} &
\instbitrange{19}{15} &
\instbitrange{14}{12} &
\instbitrange{11}{7} &
\instbitrange{6}{0} \\
\hline
\multicolumn{1}{|c|}{fm} &
\multicolumn{1}{c|}{PI} &
\multicolumn{1}{c|}{PO} &
\multicolumn{1}{c|}{PR} &
\multicolumn{1}{c|}{PW} &
\multicolumn{1}{|c|}{SI} &
\multicolumn{1}{c|}{SO} &
\multicolumn{1}{c|}{SR} &
\multicolumn{1}{c|}{SW} &
\multicolumn{1}{c|}{rs1} &
\multicolumn{1}{c|}{funct3} &
\multicolumn{1}{c|}{rd} &
\multicolumn{1}{c|}{opcode} \\
\hline
4 & 1 & 1 & 1 & 1 & 1 & 1 & 1 & 1 & 5 & 3 & 5 & 7 \\
FM & \multicolumn{4}{c}{前驱} & \multicolumn{4}{c}{后继} & 0 & FENCE & 0 & MISC-MEM \\
\end{tabular}
\end{center}

FENCE指令被用于为其它RISC-V硬件线程和外部设备或协处理器所看到的设备I/O和内存访问进行排序。
设备输入(I)、设备输出(O)、内存读(R)和内存写(W)的任意组合可以与其他同样这些的任意组合进行排序。
可以非正式地认为,没有其它的RISC-V硬件线程或外部设备可以在FENCE之前的{\em 前驱(predecessor)}集合中的任何操作之前,
观察到FENCE之后的{\em 后继(successor)}集合中的任何操作。第17章~\ref{ch:memorymodel}提供了RISC-V内存一致性模型的一个精确的描述。
% The FENCE instruction is used to order device I/O and
% memory accesses as viewed by other RISC-V harts and external devices
% or coprocessors.  Any combination of device input (I), device output
% (O), memory reads (R), and memory writes (W) may be ordered with
% respect to any combination of the same.  Informally, no other RISC-V
% hart or external device can observe any operation in the {\em
%   successor} set following a FENCE before any operation in the {\em
%   predecessor} set preceding the FENCE.
% Chapter~\ref{ch:memorymodel} provides a precise description of the
% RISC-V memory consistency model.

如同所观察到的对那些外部设备发起的内存读写进行排序,FENCE指令也对硬件线程发起的内存读和内存写进行排序。然而,FENCE不对外部设备使用任何其它信号机制发起的观察事件进行排序。
% The FENCE instruction also orders memory reads and writes made by the
% hart as observed by memory reads and writes made by an external
% device.  However, FENCE does not order observations of events made by
% an external device using any other signaling mechanism.

\begin{commentary}

  一个设备可能通过某些外部通信机制(例如,一个为中断控制器驱动中断信号的内存映射控制寄存器)观察到对一个内存位置的访问。这个通信是在FENCE排序机制的视野之外的,因此,FENCE指令不能保证中断信号变化何时能对中断控制器可见。特定的设备可以提供额外的排序保证以减少软件负载,但属于RISC-V内存模型的范畴之外的话题。

一个设备可能通过某些外部通信机制(例如,一个为中断控制器驱动中断信号的内存映射控制寄存器)观察到对一个内存位置的访问。
这个通信是在FENCE排序机制的视野之外的,因此,FENCE指令不能提供保证,中断信号的变化何时能对中断控制器可见。
特定的设备可以提供额外的排序保证以减小软件负载,但是那些机制属于RISC-V内存模型的范畴之外了。
% A device might observe an access to a memory location via some
% external communication mechanism, e.g., a memory-mapped control
% register that drives an interrupt signal to an interrupt controller.
% This communication is outside the scope of the FENCE ordering
% mechanism and hence the FENCE instruction can provide no guarantee on
% when a change in the interrupt signal is visible to the interrupt
% controller.  Specific devices might provide additional ordering
% guarantees to reduce software overhead but those are outside the scope
% of the RISC-V memory model.
\end{commentary}

EEI将定义什么样的I/O操作是允许的,特别是当被加载和存储指令访问时,分别有哪些内存地址将被视为设备输入和设备输出操作、而不是内存读取和写入操作,并以此排序。例如,内存映射I/O设备通常被未缓存的加载和存储访问,这些访问使用I和O位而不是R和W位进行排序。指令集扩展也可以在FENCE中规定同样使用I和O位排序的新的I/O指令。

% The EEI will define what I/O operations are possible, and in
% particular, which memory addresses when accessed by load and store instructions will be treated and
% ordered as device input and device output operations respectively
% rather than memory reads and writes.  For example, memory-mapped I/O
% devices will typically be accessed with uncached loads and stores that
% are ordered using the I and O bits rather than the R and W bits.
% Instruction-set extensions might also describe new I/O
% instructions that will also be ordered using the I and O bits in a
% FENCE.

\begin{table}[htp]
\begin{small}
\begin{center}
\begin{tabular}{|c|c|l|}
\hline
{\em fm} 域 & 助记符 & 含义 \\
\hline
0000 & \em 无 & 一般的屏障 \\
\hline
\multirow{2}{*}{1000} & \multirow{2}{*}{TSO} & 带有FENCE RW, RW:排除“写到读”的次序 \\
                      &                      & 其它的:\em 保留供未来使用。 \\
\hline
\multicolumn{2}{|c|}{\em 其他} & \em 保留供未来使用。 \\
\hline
\end{tabular}
\end{center}
\end{small}
\caption{屏障模式编码}
\label{fm}
\end{table}

屏障模式域{\em fm}定义了FENCE的语义。一个{\em fm}=0000的FENCE把它的前驱集合中的所有内存操作,
排在它的后继集合的所有内存操作之前。
% The fence mode field {\em fm} defines the semantics of the FENCE.  A
% FENCE with {\em fm}=0000 orders all memory operations in its
% predecessor set before all memory operations in its successor set. 

FENCE.TSO指令被编码为{\em fm}=1000、{\em 前驱}RW、以及{\em 后继}=RW的FENCE指令。
FENCE.TSO把它前驱集合中的所有加载操作排在它后继集合中的所有内存操作之前,
并把它前驱集合中的所有存储操作排在它后继集合中的所有存储操作之前。
这使得FENCE.TSO的前驱集合中的非AMO存储操作与它的后继集合中的非AMO加载操作不再有序。
% The FENCE.TSO instruction is encoded as a FENCE instruction
% with {\em fm}=1000, {\em predecessor}=RW, and {\em successor}=RW.
% FENCE.TSO orders all load
% operations in its predecessor set before all memory operations in its
% successor set, and all store operations in its predecessor set before
% all store operations in its successor set.  This leaves non-AMO store
% operations in the FENCE.TSO's predecessor set unordered with non-AMO
% loads in its successor set.

\begin{commentary}
  因为\mbox{FENCE RW,RW}所施加的排序是FENCE.TSO所施加排序的一个超集,
  所以忽略{\em fm}域并把FENCE.TSO作为\mbox{FENCE RW,RW}实现是正确的。
  % Because \mbox{FENCE RW,RW} imposes a superset of the orderings that
  % FENCE.TSO imposes, it is correct to ignore the {\em fm} field and
  % implement FENCE.TSO as \mbox{FENCE RW,RW}.
\end{commentary}

FENCE指令中的未使用的域——{\em rs1}和{\em rd}——被保留用于未来扩展中的更细粒度的屏障。
为了向前兼容,基础实现应当忽略这些域,而标准软件应当把这些域置为零。
同样地,表~\ref{fm}中的许多{\em fm}和前驱/后继集合设置也被保留供将来使用。
基础实现应当把所有这些保留的配置视为普通的{\em fm}=0000的屏障,而标准软件应当只使用非保留的配置。
% The unused fields in the FENCE instructions---{\em rs1} and {\em rd}---are
% reserved for finer-grain fences in future extensions.  For forward
% compatibility, base implementations shall ignore these fields, and standard
% software shall zero these fields.  Likewise, many {\em fm} and
% predecessor/successor set settings in Table~\ref{fm} are also reserved
% for future use.  Base implementations shall treat all such reserved
% configurations as normal fences with {\em fm}=0000, and standard
% software shall use only non-reserved configurations.

\begin{commentary}
我们选择了一个宽松的内存模型以允许从简单的机器实现和可能的未来协处理器或加速器扩展获得高性能。
我们从内存R/W排序中分离了I/O排序以避免在一个设备驱动硬件线程中进行不必要的序列化,
而且也支持备用的非内存路径来控制额外增加的协处理器或I/O设备。此外,简单的实现还可以忽略{\em 前驱}和{\em 后继}的域,
而总是在所有的操作上执行保守的屏障。
% We chose a relaxed memory model to allow high performance from simple
% machine implementations and from likely future
% coprocessor or accelerator extensions.  We separate out I/O ordering
% from memory R/W ordering to avoid unnecessary serialization within a
% device-driver hart and also to support alternative non-memory paths
% to control added coprocessors or I/O devices.  Simple implementations
% may additionally ignore the {\em predecessor} and {\em successor}
% fields and always execute a conservative fence on all operations.
\end{commentary}

\section{环境调用和断点}

SYSTEM指令被用于访问那些可能需要访问特权的系统功能,并且使用I类型指令格式进行编码。
这些指令可以被划分为两个主要的类别:那些原子性的“读-修改-写”控制和状态寄存器(CSR)相关指令,和所有其它潜在的特权指令。
CSR指令在第~\ref{csrinsts}章描述,而基础非特权指令在接下来的小节中描述。
% SYSTEM instructions are used to access system functionality that might
% require privileged access and are encoded using the I-type instruction
% format.  These can be divided into two main classes: those that
% atomically read-modify-write control and status registers (CSRs), and
% all other potentially privileged instructions. CSR instructions are
% described in Chapter~\ref{csrinsts}, and the base unprivileged instructions
% are described in the following section.

\begin{commentary}

  在简单的实现中,SYSTEM指令被定义为允许更简单的实现总是陷入到一个单独的软件陷入处理程序。
更复杂的实现可能在硬件中执行更多的各系统指令。
% The SYSTEM instructions are defined to allow simpler implementations
% to always trap to a single software trap handler.  More sophisticated
% implementations might execute more of each system instruction in
% hardware.
\end{commentary}

\vspace{-0.2in}
\begin{center}
\begin{tabular}{M@{}R@{}F@{}R@{}S}
\\
\instbitrange{31}{20} &
\instbitrange{19}{15} &
\instbitrange{14}{12} &
\instbitrange{11}{7} &
\instbitrange{6}{0} \\
\hline
\multicolumn{1}{|c|}{funct12} &
\multicolumn{1}{c|}{rs1} &
\multicolumn{1}{c|}{funct3} &
\multicolumn{1}{c|}{rd} &
\multicolumn{1}{c|}{opcode} \\
\hline
12 & 5 & 3 & 5 & 7 \\
ECALL   & 0 & PRIV & 0 & SYSTEM \\
EBREAK  & 0 & PRIV & 0 & SYSTEM \\
\end{tabular}
\end{center}

这两个指令对支持的执行环境引发了一个精确的请求陷入。
% These two instructions cause a precise requested trap to the
% supporting execution environment.

ECALL指令被用于向执行环境发起一个服务请求。EEI将定义服务请求参数传递的方式,
但是通常这些参数将处于整数寄存器文件中已定义的位置。
% The ECALL instruction is used to make a service request to the
% execution environment.  The EEI will define how parameters for the
% service request are passed, but usually these will be in defined
% locations in the integer register file.

EBREAK指令被用于将控制返回到调试环境。
% The EBREAK instruction is used to return control to a debugging
% environment.

\begin{commentary}
ECALL和EBREAK之前被命名为SCALL和SBREAK。这些指令有相同的功能和编码,但是被重命名了,
是为了反映它们可以更一般化地使用,而不只是调用一个监管器级别的操作系统或者调试器。
% ECALL and EBREAK were previously named SCALL and SBREAK.  The
% instructions have the same functionality and encoding, but were
% renamed to reflect that they can be used more generally than to call a
% supervisor-level operating system or debugger.
\end{commentary}

\begin{commentary}
  EBREAK被主要设计为供调试器使用的,以引发执行停止和返回到调试器中。
  EBREAK也被标准gcc编译器用来标记可能不会被执行的代码路径。
  % EBREAK was primarily designed to be used by a debugger to cause
  % execution to stop and fall back into the debugger. EBREAK is also
  % used by the standard gcc compiler to mark code paths that should not
  % be executed.

  EBREAK的另一个用处是支持“半宿主”,即,包含调试器的执行环境可以通过围绕EBREAK指令构建一套备用系统调用接口来提供服务。
  因为RISC-V基础ISA没有提供更多的(多于一个的)EBREAK指令,
  RISC-V半宿主使用一个特殊的指令序列来将半宿主EBREAK与调试器插入的EBREAK进行区分。
  % Another use of EBREAK is to support ``semihosting'', where the
  % execution environment includes a debugger that can provide services
  % over an alternate system call interface built around the EBREAK
  % instruction.  Because the RISC-V base ISAs do not provide more than
  % one EBREAK instruction, RISC-V semihosting uses a special sequence of
  % instructions to distinguish a semihosting EBREAK from a debugger
  % inserted EBREAK.
\begin{verbatim}
    slli x0, x0, 0x1f   # 入口 NOP
    ebreak              # 中断到调试器
    srai x0, x0, 7      # NOP编码编号为7的半宿主调用
\end{verbatim}
  注意这三个指令都必须是32位宽的指令,也就是说,它们必须不能出现在第~\ref{compressed}章里描述的压缩16位指令之中。
  %  Note that these three instructions must be 32-bit-wide instructions,
  %  i.e., they mustn't be among the compressed 16-bit instructions
  %  described in Chapter~\ref{compressed}.
  
   移位NOP指令仍然被认为可以用作HINT
  %  The shift NOP instructions are still considered available for use as
  %  HINTs.

   半宿主是一种服务调用的形式,它将更自然地使用现有ABI被编码为ECALL,但是这将要求调试器有能力拦截ECALL,
   那是对调试标准的一个较新的补充。我们试图改为使用带有标准ABI的ECALL,这种情况中,半宿主可以与现有标准分享服务ABI。
  %  Semihosting is a form of service call and would be more naturally
  %  encoded as an ECALL using an existing ABI, but this would require
  %  the debugger to be able to intercept ECALLs, which is a newer
  %  addition to the debug standard.  We intend to move over to using
  %  ECALLs with a standard ABI, in which case, semihosting can share a
  %  service ABI with an existing standard.
  
   我们注意到,ARM处理器在较新的设计中,对于半宿主调用,也已经转为使用了SVC而不再是BKPT。
  %  We note that ARM processors have also moved to using SVC instead of
  %  BKPT for semihosting calls in newer designs.
\end{commentary}

\section{“提示”指令}
\label{sec:rv32i-hints}

RV32I保留了大量的编码空间用于HINT指令,这些通常被用于与微架构进行性能提示的交互。
像NOP指令,除了提升{\tt pc}和任何适用的性能计数器,HINT不改变任何架构上的可视状态。实现中总是被允许忽略已编码的提示。
% RV32I reserves a large encoding space for HINT instructions, which are
% usually used to communicate performance hints to the
% microarchitecture.
% Like the NOP instruction, HINTs do not change any architecturally visible
% state, except for advancing the {\tt pc} and any applicable performance
% counters.
% Implementations are always allowed to ignore the encoded hints.

大多数RV32I HINT被编码为{\em rd}={\tt x0}的整数运算指令。
其余RV32I HINT被编码为没有前驱集和后继集且{\em fm}=0的FENCE指令。
% Most RV32I HINTs are encoded as integer computational instructions with
% {\em rd}={\tt x0}.
% The other RV32I HINTs are encoded as FENCE instructions with a null
% predecessor or successor set and with {\em fm}=0.

\begin{commentary}
选择这样的HINT编码是为了简单的实现可以完全忽略HINT,而把HINT作为一个常规的、但是恰好不改变架构状态的指令。
例如,如果目的寄存器是{\tt x0},那么ADD就是一个HINT;五位的{\em rs1}和{\em rs2}域编码了HINT的参数。
然而,简单的实现中可以简单地把HINT执行为把{\em rs1}加{\em rs2}写入{\tt x0}的ADD指令,这种没有架构上可见的影响。
% These HINT encodings have been chosen so that simple implementations can ignore
% HINTs altogether, and instead execute a HINT as a regular
% instruction that happens not to mutate the architectural state.  For example, ADD is
% a HINT if the destination register is {\tt x0}; the five-bit {\em rs1} and {\em
% rs2} fields encode arguments to the HINT.  However, a simple implementation can
% simply execute the HINT as an ADD of {\em rs1} and {\em rs2} that writes {\tt
% x0}, which has no architecturally visible effect.

作为另一个例子,一个{\em pred}域为零且{\em fm} 域为零的FENCE指令是一个HINT;
{\em succ}域, {\em rs1}域, and {\em rd}域编码了HINT的参数。
一个简单的实现可以把HINT作为一个FENCE简单地执行,即,在任何被编码在{\em succ}域中的后续内存访问之前,
对先前内存访问的空集进行排序。由于前驱集和后继集的交集为空,该指令不会施加内存排序,因此它没有架构可见的影响。
% As another example, a FENCE instruction with a zero {\em pred} field and
% a zero {\em fm} field is a HINT; the {\em succ}, {\em rs1}, and {\em rd}
% fields encode the arguments to the HINT.
% A simple implementation can simply execute the HINT as a FENCE that orders the
% null set of prior memory accesses before whichever subsequent memory accesses
% are encoded in the {\em succ} field.
% Since the intersection of the predecessor and successor sets is null, the
% instruction imposes no memory orderings, and so it has no architecturally
% visible effect.
\end{commentary}

表~\ref{tab:rv32i-hints}列出了所有的RV32I HINT代码点。91\%的HINT空间被保留用于标准HINT。
剩余的HINT空间被指定用于自定义的HINT:在这个子空间中,将永远不会定义标准HINT。
% Table~\ref{tab:rv32i-hints} lists all RV32I HINT code points.  91\% of the HINT
% space is reserved for standard HINTs.  The
% remainder of the HINT space is designated for custom HINTs: no standard HINTs
% will ever be defined in this subspace.

\begin{commentary}
我们预计标准的提示最终会包含内存系统空间和时间的局部性提示、分支预测提示、线程调度提示、安全性标签、和用于模拟/仿真的仪器标志。
% We anticipate
% standard hints to eventually include memory-system spatial and
% temporal locality hints, branch prediction hints, thread-scheduling
% hints, security tags, and instrumentation flags for
% simulation/emulation.
\end{commentary}

\begin{table}[hbt]
\centering
\begin{tabular}{|l|l|c|l|}
  \hline
  指令                   & 约束                                        & 代码点                       & 目的\\ \hline \hline
  LUI                   & {\em rd}={\tt x0}                           & $2^{20}$                    & \multirow{10}{*}{\em 保留供未来标准使用} \\ \cline{1-3}
  AUIPC                 & {\em rd}={\tt x0}                           & $2^{20}$                    & \\ \cline{1-3}
  \multirow{2}{*}{ADDI} & {\em rd}={\tt x0}, 并且要么                  & \multirow{2}{*}{$2^{17}-1$} & \\
                        & {\em rs1}$\neq${\tt x0} 要么 {\em imm}$\neq$0 &                             & \\ \cline{1-3}
  ANDI                  & {\em rd}={\tt x0}                           & $2^{17}$                    & \\ \cline{1-3}
  ORI                   & {\em rd}={\tt x0}                           & $2^{17}$                    & \\ \cline{1-3}
  XORI                  & {\em rd}={\tt x0}                           & $2^{17}$                    & \\ \cline{1-3}
  ADD                   & {\em rd}={\tt x0}, {\em rs1}$\neq${\tt x0}  & $2^{10}-32$                 & \\ \cline{1-3}
  \multirow{2}{*}{ADD}  & {\em rd}={\tt x0}, {\em rs1}={\tt x0},      & \multirow{2}{*}{$28$}       & \\
                        & {\em rs2}$\neq${\tt x2}--{\tt x5}           &                             & \\ \hline
  \multirow{4}{*}{ADD}  & \multirow{4}{*}{\shortstack[l]{{\em rd}={\tt x0}, {\em rs1}={\tt x0}, \\{\em rs2}={\tt x2}--{\tt x5}}}
                                                                      & \multirow{4}{*}{$4$}        & ({\em rs2}={\tt x2}) NTL.P1 \\
                        &                                             &                             & ({\em rs2}={\tt x3}) NTL.PALL \\
                        &                                             &                             & ({\em rs2}={\tt x4}) NTL.S1 \\
                        &                                             &                             & ({\em rs2}={\tt x5}) NTL.ALL \\ \hline
  SUB                   & {\em rd}={\tt x0}                           & $2^{10}$                    & \multirow{17}{*}{\em 保留供未来标准使用} \\ \cline{1-3}
  AND                   & {\em rd}={\tt x0}                           & $2^{10}$                    & \\ \cline{1-3}
  OR                    & {\em rd}={\tt x0}                           & $2^{10}$                    & \\ \cline{1-3}
  XOR                   & {\em rd}={\tt x0}                           & $2^{10}$                    & \\ \cline{1-3}
  SLL                   & {\em rd}={\tt x0}                           & $2^{10}$                    & \\ \cline{1-3}
  SRL                   & {\em rd}={\tt x0}                           & $2^{10}$                    & \\ \cline{1-3}
  SRA                   & {\em rd}={\tt x0}                           & $2^{10}$                    & \\ \cline{1-3}
  \multirow{3}{*}{FENCE}& {\em rd}={\tt x0}, {\em rs1}$\neq${\tt x0}, & \multirow{3}{*}{$2^{10}-63$}& \\
                        & {\em fm}=0,且 {\em pred}=0 或者 {\em succ}=0                 &                             & \\ \cline{1-3} 
                   %     &                &                             & \\ 
  \multirow{3}{*}{FENCE}& {\em rd}$\neq${\tt x0}, {\em rs1}={\tt x0}, & \multirow{3}{*}{$2^{10}-63$}& \\
                        & {\em fm}=0, 且{\em pred}=0 或者 {\em succ}=0            &                             & \\ \cline{1-3}
              %          &               &                             & \\ 
  \multirow{2}{*}{FENCE}& {\em rd}={\em rs1}={\tt x0}, {\em fm}=0,    & \multirow{2}{*}{15}         & \\
                        & {\em pred}=0, {\em succ}$\neq$0             &                             & \\ \cline{1-3}
  \multirow{2}{*}{FENCE}& {\em rd}={\em rs1}={\tt x0}, {\em fm}=0,    & \multirow{2}{*}{15}         & \\
                        & {\em pred}$\neq$W, {\em succ}=0             &                             & \\ \hline
  \multirow{2}{*}{FENCE}& {\em rd}={\em rs1}={\tt x0}, {\em fm}=0,    & \multirow{2}{*}{1}          & \multirow{2}{*}{暂停} \\
                        & {\em pred}=W, {\em succ}=0                  &                             & \\ \hline \hline
  SLTI                  & {\em rd}={\tt x0}                           & $2^{17}$                    & \multirow{7}{*}{\em 指定供自定义使用} \\ \cline{1-3}
  SLTIU                 & {\em rd}={\tt x0}                           & $2^{17}$                    & \\ \cline{1-3}
  SLLI                  & {\em rd}={\tt x0}                           & $2^{10}$                    & \\ \cline{1-3}
  SRLI                  & {\em rd}={\tt x0}                           & $2^{10}$                    & \\ \cline{1-3}
  SRAI                  & {\em rd}={\tt x0}                           & $2^{10}$                    & \\ \cline{1-3}
  SLT                   & {\em rd}={\tt x0}                           & $2^{10}$                    & \\ \cline{1-3}
  SLTU                  & {\em rd}={\tt x0}                           & $2^{10}$                    & \\ \hline
\end{tabular}
\caption{RV32I 提示指令。}
\label{tab:rv32i-hints}
\end{table}


\chapter{“Zifencei”指令获取屏障(2.0版本)}
\label{chap:zifencei}

这章定义了“Zifencei”扩展,它包括了FENCE.I指令,该指令提供了在相同硬件线程上进行的写指令内存与指令获取之间的显式同步。
目前,这个指令是确保对硬件线程可见的存储也将对它的指令获取可见的唯一标准机制。
% This chapter defines the ``Zifencei'' extension, which includes the
% FENCE.I instruction that provides explicit synchronization between
% writes to instruction memory and instruction fetches on the same hart.
% Currently, this instruction is the only standard mechanism to ensure
% that stores visible to a hart will also be visible to its instruction
% fetches.

\begin{commentary}
我们考虑过、但是没有包括“存储指令字”指令(像在MAJC中那样~\cite{majc})。
JIT编译器可以在单个的FENCE.I之前生成一大段对指令的追踪,
并且通过把翻译过的指令写到已知的没有保留在I-缓存中的内存区域,分摊任何指令缓存的嗅探/失效负载。
% We considered but did not include a ``store instruction word''
% instruction (as in MAJC~\cite{majc}).  JIT compilers may generate a
% large trace of instructions before a single FENCE.I, and amortize any
% instruction cache snooping/invalidation overhead by writing translated
% instructions to memory regions that are known not to reside in the
% I-cache.
\end{commentary}

\begin{commentary}
FENCE.I指令被设计为支持多种实现。简单的实现可以在FENCE.I被执行的时候冲刷本地指令缓存和指令流水线。
更加复杂的实现可以在每个数据(指令)缓存缺失的时候嗅探(snoop)指令(数据)缓存,
或者在主指令缓存中的某些行正在被本地存储指令写入时,使用一个包容统一的私有L2缓存使其无效化。
如果指令和数据缓存以这种方式保持一致性,或者如果内存系统只由未缓存的RAM组成,那么只有获取流水线需要在FENCE.I被冲刷。
% The FENCE.I instruction was designed to support a wide variety of
% implementations.  A simple implementation can flush the local
% instruction cache and the instruction pipeline when the FENCE.I is
% executed.  A more complex implementation might snoop the instruction
% (data) cache on every data (instruction) cache miss, or use an
% inclusive unified private L2 cache to invalidate lines from the
% primary instruction cache when they are being written by a local store
% instruction.  If instruction and data caches are kept coherent in this
% way, or if the memory system consists of only uncached RAMs, then just
% the fetch pipeline needs to be flushed at a FENCE.I.

FENCE.I指令曾是基础I指令集的前一部分。
受到两个主要问题的驱使,尽管在编写本手册时它仍然是保持指令获取一致性的仅有的标准方法,它还是被移出了强制性基础指令集。
% The FENCE.I instruction was previously part of the base I instruction
% set.  Two main issues are driving moving this out of the mandatory
% base, although at time of writing it is still the only standard method
% for maintaining instruction-fetch coherence.

首先,我们已经认识到,在一些系统上,FENCE.I的实现将是昂贵的,在内存模型任务组中正在讨论替代它的机制。
特别地,对于拥有非一致性指令缓存和非一致性数据缓存、或者指令缓存的重新填充不会嗅探(snoop)一致性数据缓存的设计,
在遇到一个FENCE.I指令时,这两个缓存都必须完全被冲刷。
当在一个统一的缓存或较外层内存系统之前有多个级别的I缓存和D缓存时,这个问题将更加严重。
% First, it has been recognized that on some systems, FENCE.I will be
% expensive to implement and alternate mechanisms are being discussed in
% the memory model task group.  In particular, for designs that have an
% incoherent instruction cache and an incoherent data cache, or where
% the instruction cache refill does not snoop a coherent data cache,
% both caches must be completely flushed when a FENCE.I instruction is
% encountered.  This problem is exacerbated when there are multiple
% levels of I and D cache in front of a unified cache or outer memory
% system.

第二,该指令并非足够强力能在一个像Unix那样的操作系统环境中的用户级别可用。
FENCE.I只同步本地硬件线程,而OS可以在FENCE.I之后把用户硬件线程重新调度到一个不同的物理硬件线程。
这将需要OS执行一个额外的FENCE.I作为每个上下文迁移的一部分。
出于这个原因,标准Linux ABI已经从用户级别中移除了FENCE.I,现在是需要一个系统调用来保持指令获取的一致性,
这允许OS最小化当前系统上需要执行的FENCE.I的数目,并为将来改进的指令获取一致性机制提供向前兼容性。
% Second, the instruction is not powerful enough to make available at
% user level in a Unix-like operating system environment.  The FENCE.I
% only synchronizes the local hart, and the OS can reschedule the user
% hart to a different physical hart after the FENCE.I.  This would
% require the OS to execute an additional FENCE.I as part of every
% context migration.  For this reason, the standard Linux ABI has
% removed FENCE.I from user-level and now requires a system call to
% maintain instruction-fetch coherence, which allows the OS to minimize
% the number of FENCE.I executions required on current systems and
% provides forward-compatibility with future improved instruction-fetch
% coherence mechanisms.

正在讨论的未来的指令获取一致性方法包括,提供更加严格的FENCE.I版本,它只把{\em rs1}中指定的地址作为目标,
并/或者允许软件使用依赖于机器模式缓存维护操作的ABI。
% Future approaches to instruction-fetch coherence under discussion
% include providing more restricted versions of FENCE.I that only target
% a given address specified in {\em rs1}, and/or allowing software to use an
% ABI that relies on machine-mode cache-maintenance operations.
\end{commentary}

\vspace{-0.4in}
\begin{center}
\begin{tabular}{M@{}R@{}S@{}R@{}O}
\\
\instbitrange{31}{20} &
\instbitrange{19}{15} &
\instbitrange{14}{12} &
\instbitrange{11}{7} &
\instbitrange{6}{0} \\
\hline
\multicolumn{1}{|c|}{imm[11:0]} &
\multicolumn{1}{c|}{rs1} &
\multicolumn{1}{c|}{funct3} &
\multicolumn{1}{c|}{rd} &
\multicolumn{1}{c|}{opcode} \\
\hline
12 & 5 & 3 & 5 & 7 \\
0 & 0 & FENCE.I & 0 & MISC-MEM \\
\end{tabular}
\end{center}

FENCE.I指令被用于同步指令和数据流。
在硬件线程执行FENCE.I指令以前,RISC-V不保证到指令内存的存储将对RISC-V硬件线程上的指令获取可见。
FENCE.I指令确保RISC-V硬件线程上后续的指令获取将能看到已经对同一RISC-V硬件线程可见的任何先前的数据存储。
在一个多处理器系统中,FENCE.I不确保其它RISC-V硬件线程的指令获取也将能看到本地硬件线程的存储。
为了让对指令内存的存储对于所有的RISC-V硬件线程可见,
正在写的硬件线程也必须在请求所有的远程RISC-V硬件线程执行FENCE.I之前执行一次数据FENCE。
% The FENCE.I instruction is used to synchronize the instruction and
% data streams.  RISC-V does not guarantee that stores to instruction
% memory will be made visible to instruction fetches on a RISC-V
% hart until that hart executes a FENCE.I instruction.  A FENCE.I instruction
% ensures that a subsequent instruction fetch on a RISC-V hart
% will see any previous data stores already visible to the same RISC-V
% hart.  FENCE.I does {\em not} ensure that other RISC-V harts'
% instruction fetches will observe the local hart's stores in a
% multiprocessor system. To make a store to instruction memory visible
% to all RISC-V harts, the writing hart also has to execute a data FENCE
% before requesting that all remote RISC-V harts execute a FENCE.I.

FENCE.I指令中的未使用的域,imm[11:0]、rs1和rd,被保留用于未来扩展中的更细粒度的屏障功能。
为了向前兼容,基础实现应当忽略这些域,而标准软件应当把这些域置为零。
% The unused fields in the FENCE.I instruction, {\em imm[11:0]}, {\em rs1}, and
% {\em rd}, are reserved for finer-grain fences in future extensions.  For
% forward compatibility, base implementations shall ignore these fields, and
% standard software shall zero these fields.

\begin{commentary}
因为FENCE.I只使用硬件线程自己的指令获取来给存储排序,
如果应用程序线程将不会被迁移到不同的硬件线程,那么应用程序代码应当只依赖FENCE.I。
EEI可以提供有效的多处理器指令流同步机制。
% Because FENCE.I only orders stores with a hart's own instruction
% fetches, application code should only rely upon FENCE.I if the
% application thread will not be migrated to a different hart.  The EEI
% can provide mechanisms for efficient multiprocessor instruction-stream
% synchronization.
\end{commentary}



\chapter{“Zihintntl”非时间局部性提示(0.2版本)}
\label{chap:zihintntl}

NTL指令是一种HINT,它表示直接后继指令(下称“目标指令”)的显式内存访问显现出较差的引用时间局部性。
NTL指令既不改变架构状态,也确实不改变目标指令的架构可见的影响。它提供四种变体:
% The NTL instructions are HINTs that indicate that the explicit memory accesses of the immediately subsequent
% instruction (henceforth ``target instruction'') exhibit poor temporal locality of reference.
% The NTL instructions do not change architectural state, nor do they alter the
% architecturally visible effects of the target instruction.
% Four variants are provided:

NTL.P1指令表示目标指令在内存层次的最内层私有缓存的容量内没有显现出时间局部性。NTL.P1被编码为\mbox{ADD {\em x0, x0, x2}}。
% The NTL.P1 instruction indicates that the target instruction
% does not exhibit temporal locality within the capacity of the innermost level
% of private cache in the memory hierarchy.
% NTL.P1 is encoded as \mbox{ADD {\em x0, x0, x2}}.

NTL.PALL指令表示目标指令在内存层次的任何私有缓存层次的容量内都没有显现出时间局部性。NTL.PALL被编码为\mbox{ADD {\em x0, x0, x3}}。
% The NTL.PALL instruction indicates that the target instruction
% does not exhibit temporal locality within the capacity of any
% level of private cache in the memory hierarchy.
% NTL.PALL is encoded as \mbox{ADD {\em x0, x0, x3}}.

NTL.S1指令表示目标指令在内存层次的最内层共享缓存的容量内没有显现出时间局部性。NTL.S1被编码为\mbox{ADD {\em x0, x0, x4}}。
% The NTL.S1 instruction indicates that the target instruction
% does not exhibit temporal locality within the capacity of the innermost level
% of shared cache in the memory hierarchy.
% NTL.S1 is encoded as \mbox{ADD {\em x0, x0, x4}}.

NTL.ALL指令表示目标指令在内存层次的任何缓存层次的容量内都没有显现出时间局部性。NTL.ALL被编码为\mbox{ADD {\em x0, x0, x5}}。
% The NTL.ALL instruction indicates that the target
% instruction does not exhibit temporal locality within the capacity of any
% level of cache in the memory hierarchy.
% NTL.ALL is encoded as \mbox{ADD {\em x0, x0, x5}}.

\begin{commentary}
NTL指令可以被用于在数据流动、或遍历大型数据结构时,避免缓存污染,或者减少生产者-消费者交互中的延迟。
% The NTL instructions can be used to avoid cache pollution when streaming data
% or traversing large data structures, or to reduce latency in producer-consumer
% interactions.

微架构可能使用NTL指令来通知缓存替换策略,或者决定分配到哪块缓存,或者避免缓存分配。
例如,NTL.P1可以表示一个实现不应当申请私有L1缓存中的一行,但应当在L2中(不论私有或共享)申请。
在另一个实现中,NTL.P1可以申请L1中的行,但是处于最近最少使用(LRU)的状态。
% A microarchitecture might use the NTL instructions to inform the cache
% replacement policy, or to decide which cache to allocate into, or to avoid
% cache allocation altogether.
% For example, NTL.P1 might indicate that an implementation should not allocate
% a line in a private L1 cache, but should allocate in L2 (whether private or
% shared).
% In another implementation, NTL.P1 might allocate the line in L1, but in
% the least-recently used state.

NTL.ALL通常将通知实现不要申请缓存层次中的任何位置。编程人员应当为那些没有可利用的时间局部性的访问使用NTL.ALL。
% NTL.ALL will typically inform implementations not to allocate anywhere in the
% cache hierarchy.
% Programmers should use NTL.ALL for accesses that have no exploitable temporal
% locality.

像任何HINT一样,这些指令可以被自由地忽略。因此,尽管它们是以基于缓存的内存层次的角度描述的,它们并不强制要求提供缓存。
% Like any HINTs, these instructions may be freely ignored.
% Hence, although they are described in terms of cache-based memory hierarchies,
% they do not mandate the provision of caches.

一些实现的某些内存可能遵从这些HINT,而对其它内存访问忽略它们:
例如,通过在L1中以独占状态获取一个缓存行来实现LR/SC的实现可能忽略在LR和SC上的NTL指令,
但是可能遵从AMO和常规加载与存储的NTL指令。
% Some implementations might respect these HINTs for some memory accesses but
% not others: e.g., implementations that implement LR/SC by acquiring a
% cache line in the exclusive state in L1 might ignore NTL instructions
% on LR and SC, but might respect NTL instructions for
% AMOs and regular loads and stores.
\end{commentary}

表~\ref{tab:ntl-portable}列出了一些软件使用情况,
以及{\em 可移植} 软件(即,不会针对任何特定实现的内存层次进行调整的软件)在各情况中应当使用的推荐的NTL变体。
% Table~\ref{tab:ntl-portable} lists several software use cases and the
% recommended NTL variant that {\em portable} software---i.e., software not
% tuned for any specific implementation's memory hierarchy---should use in each
% case.

\begin{table}[h!]
\begin{center}
\begin{tabular}{|l|l|}
\hline
场景 & 推荐的NTL变体 \\
\hline
访问一个\wunits{64}{KiB}~\wunits{256}{KiB}尺寸的工作集 & NTL.P1 \\
访问一个\wunits{256}{KiB}~\wunits{1}{MiB}尺寸的工作集  & NTL.PALL \\
访问一个尺寸超过\wunits{1}{MiB}的工作集        & NTL.S1 \\
没有可利用的时间局部性的访问(例如,流)                & NTL.ALL \\
访问一个竞争的同步变量                                & NTL.PALL \\
\hline
\end{tabular}
\end{center}
\caption{为可移植软件推荐的在各种场景中采用的NTL变体。}
\label{tab:ntl-portable}
\end{table}

\begin{commentary}
缓存尺寸将在实现之间明显变化,因此表\ref{tab:ntl-portable}中列出的工作集尺寸仅仅是粗略的指导意见。
% Cache sizes will obviously vary between implementations, and so the working-set
% sizes listed in Table~\ref{tab:ntl-portable} are merely rough guidelines.
\end{commentary}

表~\ref{tab:ntl}列出了一些样例内存层次,以及各NTL变体如何映射到各缓存层次的建议。
该表也推荐了为实现所调整的软件在申请特定层次缓存时应当避免的NTL变体。
例如,对于一个具有私有L1和共享L2的系统,表格推荐NTL.P1和NTL.PALL,表示时间局部性不能被L1利用,
而NTL.S1和NTL.ALL表示时间局部性不能被L2利用。
进一步地,为这种系统所调整的软件应当使用NTL.P1,以表示缺少可以被L1利用的时间局部性,或者应当使用NTL.ALL表示缺少可以被L2利用的时间局部性。
% Table~\ref{tab:ntl} lists several sample memory hierarchies and recommends
% how each NTL variant maps onto each cache level.
% The table also recommends which NTL variant that implementation-tuned
% software should use to avoid allocating in a particular cache level.
% For example, for a system with a private L1 and a shared L2, it is recommended
% that NTL.P1 and NTL.PALL indicate that temporal locality cannot be exploited by
% the L1, and that NTL.S1 and NTL.ALL indicate that temporal locality cannot be
% exploited by the L2.
% Furthermore, software tuned for such a system should use NTL.P1 to indicate
% a lack of temporal locality exploitable by the L1, or should use NTL.ALL
% indicate a lack of temporal locality exploitable by the L2.

\begin{table}[h!]
\begin{center}
\scalebox{0.95}{
\begin{tabular}{|l|WWWW|WWWW|}
\hline
内存层次 & \multicolumn{4}{c|}{NTL变体到实际缓存层次的推荐映射}    & \multicolumn{4}{c|}{为显式缓存管理推荐的NTL变体} \\
%                 & \multicolumn{4}{c|}{variant to actual cache level} & \multicolumn{4}{c|}{explicit cache management} \\
\hline
                                 & P1  & PALL& S1  & ALL& L1  & L2  & L3  & L4/L5  \\
\hline
\multicolumn{9}{|c|}{常见场景} \\
\hline
无缓存            & \multicolumn{4}{c|}{---} & \multicolumn{4}{c|}{\em none} \\
\hline
仅有私有L1                  & L1  & L1  & L1  & L1 & ALL & --- & --- & --- \\
私有L1,共享L2            & L1  & L1  & L2  & L2 & P1  & ALL & --- & --- \\
私有L1,共享L2/L3         & L1  & L1  & L2  & L3 & P1  & S1  & ALL & --- \\
私有 L1/L2                    & L1  & L2  & L2  & L2 & P1  & ALL & --- & --- \\
私有 L1/L2; 共享 L3         & L1  & L2  & L3  & L3 & P1  & PALL& ALL & --- \\
私有 L1/L2; 共享 L3/L4      & L1  & L2  & L3  & L4 & P1  & PALL& S1  & ALL \\
\hline
\multicolumn{9}{|c|}{不常见的场景} \\
\hline
私有 L1/L2/L3; 共享 L4      & L1  & L3  & L4  & L4 & P1  & P1  & PALL& ALL \\
私有 L1; 共享 L2/L3/L4      & L1  & L1  & L2  & L4 & P1  & S1  & ALL & ALL \\
私有 L1/L2; 共享 L3/L4/L5   & L1  & L2  & L3  & L5 & P1  & PALL& S1  & ALL \\
私有 L1/L2/L3; 共享 L4/L5   & L1  & L3  & L4  & L5 & P1  & P1  & PALL& ALL \\
\hline
\end{tabular}}
\end{center}
\caption{NTL变体到各种内存层次的映射。}
\label{tab:ntl}
\end{table}

如果提供了C扩展,也会提供这些HINT的压缩变体:
C.NTL.P1被编码为\mbox{C.ADD {\em x0, x2}};
C.NTL.PALL被编码为\mbox{C.ADD {\em x0, x3}};
C.NTL.S1 被编码为\mbox{C.ADD {\em x0, x4}};
以及C.NTL.ALL被编码为\mbox{C.ADD {\em x0, x5}}。
% If the C extension is provided, compressed variants of these HINTs are also
% provided:
% C.NTL.P1 is encoded as \mbox{C.ADD {\em x0, x2}};
% C.NTL.PALL is encoded as \mbox{C.ADD {\em x0, x3}};
% C.NTL.S1 is encoded as \mbox{C.ADD {\em x0, x4}};
% and C.NTL.ALL is encoded as \mbox{C.ADD {\em x0, x5}}.

NTL指令影响除Zicbom扩展中缓存管理指令外的所有内存访问指令。
% The NTL instructions affect all memory-access instructions except the
% cache-management instructions in the Zicbom extension.

\begin{commentary}
在撰写本文时,对于这条规则还没有其它的例外,
因此NTL指令会影响在基础ISA和A、F、D、Q、C及V标准扩展中定义的所有内存访问指令,
也会影响那些在卷II中hypervisor扩展中定义的内存访问指令。
% As of this writing, there are no other exceptions to this rule, and so
% the NTL instructions affect all memory-access instructions
% defined in the base ISAs and the A, F, D, Q, C, and V standard extensions,
% as well as those defined within the hypervisor extension in Volume II.

NTL指令可以影响除Zicbom以外的缓存管理操作。
例如,NTL.PALL后跟CBO.ZERO可以表示该行应该在L3中分配并被清零,但是不能在L1或L2中分配。
% The NTL instructions can affect cache-management operations other than those
% in the Zicbom extension.
% For example, NTL.PALL followed by CBO.ZERO might indicate
% that the line should be allocated in L3 and zeroed, but not allocated in
% L1 or L2.
\end{commentary}

当一个NTL指令被应用到Zicbop扩展中的预取提示时,它表示缓存行应当被预取到比NTL所指定的层次更外层的缓存中。
% When an NTL instruction is applied to a prefetch hint in the Zicbop extension,
% it indicates that a cache line should be prefetched into a cache that is
% {\em outer} from the level specified by the NTL.

\begin{commentary}
例如,在一个具有私有L1和共享L2的系统中,NTL.P1后跟PREFETCH.R可以以读意图预取到L2中。
% For example, in a system with a private L1 and shared L2, NTL.P1 followed by
% PREFETCH.R might prefetch into L2 with read intent.

为了预取到最内层的缓存中,不要将NTL指令作为预取指令的前缀。
% To prefetch into the innermost level of cache, do not prefix the prefetch
% instruction with an NTL instruction.

在某些系统中,NTL.ALL后跟一条预取指令可以预取到内存控制器内部的缓存中或者预取缓冲区中。
% In some systems, NTL.ALL followed by a prefetch instruction might prefetch
% into a cache or prefetch buffer internal to a memory controller.
\end{commentary}

不鼓励软件在一条NTL指令之后跟一条并不明确访问内存的指令。不遵守此建议可能会降低性能,但是没有架构上可见的影响。
% Software is discouraged from following an NTL instruction with an
% instruction that does not explicitly access memory.
% Nonadherence to this recommendation might reduce performance but
% otherwise has no architecturally visible effect.

在目标指令上发生陷入的事件中,不鼓励实现将NTL应用到陷入处理器中的第一条指令。相反,在这种情况下,建议实现忽略HINT。
% In the event that a trap is taken on the target instruction,
% implementations are discouraged from applying the NTL to the first instruction
% in the trap handler.
% Instead, implementations are recommended to ignore the HINT in this case.

\begin{commentary}
如果在一条NTL指令与它的目标指令之间发生了中断,那么执行通常将在目标指令处恢复。不被重新执行的NTL指令并不会改变程序的语义。
% If an interrupt occurs between the execution of an NTL instruction and its
% target instruction, execution will normally resume at the
% target instruction.
% That the NTL instruction is not reexecuted does not change the semantics of
% the program.

某些实现可能希望在目标指令被发现之前不处理NTL指令(例如,这样可以使NTL与其修改的内存访问相融合)。
这种实现可能优先在NTL之前进行中断,而不是在NTL与内存访问之间中断。
% Some implementations might prefer not to process the NTL instruction until the
% target instruction is seen (e.g., so that the NTL can be
% fused with the memory access it modifies).
% Such implementations might preferentially take the interrupt before the NTL,
% rather than between the NTL and the memory access.
\end{commentary}

\begin{commentary}
由于NTL指令被编码为ADD,因此它们可以在LR/SC循环中使用,而不回避向前执行保证。
但是,由于在LR/SC循环中使用其它的加载和存储确实会避开向前执行保证,因此在这种循环中使用NTL的唯一原因是修改LR或SC。
% Since the NTL instructions are encoded as ADDs, they can be used within LR/SC
% loops without voiding the forward-progress guarantee.
% But, since using other loads and stores within an LR/SC loop {\em does}
% void the forward-progress guarantee, the only reason to use an NTL
% within such a loop is to modify the LR or the SC.
\end{commentary}

\chapter{“Zihintpause”暂停提示(2.0版本)}
\label{chap:zihintpause}

PAUSE指令是一个表示当前硬件线程的指令引退率应当暂时减少或暂停的HINT。
它影响的持续时间必须是有界的,可以是零。没有架构状态被改变。
% The PAUSE instruction is a HINT that indicates the current hart's rate of
% instruction retirement should be temporarily reduced or paused.  The duration of its
% effect must be bounded and may be zero.  No architectural state is changed.

\begin{commentary}
软件可以在执行自旋等待的代码序列时,使用PAUSE指令来减少能耗。在执行PAUSE时,多线程核心可以暂时地放弃执行资源,让给其它硬件线程。
建议使PAUSE指令通常性地包含在自旋等待循环的代码序列中。
% Software can use the PAUSE instruction to reduce energy consumption while
% executing spin-wait code sequences.  Multithreaded cores might temporarily
% relinquish execution resources to other harts when PAUSE is executed.
% It is recommended that a PAUSE instruction generally be included in the code
% sequence for a spin-wait loop.

未来的扩展可能添加类似于x86 MONITOR/MWAIT指令的原语,这提供了一种更有效的机制来等待对特定内存位置的写入。
然而,这些指令不会取代PAUSE。当轮询非内存事件时、轮询多重事件时、或者软件不确切地知道它正在轮询什么事件时,PAUSE是更合适的。
% A future extension might add primitives similar to the x86 MONITOR/MWAIT
% instructions, which provide a more efficient mechanism to wait on writes to
% a specific memory location.
% However, these instructions would not supplant PAUSE.
% PAUSE is more appropriate when polling for non-memory events, when polling for
% multiple events, or when software does not know precisely what events it is
% polling for.

PAUSE指令效果的持续时间,在实现内部和实现之间可以有显著变化。
在典型的实现中,该持续时间应当远小于执行一次上下文切换的时间,
可能多于一次片上缓存未命中延迟或一次对主内存无缓存访问的粗略次序。
% The duration of a PAUSE instruction's effect may vary significantly within and
% among implementations.
% In typical implementations this duration should be much less than the time to
% perform a context switch, probably more on the rough order of an on-chip cache
% miss latency or a cacheless access to main memory.

一系列PAUSE指令可以被用于创建与PAUSE指令数目粗略成比例的累积延迟。
然而,在可移植代码的自旋等待循环中,在重新评估循环条件之前应当仅使用一条PAUSE指令,
否则硬件线程可能在某些实现中拖延比最优更长的时间,从而降低系统的性能。
% A series of PAUSE instructions can be used to create a cumulative delay loosely
% proportional to the number of PAUSE instructions.
% In spin-wait loops in portable code, however, only one PAUSE instruction should
% be used before re-evaluating loop conditions, else the hart might stall longer
% than optimal on some implementations, degrading system performance.
\end{commentary}

PAUSE被编码为{\em pred}=W, {\em succ}=0, {\em fm}=0, {\em rd}={\tt x0},和 {\em rs1}={\tt x0}。
% PAUSE is encoded as a FENCE instruction with {\em pred}=W, {\em succ}=0,
% {\em fm}=0, {\em rd}={\tt x0}, and {\em rs1}={\tt x0}.

\begin{commentary}
PAUSE被编码为FENCE操作码中的一条提示,因为某些实现有可能故意拖延PAUSE指令,直到完成尚未完成的内存事务。
然而,由于后继集为空,PAUSE并不强制要求任何特定的内存次序——因此,它确实是一个HINT。
% PAUSE is encoded as a hint within the FENCE opcode because some
% implementations are expected to deliberately stall the PAUSE instruction until outstanding
% memory transactions have completed.
% Because the successor set is null, however, PAUSE does not {\em mandate} any
% particular memory ordering---hence, it truly is a HINT.

像其它FENCE指令一样,PAUSE不能被用在LR/SC序列中而不避开向前执行保证。
% Like other FENCE instructions, PAUSE cannot be used within LR/SC sequences
% without voiding the forward-progress guarantee.

前驱集W的选择是任意的,因为后继集为空。其它类似于PAUSE的HINT可能与其它前驱集一同编码。
% The choice of a predecessor set of W is arbitrary, since the successor set is
% null.
% Other HINTs similar to PAUSE might be encoded with other predecessor sets.
\end{commentary}

\chapter{RV32E和RV64E基础整数指令集(1.95版本)}
\label{rv32e}

这章描述了一个RV32E和RV64E基础整数指令集的建议草案,它们是为嵌入式系统的微控制器设计的。
RV32E和RV64E分别是RV32I和RV64I的简化版本:
仅有的改变是把整数寄存器的数目减少到了16个。
这章仅仅概述了RV32E/RV64E和RV32I/RV64I之间的不同,并因此应当被放在第~\ref{rv32}章和第~\ref{rv64}章之后阅读。
% This chapter describes a proposal for the RV32E and RV64E base integer
% instruction sets, designed for microcontrollers in embedded systems.
% RV32E and RV64E are reduced versions of RV32I and RV64I, respectively:
% the only change is to reduce the number of integer registers to 16.
% This chapter only outlines the differences between RV32E/RV64E and
% RV32I/RV64I, and so should be read after Chapters~\ref{rv32} and
% \ref{rv64}.

\begin{commentary}
RV32E被设计为,为嵌入式微控制器提供一个更小的基础核心。
RV64E也有兴趣用于大型SoC设计中的微控制器、以及减少高线程(highly thread)64位处理器的上下文状态。
% RV32E was designed to provide an even smaller base core for embedded
% microcontrollers.  There is also interest in RV64E for
% microcontrollers within large SoC designs, and to reduce context state
% for highly threaded 64-bit processors.

RV32E和RV64E可以与所有当前的标准扩展相组合。
% RV32E and RV64E can be combined with all current standard extensions.
\end{commentary}

\section{RV32E和RV64E编程模型}

RV32E和RV64E把整数寄存器的数目减少到16个通用目的寄存器,({\tt x0}--{\tt x15}),这里{\tt x0}是一个专用的零寄存器。
% RV32E and RV64E reduce the integer register count to 16
% general-purpose registers, ({\tt x0}--{\tt x15}), where {\tt x0} is a
% dedicated zero register.

\begin{commentary}
我们已经发现,在小型RV32I内核设计中,较高的16个寄存器消费了除内存外的所有内核区域的大约四分之一,
因此它们的移除节省了大约25\%的内存区域,而内核的电量也相应地减少了。
% We have found that in the small RV32I core implementations, the upper
% 16 registers consume around one quarter of the total area of the core
% excluding memories, thus their removal saves around 25\% core area
% with a corresponding core power reduction.
\end{commentary}

\section{RV32E和RV64E指令集编码}

RV32E及RV64E分别使用与RV32I和RV64I相同的指令集编码,但是只提供寄存器{\tt x0} - {\tt x15}。
所有指定其它寄存器{\tt x16} - {\tt x31}的编码都是保留的
% RV32E and RV64E use the same instruction-set encoding as RV32I and
% RV64I respectively, except that only registers {\tt x0}--{\tt x15} are
% provided.  All encodings specifying the other registers {\tt x16}--{\tt
%   x31} are reserved.

\begin{commentary}
本章的前一稿将所有使用{\tt x16} - {\tt x31}寄存器的编码都可用于自定义的。
这一版本采用了一种更加保守的方法,让这些编码被保留,以便以后可以在自定义空间或新的标准编码之间分配它们。
% The previous draft of this chapter made all encodings using the {\tt
%   x16}--{\tt x31} registers available as custom.  This version takes a
% more conservative approach, making these reserved so that they can be
% allocated between custom space or new standard encodings at a later
% date.
\end{commentary}

\chapter{RV64I基础整数指令集(2.1版本)}
\label{rv64}

这章描述了RV64I基础整数指令集,它是在第~\ref{rv32}章中描述的RV32I变体之上构建的。
这章只呈现了与RV32I的不同,所以应当与那篇更早的章节结合着阅读。
% This chapter describes the RV64I base integer instruction set, which
% builds upon the RV32I variant described in Chapter~\ref{rv32}.  This
% chapter presents only the differences with RV32I, so should be read in
% conjunction with the earlier chapter.

\section{寄存器状态}

RV64I把整数寄存器和所支持的用户地址空间拓宽到64位(表~\ref{gprs}中XLEN=64)。
% RV64I widens the integer registers and supported user address space to
% 64 bits (XLEN=64 in Figure~\ref{gprs}).

\section{整数运算指令}

大多数整数运算指令在XLEN位的值上操作。在RV64I中提供了额外的指令变体来操作32位的值,通过在操作码上添加“W”后缀来表示。
这些“*W”指令忽略了它们的输入的高32位,并且总是产生32位有符号的值、把它们符号扩展到64位,也就是说,从位XLEN-1位到位31是相等的。
% Most integer computational instructions operate on XLEN-bit values.
% Additional instruction variants are provided to manipulate 32-bit
% values in RV64I, indicated by a `W' suffix to the opcode.  These
% ``*W'' instructions ignore the upper 32 bits of their inputs and
% always produce 32-bit signed values, sign-extending them to 64 bits,
% i.e. bits XLEN-1 through 31 are equal.

\begin{commentary}
编译器和调用约定维持了一种不变性,即在64位寄存器中,所有的32位值都以一种符号扩展的格式被保持。
甚至32位无符号整数也会把位31扩展到位63~32。
因此,在无符号32位整数和有符号32位整数之间的转换是一个no-op,从一个有符号32位整数转换到一个有符号64位整数也是如此。
在这种不变性下,现有的64位宽SLTU和无符号分支比较仍然能正确地操作无符号32位整数。
类似地,现有的在32位符号扩展整数上的64位宽逻辑操作保留了符号扩展属性。加法和移位需要少量的新指令(ADD[I]W/SUBW/SxxW),
以确保32位值的合理的性能。
% The compiler and calling convention maintain an invariant that all 32-bit
% values are held in a sign-extended format in 64-bit registers.  Even 32-bit
% unsigned integers extend bit 31 into bits 63 through 32.  Consequently,
% conversion between unsigned and signed 32-bit integers is a no-op,
% as is conversion from a signed 32-bit integer to a signed 64-bit
% integer.  Existing 64-bit wide SLTU and unsigned branch compares still operate
% correctly on unsigned 32-bit integers under this invariant.  Similarly,
% existing 64-bit wide logical operations on 32-bit sign-extended integers
% preserve the sign-extension property.  A few new instructions
% (ADD[I]W/SUBW/SxxW) are required for addition and shifts to ensure reasonable
% performance for 32-bit values.
\end{commentary}

\newpage
\subsubsection*{整数寄存器 - 立即数指令}
\vspace{-0.4in}
\begin{center}
\begin{tabular}{M@{}R@{}S@{}R@{}O}
\\
\instbitrange{31}{20} &
\instbitrange{19}{15} &
\instbitrange{14}{12} &
\instbitrange{11}{7} &
\instbitrange{6}{0} \\
\hline
\multicolumn{1}{|c|}{imm[11:0]} &
\multicolumn{1}{c|}{rs1} &
\multicolumn{1}{c|}{funct3} &
\multicolumn{1}{c|}{rd} &
\multicolumn{1}{c|}{opcode} \\
\hline
12 & 5 & 3 & 5 & 7 \\
I-立即数[11:0] & src & ADDIW  & dest & OP-IMM-32 \\
\end{tabular}
\end{center}

ADDIW是一个RV64I指令,它把符号扩展的12位立即数加到寄存器{\em rs1},并在{\em rd}中产生合适的32位符号扩展的结果。
运算结果的低32位符号扩展到64位作为结果,而忽略了溢出。
注意,{\em rd, rs1, 0}把寄存器{\em rs1}的低32位的符号扩展写入寄存器{\em rd}(汇编器伪指令SEXT.W)。
% ADDIW is an RV64I instruction that adds the sign-extended 12-bit
% immediate to register {\em rs1} and produces the proper sign-extension
% of a 32-bit result in {\em rd}.  Overflows are ignored and the result
% is the low 32 bits of the result sign-extended to 64 bits.  Note,
% ADDIW {\em rd, rs1, 0} writes the sign-extension of the lower 32 bits
% of register {\em rs1} into register {\em rd} (assembler pseudoinstruction
% SEXT.W).

\vspace{-0.4in}
\begin{center}
\begin{tabular}{R@{}W@{}R@{}R@{}R@{}R@{}O}
\\
\instbitrange{31}{26} &
\multicolumn{1}{c}{\instbit{25}} &
\instbitrange{24}{20} &
\instbitrange{19}{15} &
\instbitrange{14}{12} &
\instbitrange{11}{7} &
\instbitrange{6}{0} \\
\hline
\multicolumn{1}{|c|}{imm[11:6]} &
\multicolumn{1}{|c|}{imm[5]} &
\multicolumn{1}{|c|}{imm[4:0]} &
\multicolumn{1}{c|}{rs1} &
\multicolumn{1}{c|}{funct3} &
\multicolumn{1}{c|}{rd} &
\multicolumn{1}{c|}{opcode} \\
\hline
6 & \multicolumn{1}{c}{1} & 5 & 5 & 3 & 5 & 7 \\
000000 & shamt[5] & shamt[4:0]  & src & SLLI & dest & OP-IMM \\
000000 & shamt[5] & shamt[4:0]  & src & SRLI & dest & OP-IMM \\
010000 & shamt[5] & shamt[4:0]  & src & SRAI & dest & OP-IMM \\
000000 &       0  & shamt[4:0]  & src & SLLIW & dest & OP-IMM-32 \\
000000 &       0  & shamt[4:0]  & src & SRLIW & dest & OP-IMM-32 \\
010000 &       0  & shamt[4:0]  & src & SRAIW & dest & OP-IMM-32 \\
\end{tabular}
\end{center}

按常量移位被编码为一种专门化的I类型格式,它使用与RV32I相同的指令操作码。
对于RV64I,被移位的操作数在{\em rs1}中,移位的数目被编码在I立即数域的低6位中。
右移类型被编码在底第30位上。
SLLI是逻辑左移(移位后低位补零);SRLI是逻辑右移(移位后高位补零);而SRAI是算数右移(原始符号位被复制到空白的高位中)。
% Shifts by a constant are encoded as a specialization of the I-type
% format using the same instruction opcode as RV32I.  The operand to be
% shifted is in {\em rs1}, and the shift amount is encoded in the lower
% 6 bits of the I-immediate field for RV64I.  The right shift type is
% encoded in bit 30.  SLLI is a logical left shift (zeros are shifted
% into the lower bits); SRLI is a logical right shift (zeros are shifted
% into the upper bits); and SRAI is an arithmetic right shift (the
% original sign bit is copied into the vacated upper bits).

SLLIW、SRLIW和SRAIW是RV64I中独有的指令,它们的定义类似,但是在32位值上操作,并把它们的32位结果符号扩展到64位。
带有$imm[5] \neq 0$的SLLIW、SRLIW和SRAIW的编码是保留的。
% SLLIW, SRLIW, and SRAIW are RV64I-only instructions that are
% analogously defined but operate on 32-bit values and
% sign-extend their 32-bit results to 64 bits.
% SLLIW, SRLIW, and SRAIW encodings with $imm[5] \neq 0$ are reserved.

\begin{commentary}
  先前,$imm[5] \neq 0$的SLLIW、SRLIW和SRAIW被定义为:引发非法指令异常,而现在它们被标记为保留的。
  这是一个向后兼容的改变。
  % Previously, SLLIW, SRLIW, and SRAIW with $imm[5] \neq 0$ were defined to
  % cause illegal instruction exceptions, whereas now they are marked as
  % reserved.  This is a backwards-compatible change.
\end{commentary}

\vspace{-0.2in}
\begin{center}
\begin{tabular}{U@{}R@{}O}
\\
\instbitrange{31}{12} &
\instbitrange{11}{7} &
\instbitrange{6}{0} \\
\hline
\multicolumn{1}{|c|}{imm[31:12]} &
\multicolumn{1}{c|}{rd} &
\multicolumn{1}{c|}{opcode} \\
\hline
20 & 5 & 7 \\
U-立即数[31:12] & dest & LUI \\
U-立即数[31:12] & dest & AUIPC
\end{tabular}
\end{center}

LUI(加载高位立即数)使用与RV32I相同的操作码。
LUI把32位的U立即数放进寄存器{\em rd}中,并把最低的12位填充为零。该32位结果被符号扩展到64位。
% LUI (load upper immediate) uses the same opcode as RV32I.  LUI places
% the 32-bit U-immediate into register {\em rd}, filling in the lowest 12
% bits with zeros.
% The 32-bit result is sign-extended to 64 bits.

AUIPC(加高位立即数到{\tt pc})使用与RV32I相同的操作码。
AUIPC被用于构建关于{\tt pc}的相对地址,并使用U类型格式。
AUIPC从U立即数形成32位偏移量,并把最低的12位填充为零,把结果符号扩展到64位,把它加到AUIPC指令的地址,
然后把结果放进寄存器{\em rd}中。
% AUIPC (add upper immediate to {\tt pc}) uses the same opcode as RV32I.
% AUIPC is used to build {\tt
%   pc}-relative addresses and uses the U-type format.  AUIPC forms a 32-bit
% offset from the U-immediate, filling in the lowest 12 bits with zeros,
% sign-extends the result to 64 bits,
% adds it to the address of the AUIPC instruction,
% then places the result in register {\em rd}.

\begin{commentary}
注意,可以通过将LUI与LD配对、将AUIPC与JALR配对等方式形成的偏移量的地址集是[${-}2^{31}{-}2^{11}$, $2^{31}{-}2^{11}{-}1$]。
% Note that the set of address offsets that can be formed by pairing LUI
% with LD, AUIPC with JALR, etc.\@ in RV64I is
% [${-}2^{31}{-}2^{11}$, $2^{31}{-}2^{11}{-}1$].
\end{commentary}

\subsubsection*{整数寄存器—寄存器操作}

\vspace{-0.2in}
\begin{center}
\begin{tabular}{S@{}R@{}R@{}S@{}R@{}O}
\\
\instbitrange{31}{25} &
\instbitrange{24}{20} &
\instbitrange{19}{15} &
\instbitrange{14}{12} &
\instbitrange{11}{7} &
\instbitrange{6}{0} \\
\hline
\multicolumn{1}{|c|}{funct7} &
\multicolumn{1}{c|}{rs2} &
\multicolumn{1}{c|}{rs1} &
\multicolumn{1}{c|}{funct3} &
\multicolumn{1}{c|}{rd} &
\multicolumn{1}{c|}{opcode} \\
\hline
7 & 5 & 5 & 3 & 5 & 7 \\
0000000 & src2 & src1 & SLL/SRL     & dest & OP    \\
0100000 & src2 & src1 & SRA         & dest & OP    \\
0000000 & src2 & src1 & ADDW        & dest & OP-32    \\
0000000 & src2 & src1 & SLLW/SRLW   & dest & OP-32    \\
0100000 & src2 & src1 & SUBW/SRAW   & dest & OP-32    \\
\end{tabular}
\end{center}

ADDW和SUBW是RV64I独有的指令,它们的定义类似于ADD和SUB,但是在32位值上操作,并产生有符号的32位结果。
溢出被忽略,且结果的低32位被符号扩展到64位,并写到目的寄存器。
% ADDW and SUBW are RV64I-only instructions that are defined analogously
% to ADD and SUB but operate on 32-bit values and produce signed 32-bit
% results.  Overflows are ignored, and the low 32-bits of the result is
% sign-extended to 64-bits and written to the destination register.

SLL、SRL和SRA对寄存器{\em rs1}中的值实施逻辑左移、逻辑右移和算数右移,
移位的数目保持在寄存器{\em rs2}中。在RV64I中,只有{\em rs2}的低6位被考虑用于移位数目。
% SLL, SRL, and SRA perform logical left, logical right, and arithmetic
% right shifts on the value in register {\em rs1} by the shift amount
% held in register {\em rs2}.  In RV64I, only the low 6 bits of {\em
%   rs2} are considered for the shift amount.

SLLW、SRW和SRAW是RV64I独有的指令,它们的定义类似,但是在32位值上操作,并把它们的32位结果符号扩展到64位。
移位数量由{\em rs2[4:0]}给出。
% SLLW, SRLW, and SRAW are RV64I-only instructions that are analogously
% defined but operate on 32-bit values and
% sign-extend their 32-bit results to 64 bits.
% The shift amount is given by {\em rs2[4:0]}.

\section{加载和存储指令}

RV64I把地址空间扩展到了64位。执行环境将定义地址空间的哪部分对于访问是合法的。
% RV64I extends the address space to 64 bits.  The execution environment
% will define what portions of the address space are legal to access.

\vspace{-0.4in}
\begin{center}
\begin{tabular}{M@{}R@{}S@{}R@{}O}
\\
\instbitrange{31}{20} &
\instbitrange{19}{15} &
\instbitrange{14}{12} &
\instbitrange{11}{7} &
\instbitrange{6}{0} \\
\hline
\multicolumn{1}{|c|}{imm[11:0]} &
\multicolumn{1}{c|}{rs1} &
\multicolumn{1}{c|}{funct3} &
\multicolumn{1}{c|}{rd} &
\multicolumn{1}{c|}{opcode} \\
\hline
12 & 5 & 3 & 5 & 7 \\
offset[11:0] & base & width & dest & LOAD \\
\end{tabular}
\end{center}

\vspace{-0.2in}
\begin{center}
\begin{tabular}{O@{}R@{}R@{}S@{}R@{}O}
\\
\instbitrange{31}{25} &
\instbitrange{24}{20} &
\instbitrange{19}{15} &
\instbitrange{14}{12} &
\instbitrange{11}{7} &
\instbitrange{6}{0} \\
\hline
\multicolumn{1}{|c|}{imm[11:5]} &
\multicolumn{1}{c|}{rs2} &
\multicolumn{1}{c|}{rs1} &
\multicolumn{1}{c|}{funct3} &
\multicolumn{1}{c|}{imm[4:0]} &
\multicolumn{1}{c|}{opcode} \\
\hline
7 & 5 & 5 & 3 & 5 & 7 \\
offset[11:5] & src & base & width & offset[4:0] & STORE \\
\end{tabular}
\end{center}

对于RV64I,LD指令从内存加载一个64位的值到寄存器{\em rd}。
% The LD instruction loads a 64-bit value from memory into register {\em
%   rd} for RV64I.

对于RV64I,LW指令从内存加载一个32位的值,并把它符号扩展到64位,然后将其存储到寄存器{\em rd}。
另一方面,RV64I的LWU指令则会对内存中的32位值使用零扩展。
类似地,LH和LHU被定义用于16位值,以及LB和LBU用于8位值。
SD、SW、SH和SB指令分别把寄存器{\em rs2}的低64位、32位、16位和8位值存储到内存。
% The LW instruction loads a 32-bit value from memory and sign-extends
% this to 64 bits before storing it in register {\em rd} for RV64I.  The
% LWU instruction, on the other hand, zero-extends the 32-bit value from
% memory for RV64I.  LH and LHU are defined analogously for 16-bit
% values, as are LB and LBU for 8-bit values.  The SD, SW, SH, and SB
% instructions store 64-bit, 32-bit, 16-bit, and 8-bit values from the
% low bits of register {\em rs2} to memory respectively.

\section{“提示”指令}
\label{sec:rv64i-hints}

所有在RV32I中作为微架构HINT的指令(见~\ref{sec:rv32i-hints}节)也是RV64I中的HINT。
RV64I中的额外的运算指令同时扩展了标准HINT和自定义HINT的编码空间。
% All instructions that are microarchitectural HINTs in RV32I (see
% Section~\ref{sec:rv32i-hints}) are also HINTs in RV64I.  The
% additional computational instructions in RV64I expand both the standard and
% custom HINT encoding spaces.

表~\ref{tab:rv64i-hints}列出了所有的RV64I HINT代码点。91\%的HINT空间被保留用于标准HINT,
但是目前还没有被定义。其余的HINT空间被指定用于自定义HINT:不会有标准HINT将被定义在这个子空间中。
% Table~\ref{tab:rv64i-hints} lists all RV64I HINT code points.  91\% of the HINT
% space is reserved for standard HINTs.  The
% remainder of the HINT space is designated for custom HINTs: no standard HINTs
% will ever be defined in this subspace.

\begin{table}[hbt]
\centering
\begin{tabular}{|l|l|c|l|}
  \hline
  指令                  & 约束                                        & 代码点 & 目的\\ \hline \hline
  LUI                   & {\em rd}={\tt x0}                           & $2^{20}$                    & \multirow{11}{*}{\em 保留供未来标准使用} \\ \cline{1-3}
  AUIPC                 & {\em rd}={\tt x0}                           & $2^{20}$                    & \\ \cline{1-3}
  \multirow{2}{*}{ADDI} & {\em rd}={\tt x0}, 要么               & \multirow{2}{*}{$2^{17}-1$} & \\
                        & {\em rs1}$\neq${\tt x0} 要么 {\em imm}$\neq$0 &                             & \\ \cline{1-3}
  ANDI                  & {\em rd}={\tt x0}                           & $2^{17}$                    & \\ \cline{1-3}
  ORI                   & {\em rd}={\tt x0}                           & $2^{17}$                    & \\ \cline{1-3}
  XORI                  & {\em rd}={\tt x0}                           & $2^{17}$                    & \\ \cline{1-3}
  ADDIW                 & {\em rd}={\tt x0}                           & $2^{17}$                    & \\ \cline{1-3}
  ADD                   & {\em rd}={\tt x0}, {\em rs1}$\neq${\tt x0}  & $2^{10}-32$                 & \\ \cline{1-3}
  \multirow{2}{*}{ADD}  & {\em rd}={\tt x0}, {\em rs1}={\tt x0},      & \multirow{2}{*}{$28$}       & \\
                        & {\em rs2}$\neq${\tt x2}--{\tt x5}           &                             & \\ \hline
  \multirow{4}{*}{ADD}  & \multirow{4}{*}{\shortstack[l]{{\em rd}={\tt x0}, {\em rs1}={\tt x0}, \\{\em rs2}={\tt x2}--{\tt x5}}}
                                                                      & \multirow{4}{*}{$4$}        & ({\em rs2}={\tt x2}) NTL.P1 \\
                        &                                             &                             & ({\em rs2}={\tt x3}) NTL.PALL \\
                        &                                             &                             & ({\em rs2}={\tt x4}) NTL.S1 \\
                        &                                             &                             & ({\em rs2}={\tt x5}) NTL.ALL \\ \hline
  SUB                   & {\em rd}={\tt x0}                           & $2^{10}$                    & \multirow{22}{*}{\em 保留供未来标准使用} \\ \cline{1-3}
  AND                   & {\em rd}={\tt x0}                           & $2^{10}$                    & \\ \cline{1-3}
  OR                    & {\em rd}={\tt x0}                           & $2^{10}$                    & \\ \cline{1-3}
  XOR                   & {\em rd}={\tt x0}                           & $2^{10}$                    & \\ \cline{1-3}
  SLL                   & {\em rd}={\tt x0}                           & $2^{10}$                    & \\ \cline{1-3}
  SRL                   & {\em rd}={\tt x0}                           & $2^{10}$                    & \\ \cline{1-3}
  SRA                   & {\em rd}={\tt x0}                           & $2^{10}$                    & \\ \cline{1-3}
  ADDW                  & {\em rd}={\tt x0}                           & $2^{10}$                    & \\ \cline{1-3}
  SUBW                  & {\em rd}={\tt x0}                           & $2^{10}$                    & \\ \cline{1-3}
  SLLW                  & {\em rd}={\tt x0}                           & $2^{10}$                    & \\ \cline{1-3}
  SRLW                  & {\em rd}={\tt x0}                           & $2^{10}$                    & \\ \cline{1-3}
  SRAW                  & {\em rd}={\tt x0}                           & $2^{10}$                    & \\ \cline{1-3}
  \multirow{3}{*}{FENCE}& {\em rd}={\tt x0}, {\em rs1}$\neq${\tt x0}, & \multirow{3}{*}{$2^{10}-63$}& \\
                        & {\em fm}=0, 且 {\em pred}=0 或者 {\em succ}=0                      &                             & \\ \cline{1-3}
  %                      &                &                             & \\ 
  \multirow{3}{*}{FENCE}& {\em rd}$\neq${\tt x0}, {\em rs1}={\tt x0}, & \multirow{3}{*}{$2^{10}-63$}& \\
                        & {\em fm}=0, 且 {\em pred}=0 或者 {\em succ}=0                     &                             & \\ \cline{1-3}
  %                      &                 &                             & \\ 
  \multirow{2}{*}{FENCE}& {\em rd}={\em rs1}={\tt x0}, {\em fm}=0,    & \multirow{2}{*}{15}         & \\
                        & {\em pred}=0, {\em succ}$\neq$0             &                             & \\ \cline{1-3}
  \multirow{2}{*}{FENCE}& {\em rd}={\em rs1}={\tt x0}, {\em fm}=0,    & \multirow{2}{*}{15}         & \\
                        & {\em pred}$\neq$W, {\em succ}=0             &                             & \\ \hline
  \multirow{2}{*}{FENCE}& {\em rd}={\em rs1}={\tt x0}, {\em fm}=0,    & \multirow{2}{*}{1}          & \multirow{2}{*}{暂停} \\
                        & {\em pred}=W, {\em succ}=0                  &                             & \\ \hline
  SLTI                  & {\em rd}={\tt x0}                           & $2^{17}$                    & \multirow{10}{*}{\em 指定供自定义使用} \\ \cline{1-3}
  SLTIU                 & {\em rd}={\tt x0}                           & $2^{17}$                    & \\ \cline{1-3}
  SLLI                  & {\em rd}={\tt x0}                           & $2^{11}$                    & \\ \cline{1-3}
  SRLI                  & {\em rd}={\tt x0}                           & $2^{11}$                    & \\ \cline{1-3}
  SRAI                  & {\em rd}={\tt x0}                           & $2^{11}$                    & \\ \cline{1-3}
  SLLIW                 & {\em rd}={\tt x0}                           & $2^{10}$                    & \\ \cline{1-3}
  SRLIW                 & {\em rd}={\tt x0}                           & $2^{10}$                    & \\ \cline{1-3}
  SRAIW                 & {\em rd}={\tt x0}                           & $2^{10}$                    & \\ \cline{1-3}
  SLT                   & {\em rd}={\tt x0}                           & $2^{10}$                    & \\ \cline{1-3}
  SLTU                  & {\em rd}={\tt x0}                           & $2^{10}$                    & \\ \hline
\end{tabular}
\caption{RV64I “提示”指令}
\label{tab:rv64i-hints}
\end{table}

\chapter{RV128I基础整数指令集(1.7版本)}
\label{rv128}

\begin{quote}
{\em  “在计算机设计中只可能发生一个难以恢复的错误——没有足够的地址位用于内存编址和内存管理。”} -- Bell和Strecker,ISCA-3,1976年。
% {\em ``There is only one mistake that can be made in computer design that is
% difficult to recover from---not having enough address bits for memory
% addressing and memory management.''} Bell and Strecker, ISCA-3, 1976.
\end{quote}

这章描述了RV128I,一个支持扁平128位地址空间的RISC-V ISA的变体。该变体是对现有的RV32I和RV64I设计的一种直接的外扩。
% This chapter describes RV128I, a variant of the RISC-V ISA
% supporting a flat 128-bit address space.  The variant is a
% straightforward extrapolation of the existing RV32I and RV64I designs.

\begin{commentary}
扩展整数寄存器宽度的主要原因是为了支持更大的地址空间。
还不清楚什么时候将会需要大于64位的扁平地址空间。
在编写本手册时,世界上最快的超级计算机,经Top500基准的衡量,有超过\wunits{1}{PB}的DRAM,
而且如果所有的DRAM都保留在单一地址空间中,将需要超过50位的地址空间。
一些仓储级(warehouse-scale)的计算机甚至已经包含了更大数量的DRAM,
且新型高密度固态非易失性存储器和快速互联技术可能驱使着甚至更大内存空间的需求。
超规模系统的研究把\wunits{100}{PB}的内存系统作为目标,它占据了57位地址空间。
根据历史的增长率,很可能在2030年以前就需要超过64位的地址空间了。
% The primary reason to extend integer register width is to support
% larger address spaces.  It is not clear when a flat address space larger
% than 64 bits will be required.  At the time of writing, the fastest
% supercomputer in the world as measured by the Top500 benchmark had
% over \wunits{1}{PB} of DRAM, and would require over 50 bits of address
% space if all the DRAM resided in a single address space.  Some
% warehouse-scale computers already contain even larger quantities of
% DRAM, and new dense solid-state non-volatile memories and fast
% interconnect technologies might drive a demand for even larger memory
% spaces.  Exascale systems research is targeting \wunits{100}{PB}
% memory systems, which occupy 57 bits of address space.  At historic
% rates of growth, it is possible that greater than 64 bits of address
% space might be required before 2030.

历史表明,无论何时,只要对超过64位地址空间的需要变得明确,架构师们都将重复关于替代扩展地址空间的激烈辩论,
包括分段、96位地址空间、和软件工作环境,直到最终,扁平128位地址空间被采纳为最简单和最佳的解决方案。
% History suggests that whenever it becomes clear that more than 64 bits
% of address space is needed, architects will repeat intensive debates
% about alternatives to extending the address space, including
% segmentation, 96-bit address spaces, and software workarounds, until,
% finally, flat 128-bit address spaces will be adopted as the simplest
% and best solution.

这时我们还没有冻结RV128规范,因为基于128位地址空间的实际用途,可能还有需要演化该设计。
% We have not frozen the RV128 spec at this time, as there might be need
% to evolve the design based on actual usage of 128-bit address spaces.
\end{commentary}

RV128I以与RV32I上构建RV64I的相同的方法构建于RV64I之上,把整数寄存器扩展到128位(也就是说,XLEN=128)。
大多数整数运算指令是没有变化的,因为它们被定义为在XLEN位上操作。
保留了RV64I在寄存器低位的32位值上操作的“*W”整数指令,但是现在把它们的结果从位31符号扩展到位127了。
添加了一个新的“*D”整数指令集,它在128位整数寄存器的低位中的64位值上进行操作,并把结果从位63符号扩展到位127。
“*D”指令消耗了标准32位编码中的两个主要的操作码(OP-IMM-64和OP-64)。
% RV128I builds upon RV64I in the same way RV64I builds upon RV32I, with
% integer registers extended to 128 bits (i.e., XLEN=128).  Most integer
% computational instructions are unchanged as they are defined to
% operate on XLEN bits.  The RV64I ``*W'' integer instructions that
% operate on 32-bit values in the low bits of a register are retained
% but now sign extend their results from bit 31 to bit 127. A new set of
% ``*D'' integer instructions are added that operate on 64-bit values
% held in the low bits of the 128-bit integer registers and sign extend
% their results from bit 63 to bit 127.  The ``*D'' instructions consume
% two major opcodes (OP-IMM-64 and OP-64) in the standard 32-bit
% encoding.

\begin{commentary}
  为了提升对RV64的兼容性,与处理RV32到RV64的做法相反,我们可以改变解码方式,
  比如把RV64I的ADD重命名为64位的ADDD,并在先前的OP-64主操作码(现在重命名为OP-128主操作码)中添加一个128位的ADDQ。
  % To improve compatibility with RV64, in a reverse of how RV32 to RV64
  % was handled, we might change the decoding around to rename RV64I ADD
  % as a 64-bit ADDD, and add a 128-bit ADDQ in what was previously the
  % OP-64 major opcode (now renamed the OP-128 major opcode).
\end{commentary}

按立即数移位(SLLI/SRLI/SRAI)现在使用I立即数的低7位进行编码,
而可变的移位(SLL/SRL/SRA)使用移位数目源寄存器的低7位进行编码。
% Shifts by an immediate (SLLI/SRLI/SRAI) are now encoded using the low
% 7 bits of the I-immediate, and variable shifts (SLL/SRL/SRA) use the
% low 7 bits of the shift amount source register.

使用现有的LOAD主操作码添加了LDU(双无符号加载)指令,随着新的LQ和SQ指令一起加载和存储四字值。
SQ被添加到STORE主操作码,同时LQ被添加到MISC-MEM主操作码。
% A LDU (load double unsigned) instruction is added using the existing
% LOAD major opcode, along with new LQ and SQ instructions to load and
% store quadword values.  SQ is added to the STORE major opcode, while
% LQ is added to the MISC-MEM major opcode.

浮点指令集没有变化,尽管128位Q浮点扩展现在可以支持FMV.X.Q和FMV.Q.X指令,以及来往于T(128位)整数格式的额外的FCVT指令。
% The floating-point instruction set is unchanged, although the 128-bit
% Q floating-point extension can now support FMV.X.Q and FMV.Q.X
% instructions, together with additional FCVT instructions to and from
% the T (128-bit) integer format.



\chapter{用于乘法和除法的“M”标准扩展(2.0版本)}

这章描述了标准整数乘法和除法指令扩展,命名为“M”,包含了将两个整数寄存器中持有的值相乘或相除的指令.
% This chapter describes the standard integer multiplication and
% division instruction extension, which is named ``M'' and contains
% instructions that multiply or divide values held in two integer
% registers.

\begin{commentary}

  我们将整数乘法和除法从基础中分离出来,以简化低端的实现,或者用于那些整数乘法和除法操作并不频繁的、或在相接的加速器中能更好地处理的应用。
% We separate integer multiply and divide out from the base to simplify
% low-end implementations, or for applications where integer multiply
% and divide operations are either infrequent or better handled in
% attached accelerators.
\end{commentary}

\section{乘法操作}
\label{multiplication-operations}

\vspace{-0.2in}
\begin{center}
\begin{tabular}{S@{}R@{}R@{}S@{}R@{}O}
\\
\instbitrange{31}{25} &
\instbitrange{24}{20} &
\instbitrange{19}{15} &
\instbitrange{14}{12} &
\instbitrange{11}{7} &
\instbitrange{6}{0} \\
\hline
\multicolumn{1}{|c|}{funct7} &
\multicolumn{1}{c|}{rs2} &
\multicolumn{1}{c|}{rs1} &
\multicolumn{1}{c|}{funct3} &
\multicolumn{1}{c|}{rd} &
\multicolumn{1}{c|}{opcode} \\
\hline
7 & 5 & 5 & 3 & 5 & 7 \\
MULDIV & 乘数 & 被乘数 & MUL/MULH[[S]U] & dest & OP    \\
MULDIV & 乘数 & 被乘数 & MULW           & dest & OP-32 \\
\end{tabular}
\end{center}

MUL在{\em rs1}和{\em rs2}上执行XLEN位$\times$XLEN位乘法,
并将低XLEN位放入目的寄存器。MULH、MULHU和MULHSU分别针对有符号数$\times$有符号数、无符号数$\times$无符号数、
有符号rs1$\times$无符号rs2执行相同的乘法,但是返回2$\times$XLEN位结果的高XLEN位。
如果同一个乘积的高位和低位都需要,那么推荐的代码序列是:MULH[[S]U] {\em rdh,rs1, rs2}; MUL {\em rdl, rs1, rs2}
(源寄存器标识符必须次序相同,且{\em rdh}不能与{\em rs1}或{\em rs2}相同)。然后微架构可以把这些代码融合进单独的一次乘法操作,而不是两次分离的乘法。

% MUL performs an XLEN-bit$\times$XLEN-bit multiplication
% of {\em rs1} by {\em rs2} and places the
% lower XLEN bits in the destination register.  MULH, MULHU, and MULHSU
% perform the same multiplication but return the upper XLEN bits of the
% full 2$\times$XLEN-bit product, for signed$\times$signed,
% unsigned$\times$unsigned, and \wunits{signed}{\em rs1}$\times$\wunits{unsigned}{\em rs2} multiplication,
% respectively.  If both the high and low bits of the same product are
% required, then the recommended code sequence is: MULH[[S]U] {\em rdh,
%   rs1, rs2}; MUL {\em rdl, rs1, rs2} (source register specifiers must
% be in same order and {\em rdh} cannot be the same as {\em rs1} or {\em
%   rs2}).  Microarchitectures can then fuse these into a single
% multiply operation instead of performing two separate multiplies.

\begin{commentary}
  MULHSU被用在多字有符号乘法中,将被乘数(包含符号位)的最高位有效字与乘数(无符号的)较低位有效字相乘。
% MULHSU is used in multi-word signed multiplication to multiply the
% most-significant word of the multiplicand (which contains the sign bit)
% with the less-significant words of the multiplier (which are unsigned).
\end{commentary}

MULW是一个RV64指令,它将源寄存器的低32位相乘,把符号扩展的低32位结果放入目的寄存器。
% MULW is an RV64 instruction that multiplies the lower 32 bits of the source
% registers, placing the sign-extension of the lower 32 bits of the result
% into the destination register.

\begin{commentary}
  在RV64中,MUL可以被用于获得64位乘积的高32位,但是有符号参数必须是合适的32位有符号值,
  反之无符号参数必须清除它们的高32位。如果参数不知道是符号扩展还是零扩展的,一个备选方案是把两个参数都向左移位32位,
  然后使用MULH[[S]U]。
% In RV64, MUL can be used to obtain the upper 32 bits of the 64-bit product,
% but signed arguments must be proper 32-bit signed values, whereas unsigned
% arguments must have their upper 32 bits clear.  If the
% arguments are not known to be sign- or zero-extended, an alternative is to
% shift both arguments left by 32 bits, then use MULH[[S]U].
\end{commentary}

\section{除法操作
  % Division Operations
  }

\vspace{-0.2in}
\begin{center}
\begin{tabular}{S@{}R@{}R@{}O@{}R@{}O}
\\
\instbitrange{31}{25} &
\instbitrange{24}{20} &
\instbitrange{19}{15} &
\instbitrange{14}{12} &
\instbitrange{11}{7} &
\instbitrange{6}{0} \\
\hline
\multicolumn{1}{|c|}{funct7} &
\multicolumn{1}{c|}{rs2} &
\multicolumn{1}{c|}{rs1} &
\multicolumn{1}{c|}{funct3} &
\multicolumn{1}{c|}{rd} &
\multicolumn{1}{c|}{opcode} \\
\hline
7 & 5 & 5 & 3 & 5 & 7 \\
MULDIV & 除数 & 被除数 & DIV[U]/REM[U]   & dest & OP    \\
MULDIV & 除数 & 被除数 & DIV[U]W/REM[U]W & dest & OP-32 \\
\end{tabular}
\end{center}

DIV和DIVU在{\em rs1}和{\em rs2}上执行XLEN位与XLEN位的有符号和无符号整数除法,结果向零取整。
REM和REMU提供了对应的除法操作的余数。对于REM,结果的符号等于被除数的符号。
% DIV and DIVU perform an XLEN bits by XLEN bits signed and unsigned integer
% division of {\em rs1} by {\em rs2}, rounding towards zero.
% REM and REMU provide the remainder of the corresponding division operation.
% For REM, the sign of the result equals the sign of the dividend.

\begin{commentary}
  对于有符号除法和无符号除法,都有
  \mbox{$\textrm{被除数} = \textrm{除数} \times \textrm{商} + \textrm{余数}$}
% For both signed and unsigned division, it holds that
% \mbox{$\textrm{dividend} = \textrm{divisor} \times \textrm{quotient} + \textrm{remainder}$}.
\end{commentary}

如果同一个除法的商和余数都需要,推荐的代码序列是:DIV[U] {\em rdq, rs1, rs2}; REM[U] {\em rdr, rs1, rs2} 
({\em rdq}不能与{\em rs1}或{\em rs2}相同)。然后微架构可以把这些代码融合为单独的一次除法操作,而不是执行两次分离的除法。
% If both the quotient and remainder
% are required from the same division, the recommended code sequence is:
% DIV[U] {\em rdq, rs1, rs2}; REM[U] {\em rdr, rs1, rs2} ({\em rdq}
% cannot be the same as {\em rs1} or {\em rs2}).  Microarchitectures can
% then fuse these into a single divide operation instead of performing
% two separate divides.

DIVW和DIVUW是RV64的指令,它们将{\em rs1}的低32位与{\em rs2}的低32位分别视为有符号整数和无符号整数并相除,
把32位商放入{\em rd},并符号扩展到64位。REMW和REMUW是RV64的指令,它们分别提供对应的有符号余数和无符号余数操作。
REMW和REMUW都总是把32位结果符号扩展到64位,包括除数为零时。
% DIVW and DIVUW are RV64 instructions that divide the
% lower 32 bits of {\em rs1} by the lower 32 bits of {\em rs2}, treating
% them as signed and unsigned integers respectively, placing the 32-bit
% quotient in {\em rd}, sign-extended to 64 bits.  REMW and REMUW
% are RV64 instructions that provide the corresponding
% signed and unsigned remainder operations respectively. Both REMW and
% REMUW always sign-extend the 32-bit result to 64 bits, including on a
% divide by zero.

表~\ref{tab:divby0}中总结了除数为零和除法溢出的语义。如果除数为零,商的所有位都被设置为1,余数等于被除数。
有符号的除法溢出只发生在最大负数被$-1$除的时候。溢出的有符号除法的商等于被除数,而余数为零。无符号除法不可能发生溢出。
% The semantics for division by zero and division overflow are summarized in
% Table~\ref{tab:divby0}.  The quotient of division by zero has all bits set, and
% the remainder of division by zero equals the dividend.  Signed division overflow
% occurs only when the most-negative integer is divided by $-1$.  The quotient of
% a signed division with overflow is equal to the dividend, and the remainder is
% zero. Unsigned division overflow cannot occur.

\begin{table}[h]
\center
\begin{tabular}{|l|c|c||c|c|c|c|}
\hline
Condition              & 被除数   & 除数 & DIVU[W]   & REMU[W] & DIV[W]     & REM[W] \\ \hline
Division by zero       & $x$        & 0       & $2^{L}-1$ & $x$     & $-1$       & $x$    \\
Overflow (signed only) & $-2^{L-1}$ & $-1$    & --        & --      & $-2^{L-1}$ & 0      \\
\hline
\end{tabular}
\caption{除数为零和除法溢出的语义。L是操作数的位宽度:或者是XLEN(对于DIV[U]和REM[U]),或者是32(对于DIV[U]W和REM[U]W)
% Semantics for division by zero and division overflow.
% L is the width of the operation in bits: XLEN for DIV[U] and REM[U], or
% 32 for DIV[U]W and REM[U]W.
}
\label{tab:divby0}
\end{table}

\begin{commentary}

  我们考虑过在整数被零除的时候产生异常,这些异常在大多数执行环境中会引发一个陷入。
  然而,这将是标准ISA中仅有的算数陷入(浮点异常会设置标志和写默认值,但是不会引起陷入),
  而对于这种情况,将需要语言解释器来与执行环境的陷入处理程序进行交互。此外,如果语言标准强制要求除数为零的异常必须引起控制流的立即改变,
  那么只需要为每个除法操作添加一条分支指令,而这个分支指令可以在除法之后被插入,并且通常应当非常大概率地被预测为不执行,
  几乎不增加运行时的负载。
% We considered raising exceptions on integer divide by zero, with these
% exceptions causing a trap in most execution environments.  However,
% this would be the only arithmetic trap in the standard ISA
% (floating-point exceptions set flags and write default values, but do
% not cause traps) and would require language implementors to interact
% with the execution environment's trap handlers for this case.
% Further, where language standards mandate that a divide-by-zero
% exception must cause an immediate control flow change, only a single
% branch instruction needs to be added to each divide operation, and
% this branch instruction can be inserted after the divide and should
% normally be very predictably not taken, adding little runtime
% overhead.

为了简化除法器电路,除数为零时,无论无符号还是有符号除法都返回所有位被设置为1的值。
全1值既是无符号除法返回的自然值,代表了最大的无符号数,也是简单无符号除法器实现的自然结果。
有符号除法经常使用无符号除法电路实现,并指定了相同的溢出结果来简化硬件。

% The value of all bits set is returned for both unsigned and signed
% divide by zero to simplify the divider circuitry.  The value of all 1s
% is both the natural value to return for unsigned divide, representing
% the largest unsigned number, and also the natural result for simple
% unsigned divider implementations.  Signed division is often
% implemented using an unsigned division circuit and specifying the same
% overflow result simplifies the hardware.
\end{commentary}

\section{Zmmul扩展(1.0版本)}

Zmmul扩展实现了M扩展的乘法子集。它添加了定义在第9.1节~\ref{multiplication-operations}中的所有指令,
即:MUL、MULH、MULHU、MULHSU、和(仅用于RV64的)MULW。这些编码与那些对应的M扩展的指令是相同的。
% The Zmmul extension implements the multiplication subset of the M extension.
% It adds all of the instructions defined in Section~\ref{multiplication-operations},
% namely: MUL, MULH, MULHU, MULHSU, and (for RV64 only) MULW.
% The encodings are identical to those of the corresponding M-extension instructions.

\begin{commentary}

  Zmmul扩展支持那些需要乘法操作但不需要除法操作的低成本实现。对于许多微控制器应用,除法操作太不频繁,
  导致除法器硬件的开销无法证明是正当的。相比之下,乘法操作更加频繁,使得乘法器硬件的开销更加正当。
  消除除法但保留乘法,令简单的FPGA软核特别获益,因为许多FPGA提供硬布线的乘法器,但是需要在软逻辑中实现除法器。
% The Zmmul extension enables low-cost implementations that require
% multiplication operations but not division.
% For many microcontroller applications, division operations are too
% infrequent to justify the cost of divider hardware.
% By contrast, multiplication operations are more frequent, making the cost of
% multiplier hardware more justifiable.
% Simple FPGA soft cores particularly benefit from eliminating division but
% retaining multiplication, since many FPGAs provide hardwired multipliers
% but require dividers be implemented in soft logic.
\end{commentary}

\chapter{用于原子指令的“A”标准扩展(2.1版本)}
\label{atomics}

命名为“A”的标准原子指令扩展包含了原子性的读-修改-写内存指令,
以支持运行在相同内存空间中的多个RISC-V硬件线程之间的同步。该扩展提供了两种形式的原子指令,
即“加载-保留/存储-条件”指令和“原子性获取并操作内存”指令。
原子指令的这两种形式都支持各种内存一致性次序,包括无序的、获取的、释放的、和顺序的一致性语义。
这些指令允许RISC-V支持RCsc内存一致性模型~\cite{Gharachorloo90memoryconsistency}。
% The standard atomic-instruction extension, named ``A'',
% contains instructions that atomically
% read-modify-write memory to support synchronization between multiple
% RISC-V harts running in the same memory space.  The two forms of
% atomic instruction provided are load-reserved/store-conditional
% instructions and atomic fetch-and-op memory instructions.  Both types
% of atomic instruction support various memory consistency orderings
% including unordered, acquire, release, and sequentially consistent
% semantics.  These instructions allow RISC-V to support the RCsc memory
% consistency model~\cite{Gharachorloo90memoryconsistency}.

\begin{commentary}

  在大量辩论之后,语言社区和架构社区似乎最终商定了将释放一致性作为标准内存一致性模型,
  并因此RISC-V原子性支持就是围绕这个模型构建的。
% After much debate, the language community and architecture community
% appear to have finally settled on release consistency as the standard
% memory consistency model and so the RISC-V atomic support is built
% around this model.
\end{commentary}

\section{指定原子指令的次序}

基础的RISC-V ISA有一个宽松的内存模型,其中FENCE指令被用于强制执行额外的次序约束。
地址空间被执行环境划分为内存领域和I/O领域,而FENCE指令提供了选项,以对这两个地址领域中的一个、或两者同时的访问进行排序。
% The base RISC-V ISA has a relaxed memory model, with the FENCE
% instruction used to impose additional ordering constraints.  The
% address space is divided by the execution environment into memory and
% I/O domains, and the FENCE instruction provides options to order
% accesses to one or both of these two address domains.

为了对释放一致性~\cite{Gharachorloo90memoryconsistency}提供更有效的支持,每个原子指令有两个位,{\em aq}和{\em rl},
用于指定额外的内存次序约束,正如其它RISC-V硬件线程所看到的那样。位次序访问两个地址领域——内存或I/O——中的哪一个,
依赖于原子指令正在访问的地址领域。对另一个领域的访问不隐含次序约束,而应当使用FENCE指令跨两个领域排序。
% To provide more efficient support for release
% consistency~\cite{Gharachorloo90memoryconsistency}, each atomic
% instruction has two bits, {\em aq} and {\em rl}, used to specify
% additional memory ordering constraints as viewed by other RISC-V
% harts.  The bits order accesses to one of the two address domains,
% memory or I/O, depending on which address domain the atomic
% instruction is accessing.  No ordering constraint is implied to
% accesses to the other domain, and a FENCE instruction should be used
% to order across both domains.

如果两个位都被清除,则在原子内存操作上不会被施加额外的次序约束。如果只设置了{\em aq}位,
原子内存操作被视为一次{\em 获取}访问,也就是说,在这个RISC-V硬件线程上,在获取内存操作之前不会观测到随后发生的内存操作。
如果只设置了{\em rl}位,原子内存操作被视为一次释放访问,也就是说,在这个RISC-V硬件线程上,
在任何更早的内存操作之前不会观测到释放内存操作发生。如果{\em aq}位和{\em rl}位都被设置了,原子内存操作是{\em 顺序一致性的},
在同一个RISC-V硬件线程中,对于同一个地址领域,不能在任何更早的内存操作之前、
或在任何更迟的内存操作之后观测到该原子内存操作发生。
% If both bits are clear, no additional ordering constraints are imposed
% on the atomic memory operation.  If only the {\em aq} bit is set, the
% atomic memory operation is treated as an {\em acquire} access, i.e.,
% no following memory operations on this RISC-V hart can be observed
% to take place before the acquire memory operation.  If only the {\em
%   rl} bit is set, the atomic memory operation is treated as a {\em
%   release} access, i.e., the release memory operation cannot be
% observed to take place before any earlier memory operations on this
% RISC-V hart.  If both the {\em aq} and {\em rl} bits are set, the
% atomic memory operation is {\em sequentially consistent} and cannot be
% observed to happen before any earlier memory operations or after any
% later memory operations in the same RISC-V hart and to the same
% address domain.

\section{加载-保留/存储-条件指令}
% \section{Load-Reserved/Store-Conditional Instructions}
\label{sec:lrsc}

\vspace{-0.2in}
\begin{center}
\begin{tabular}{R@{}W@{}W@{}R@{}R@{}F@{}R@{}O}
\\
\instbitrange{31}{27} &
\instbit{26} &
\instbit{25} &
\instbitrange{24}{20} &
\instbitrange{19}{15} &
\instbitrange{14}{12} &
\instbitrange{11}{7} &
\instbitrange{6}{0} \\
\hline
\multicolumn{1}{|c|}{funct5} &
\multicolumn{1}{c|}{aq} &
\multicolumn{1}{c|}{rl} &
\multicolumn{1}{c|}{rs2} &
\multicolumn{1}{c|}{rs1} &
\multicolumn{1}{c|}{funct3} &
\multicolumn{1}{c|}{rd} &
\multicolumn{1}{c|}{opcode} \\
\hline
5 & 1 & 1 & 5 & 5 & 3 & 5 & 7 \\
LR.W/D & \multicolumn{2}{c}{ordering} & 0   & addr & width & dest & AMO    \\
SC.W/D & \multicolumn{2}{c}{ordering} & src & addr & width & dest & AMO  \\
\end{tabular}
\end{center}

加载-保留(LR)和存储-条件(SC)指令在一个内存字或双字上实施复杂的原子内存操作。
LR.W从{\em rs1}中的地址处加载一个字,把符号扩展的值放入{\em rd},并注册一个保留集——归入地址字的字节的一个字节集合。
SC.W有条件地将{\em rs2}中的字写到{\em rs1}中的地址:当且仅当保留仍然有效且保留集包含正在写的字节时,
SC.W成功。如果SC.W成功,指令把{\em rs2}中的字写到内存,并把{\em rd}写零。如果SC.W失败,指令不会写内存,
它向{\em rd}写一个非零值。不管成功还是失败,执行一次SC.W指令都会使这个硬件线程持有的任何保留失效。
LR.D和SC.D对双字采取类似的行为,并只在RV64上可用。对于RV64,LR.W和SC.W对放入{\em rd}的值进行符号扩展。

% Complex atomic memory operations on a single memory word or doubleword are performed
% with the load-reserved (LR) and store-conditional (SC) instructions.
% LR.W loads a word from the address in {\em rs1}, places the sign-extended
% value in {\em rd}, and registers a {\em reservation set}---a set of bytes
% that subsumes the bytes in the addressed word.
% SC.W conditionally writes a word in {\em rs2} to the address in {\em rs1}: the
% SC.W succeeds only if the reservation is still valid and the reservation set
% contains the bytes being written.
% If the SC.W succeeds, the instruction writes the word in {\em rs2} to memory,
% and it writes zero to {\em rd}.
% If the SC.W fails, the instruction does not write to memory, and it writes
% a nonzero value to {\em rd}.
% Regardless of success or failure, executing an SC.W instruction invalidates
% any reservation held by this hart.
% LR.D and SC.D act analogously on doublewords and are only available on RV64.
% For RV64, LR.W and SC.W sign-extend the value placed in {\em rd}.

\begin{commentary}
  “比较并交换”(CAS)和LR/SC都能够被用于构建无锁的数据结构。
  在紧张的讨论之后,我们选择了LR/SC,是由于几个原因:
  1)LR/SC因为监视所有的对地址的访问、而不只是检查数值的变化,可以避免CAS存在的ABA问题;
  2)在已经需要一个不同于内存系统的消息格式的基础上,CAS还将需要一个新的整数指令格式来支持三个源操作数(地址、比较的值、交换的值),这将使微架构复杂化;
  3)更进一步地,为了避免ABA问题,其它系统提供了一个双宽度CAS(DW-CAS),以允许计数器可以沿着数据字测试和增长。这需要在一条指令中读五个寄存器并写两个寄存器,而且也需要一个新的更大的内存系统消息格式,远比实现要复杂;
  4)LR/SC为许多原语提供了一个更加有效的实现,因为它只需要加载一次,与之相反的是,CAS需要加载两次(一次在CAS指令之前加载以获得推测计算的值,然后第二次加载作为CAS指令的一部分以检查值是否在更新之前被改变了)。
% Both compare-and-swap (CAS) and LR/SC can be used to build lock-free
% data structures.  After extensive discussion, we opted for LR/SC for
% several reasons: 1) CAS suffers from the ABA problem, which LR/SC
% avoids because it monitors all writes to the address rather than
% only checking for changes in the data value; 2) CAS would also require
% a new integer instruction format to support three source operands
% (address, compare value, swap value) as well as a different memory
% system message format, which would complicate microarchitectures; 3)
% Furthermore, to avoid the ABA problem, other systems provide a
% double-wide CAS (DW-CAS) to allow a counter to be tested and
% incremented along with a data word. This requires reading five
% registers and writing two in one instruction, and also a new larger
% memory system message type, further complicating implementations; 4)
% LR/SC provides a more efficient implementation of many primitives as
% it only requires one load as opposed to two with CAS (one load before
% the CAS instruction to obtain a value for speculative computation,
% then a second load as part of the CAS instruction to check if value is
% unchanged before updating).

与CAS相比,LR/SC的主要劣势是活锁,我们在特定环境下通过最终推进的一种结构化的保证来避免它,
如下文所描述的那样。另一个问题是,当前x86架构和它的DW-CAS的影响是否会将同步库和其它假定DW-CAS是基本机器原语的软件的移植复杂化。
一个可能的缓解因素是,当前向x86添加的事务内存指令,它可能导致一次从DW-CAS的迁移。
% The main disadvantage of LR/SC over CAS is livelock, which we avoid,
% under certain circumstances,
% with an architected guarantee of eventual forward progress as
% described below.  Another concern is whether the influence of the
% current x86 architecture, with its DW-CAS, will complicate porting of
% synchronization libraries and other software that assumes DW-CAS is
% the basic machine primitive.  A possible mitigating factor is the
% recent addition of transactional memory instructions to x86, which
% might cause a move away from DW-CAS.

更一般地,多字原子原语是令人向往的,但是关于这应当采取什么形式仍然有相当大的争议,
并且保证向前进度会增加系统的复杂性。我们当前的想法是,沿着原始事务内存提案的内容,
包括一个小型的有限容量的事务内存缓冲,作为一个可选的标准扩展“T”。
% More generally, a multi-word atomic primitive is desirable, but there is
% still considerable debate about what form this should take, and
% guaranteeing forward progress adds complexity to a system.
\end{commentary}

失败代码值1编码了一个未指定的失败。其它的失败代码目前被保留,可移植的软件应当仅仅假定,失败代码将是非零的。
% The failure code with value 1 encodes an unspecified failure.
% Other failure codes are reserved at this time.
% Portable software should only assume the failure code will be non-zero.

\begin{commentary}
  我们保留了一个失败代码1来表示“未指定的”,这样简单的实现可以使用SLT/SLTU指令所需的现有的mux来返回这个值。
  ISA的未来版本或扩展中可能会定义更具体的失败代码。
% We reserve a failure code of 1 to mean ``unspecified'' so that simple
% implementations may return this value using the existing mux required
% for the SLT/SLTU instructions.  More specific failure codes might be
% defined in future versions or extensions to the ISA.
\end{commentary}

对于LR和SC,A扩展需要{\em rs1}中持有的地址自然对齐到操作数的尺寸(也就是说,64位字对齐到8字节,32位字对齐到4字节)。
如果地址没有自然对齐,将产生一个地址未对齐异常或者一个访问故障异常。如果未对齐的访问不宜被模拟,
而除了未对齐之外内存访问都能完成,那么可以为内存访问生成访问故障异常。
% For LR and SC, the A extension requires that the address held in {\em
%   rs1} be naturally aligned to the size of the operand (i.e.,
% eight-byte aligned for 64-bit words and four-byte aligned for 32-bit
% words).  If the address is not naturally aligned, an address-misaligned
% exception or an access-fault exception will be generated.  The access-fault
% exception can be generated for a memory access that would otherwise be
% able to complete except for the misalignment, if the misaligned access
% should not be emulated.

\begin{commentary}
  在大多数系统中,模拟未对齐的LR/SC序列是不实际的。
% Emulating misaligned LR/SC sequences is impractical in most systems.

  未对齐的LR/SC序列也增加了一次访问多个保留集的可能性,这是现有的定义没有提供的。
% Misaligned LR/SC sequences also raise the possibility of accessing multiple
% reservation sets at once, which present definitions do not provide for.
\end{commentary}

实现可以在每个LR上注册一个任意大的保留集,提供包括被编址的数据字或双字的所有字节的保留。
SC可以只与程序次序中最相近的LR配对。SC可能成功,当前仅当:没有从另一个硬件线程到该保留集的存储能够在LR和SC之间被观察到,
且以程序次序,在其LR和它自己之间没有其它的SC。SC可能成功,当且仅当:在LR和SC之间不会观察到,
发生从硬件线程以外的设备到被LR指令所访问的字节的写入。注意这个LR的有效地址和数据尺寸可能已经不同了,
但是保留了SC的地址,将之作为保留集的一部分。
% An implementation can register an arbitrarily large reservation set on each
% LR, provided the reservation set includes all bytes of the addressed data word
% or doubleword.
% An SC can only pair with the most recent LR in program order.
% An SC may succeed only if no store from another hart
% to the reservation set can be observed to have occurred between the LR
% and the SC, and if there is no other SC between the LR and itself in program
% order.
% An SC may succeed only if no write from a device other than a hart
% to the bytes accessed by the LR instruction can be observed to have occurred
% between the LR and SC.
% Note this LR might have had a different effective address and data size, but
% reserved the SC's address as part of the reservation set.
\begin{commentary}
  根据这个模型,在带有内存事务的系统中,如果更早的LR使用不同的虚拟地址别名预约了相同的位置,那么允许SC成功;
  但是如果虚拟地址是不同的,那么也允许失败。
% Following this model, in systems with memory translation, an SC is allowed to
% succeed if the earlier LR reserved the same location using an alias with
% a different virtual address, but is also allowed to fail if the virtual
% address is different.
\end{commentary}
\begin{commentary}
  为了顾及遗留的设备和总线,从RISC-V硬件线程以外的设备的写只需要在它们与由LR所访问的字节重叠时,
  将预约无效化即可。当它们访问保留集中的其它字节时,这些写不需要将预约无效化。
% To accommodate legacy devices and buses, writes from devices other than RISC-V
% harts are only required to invalidate reservations when they overlap the bytes
% accessed by the LR.  These writes are not required to invalidate the
% reservation when they access other bytes in the reservation set.
\end{commentary}

SC必定失败,如果:以程序次序,地址没有在最近的LR的保留集中。
SC必定失败,如果:在LR和SC之间可以观察到有从其它硬件线程到保留集的存储发生。
SC必定失败,如果:在LR和SC之间可以观察到有从其它设备到LR所访问的字节的写发生。
(如果这个设备写了保留集但是没有写由LR所访问的字节,SC可能失败,也可能不失败。)
SC必定失败,如果:以程序次序,在LR和SC之间有另一个SC(对任何地址)。
在~\ref{sec:rvwmo}节中,“原子性公理”定义了对成功的LR/SC序列的原子性需求的精确陈述。
% The SC must fail if the address is not within the reservation set of the most
% recent LR in program order.
% The SC must fail if a store to the reservation set from another hart can be
% observed to occur between the LR and SC.
% The SC must fail if a write from some other device to the bytes accessed by
% the LR can be observed to occur between the LR and SC.
% (If such a device writes the reservation set but does not write the bytes
% accessed by the LR, the SC may or may not fail.)
% An SC must fail if there is another SC (to any address) between the LR and the
% SC in program order.
% The precise statement of the atomicity requirements for successful LR/SC
% sequences is defined by the Atomicity Axiom in Section~\ref{sec:rvwmo}.

\begin{commentary}
  平台应当提供一种定义保留集的尺寸和形状的方法。
% The platform should provide a means to determine the size and shape of the
% reservation set.
  一个平台的规范可能约束保留指令集的尺寸和形状。
% A platform specification may constrain the size and shape of the reservation
% set.
\end{commentary}

\begin{commentary}
  对内存的划痕字的存储-条件指令应当被用于使任何现有的加载保留强制失效:
% A store-conditional instruction to a scratch word of memory should be used
% to forcibly invalidate any existing load reservation:
\begin{itemize}
\item 在抢占式上下文切换期间,和   during a preemptive context switch, and
\item 如果有必要,在改变虚拟地址到物理地址映射时,例如,当迁移可能包含一个活动的保留的页时。  if necessary when changing virtual to physical address mappings,
  such as when migrating pages that might contain an active reservation.
\end{itemize}
\end{commentary}

\begin{commentary}
  如果一个LR或SC暗示,硬件线程当时只持有一个保留,并且以程序次序,SC只能与最接近的LR配对,
  且LR只能与接下来的下一个SC配对,那么当硬件线程执行该LR或SC时,硬件线程的预约被无效化。
  这是对~\ref{sec:rvwmo}节中原子性公理的一个限制,该公理确保软件正确地运行在以这种方式操作的期望的常见实现上。
% The invalidation of a hart's reservation when it executes an LR or SC
% imply that a hart can only hold one reservation at a time, and that
% an SC can only pair with the most recent LR, and LR with the next
% following SC, in program order.  This is a restriction to the
% Atomicity Axiom in Section~\ref{sec:rvwmo} that ensures software runs
% correctly on expected common implementations that operate in this manner.
\end{commentary}

在建立了保留的LR指令之前,其它RISC-V硬件线程永远不可以观测到SC指令。
通过设置LR指令的{\em aq}位,可以赋予LR/SC序列获取的语义。通过设置SC指令的{\em rl}位,
可以赋予LR/SC序列释放的语义。设置LR指令的{\em aq}位,并设置SC指令的{\em aq}位和{\em rl}位,
使LR/SC序列顺序一致,意味着它不能被相同硬件线程的更早的或更迟的内存操作重新排序。
% An SC instruction can never be observed by another RISC-V hart
% before the LR instruction that established the reservation.
% The LR/SC
% sequence can be given acquire semantics by setting the {\em aq} bit on
% the LR instruction.  The LR/SC sequence can be given release semantics
% by setting the {\em rl} bit on the SC instruction.  Setting the {\em
%   aq} bit on the LR instruction, and setting both the {\em aq} and the {\em
%   rl} bit on the SC instruction makes the LR/SC sequence sequentially
% consistent, meaning that it cannot be reordered with earlier or
% later memory operations from the same hart.

如果在LR和SC上都没有设置任何位,可以在来自同一RISC-V硬件线程的周围的内存操作之前或之后观测到LR/SC序列的发生。
当LR/SC序列被用于实现并行规约操作时,这可以是合适的。
% If neither bit is set on both LR and SC, the LR/SC sequence can be
% observed to occur before or after surrounding memory operations from
% the same RISC-V hart.  This can be appropriate when the LR/SC
% sequence is used to implement a parallel reduction operation.

软件不应当设置LR指令的{\em rl}位,除非也设置了{\em aq}位;软件也不应当设置SC指令的{\em aq}位,
除非也设置了{\em rl}位。LR.{\em rl}和SC.{\em aq}指令不保证提供任何比那些位都被清除的指令更强的次序,但是可能会导致更低的效率。
% Software should not set the {\em rl} bit on an LR instruction unless the {\em
% aq} bit is also set, nor should software set the {\em aq} bit on an SC
% instruction unless the {\em rl} bit is also set.  LR.{\em rl} and SC.{\em aq}
% instructions are not guaranteed to provide any stronger ordering than those
% with both bits clear, but may result in lower performance.

\begin{figure}[h!]
\begin{center}
\begin{verbatim}
        # a0 持有内存位置的地址  holds address of memory location
        # a1 持有期望的(expected)值  holds expected value
        # a2 持有需要的(desired)值  holds desired value
        # a0 持有返回值,如果成功则为零,否则为非零。  holds return value, 0 if successful, !0 otherwise
    cas:
        lr.w t0, (a0)        # 加载原始值。  Load original value.
        bne t0, a1, fail     # 不匹配,所以失败。  Doesn't match, so fail.
        sc.w t0, a2, (a0)    # 尝试更新。  Try to update.
        bnez t0, cas         # 如果存储条件失败,那么重试  Retry if store-conditional failed.
        li a0, 0             # 设置返回值为成功。  Set return to success.
        jr ra                # 返回。 Return.
    fail:
        li a0, 1             # 设置返回值为失败。  Set return to failure.
        jr ra                # 返回。  Return.
\end{verbatim}
\end{center}
\caption{使用LR/SC的比较与交换功能的样例代码  Sample code for compare-and-swap function using LR/SC.}
\label{cas}
\end{figure}

LR/SC可以被用于构造无锁的数据结构。图~\ref{cas}显示了一个使用LR/SC来实现比较与交换功能的例子。
如果被内联,比较与交换功能只需要采用四个指令。
% LR/SC can be used to construct lock-free data structures.  An example
% using LR/SC to implement a compare-and-swap function is shown in
% Figure~\ref{cas}.  If inlined, compare-and-swap functionality need
% only take four instructions.

\section{存储-条件指令的最终正确完成}
% \section{Eventual Success of Store-Conditional Instructions}
\label{sec:lrscseq}

标准A扩展定义了受约束的{\em LR/SC循环},它有如下的性质:
% The standard A extension defines {\em constrained LR/SC loops}, which have
% the following properties:
\vspace{-0.2in}
\begin{itemize}
\parskip 0pt
\itemsep 1pt
\item 该循环只包含了一个LR/SC序列,和在失败的情况下进行重新尝试该序列的代码,并且必须包含至多16条在内存中顺序放置的指令。 
  % The loop comprises only an LR/SC sequence and code to retry the sequence
  % in the case of failure, and must comprise at most 16 instructions placed
  % sequentially in memory.
\item 一个LR/SC序列以一条LR指令开始,以一条SC指令结束。在LR和SC指令之间执行的动态代码只能包含来自基础“I”指令集的指令,
  不包括加载、存储、向后跳转、执行向后分支、JALR、FENCE和SYSTEM指令。
  如果支持“C”扩展,那么前面提到的“I”指令的压缩形式也是被允许的。

  % An LR/SC sequence begins with an LR instruction and ends with an SC
  % instruction.  The dynamic code executed between the LR and SC instructions
  % can only contain instructions from the base ``I'' instruction set, excluding
  % loads, stores, backward jumps, taken backward branches, JALR, FENCE,
  % and SYSTEM instructions.
  % If the ``C'' extension is supported, then compressed
  % forms of the aforementioned ``I'' instructions are also permitted.
\item 重新尝试一次失败的LR/SC序列的代码可以包含向后跳转和/或分支以重复LR/SC序列,
  但如果不包含,那么与LR和SC之间的代码受到相同的约束。
  % The code to retry a failing LR/SC sequence can contain backwards jumps
  % and/or branches to repeat the LR/SC sequence, but otherwise has the same
  % constraint as the code between the LR and SC.
\item LR和SC的地址必须列于带有{\em LR/SC最终属性的}内存区域之中。具有这种属性的区域的通信由执行环境负责。
  % The LR and SC addresses must lie within a memory region with the
  % {\em LR/SC eventuality} property.  The execution environment is responsible
  % for communicating which regions have this property.
\item SC必须与同一硬件线程所执行的最近一次LR具有相同的有效地址和相同的数据尺寸。
% The SC must be to the same effective address and of the same data size as
%   the latest LR executed by the same hart.
\end{itemize}

不在受约束的LR/SC循环中的LR/SC序列是{\em 不受约束的}。
不受约束的LR/SC序列可能在某些实现的某些尝试上成功,但是可能在其它实现上永远不成功。

% LR/SC sequences that do not lie within constrained LR/SC loops are {\em
% unconstrained}.  Unconstrained LR/SC sequences might succeed on some attempts
% on some implementations, but might never succeed on other implementations.

\begin{commentary}
  我们限制了LR/SC循环的长度来适合基础ISA中的64个连续指令字节,以避免对指令缓存、TLB尺寸和关联性的过度限制。
  类似地,在追踪自由缓存中的预约的简单实现中,我们不允许循环中有其它的加载和存储,以避免限制数据-缓存的关联性。
  对分支和跳转的约束限制了本可以花费在序列中的时间。在缺少合适的硬件支持的实现上,不允许浮点操作和整数乘法/除法,
  以简化操作系统对这些指令的模拟。
% We restricted the length of LR/SC loops to fit within 64 contiguous
% instruction bytes in the base ISA to avoid undue restrictions on instruction
% cache and TLB size and associativity.
% Similarly, we disallowed other loads and stores within the loops to avoid
% restrictions on data-cache associativity in simple implementations that track
% the reservation within a private cache.
% The restrictions on branches and jumps limit the time that
% can be spent in the sequence.  Floating-point operations and integer
% multiply/divide were disallowed to simplify the operating system's emulation
% of these instructions on implementations lacking appropriate hardware support.

不禁止软件使用不受约束的LR/SC序列,但是可移植的软件必须检测序列重复失败的情况,
然后退回到不依赖于不受约束的LR/SC序列的备用代码序列。实现可以无条件地令任何不受约束的LR/SC序列失败。
% Software is not forbidden from using unconstrained LR/SC sequences, but
% portable software must detect the case that the sequence repeatedly fails,
% then fall back to an alternate code sequence that does not rely on an
% unconstrained LR/SC sequence.  Implementations are permitted to
% unconditionally fail any unconstrained LR/SC sequence.
\end{commentary}

如果一个硬件线程{\em H}进入了一个受约束的LR/SC循环,执行环境必须保证下列事件之一能最终发生:
% If a hart {\em H} enters a constrained LR/SC loop, the execution environment
% must guarantee that one of the following events eventually occurs:
\vspace{-0.2in}
\begin{itemize}
\parskip 0pt
\itemsep 1pt
\item {\em H}或某些其它的硬件线程对{\em H}的受约束的LR/SC循环中的LR指令的保留集执行了一次成功的SC。 
% {\em H} or some other hart executes a successful SC to the reservation
%   set of the LR instruction in {\em H}'s constrained LR/SC loops.
\item 某些其它硬件线程对{\em H}的受约束的LR/SC循环中的LR指令的保留集执行了一次无条件存储或AMO指令,或者系统中的某些其它设备写了该保留集。
% Some other hart executes an unconditional store or AMO instruction to
%   the reservation set of the LR instruction in {\em H}'s constrained LR/SC
%   loop, or some other device in the system writes to that reservation set.
\item {\em H} 执行了一次分支或跳转而退出了受约束的LR/SC循环。
  {\em H} executes a branch or jump that exits the constrained LR/SC loop.
\item {\em H} 陷入。
  {\em H} traps.
\end{itemize}

\begin{commentary}
  注意,只要不违背前面提到的各项保证,这些定义允许实现偶尔以任何原因令SC指令失败。
% Note that these definitions permit an implementation to fail an SC instruction
% occasionally for any reason, provided the aforementioned guarantee is not
% violated.
\end{commentary}

\begin{commentary}
  作为终结性保证的结果,如果执行环境的某些硬件线程正在执行受约束的LR/SC循环,
  而没有其它硬件线程或设备在该执行环境中对保留集执行无条件存储或AMO,
  那么至少一个硬件线程将最终退出它的受约束的LR/SC循环。
  反之,如果其它硬件线程或设备持续写保留集,不保证任何硬件线程将退出它的LR/SC循环。
% As a consequence of the eventuality guarantee, if some harts in an execution
% environment are executing constrained LR/SC loops, and no other harts or
% devices in the execution environment execute an unconditional store or AMO to
% that reservation set, then at least one hart will eventually exit its
% constrained LR/SC loop.
% By contrast, if other harts or devices continue to write to that reservation
% set, it is not guaranteed that any hart will exit its LR/SC loop.

加载和加载-保留指令本身不会阻碍其它硬件线程的LR/SC序列的进程。
我们注意到这个约束意味着,除此之外,其它硬件线程执行的加载和加载保留指令(可能在同一个核中)不能无限阻碍LR/SC进程。
例如,由共享缓存的另一个硬件线程引起的缓存收回不能无限地阻碍LR/SC进程。
典型地,这意味着对保留的追踪是独立于任何共享缓存的收回的。
类似地,由硬件线程内的推测性执行引起的缓存缺失不能无限地阻碍LR/SC进程。
% Loads and load-reserved instructions do not by themselves impede the progress
% of other harts' LR/SC sequences.
% We note this constraint implies, among other things, that loads and
% load-reserved instructions executed by other harts (possibly within the same
% core) cannot impede LR/SC progress indefinitely.
% For example, cache evictions caused by another hart sharing the cache cannot
% impede LR/SC progress indefinitely.
% Typically, this implies reservations are tracked independently of
% evictions from any shared cache.
% Similarly, cache misses caused by speculative execution within a hart cannot
% impede LR/SC progress indefinitely.

这些定义承认,即使进程最终完成,SC指令也可能由于实现的原因而貌似失败。
% These definitions admit the possibility that SC instructions may spuriously
% fail for implementation reasons, provided progress is eventually made.
\end{commentary}

\begin{commentary}
  CAS的一个优势是,它保证了某些硬件线程最终完成进程,尽管在某些系统上LR/SC原子性序列可能无限期地活锁。
  为了避免这个问题,我们为特定的LR/SC序列添加了一个活锁自由的结构性保证。
% One advantage of CAS is that it guarantees that some hart eventually
% makes progress, whereas an LR/SC atomic sequence could livelock
% indefinitely on some systems.  To avoid this concern, we added an
% architectural guarantee of livelock freedom for certain LR/SC sequences.

这个规范的更早的版本推行了一个更强的饥饿-自由保证。
然而,较弱的活锁-自由保证对于实现C11和C++11语言已经足够,并且在某些微架构样式中相当更容易被提供。
% Earlier versions of this specification imposed a stronger starvation-freedom
% guarantee.  However, the weaker livelock-freedom guarantee is sufficient to
% implement the C11 and C++11 languages, and is substantially easier to provide
% in some microarchitectural styles.
\end{commentary}


\section{原子内存操作}
% \section{Atomic Memory Operations}
\label{sec:amo}

\vspace{-0.2in}
\begin{center}
\begin{tabular}{O@{}W@{}W@{}R@{}R@{}F@{}R@{}R}
\\
\instbitrange{31}{27} &
\instbit{26} &
\instbit{25} &
\instbitrange{24}{20} &
\instbitrange{19}{15} &
\instbitrange{14}{12} &
\instbitrange{11}{7} &
\instbitrange{6}{0} \\
\hline
\multicolumn{1}{|c|}{funct5} &
\multicolumn{1}{c|}{aq} &
\multicolumn{1}{c|}{rl} &
\multicolumn{1}{c|}{rs2} &
\multicolumn{1}{c|}{rs1} &
\multicolumn{1}{c|}{funct3} &
\multicolumn{1}{c|}{rd} &
\multicolumn{1}{c|}{opcode} \\
\hline
5 & 1 & 1 & 5 & 5 & 3 & 5 & 7 \\
AMOSWAP.W/D & \multicolumn{2}{c}{ordering} & src & addr & width & dest & AMO  \\
AMOADD.W/D & \multicolumn{2}{c}{ordering} & src & addr & width & dest & AMO  \\
AMOAND.W/D & \multicolumn{2}{c}{ordering} & src & addr & width & dest & AMO  \\
AMOOR.W/D & \multicolumn{2}{c}{ordering} & src & addr & width & dest & AMO  \\
AMOXOR.W/D & \multicolumn{2}{c}{ordering} & src & addr & width & dest & AMO  \\
AMOMAX[U].W/D & \multicolumn{2}{c}{ordering} & src & addr & width & dest & AMO  \\
AMOMIN[U].W/D & \multicolumn{2}{c}{ordering} & src & addr & width & dest & AMO  \\
\end{tabular}
\end{center}

\vspace{-0.1in} 原子内存操作(AMO)指令为多处理器同步执行读-修改-写操作,并被编码为R类型指令格式。
这些AMO指令从{\em rs1}中的地址处原子性地加载一个数据值,把该值放进寄存器{\em rd}中,对被加载的值和{\em rs2}中的原有值使用一个二进制操作符,
然后把结果存回{\em rs1}中的原始地址。AMO既可以操作在内存中的64位字上(仅限RV64),
也可以操作在32位字上。对于RV64,32位的AMO总是把放入rd中的值进行符号扩展,并忽略{\em rs2}的原始值的高32位。
% The atomic memory operation (AMO) instructions perform
% read-modify-write operations for multiprocessor synchronization and
% are encoded with an R-type instruction format.  These AMO instructions
% atomically load a data value from the address in {\em rs1}, place the
% value into register {\em rd}, apply a binary operator to the loaded
% value and the original value in {\em rs2}, then store the result back
% to the original address in {\em rs1}. AMOs can either operate on 64-bit (RV64
% only) or 32-bit words in memory.  For RV64, 32-bit AMOs always
% sign-extend the value placed in {\em rd}, and ignore the upper 32 bits
% of the original value of {\em rs2}.

对于AMO,A扩展需要{\em rs1}中持有的地址被自然地对齐到操作数的尺寸(也就是说,对于64位字是8字节对齐,对于32位字是4字节对齐)。
如果地址没有自然对齐,将生成一个地址未对齐异常或一个访问故障异常。如果未对齐的访问不宜被模拟,
而除了未对齐之外的内存访问都能完成,那么可以为内存访问生成访问故障异常。
第~\ref{sec:zam}章里描述的“Zam”扩展放松了这个需求并指定了未对齐的AMO的语义。
% For AMOs, the A extension requires that the address held in {\em rs1}
% be naturally aligned to the size of the operand (i.e., eight-byte
% aligned for 64-bit words and four-byte aligned for 32-bit words).  If
% the address is not naturally aligned, an address-misaligned exception
% or an access-fault exception will be generated.  The access-fault exception can be
% generated for a memory access that would otherwise be able to complete
% except for the misalignment, if the misaligned access should not be
% emulated.  The ``Zam'' extension, described in Chapter~\ref{sec:zam},
% relaxes this requirement and specifies the semantics of misaligned
% AMOs.

支持的操作有:交换、整数加法、按位AND、按位OR、按位XOR、有符号/无符号整数的取最大值/取最小值。
如果没有次序约束,这些AMO可以被用于实现并行规约操作,通常情况下,返回值将通过写到{\tt x0}而被弃置。
% The operations supported are swap, integer add, bitwise AND, bitwise
% OR, bitwise XOR, and signed and unsigned integer maximum and minimum.
% Without ordering constraints, these AMOs can be used to implement
% parallel reduction operations, where typically the return value would
% be discarded by writing to {\tt x0}.

\begin{commentary}
  我们提供了“获取并操作”样式的原子性原语,因为它们比LR/SC或CAS更适合高度并行化的系统。
  一个简单的微架构可以使用LR/SC原语实现AMO,如果实现提供AMO最终完成的保证。
  更复杂的实现也可以在内存控制器出实现AMO,并且能够当目的寄存器是{\tt x0}时,对获取原始值进行优化。
% We provided fetch-and-op style atomic primitives as they scale to
% highly parallel systems better than LR/SC or CAS.
% A simple microarchitecture can implement AMOs using the LR/SC primitives,
% provided the implementation can guarantee the AMO eventually completes.
% More complex implementations might also implement AMOs at memory
% controllers, and can optimize away fetching the original value when
% the destination is {\tt x0}.

选择AMO的集合以有效地支持C11/C++11原子性内存操作,也为了支持内存中的并行规约。
AMO的另一个使用是提供对I/O空间中的内存映射设备寄存器的原子性更新(例如,设置位、清除位、或者切换位)。
% The set of AMOs was chosen to support the C11/C++11 atomic memory
% operations efficiently, and also to support parallel reductions in
% memory.  Another use of AMOs is to provide atomic updates to
% memory-mapped device registers (e.g., setting, clearing, or toggling
% bits) in the I/O space.
\end{commentary}

为了帮助实现多处理器同步,AMO有选择地提供了释放一致性语义。
如果设置了{\em aq}位,那么这个RISC-V硬件线程中,不会观测到有比AMO更迟的内存操作在AMO之前发生。
相对地,如果设置了{\em rl}位,那么其它RISC-V硬件线程将不会在这个RISC-V硬件线程中比AMO更早的内存访问之前观测到AMO。
在一个AMO上同时设置{\em aq}q和{\em rl}位会使序列具有顺序一致性,意味着它不能与来自相同硬件线程的更早的或更迟的内存操作被重新排序。
% To help implement multiprocessor synchronization, the AMOs optionally
% provide release consistency semantics.  If the {\em aq} bit is set,
% then no later memory operations in this RISC-V hart can be observed
% to take place before the AMO.
% Conversely, if the {\em rl} bit is set, then other
% RISC-V harts will not observe the AMO before memory accesses
% preceding the AMO in this RISC-V hart.  Setting both the {\em aq} and the {\em
% rl} bit on an AMO makes the sequence sequentially consistent, meaning that
% it cannot be reordered with earlier or later memory operations from the same
% hart.

\begin{commentary}

  AMO为高效地实现C11和C++11内存模型而设计。尽管FENCE R, RW指令足以实现获取操作,以及FENCE RW, W足以实现释放操作,
  但与设置对应的{\em aq}或{\em rl}位的AMO相比,它们都意味着额外的不必要的排序。
% The AMOs were designed to implement the C11 and C++11 memory models
% efficiently.  Although the FENCE R, RW instruction suffices to
% implement the {\em acquire} operation and FENCE RW, W suffices to
% implement {\em release}, both imply additional unnecessary ordering as
% compared to AMOs with the corresponding {\em aq} or {\em rl} bit set.
\end{commentary}

图~\ref{critical}中显示了一个通过“测试与测试与设置”自旋锁来保护关键节的示例代码序列。
注意第一个AMO被标记了{\em aq},是为了将锁的获得排在关键节之前,而第二个AMO被标记了{\em rl},是为了将关键节排在锁的释放之前。
% An example code sequence for a critical section guarded by a
% test-and-test-and-set spinlock is shown in Figure~\ref{critical}.  Note the
% first AMO is marked {\em aq} to order the lock acquisition before the
% critical section, and the second AMO is marked {\em rl} to order
% the critical section before the lock relinquishment.

\begin{figure}[h!]
\begin{center}
\begin{verbatim}
        li           t0, 1        # 初始化交换的值。 Initialize swap value.
    again:
        lw           t1, (a0)     # 检查锁是否被占用。 Check if lock is held.
        bnez         t1, again    # 如果锁被占用则重试。 Retry if held.
        amoswap.w.aq t1, t0, (a0) # 尝试获取锁。 Attempt to acquire lock.
        bnez         t1, again    # 如果锁被占用则重试。 Retry if held.
        # ...
        # 关键小节 Critical section.
        # ...
        amoswap.w.rl x0, x0, (a0) # 通过存储0来释放锁。 Release lock by storing 0.
\end{verbatim}
\end{center}
\caption{互斥的样例代码。{\tt a0}包含了锁的地址。 Sample code for mutual exclusion.  {\tt a0} contains the address of the lock.}
\label{critical}
\end{figure}

\begin{commentary}
  我们推荐为锁的获取和释放使用上面显示的AMO交换用语,以简化推测锁省略~\cite{Rajwar:2001:SLE}的实现。
% We recommend the use of the AMO Swap idiom shown above for both lock
% acquire and release to simplify the implementation of speculative lock
% elision~\cite{Rajwar:2001:SLE}.
\end{commentary}

“A”扩展中的指令也可以被用于提供顺序一致性的加载和存储。顺序一致性加载可以用一个设置了{\em aq}和 {\em rl}的LR实现。
顺序一致性存储可以用一个AMOSWAP实现,它把旧的值写到x0,并设置{\em aq}和 {\em rl}。
% The instructions in the ``A'' extension can also be used to provide
% sequentially consistent loads and stores.  A sequentially consistent load can
% be implemented as an LR with both {\em aq} and {\em rl} set. A sequentially
% consistent store can be implemented as an AMOSWAP that writes the old value to
% x0 and has both {\em aq} and {\em rl} set.


\chapter{控制与状态寄存器(CSR)指令“Zicsr”(2.0版本)}
%\chapter{``Zicsr'', Control and Status Register (CSR) Instructions, Version 2.0}
\label{csrinsts}

RISC-V定义了一个独立的地址空间,包含与各硬件线程相关联的4096个控制和状态寄存器。
这章定义了操作在这些CSR上的CSR指令的完整集合。
% RISC-V defines a separate address space of 4096 Control and Status
% registers associated with each hart.  This chapter defines the full
% set of CSR instructions that operate on these CSRs.

\begin{commentary}
  CSR主要被用于特权架构,但同时也在非特权代码中有一些使用,包括用于计数器和计时器,以及浮点状态。
  % While CSRs are primarily used by the privileged architecture, there
  % are several uses in unprivileged code including for counters and
  % timers, and for floating-point status.

  计数器和计时器不再被认为是标准基础ISA的强制性部分,因此访问它们所需要的CSR指令已经从基础ISA章节~\ref{rv32}被移出,进入了这个独立的章节。
  % The counters and timers are no longer considered mandatory parts of
  % the standard base ISAs, and so the CSR instructions required to
  % access them have been moved out of Chapter~\ref{rv32} into this
  % separate chapter.
\end{commentary}

\section{CSR指令}

所有的CSR指令自动地读取-修改-写入一个单独的CSR,指令的位31-20持有的12位{\em csr}域中编码了CSR的标识符。
立即数形式使用5位的零扩展立即数,编码在rs1域中。
% All CSR instructions atomically read-modify-write a single CSR, whose
% CSR specifier is encoded in the 12-bit {\em csr} field of the
% instruction held in bits 31--20.  The immediate forms use a 5-bit
% zero-extended immediate encoded in the {\em rs1} field.

\vspace{-0.2in}
\begin{center}
\begin{tabular}{M@{}R@{}F@{}R@{}S}
\\
\instbitrange{31}{20} &
\instbitrange{19}{15} &
\instbitrange{14}{12} &
\instbitrange{11}{7} &
\instbitrange{6}{0} \\
\hline
\multicolumn{1}{|c|}{csr} &
\multicolumn{1}{c|}{rs1} &
\multicolumn{1}{c|}{funct3} &
\multicolumn{1}{c|}{rd} &
\multicolumn{1}{c|}{opcode} \\
\hline
12 & 5 & 3 & 5 & 7 \\
source/dest  & source & CSRRW  & dest & SYSTEM \\
source/dest  & source & CSRRS  & dest & SYSTEM \\
source/dest  & source & CSRRC  & dest & SYSTEM \\
source/dest  & uimm[4:0]   & CSRRWI & dest & SYSTEM \\
source/dest  & uimm[4:0]   & CSRRSI & dest & SYSTEM \\
source/dest  & uimm[4:0]   & CSRRCI & dest & SYSTEM \\
\end{tabular}
\end{center}

CSRRW(原子性读/写CSR)指令自动地交换CSR和整数寄存器中的值。
CSRRW读取CSR的旧值,把该值零扩展到XLEN位,然后把它写到整数寄存器{\em rd}。{\em rs1}中的初始值被写到CSR。
如果{\em rd}={\tt x0},那么指令不应当读CSR,也不应当引起任何可能在读CSR时发生的副作用。
% The CSRRW (Atomic Read/Write CSR) instruction atomically swaps values
% in the CSRs and integer registers. CSRRW reads the old value of the
% CSR, zero-extends the value to XLEN bits, then writes it to integer
% register {\em rd}.  The initial value in {\em rs1} is written to the
% CSR.  If {\em rd}={\tt x0}, then the instruction shall not read the CSR
% and shall not cause any of the side effects that might occur on a CSR
% read.

CSRRS(原子性读和设置CSR位)指令读取CSR的值,把该值零扩展到XLEN位,然后把它写到整数寄存器{\em rd}。
整数寄存器{\em rs1}中的初始值被视为位掩码,它指定要在CSR中设置的位的位置。
任何在{\em rs1}中为高的位将引起CSR中对应的位(如果它可写的话)被设置。CSR中的其它位不会被显式地写入。
% The CSRRS (Atomic Read and Set Bits in CSR) instruction reads the
% value of the CSR, zero-extends the value to XLEN bits, and writes it
% to integer register {\em rd}.  The initial value in integer register
% {\em rs1} is treated as a bit mask that specifies bit positions to be
% set in the CSR.  Any bit that is high in {\em rs1} will cause the
% corresponding bit to be set in the CSR, if that CSR bit is writable.
% Other bits in the CSR are not explicitly written.

CSRRC(原子性读和清除CSR位)指令读取CSR的值,把该值零扩展到XLEN位,然后把它写到整数寄存器{\em rd}。
整数寄存器{\em rs1}中的初始值被视为位掩码,它指定了要在CSR中被清除的位的位置。
任何在{\em rs1}中为高的位将引起CSR中对应的位(如果它是可写的话)被清除。CSR中的其它位不会被显式地写入。
% The CSRRC (Atomic Read and Clear Bits in CSR) instruction reads the
% value of the CSR, zero-extends the value to XLEN bits, and writes it
% to integer register {\em rd}.  The initial value in integer register
% {\em rs1} is treated as a bit mask that specifies bit positions to be
% cleared in the CSR.  Any bit that is high in {\em rs1} will cause the
% corresponding bit to be cleared in the CSR, if that CSR bit is writable.
% Other bits in the CSR are not explicitly written.

对于CSRRS和CSRRC,如果{\em rs1}={\tt x0},那么指令将完全不会写CSR,
并因此也应当既不会引起任何只可能在写CSR时发生的副作用,也不会在访问只读CSR时产生非法指令异常。
不管{\em rs1}和{\em rd}域如何设置,CSRRS和CSRRC都总是读取已编址的CSR,并引起任何读的副作用。
注意如果{\em rs1}指定了一个持有{\tt x0}以外的零值的寄存器,那么指令将仍然尝试把未修改的值写回到CSR,
并将引起任何随之而来的副作用。一个{\em rs1}={\tt x0}的CSSRW将尝试向目的CSR写入零。
% For both CSRRS and CSRRC, if {\em rs1}={\tt x0}, then the instruction
% will not write to the CSR at all, and so shall not cause any of the
% side effects that might otherwise occur on a CSR write, nor
% raise illegal instruction exceptions on accesses to read-only CSRs.
% Both CSRRS and CSRRC always read the addressed CSR and cause any read
% side effects regardless of {\em rs1} and {\em rd} fields.  Note that
% if {\em rs1} specifies a register holding a zero value other than {\tt
%   x0}, the instruction will still attempt to write the unmodified
% value back to the CSR and will cause any attendant side effects.  A
% CSRRW with {\em rs1}={\tt x0} will attempt to write zero to the
% destination CSR.

CSRRWI、CSRRSI和CSRRCI变体分别与CSRRW、CSRRS和CSRRC相似,
除了它们使用一个XLEN位的值来更新CSR,这个值通过零扩展编码在{\em rs1}中的一个5位的无符号立即数(uimm[4:0])域得到,
而不是来自一个整数寄存器。对于CSRRI和CSRRCI,如果uimm[4:0]域是零,那么这些指令将不会写CSR,
并且应当既不引起任何只可能在写CSR时发生的副作用,也不在访问只读CSR时产生非法的指令异常。
对于CSRRWI,如果{\em rd}={\tt x0},那么指令不应当读CSR,也不应当跟引起任何可能在读CSR时发生的副作用。
不论{\em rd}和{\em rs1}域如何,CSRRSI和CSRRCI都将总是读CSR,和引起任何读的副作用。
% The CSRRWI, CSRRSI, and CSRRCI variants are similar to CSRRW, CSRRS,
% and CSRRC respectively, except they update the CSR using an XLEN-bit
% value obtained by zero-extending a 5-bit unsigned immediate (uimm[4:0]) field
% encoded in the {\em rs1} field instead of a value from an integer
% register.  For CSRRSI and CSRRCI, if the uimm[4:0] field is zero, then
% these instructions will not write to the CSR, and shall not cause any
% of the side effects that might otherwise occur on a CSR write, nor raise
% illegal instruction exceptions on accesses to read-only CSRs.
% For CSRRWI, if {\em rd}={\tt x0}, then the instruction shall not read the
% CSR and shall not cause any of the side effects that might occur on a
% CSR read.  Both CSRRSI and CSRRCI will always read the CSR and cause
% any read side effects regardless of {\em rd} and {\em rs1} fields.

\begin{table}
  \centering
  \begin{tabular}{|l|c|c|c|c|}
    \hline
    \multicolumn{5}{|c|}{寄存器操作数} \\
    \hline
    指令 & \textit{rd} 是 \texttt{x0}
                      & \textit{rs1} 是 \texttt{x0}
                            & 读 CSR & 写 CSR \\
    \hline
    CSRRW       & 是 & --  & 否  & 是 \\
    CSRRW       & 否  & --  & 是 & 是 \\
    CSRRS/CSRRC & --  & 是 & 是 & 否 \\
    CSRRS/CSRRC & --  & 否  & 是 & 是 \\
    \hline
    \multicolumn{5}{|c|}{立即数操作数} \\
    \hline
    指令 & \textit{rd} 是 \texttt{x0}
                        & \textit{uimm}$=$0
                              & 读 CSR & 写 CSR \\
    \hline
    CSRRWI        & 是 & --  & 否  & 是 \\
    CSRRWI        & 否  & --  & 是 & 是 \\
    CSRRSI/CSRRCI & --  & 是 & 是 & 否 \\
    CSRRSI/CSRRCI & --  & 否  & 是 & 是 \\
    \hline
  \end{tabular}
  \caption{决定一条CSR指令是否读或写指定CSR的条件。  
  % Conditions determining whether a CSR instruction reads or writes
    % the specified CSR.
    }
  \label{tab:csrsideeffects}
\end{table}

表~\ref{tab:csrsideeffects}总结了CSR指令在它们是否读和/或写CSR方面的行为。
% Table~\ref{tab:csrsideeffects} summarizes the behavior of the CSR
% instructions with respect to whether they read and/or write the CSR.

对于任何由于具有特定值的CSR而发生事件或后果,如果一次写入CSR给了它该值,那么导致的事件或后果被称为该写入的\emph{间接效果}。
RISC-V ISA不认为一个CSR写入的间接效果是该写入的副作用。
% For any event or consequence that occurs due to a CSR having a particular
% value, if a write to the CSR gives it that value, the resulting event or
% consequence is said to be an \emph{indirect effect} of the write.
% Indirect effects of a CSR write are not considered by the RISC-V ISA to
% be side effects of that write.

\begin{commentary}
  CSR访问的一个副作用的例子是,如果从一个指定CSR读取将导致灯泡打开,而向同一个CSR写入一个奇数值将导致灯泡关闭。
  假定写入一个偶数值没有影响。在这种情况下,读取和写入都有控制灯泡是否点亮的副作用,因为这个条件不仅仅根据CSR的值决定。
  (注意,在将一个奇数值写入CSR来关闭灯之后,再读来打开灯,重新写相同的奇数值会导致灯再次关闭。
  因此,在最后一次写入时,不是CSR值的变化关掉了灯。)
  % An example of side effects for CSR accesses would be if reading from a
  % specific CSR causes a light bulb to turn on, while writing an odd value
  % to the same CSR causes the light to turn off.
  % Assume writing an even value has no effect.
  % In this case, both the read and write have side effects controlling
  % whether the bulb is lit, as this condition is not determined solely
  % from the CSR value.
  % (Note that after writing an odd value to the CSR to turn off the light,
  % then reading to turn the light on, writing again the same odd value
  % causes the light to turn off again.
  % Hence, on the last write, it is not a change in the CSR value that
  % turns off the light.)

  另一方面,如果在某个特定CSR的值为奇数时操纵灯泡,那么打开和关闭灯泡不会被视为写入CSR的副作用,仅仅是这种写入的间接效果。
  % On the other hand, if a bulb is rigged to light whenever the value
  % of a particular CSR is odd, then turning the light on and off is not
  % considered a side effect of writing to the CSR but merely an indirect
  % effect of such writes.

  更具体地,第二卷中定义的RISC-V特权架构表明,CSR值的特定组合会导致陷入的发生。
  当对CSR的一次显式写入创造了触发陷入的条件时,该陷入不被认为是写入的副作用,而仅仅是间接效果。
  % More concretely, the RISC-V privileged architecture defined in
  % Volume~II specifies that certain combinations of CSR values cause a
  % trap to occur.
  % When an explicit write to a CSR creates the conditions that trigger the
  % trap, the trap is not considered a side effect of the write but merely
  % an indirect effect.

  标准CSR没有任何关于读的副作用。标准CSR可能有关于写的副作用。自定义扩展可能添加某些CSR,其访问具有关于读或写的副作用。
  % Standard CSRs do not have any side effects on reads.
  % Standard CSRs may have side effects on writes.
  % Custom extensions might add CSRs for which accesses have side effects
  % on either reads or writes.
\end{commentary}

一些CSR,例如指令引退计数器、{\tt 指令返回}(instret),可能因为指令执行的副作用而被修改。
在这些情况中,如果一条CSR访问指令读了一个CSR,它读取值要优先于指令的执行。
如果一条CSR访问指令写这样的一个CSR,那么写被完成,而不是执行自增。
特别地,被一条指令写到{\tt instret}的值将是下一条指令所读到的值。
% Some CSRs, such as the instructions-retired counter, {\tt instret},
% may be modified as side effects of instruction execution.  In these
% cases, if a CSR access instruction reads a CSR, it reads the value
% prior to the execution of the instruction.  If a CSR access
% instruction writes such a CSR, the write is done instead of the
% increment.  In particular, a value written to {\tt instret} by one
% instruction will be the value read by the following instruction.

读CSR的汇编器伪指令,CSRR {\em rd, csr},被编码为CSRRS {\em rd, csr, x0}。
写CSR的汇编器伪指令,CSRW {\em csr, rs1},被编码为CSRRW {\em x0, csr, rs1},
同时{\em csr, uimm}被编码为CSRRWI {\em x0, csr, uimm}。
% The assembler pseudoinstruction to read a CSR, CSRR {\em rd, csr}, is
% encoded as CSRRS {\em rd, csr, x0}.  The assembler pseudoinstruction
% to write a CSR, CSRW {\em csr, rs1}, is encoded as CSRRW {\em x0, csr,
%   rs1}, while CSRWI {\em csr, uimm}, is encoded as CSRRWI {\em x0,
%   csr, uimm}.

当旧的值不需要的时候,进一步定义了设置和清除CSR中的位的汇编器伪指令:CSRS/CSRC {\em csr, rs1}; CSRSI/SCRCI {\em csr, uimm}。
% Further assembler pseudoinstructions are defined to set and clear
% bits in the CSR when the old value is not required: CSRS/CSRC {\em
%   csr, rs1}; CSRSI/CSRCI {\em csr, uimm}.


\subsection*{CSR访问排序}

每个RISC-V硬件线程通常按照程序的执行次序观察自己的CSR访问、包括隐式CSR访问。
特别地,除非另有规定,CSR访问的执行是在任何按程序次序在先的、其行为将更改CSR状态或被CSR状态更改的指令被执行之后,
且在任何按程序次序后继的、其行为将更改CSR状态或被CSR状态更改的指令被执行之前。
此外,显式CSR读返回指令执行前的CSR状态,而显式CSS写禁止并重写同一指令对于相同CSR的任何隐式写入或修改。
% Each RISC-V hart normally observes its own CSR accesses, including its
% implicit CSR accesses, as performed in program order.
% In particular, unless specified otherwise, a CSR access is performed
% after the execution of any prior instructions in program order whose behavior
% modifies or is modified by the CSR state and before the execution of any
% subsequent instructions in program order whose behavior modifies or is modified
% by the CSR state.
% Furthermore, an explicit CSR read returns the
% CSR state before the execution of the instruction, while an
% explicit CSR write suppresses and overrides any implicit writes or
% modifications to the same CSR by the same instruction.

同样,任何来自显式CSR访问的副作用通常都被观察到以程序次序同步发生。
除非另有规定,任何这种副作用的全部后果都可以被下一条指令观察到,并且先前指令不可能乱序地观察到任何后果。
(注意之前对于CSR写入的副作用和间接效果所做的区分。)
% Likewise, any side effects from an explicit CSR access are normally
% observed to occur synchronously in program order.
% Unless specified otherwise, the full consequences of any such side
% effects are observable by the very next instruction, and no consequences
% may be observed out-of-order by preceding instructions.
% (Note the distinction made earlier between side effects and indirect
% effects of CSR writes.)

对于RVWMO内存一致性模型(第~\ref{ch:memorymodel}章),CSR的访问默认是弱有序的,
所以其它硬件线程或设备可以以一种不同于程序次序的次序观测到CSR的访问。
另外,CSR的访问并非按照显式内存访问排序,除非CSR的访问修改了实施显式内存访问的指令的执行行为,
或者除非CSR的访问和显式内存访问被内存模型定义的语法依赖或本手册第二卷中内存排序PMA节定义的排序需求所排序。
为了在所有其它情况中强制排序,软件应当在相关访问之间执行FENCE指令。
处于FENCE指令的目的,CSR读访问被归类为设备输入(I),而CSR写访问被归类为设备输出(O)。
% For the RVWMO memory consistency model (Chapter~\ref{ch:memorymodel}),
% CSR accesses are weakly ordered by default,
% so other harts or devices may observe CSR accesses in an order
% different from program order. In addition, CSR accesses are not ordered with
% respect to explicit memory accesses, unless a CSR access modifies the execution
% behavior of the instruction that performs the explicit memory access or unless
% a CSR access and an explicit memory access are ordered by either the syntactic
% dependencies defined by the memory model or the ordering requirements defined
% by the Memory-Ordering PMAs section in Volume II of this manual. To enforce
% ordering in all other cases, software should execute a FENCE instruction
% between the relevant accesses. For the purposes of the FENCE instruction, CSR
% read accesses are classified as device input (I), and CSR write accesses are
% classified as device output (O).

\begin{commentary}
  非正式地,CSR空间扮演着一个弱排序的内存映射I/O区域,正如本手册第二卷中内存排序PMA节所定义的那样。
  因此,CSR访问的次序与所有其它访问的次序都受到相同机制的约束,该机制将内存映射I/O访问的次序约束到这样的区域内。
% Informally, the CSR space acts as a weakly ordered memory-mapped I/O region, as
% defined by the Memory-Ordering PMAs section in Volume II of this manual. As a
% result, the order of CSR accesses with respect to all other accesses is
% constrained by the same mechanisms that constrain the order of memory-mapped
% I/O accesses to such a region.

这些CSR次序约束用于支持那些对设备或其它硬件线程可见、可被影响的CSR访问中,主内存和内存映射I/O的有序访问。
例子包括{\tt time}、{\tt cycle}和{\tt mcycle} CSR,以及反映挂起中断的CSR,如{\tt mip}和{\tt sip}。
注意,这类CSR的隐式读取(例如,由于{\tt mip}的改变而采取的中断)也会被作为设备输入而排序。
% These CSR-ordering constraints are imposed to support ordering main
% memory and memory-mapped I/O accesses with respect to CSR accesses that
% are visible to, or affected by, devices or other harts.
% Examples include the {\tt time}, {\tt cycle}, and {\tt mcycle}
% CSRs, in addition to CSRs that reflect pending interrupts, like {\tt mip} and
% {\tt sip}.
% Note that implicit reads of such CSRs (e.g., taking an interrupt because of
% a change in {\tt mip}) are also ordered as device input.

大多数CSR(包括,例如,{\tt fcsr})对其它硬件线程是不可见的;
它们的访问可以按关于FENCE指令的全局内存次序自由地重新排序,而不违反本规范。
% Most CSRs (including, e.g., the {\tt fcsr}) are not visible to other harts;
% their accesses can be freely reordered in the global memory order with respect
% to FENCE instructions without violating this specification.
\end{commentary}

硬件平台可以把对特定CSR的访问定义为强排序的,就像本手册的第二卷中内存排序PMA节里定义的那样。
对强排序CSR的访问相对于对弱排序CSR的访问和内存映射I/O区域的访问,有更强的次序约束。
% The hardware platform may define that accesses to certain CSRs are
% strongly ordered, as defined by the Memory-Ordering PMAs section in Volume II
% of this manual. Accesses to strongly ordered CSRs have stronger ordering
% constraints with respect to accesses to both weakly ordered CSRs and accesses
% to memory-mapped I/O regions.

\begin{commentary}
  按全局内存次序进行CSR访问的重新排序的规则应该大概被移动至第~\ref{ch:memorymodel}章,关于RVWMO内存一致性模型。
% The rules for the reordering of CSR accesses in the global memory order
% should probably be moved to Chapter~\ref{ch:memorymodel} concerning the
% RVWMO memory consistency model.
\end{commentary}

% \chapter{``Zicntr'' and ``Zihpm'' Counters}
\chapter{“Zicntr”和“Zihpm”计数器}
\label{counters}

RISC-V ISA提供了一组至多32个64位性能计数器和计时器,
它们可以通过非特权XLEN只读CSR寄存器{\tt 0xC00}--{\tt 0xC1F}
(当XLEN=32时,高32位通过CSR寄存器{\tt 0xC80}--{\tt 0xC9F})来访问。这些计数器被划分到“Zicntr”和“Zihpm”扩展。
% RISC-V ISAs provide a set of up to thirty-two 64-bit performance counters and
% timers that are accessible via unprivileged XLEN-bit read-only CSR
% registers {\tt 0xC00}--{\tt 0xC1F} (when XLEN=32, the upper 32 bits
% are accessed via CSR registers {\tt 0xC80}--{\tt 0xC9F}).
% These counters are divided between the ``Zicntr'' and ``Zihpm'' extensions.

\section{用于基础计数器和计时器的“Zicntr”标准扩展}
% \section{``Zicntr'' Standard Extension for Base Counters and Timers}

Zicntr标准扩展包括这些计数器的前三个(CYCLE、TIME、和INSTRET),
它们具有专门的功能(分别是周期计数、实时时钟、和指令引退)。Zicntr扩展依赖于Zicsr扩展。
% The Zicntr standard extension comprises the first three of these
% counters (CYCLE, TIME, and INSTRET), which
% have dedicated functions (cycle
% count, real-time clock, and instructions retired, respectively).
% The Zicntr extension depends on the Zicsr extension.

\begin{commentary}
  我们建议在实现中提供这些基本计数器,因为它们对于基本性能分析、适应和动态优化、以及允许应用处理实时流是必需的。
  单独的Zihpm扩展中的其它计数器可以帮助诊断性能问题,应当允许用户级应用代码以较低开销访问这些计数器。
% We recommend provision of these basic counters in implementations as
% they are essential for basic performance analysis, adaptive and
% dynamic optimization, and to allow an application to work with
% real-time streams.  Additional counters in the separate Zihpm extension can
% help diagnose performance problems and these should be made accessible
% from user-level application code with low overhead.

某些执行环境可能禁止访问计数器,例如,为了阻止定时侧信道攻击。
% Some execution environments might prohibit access to counters, for
% example, to impede timing side-channel attacks.
\end{commentary}

\vspace{-0.2in}
\begin{center}
\begin{tabular}{M@{}R@{}F@{}R@{}S}
\\
\instbitrange{31}{20} &
\instbitrange{19}{15} &
\instbitrange{14}{12} &
\instbitrange{11}{7} &
\instbitrange{6}{0} \\
\hline
\multicolumn{1}{|c|}{csr} &
\multicolumn{1}{c|}{rs1} &
\multicolumn{1}{c|}{funct3} &
\multicolumn{1}{c|}{rd} &
\multicolumn{1}{c|}{opcode} \\
\hline
12 & 5 & 3 & 5 & 7 \\
RDCYCLE[H]   & 0 & CSRRS  & dest & SYSTEM \\
RDTIME[H]    & 0 & CSRRS  & dest & SYSTEM \\
RDINSTRET[H] & 0 & CSRRS  & dest & SYSTEM \\
\end{tabular}
\end{center}

对于XLEN$\geq$64的基础ISA,CSR指令可以直接访问所有的64位CSR。
特别地,RDCYCLE、RDTIME和RDINSTRET伪指令读取所有的64位{\tt cycle}、{\tt time}和{\tt instret}计数器。
% For base ISAs with XLEN$\geq$64, CSR instructions can access the full
% 64-bit CSRs directly.  In particular, the RDCYCLE, RDTIME, and
% RDINSTRET pseudoinstructions read the full 64 bits of the {\tt cycle},
% {\tt time}, and {\tt instret} counters.

\begin{commentary}
  计数器伪指令被映射到只读的{\tt csrrs rd, counter, x0}典型形式,
  但是其它的只读CSR指令形式(基于CSRRC/CSRRSI/CSRRCI)也是读取这些CSR的合法方式。
% The counter pseudoinstructions are mapped to the read-only {\tt csrrs
%   rd, counter, x0} canonical form, but the other read-only CSR
% instruction forms (based on CSRRC/CSRRSI/CSRRCI) are also legal ways
% to read these CSRs.
\end{commentary}

对于XLEN=32的基础ISA,Zicntr扩展使这三个64位只读计数器可以被以32位片段的形式访问。
RDCYCLE、RDTIME和RDINSTRET伪指令提供低32位,而RDCYCLEH、RDTIMEH和RDINSTRETH伪指令提供对应计数器的高32位。
% For base ISAs with XLEN=32, the Zicntr extension enables the three
% 64-bit read-only counters to be accessed in 32-bit pieces.
% The RDCYCLE, RDTIME, and RDINSTRET pseudoinstructions provide the lower 32
% bits, and the RDCYCLEH, RDTIMEH, and RDINSTRETH pseudoinstructions provide
% the upper 32 bits of the respective counters.

\begin{commentary}
  我们需要计数器是64位宽的——即使是在XLEN=32的时候。
  因为,否则的话,软件将很难决定值是否溢出。
  对于一个低端实现,各计数器的高32位可以使用软件计数器来实现,通过低32位的溢出触发陷入处理器进行增长。
  下面给出的样例代码显示了如何使用独立的32位宽伪指令安全地读取所有64位宽的值。
% We required the counters be 64 bits wide, even when XLEN=32, as otherwise
% it is very difficult for software to determine if values have
% overflowed.  For a low-end implementation, the upper 32 bits of each
% counter can be implemented using software counters incremented by a
% trap handler triggered by overflow of the lower 32 bits.  The sample
% code given below shows how the full 64-bit width value can be
% safely read using the individual 32-bit width pseudoinstructions.
\end{commentary}

RDCYCLE伪指令读取{\tt cycle} CSR的低XLEN位,该CSR持有时钟周期数的计数,
由从过去任意启动时间运行硬件线程的处理器核执行。RDCYCLEH仅在XLEN=32时存在,
它读取相同的cycle计数器的位63 - 32。实际中,底层64位计数器将永远不会溢出。
cycle计数器的推进率将依赖于实现和操作环境。执行环境应当提供一个决定当前cycle计数器增加的(周期/秒)率的方法。
% The RDCYCLE pseudoinstruction reads the low XLEN bits of the {\tt
%   cycle} CSR which holds a count of the number of clock cycles
% executed by the processor core on which the hart is running from an
% arbitrary start time in the past.  RDCYCLEH is only present when
% XLEN=32 and reads bits 63--32 of the same cycle
% counter.  The underlying 64-bit counter should never overflow in
% practice.  The rate at which the cycle counter advances will depend on
% the implementation and operating environment.  The execution
% environment should provide a means to determine the current rate
% (cycles/second) at which the cycle counter is incrementing.

\begin{commentary}
  RDCYCLE试图返回处理器核(而不是硬件线程)的执行周期数。
  在给定某些实现选择(例如,AMD Bulldozer)时,很难精确定义什么是“核心”。
  给定实现(包括软件模拟)的范围时,精确定义什么是“时钟周期”也是困难的,
  但是目的在于,RDCYCLE和其它性能计数器一起被用于性能监视。
  特别地,在有硬件线程/核心的地方,人们会希望有周期计数/已引退指令来测量硬件线程的CPI。
% RDCYCLE is intended to return the number of cycles executed by the
% processor core, not the hart.  Precisely defining what is a ``core'' is
% difficult given some implementation choices (e.g., AMD Bulldozer).
% Precisely defining what is a ``clock cycle'' is also difficult given the
% range of implementations (including software emulations), but the
% intent is that RDCYCLE is used for performance monitoring along with the
% other performance counters.  In particular, where there is one
% hart/core, one would expect cycle-count/instructions-retired to
% measure CPI for a hart.

完全不必要将核心暴露给软件,而且实现者可能选择让一个物理核心上的多个硬件线程假装运行在一个硬件线程/核心上的分离的多个核心上,
并为各个硬件线程提供独立的周期计数器。这在内部硬件线程的实时交互不存在或者极少的简单桶式处理器中(例如,CDC 6000外围处理器)可能是有道理的。
% Cores don't have to be exposed to software at all, and an implementor
% might choose to pretend multiple harts on one physical core are
% running on separate cores with one hart/core, and provide separate
% cycle counters for each hart.  This might make sense in a simple
% barrel processor (e.g., CDC 6600 peripheral processors) where
% inter-hart timing interactions are non-existent or minimal.

在有多于一个的硬件线程/核心和动态多线程的地方,通常不可能分离每个硬件线程的周期(尤其是有SMT时)。
或许可能定义一个独立的性能计数器,它试图捕捉一个特定的正在运行的硬件线程的周期数,
但是这个定义将不得不非常模糊以覆盖所有可能的线程实现。
例如,我们应当只计数任意被发出执行这个硬件线程的指令的周期?和/或任何失效指令的周期?
或是包含了这个硬件线程虽然正在占用机器资源、但由于其它硬件线程转入执行而暂停,导致不能执行的周期?
可能,需要“以上所有”才能获得可理解的性能统计数据。定义每个硬件线程周期计数的这种复杂性,
以及当调整多线程代码时,在任何情况中对每个核心的周期总数计数的需求,导致了每个核心的周期计数器的标准化,
这也恰好适用于常见的单硬件线程/核心的情况。
% Where there is more than one hart/core and dynamic multithreading, it
% is not generally possible to separate out cycles per hart (especially
% with SMT).  It might be possible to define a separate performance
% counter that tried to capture the number of cycles a particular hart
% was running, but this definition would have to be very fuzzy to cover
% all the possible threading implementations.  For example, should we
% only count cycles for which any instruction was issued to execution
% for this hart, and/or cycles any instruction retired, or include
% cycles this hart was occupying machine resources but couldn't execute
% due to stalls while other harts went into execution? Likely, ``all of
% the above'' would be needed to have understandable performance stats.
% This complexity of defining a per-hart cycle count, and also the need
% in any case for a total per-core cycle count when tuning multithreaded
% code led to just standardizing the per-core cycle counter, which also
% happens to work well for the common single hart/core case.

将在“睡眠”期间发生的事情标准化是不实际的,因为“睡眠”的含义不是跨执行环境标准化的,
但是如果代码整体被暂停(在深度睡眠中完全门控时钟或断电),那么时钟周期不会执行,
且周期计数每次也将不会按规格增加。这里有许多细节,例如,在处理器从断电事件中被唤醒之后,
所需要的用于重置处理器的时钟周期是否应当被计数,而这些都被认为是特定于执行环境的细节。
% Standardizing what happens during ``sleep'' is not practical given
% that what ``sleep'' means is not standardized across execution
% environments, but if the entire core is paused (entirely clock-gated
% or powered-down in deep sleep), then it is not executing clock cycles,
% and the cycle count shouldn't be increasing per the spec.  There are
% many details, e.g., whether clock cycles required to reset a processor
% after waking up from a power-down event should be counted, and these
% are considered execution-environment-specific details.

即使没有作用于全平台的精确定义,仍然有对于大多数平台都有用的版本,并且此处有一个不精确的、
常用的、“通常是正确的”标准总比没有标准要更好。RDCYCLE的意图主要是性能监视/调整,而规范在编写时考虑了此目标。
% Even though there is no precise definition that works for all
% platforms, this is still a useful facility for most platforms, and an
% imprecise, common, ``usually correct'' standard here is better than no
% standard.  The intent of RDCYCLE was primarily performance
% monitoring/tuning, and the specification was written with that goal in
% mind.
\end{commentary}

RDTIME伪指令读取{\tt time} CSR的低XLEN位,其统计了从过去任意时间开始的已经经过的墙上挂钟的真实时间。
RDTIMEH仅在XLEN=32时存在,它读取相同真实时间计数器的位63 - 32。在实际中,
底层64位计数器通过实时时钟的每次滴答来增加一,并且,对于实际的实时时钟频率,应当永远不会溢出。
执行环境应当提供一种决定计数器滴答(秒/滴答)周期的方法。该周期应当在一个小的误差范围内是恒定的。
环境应当提供一种决定时钟精度的方法(即,名义上的和实际的实时时钟周期之间的最大相关误差)。
% The RDTIME pseudoinstruction reads the low XLEN bits of the {\tt
%   time} CSR, which counts wall-clock real time that has passed from an
% arbitrary start time in the past.
% RDTIMEH is only present when XLEN=32 and reads bits 63--32 of the same
% real-time counter.
% The underlying 64-bit counter increments by one with each tick of the
% real-time clock, and, for realistic real-time clock frequencies, should never
% overflow in practice.
% The execution environment should provide a means of determining the period of
% a counter tick (seconds/tick).
% The period should be constant within a small error bound.
% The environment should provide a means to determine the accuracy of the clock
% (i.e., the maximum relative error between the nominal and actual real-time
% clock periods).

\begin{commentary}
  在一些简单的平台上,周期计数可能代表了RDTIME的一个有效的实现,在这种情况中,RDTIME和RDCYCLE可能返回相同的结果。
% On some simple platforms, cycle count might represent a valid
% implementation of RDTIME, in which case RDTIME and RDCYCLE may
% return the same result.

  由于实施平台的广泛的多样性,很难提供对时钟周期的严格的强制规定。最大误差范围应当根据平台的需求来设置。
% It is difficult to provide a strict mandate on clock period given the
% wide variety of possible implementation platforms.  The maximum error
% bound should be set based on the requirements of the platform.
\end{commentary}

所有硬件线程的实时时钟必须被同步到实时时钟的一个滴答以内。
% The real-time clocks of all harts
% must be synchronized to within one tick of the real-time clock.

\begin{commentary}
  与其它的架构强制规定一样,只要出现硬件线程“好像”被同步到实时时钟的一个滴答以内就足够了,
  即,软件无法观察到两个硬件线程上的实时时钟值之间存在更大的差异。
% As with other architectural mandates, it suffices to appear ``as if''
% harts are synchronized to within one tick of the real-time clock,
% i.e., software is unable to observe that there is a greater delta
% between the real-time clock values observed on two harts.
\end{commentary}

RDINSTRET伪指令读取{\tt instret} CSR的低XLEN位,其统计本硬件线程从过去某些任意起始点开始的已失效指令的数目。
RDINSTRETH仅在XLEN=32时存在,它读取相同指令计数器的位63 - 32。在实际中,底层64位计数器应当永远不会溢出。
% The RDINSTRET pseudoinstruction reads the low XLEN bits of the {\tt
%   instret} CSR, which counts the number of instructions retired by
% this hart from some arbitrary start point in the past.  RDINSTRETH is
% only present when XLEN=32 and reads bits 63--32 of the same
% instruction counter. The underlying 64-bit counter should never
% overflow in practice.

% 这段在word中没有,可能是当前这个tex源代码文件比官网版本多的部分。
% \begin{commentary}
% Instructions that cause synchronous exceptions, including ECALL and EBREAK,
% are not considered to retire and hence do not increment the {\tt instret} CSR.
% \end{commentary}

下面的代码序列将把一个有效的64位计数器的值读进{\tt x3}:{\tt x2},即使计数器在读取它的上半部分和下半部分之间,
溢出了它的下半部分。
% The following code sequence will read a valid 64-bit cycle counter value into
% {\tt x3}:{\tt x2}, even if the counter overflows its lower half between reading its upper
% and lower halves.

\begin{figure}[h!]
\begin{center}
\begin{verbatim}
    again:
        rdcycleh     x3
        rdcycle      x2
        rdcycleh     x4
        bne          x3, x4, again
\end{verbatim}
\end{center}
\caption{当XLEN=32时,读取64位周期计数器的样例代码。  
% Sample code for reading the 64-bit cycle counter when XLEN=32.
}
\label{rdcycle}
\end{figure}

\section{用于硬件性能计数器的“Zihpm”标准扩展}
% \section{``Zihpm'' Standard Extension for Hardware Performance Counters}

Zihpm扩展包括至多29个额外的非特权64位硬件性能计数器,{\tt hpmcounter3} - {\tt hpmcounter31}。
当XLEN=32时,这些性能计数器的高32位可以通过额外的CSR {\tt hpmcounter3h} - {\tt hpmcounter31h}进行访问。
Zihpm扩展依赖于Zicsr扩展。
% The Zihpm extension comprises up to 29 additional unprivileged 64-bit
% hardware performance counters, {\tt hpmcounter3}--{\tt hpmcounter31}.
% When XLEN=32, the upper 32 bits of these performance counters are
% accessible via additional CSRs {\tt hpmcounter3h}--{\tt
%   hpmcounter31h}.  The Zihpm extension depends on the Zicsr extension.

\begin{commentary}
  在某些应用中,能够在同时立即读取多个计数器是很重要的。
  当运行在一个多任务环境下时,用户线程在尝试读取计数器的同时可能遭遇一次上下文的切换。
  对于用户线程,一个解决方案是,事先读取真实时间计数器,然后之后读取其它计数器来决定在这个序列中是否发生了上下文切换,
  如果是发生切换的情形,可以令读取失效。我们考虑添加输出锁存器来允许用户线程自动对计数器的值进行快照,
  但是这将增加用户上下文的尺寸,尤其是对于有更多计数器集的实现来说。
% In some applications, it is important to be able to read multiple
% counters at the same instant in time.  When run under a multitasking
% environment, a user thread can suffer a context switch while
% attempting to read the counters.  One solution is for the user thread
% to read the real-time counter before and after reading the other
% counters to determine if a context switch occurred in the middle of the
% sequence, in which case the reads can be retried.  We considered
% adding output latches to allow a user thread to snapshot the counter
% values atomically, but this would increase the size of the user
% context, especially for implementations with a richer set of counters.
\end{commentary}

这些额外计数器的实现数量和宽度,以及它们所统计事件的集合,是平台相关的。
访问一个未实现的或者错误配置的计数器可能引发一个非法指令异常,或者可能返回一个常量值。
% The implemented number and width of these additional counters, and the
% set of events they count, is platform-specific.  Accessing an
% unimplemented or ill-configured counter may cause an illegal
% instruction exception or may return a constant value.

执行环境应当提供一种决定计数器实现的数目和宽度的方法,以及一个配置各计数器所统计事件的接口。
% The execution environment should provide a means to determine the
% number and width of the implemented counters, and an interface to
% configure the events to be counted by each counter.

\begin{commentary}
  对于实现RISC-V特权执行环境的平台,特权架构手册描述了通过较低特权模式来控制访问这些计数器、和把事件设置为可被计数的特权CSR。
  % For execution environments implemented on RISC-V privileged
  % platforms, the privileged architecture manual describes privileged
  % CSRs controlling access by lower privileged modes to these counters,
  % and to set the events to be counted.

  备用的执行环境(例如,仅用户级的软件性能模型)可以提供替代机制来配置被性能计数器计数的事件。
  % Alternative execution environments (e.g., user-level-only software
  % performance models) may provide alternative mechanisms to configure
  % the events counted by the performance counters.

  对于统计ISA级别的度量标准(例如浮点指令执行的数量)和可能的少量常见微架构度量标准(例如“L1指令缓存缺失”)来说,事件设置的最终标准化将是有用的。
  % It would be useful to eventually standardize event settings to count
  % ISA-level metrics, such as the number of floating-point instructions
  % executed for example, and possibly a few common microarchitectural
  % metrics, such as ``L1 instruction cache misses''.
\end{commentary}

用于单精度浮点的“F”标准扩展(2.2版本)
\chapter{用于单精度浮点的“F”标准扩展(2.2版本)}
% \chapter{``F'' Standard Extension for Single-Precision Floating-Point,
% Version 2.2}
\label{sec:single-float}

这章描述了用于单精度浮点的标准指令集扩展(其被命名为“F”),
并添加了兼容IEEE 754-2008算数标准~\cite{ieee754-2008}的单精度浮点运算指令。
F扩展依靠“Zicsr”扩展来访问控制和状态寄存器。
% This chapter describes the standard instruction-set extension for
% single-precision floating-point, which is named ``F'' and adds
% single-precision floating-point computational instructions compliant
% with the IEEE 754-2008 arithmetic standard~\cite{ieee754-2008}.
% The F extension depends on the ``Zicsr'' extension for control
% and status register access.

\section{F寄存器状态}
% \section{F Register State}

F扩展添加了32个浮点寄存器,{\tt f0}-{\tt f31},它们每个都是32位宽,
并添加了一个浮点控制和状态寄存器{\tt fcsr},它包含了浮点单元的操作模式和异常状态。
这个额外的状态被显示在表~\ref{fprs}中。我们使用术语FLEN来描述RISC-V ISA中的浮点寄存器的宽度,
而对于F单精度浮点扩展,有FLEN=32。大多数浮点指令在浮点寄存器文件中的值上进行操作。
浮点加载和存储指令在寄存器和内存之间传递浮点值。也提供了把值传入和传出整数寄存器文件的指令。
% The F extension adds 32 floating-point registers, {\tt f0}--{\tt f31},
% each 32 bits wide, and a floating-point control and status register
% {\tt fcsr}, which contains the operating mode and exception status of the
% floating-point unit.  This additional state is shown in
% Figure~\ref{fprs}.  We use the term FLEN to describe the width of the
% floating-point registers in the RISC-V ISA, and FLEN=32 for the F
% single-precision floating-point extension.  Most floating-point
% instructions operate on values in the floating-point register file.
% Floating-point load and store instructions transfer floating-point
% values between registers and memory.  Instructions to transfer values
% to and from the integer register file are also provided.

\begin{figure}[htbp]
{\footnotesize
\begin{center}
\begin{tabular}{p{2in}}
\instbitrange{FLEN-1}{0}                                    \\ \cline{1-1}
\multicolumn{1}{|c|}{\reglabel{\ \ \ \ f0\ \ \ \ \ }}       \\ \cline{1-1}
\multicolumn{1}{|c|}{\reglabel{\ \ \ \ f1\ \ \ \ \ }}       \\ \cline{1-1}
\multicolumn{1}{|c|}{\reglabel{\ \ \ \ f2\ \ \ \ \ }}       \\ \cline{1-1}
\multicolumn{1}{|c|}{\reglabel{\ \ \ \ f3\ \ \ \ \ }}       \\ \cline{1-1}
\multicolumn{1}{|c|}{\reglabel{\ \ \ \ f4\ \ \ \ \ }}       \\ \cline{1-1}
\multicolumn{1}{|c|}{\reglabel{\ \ \ \ f5\ \ \ \ \ }}       \\ \cline{1-1}
\multicolumn{1}{|c|}{\reglabel{\ \ \ \ f6\ \ \ \ \ }}       \\ \cline{1-1}
\multicolumn{1}{|c|}{\reglabel{\ \ \ \ f7\ \ \ \ \ }}       \\ \cline{1-1}
\multicolumn{1}{|c|}{\reglabel{\ \ \ \ f8\ \ \ \ \ }}       \\ \cline{1-1}
\multicolumn{1}{|c|}{\reglabel{\ \ \ \ f9\ \ \ \ \ }}       \\ \cline{1-1}
\multicolumn{1}{|c|}{\reglabel{\ \ \ f10\ \ \ \ \ }}        \\ \cline{1-1}
\multicolumn{1}{|c|}{\reglabel{\ \ \ f11\ \ \ \ \ }}        \\ \cline{1-1}
\multicolumn{1}{|c|}{\reglabel{\ \ \ f12\ \ \ \ \ }}        \\ \cline{1-1}
\multicolumn{1}{|c|}{\reglabel{\ \ \ f13\ \ \ \ \ }}        \\ \cline{1-1}
\multicolumn{1}{|c|}{\reglabel{\ \ \ f14\ \ \ \ \ }}        \\ \cline{1-1}
\multicolumn{1}{|c|}{\reglabel{\ \ \ f15\ \ \ \ \ }}        \\ \cline{1-1}
\multicolumn{1}{|c|}{\reglabel{\ \ \ f16\ \ \ \ \ }}        \\ \cline{1-1}
\multicolumn{1}{|c|}{\reglabel{\ \ \ f17\ \ \ \ \ }}        \\ \cline{1-1}
\multicolumn{1}{|c|}{\reglabel{\ \ \ f18\ \ \ \ \ }}        \\ \cline{1-1}
\multicolumn{1}{|c|}{\reglabel{\ \ \ f19\ \ \ \ \ }}        \\ \cline{1-1}
\multicolumn{1}{|c|}{\reglabel{\ \ \ f20\ \ \ \ \ }}        \\ \cline{1-1}
\multicolumn{1}{|c|}{\reglabel{\ \ \ f21\ \ \ \ \ }}        \\ \cline{1-1}
\multicolumn{1}{|c|}{\reglabel{\ \ \ f22\ \ \ \ \ }}        \\ \cline{1-1}
\multicolumn{1}{|c|}{\reglabel{\ \ \ f23\ \ \ \ \ }}        \\ \cline{1-1}
\multicolumn{1}{|c|}{\reglabel{\ \ \ f24\ \ \ \ \ }}        \\ \cline{1-1}
\multicolumn{1}{|c|}{\reglabel{\ \ \ f25\ \ \ \ \ }}        \\ \cline{1-1}
\multicolumn{1}{|c|}{\reglabel{\ \ \ f26\ \ \ \ \ }}        \\ \cline{1-1}
\multicolumn{1}{|c|}{\reglabel{\ \ \ f27\ \ \ \ \ }}        \\ \cline{1-1}
\multicolumn{1}{|c|}{\reglabel{\ \ \ f28\ \ \ \ \ }}        \\ \cline{1-1}
\multicolumn{1}{|c|}{\reglabel{\ \ \ f29\ \ \ \ \ }}        \\ \cline{1-1}
\multicolumn{1}{|c|}{\reglabel{\ \ \ f30\ \ \ \ \ }}        \\ \cline{1-1}
\multicolumn{1}{|c|}{\reglabel{\ \ \ f31\ \ \ \ \ }}        \\ \cline{1-1}
\multicolumn{1}{c}{FLEN}                                    \\

\instbitrange{31}{0}                                        \\ \cline{1-1}
\multicolumn{1}{|c|}{\reglabel{fcsr}}                       \\ \cline{1-1}
\multicolumn{1}{c}{32}                                      \\
\end{tabular}
\end{center}
}
\caption{RISC-V标准F扩展单精度浮点状态。
% RISC-V standard F extension single-precision floating-point state.
}
\label{fprs}
\end{figure}

\begin{commentary}
  为了简化软件寄存器分配和调用约定,并减少用户状态总数,
  我们考虑过为整数值和浮点值使用统一的寄存器文件。
  然而,分离的组织增加了在给定指令宽度时可访问的寄存器的总数,
  简化了为宽超标量问题进行足够regfile端口的提供,
  支持解耦的浮点单元架构,并简化了内部浮点编码技术的使用。
  编译器对分离寄存器文件架构的支持和调用约定是很好理解的,
  而且在浮点寄存器状态上使用脏位可以减少上下文切换的开销。
% We considered a unified register file for both integer and
% floating-point values as this simplifies software register allocation
% and calling conventions, and reduces total user state.  However, a
% split organization increases the total number of registers accessible
% with a given instruction width, simplifies provision of enough regfile
% ports for wide superscalar issue, supports decoupled
% floating-point-unit architectures, and simplifies use of internal
% floating-point encoding techniques.  Compiler support and calling
% conventions for split register file architectures are well understood,
% and using dirty bits on floating-point register file state can reduce
% context-switch overhead.
\end{commentary}

\clearpage

\section{浮点控制和状态寄存器}
% \section{Floating-Point Control and Status Register}

浮点控制和状态寄存器,{\tt fcsr},是一个RISC-V控制和状态寄存器(CSR)。
它是一个32位的读/写寄存器,为浮点算数操作选择动态的舍入模式,并持有累积的异常标志,如图~\ref{fcsr}中显示的那样。
% The floating-point control and status register, {\tt fcsr}, is a RISC-V
% control and status register (CSR).  It is a 32-bit read/write register that
% selects the dynamic rounding mode for floating-point arithmetic operations and
% holds the accrued exception flags, as shown in Figure~\ref{fcsr}.

\begin{figure*}[h]
{\footnotesize
\begin{center}
\begin{tabular}{K@{}E@{}ccccc}
\instbitrange{31}{8} &
\instbitrange{7}{5} &
\instbit{4} &
\instbit{3} &
\instbit{2} &
\instbit{1} &
\instbit{0} \\
\hline
\multicolumn{1}{|c|}{{\em 保留}} &
\multicolumn{1}{c|}{舍入模式({\tt frm})} &
\multicolumn{5}{c|}{累积异常({\tt fflags})} \\
\hline
\multicolumn{1}{c}{} &
\multicolumn{1}{c|}{} &
\multicolumn{1}{c|}{NV} &
\multicolumn{1}{c|}{DZ} &
\multicolumn{1}{c|}{OF} &
\multicolumn{1}{c|}{UF} &
\multicolumn{1}{c|}{NX} \\
\cline{3-7}
24 & 3 & 1 & 1 & 1 & 1 & 1 \\
\end{tabular}
\end{center}
}
\vspace{-0.1in}
\caption{浮点控制和状态寄存器。
  % Floating-point control and status register.
  }
\label{fcsr}
\end{figure*}

{\tt fcsr}寄存器可以使用FRCSR和FSCSR指令来读取和写入,它们是汇编器伪指令,
构建在底层CSR访问指令上。FRCSR通过把{\tt fcsr}复制进整数寄存器rd来读取{\tt fcsr}。
FSCSR通过把{\tt fcsr}中的原始值复制进整数寄存器{\em rd},然后把从整数寄存器{\em rs1}获得的新值写入{\tt fcsr},
来交换{\tt fcsr}中的值。
% The {\tt fcsr} register can be read and written with the FRCSR and
% FSCSR instructions, which are assembler pseudoinstructions built on the
% underlying CSR access instructions.  FRCSR reads {\tt fcsr} by copying
% it into integer register {\em rd}.  FSCSR swaps the value in {\tt
%   fcsr} by copying the original value into integer register {\em rd},
% and then writing a new value obtained from integer register {\em rs1}
% into {\tt fcsr}.

{\tt fcsr}中的域可以通过不同的CSR地址来独立地访问,并且为这些访问定义了独立的汇编器伪指令。
FRRM指令读取舍入模式域{\tt frm},并把它复制进整数寄存器rd的三个最低有效位,并把所有其它位填零。
FSRM通过把{\tt frm}域中的值复制进整数寄存器{\em rd},然后把从整数寄存器{\em rs1}的三个最低有效位中获得的新值写入{\tt frm},
来交换{\tt frm}中的值。对于加速异常标志域{\tt fflags}也类似地定义了FRFLAGS和FSFLAGS。
% The fields within the {\tt fcsr} can also be accessed individually
% through different CSR addresses, and separate assembler pseudoinstructions are
% defined for these accesses.  The FRRM instruction reads the Rounding
% Mode field {\tt frm} and copies it into the least-significant three
% bits of integer register {\em rd}, with zero in all other bits.  FSRM
% swaps the value in {\tt frm} by copying the original value into
% integer register {\em rd}, and then writing a new value obtained from
% the three least-significant bits of integer register {\em rs1} into
% {\tt frm}.  FRFLAGS and FSFLAGS are defined analogously for the
% Accrued Exception Flags field {\tt fflags}.

{\tt fcsr}的位31 - 8被保留用于其它标准扩展。
如果这些扩展尚未存在,那么实现应当忽略对这些位的写入,并在读取的时候提供零值。标准软件应当保留这些位的内容。
% Bits 31--8 of the {\tt fcsr} are reserved for other standard extensions. If
% these extensions are not present, implementations shall ignore writes to
% these bits and supply a zero value when read.  Standard software should
% preserve the contents of these bits.

浮点操作或者使用编码在指令中的静态舍入模式,或者使用{\tt frm}中持有的动态舍入模式。
表~\ref{rm}中显示了舍入模式的编码。指令的{\em rm}域中的111值选择了{\tt frm}中持有的动态舍入模式。
当使用保留的舍入模式执行时,依赖于舍入模式的浮点指令的行为是保留的,
包括静态保留舍入模式(101 - 110)和动态保留舍入模式(101 - 111)。
某些指令,包括加宽的转换,具有{\em rm}域,但是在数学上并不会被舍入模式影响;
软件应把它们的{\em rm}域设置为RNE(000),但是实现必须把{\em rm}域如常对待(特别地,在涉及解码合法vs保留编码时)。
% Floating-point operations use either a static rounding mode encoded in
% the instruction, or a dynamic rounding mode held in {\tt frm}.
% Rounding modes are encoded as shown in Table~\ref{rm}.  A value of 111
% in the instruction's {\em rm} field selects the dynamic rounding mode
% held in {\tt frm}.  The behavior of floating-point instructions that
% depend on rounding mode when executed with a reserved rounding mode is
% {\em reserved}, including both static reserved rounding modes (101--110) and
% dynamic reserved rounding modes (101--111).  Some instructions, including
% widening conversions, have the {\em rm} field but are nevertheless
% mathematically unaffected by the rounding mode; software should set their
% {\em rm} field to RNE (000) but implementations must treat the {\em rm}
% field as usual (in particular, with regard to decoding legal vs. reserved
% encodings).

\begin{table}[htp]
\begin{small}
\begin{center}
\begin{tabular}{ccl}
\hline
\multicolumn{1}{|c|}{舍入模式} &
\multicolumn{1}{c|}{助记符} &
\multicolumn{1}{c|}{含义} \\
\hline
\multicolumn{1}{|c|}{000} &
\multicolumn{1}{l|}{RNE} &
\multicolumn{1}{l|}{就近舍入,关联到偶数  \iffalse Round to Nearest, ties to Even \fi }\\
\hline
\multicolumn{1}{|c|}{001} &
\multicolumn{1}{l|}{RTZ} &
\multicolumn{1}{l|}{向零舍入  \iffalse Round towards Zero \fi }\\
\hline
\multicolumn{1}{|c|}{010} &
\multicolumn{1}{l|}{RDN} &
\multicolumn{1}{l|}{向下舍入(向$-\infty$) \iffalse Round Down (towards $-\infty$)\fi  }  \\
\hline
\multicolumn{1}{|c|}{011} &
\multicolumn{1}{l|}{RUP} &
\multicolumn{1}{l|}{向上舍入(向$+\infty$) \iffalse Round Up (towards $+\infty$)\fi }  \\
\hline
\multicolumn{1}{|c|}{100} &
\multicolumn{1}{l|}{RMM} &
\multicolumn{1}{l|}{就近舍入,关联到最大幅度  \iffalse Round to Nearest, ties to Max Magnitude\fi }   \\
\hline
\multicolumn{1}{|c|}{101} &
\multicolumn{1}{l|}{} &
\multicolumn{1}{l|}{\em 保留供未来使用。\iffalse Reserved for future use.\fi }  \\
\hline
\multicolumn{1}{|c|}{110} &
\multicolumn{1}{l|}{} &
\multicolumn{1}{l|}{\em 保留供未来使用。\iffalse Reserved for future use.\fi }  \\
\hline
\multicolumn{1}{|c|}{111} &
\multicolumn{1}{l|}{DYN} &
\multicolumn{1}{l|}{在指令的{\em rm}域中,选择动态舍入模式;  \iffalse In instruction's {\em rm} field, selects dynamic rounding mode;\fi }   \\
\multicolumn{1}{|c|}{} &
\multicolumn{1}{l|}{} &
\multicolumn{1}{l|}{在舍入模式寄存器中,{\em 保留}。 \iffalse In Rounding Mode register, {\em reserved}.\fi }    \\
\hline
\end{tabular}
\end{center}
\end{small}
\caption{舍入模式编码。}
\label{rm}
\end{table}

\begin{commentary}
  C99语言标准有效地约束了动态舍入模式寄存器的提供。在典型的实现中,写动态舍入模式CSR状态将把管道序列化。
  静态舍入模式被用于实现专门的算数操作,它们经常不得不在不同的舍入模式之间频繁切换。
% The C99 language standard effectively mandates the provision of a
% dynamic rounding mode register.  In typical implementations, writes to
% the dynamic rounding mode CSR state will serialize the pipeline.
% Static rounding modes are used to implement specialized arithmetic
% operations that often have to switch frequently between different
% rounding modes.

F规范的已批准版本强制规定,当一条指令以保留的动态舍入模式执行时,会引发一个非法指令异常。
这已经被弱化位保留的,与静态舍入模式指令的行为相匹配。当遇到保留的编码时,引发一个非法指令异常仍然是有效的行为,
所以与已批准的规范兼容的实现也是与弱化的规范相兼容的。
% The ratified version of the F spec mandated that an illegal
% instruction exception was raised when an instruction was executed with
% a reserved dynamic rounding mode.  This has been weakened to reserved,
% which matches the behavior of static rounding-mode instructions.
% Raising an illegal instruction exception is still valid behavior when
% encountering a reserved encoding, so implementations compatible with
% the ratified spec are compatible with the weakened spec.
\end{commentary}
 
累积异常标志表明了,自从软件上一次重置该域以来,在任何浮点算数指令上已经发生的异常情况,
如表~\ref{bitdef}中显示的那样。基础RISC-V ISA不支持在浮点异常标志的设置时生成陷入。
% The accrued exception flags indicate the exception conditions that
% have arisen on any floating-point arithmetic instruction since the
% field was last reset by software, as shown in Table~\ref{bitdef}.
% The base RISC-V ISA
% does not support generating a trap on the setting of a floating-point
% exception flag.

\begin{table}[htp]
\begin{small}
\begin{center}
\begin{tabular}{cl}
\hline
\multicolumn{1}{|c|}{标志助记符} &
\multicolumn{1}{c|}{标志含义} \\
\hline
\multicolumn{1}{|c|}{NV} &
\multicolumn{1}{c|}{无效的操作 \iffalse Invalid Operation \fi }\\
\hline
\multicolumn{1}{|c|}{DZ} &
\multicolumn{1}{c|}{除数为零 \iffalse Divide by Zero \fi }\\
\hline
\multicolumn{1}{|c|}{OF} &
\multicolumn{1}{c|}{溢出 \iffalse Overflow \fi }\\
\hline
\multicolumn{1}{|c|}{UF} &
\multicolumn{1}{c|}{向下溢出 \iffalse Underflow \fi}\\
\hline
\multicolumn{1}{|c|}{NX} &
\multicolumn{1}{c|}{不精确的 \iffalse Inexact \fi }\\
\hline
\end{tabular}
\end{center}
\end{small}
\caption{累积异常标志编码。
% Accrued exception flag encoding.
}
\label{bitdef}
\end{table}

\begin{commentary}
  正如标准所允许的那样,我们不支持在F扩展中的浮点异常上的陷入,但是需要显式地检查软件中的标志。
  我们考虑过添加直接通过浮点累积异常标志的内容来控制的分支,但是最终选择了忽略这些指令以保持ISA的简单。
% As allowed by the standard, we do not support traps on floating-point
% exceptions in the F extension, but instead require explicit checks of the flags
% in software.  We considered adding branches controlled directly by the
% contents of the floating-point accrued exception flags, but ultimately chose
% to omit these instructions to keep the ISA simple.
\end{commentary}

\section{NaN的生成和传播}
% \section{NaN Generation and Propagation}

除非另有说明,如果浮点操作的结果是NaN,那么它是规范的NaN。规范的NaN具有一个正号,
并且除了MSB(或者说,沉默位)以外的所有有效位都被清除。对于单精度浮点,这对应于式样{\tt 0x7fc00000}。
% Except when otherwise stated, if the result of a floating-point operation is
% NaN, it is the canonical NaN.  The canonical NaN has a positive sign and all
% significand bits clear except the MSB, a.k.a. the quiet bit.  For
% single-precision floating-point, this corresponds to the pattern {\tt
% 0x7fc00000}.

\begin{commentary}
  我们考虑过传播NaN的有效载荷,就像标准推荐的那样,但是这个决定将增加硬件开销。
  并且,由于这个特征在标准中是可选的,它不能被用于可移植的代码。
% We considered propagating NaN payloads, as is recommended by the standard,
% but this decision would have increased hardware cost.  Moreover, since this
% feature is optional in the standard, it cannot be used in portable code.

实现者可以自由地提供一个NaN有效载荷传播策略,作为被非标准操作模式启用的非标准扩展。
然而,上面描述的规范的NaN策略必须总是被支持的,并且应当成为默认模式
% Implementors are free to provide a NaN payload propagation scheme as
% a non-standard extension enabled by a non-standard operating mode.  However, the
% canonical NaN scheme described above must always be supported and should be
% the default mode.
\end{commentary}

\begin{commentary}
  在异常情况中,我们需要实现来返回标准所要求的默认值,就用户级软件而言无需进一步干预(不像Alpha ISA浮点陷入屏障那样)。
  我们相信异常情况的全硬件处理将变得更加常见,并且因此希望避免让用户级ISA复杂化,以优化其它的方法。
  实现可以总是陷入到机器模式软件处理程序来提供异常的默认值。
% We require implementations to return the standard-mandated default
% values in the case of exceptional conditions, without any further
% intervention on the part of user-level software (unlike the Alpha ISA
% floating-point trap barriers).  We believe full hardware handling of
% exceptional cases will become more common, and so wish to avoid
% complicating the user-level ISA to optimize other approaches.
% Implementations can always trap to machine-mode software handlers to
% provide exceptional default values.
\end{commentary}

\section{亚正常算法}
% \section{Subnormal Arithmetic}

关于亚正常数的操作按照IEEE754-2008标准处理。
% Operations on subnormal numbers are handled in accordance with the IEEE
% 754-2008 standard.
在IEEE标准的说法中,极小是在舍入之后检测的。
% In the parlance of the IEEE standard, tininess is detected after rounding.

\begin{commentary}
  在舍入之后检测极小导致了更少的貌似的向下溢出信号。
% Detecting tininess after rounding results in fewer spurious underflow signals.
\end{commentary}

\section{单精度加载和存储指令}
% \section{Single-Precision Load and Store Instructions}

浮点加载和存储使用相同的“基址+偏移量”编址模式,就像整数基础ISA那样,一个寄存器{\em rs1}中的基地址与一个12位的有符号字节偏移量。
FLW指令从内存加载一个单精度浮点值,并把它放入浮点寄存器{\em rd}。FSW把浮点寄存器{\em rs2}中的一个单精度值存储到内存。
% Floating-point loads and stores use the same base+offset addressing
% mode as the integer base ISAs, with a base address in register {\em
%   rs1} and a 12-bit signed byte offset.  The FLW instruction loads a
% single-precision floating-point value from memory into floating-point
% register {\em rd}.  FSW stores a single-precision value from
% floating-point register {\em rs2} to memory.

\vspace{-0.2in}
\begin{center}
\begin{tabular}{M@{}R@{}F@{}R@{}O}
\\
\instbitrange{31}{20} &
\instbitrange{19}{15} &
\instbitrange{14}{12} &
\instbitrange{11}{7} &
\instbitrange{6}{0} \\
\hline
\multicolumn{1}{|c|}{imm[11:0]} &
\multicolumn{1}{c|}{rs1} &
\multicolumn{1}{c|}{width} &
\multicolumn{1}{c|}{rd} &
\multicolumn{1}{c|}{opcode} \\
\hline
12 & 5 & 3 & 5 & 7 \\
offset[11:0] & base & W & dest & LOAD-FP \\
\end{tabular}
\end{center}

\vspace{-0.2in}
\begin{center}
\begin{tabular}{O@{}R@{}R@{}F@{}R@{}O}
\\
\instbitrange{31}{25} &
\instbitrange{24}{20} &
\instbitrange{19}{15} &
\instbitrange{14}{12} &
\instbitrange{11}{7} &
\instbitrange{6}{0} \\
\hline
\multicolumn{1}{|c|}{imm[11:5]} &
\multicolumn{1}{c|}{rs2} &
\multicolumn{1}{c|}{rs1} &
\multicolumn{1}{c|}{width} &
\multicolumn{1}{c|}{imm[4:0]} &
\multicolumn{1}{c|}{opcode} \\
\hline
7 & 5 & 5 & 3 & 5 & 7 \\
offset[11:5] & src & base & W & offset[4:0] & STORE-FP \\
\end{tabular}
\end{center}

只有在有效地址自然对齐的时候,才保证FLW和FSW的原子性执行
% FLW and FSW are only guaranteed to execute atomically if the effective address
% is naturally aligned.

FLW和FSW不修改正在被传递的位;特别地,非规范的NaN的有效载荷被保留。
% FLW and FSW do not modify the bits being transferred; in particular, the
% payloads of non-canonical NaNs are preserved.

正如第~\ref{sec:rv32:ldst}节描述的那样,EEI定义了未对齐的浮点加载和存储是被隐式地处理,还是引发一个包含的或致命的陷入。
% As described in Section~\ref{sec:rv32:ldst}, the EEI defines whether
% misaligned floating-point loads and stores are handled invisibly or raise
% a contained or fatal trap.

\section{单精度浮点运算指令}
% \section{Single-Precision Floating-Point Computational Instructions}
\label{sec:single-float-compute}

带有一个或两个源操作数的浮点算数指令使用带有OP-FP主操作码的R类型格式。
FADD.S和FMUL.S分别在{\em rs1}和{\em rs2}之间执行单精度浮点加法和乘法。
FSUB.S执行从{\em rs1}中减去{\em rs2}的单精度浮点减法。FDIV.S执行{\em rs1}除以{\em rs2}的单精度浮点除法。
FSQRT.S计算{\em rs1}的平方根。在每种情况中,结果都被写入{\em rd}。
% Floating-point arithmetic instructions with one or two source operands use the
% R-type format with the OP-FP major opcode.  FADD.S and FMUL.S perform
% single-precision floating-point addition and multiplication respectively,
% between {\em rs1} and {\em rs2}. FSUB.S performs the single-precision
% floating-point subtraction of {\em rs2} from {\em rs1}.  FDIV.S performs the
% single-precision floating-point division of {\em rs1} by {\em rs2}. FSQRT.S
% computes the square root of {\em rs1}.  In each case, the result is written to
% {\em rd}.

2位浮点格式域{\em fmt}按照表~\ref{tab:fmt}中显示的那样编码。对于F扩展中的所有指令,它都被设置为{\em S}(00)。
% The 2-bit floating-point format field {\em fmt} is encoded as shown in
% Table~\ref{tab:fmt}.  It is set to {\em S} (00) for all instructions in
% the F extension.

\begin{table}[htp]
\begin{small}
\begin{center}
\begin{tabular}{|c|c|l|}
\hline
{\em fmt} 域 &
Mnemonic &
Meaning \\
\hline
00 & S & 32位单精度\\
01 & D & 64位双精度\\
10 & H & 16位半精度\\
11 & Q & 128位四精度\\
\hline
\end{tabular}
\end{center}
\end{small}
\caption{格式域编码。
%  Format field encoding.
}
\label{tab:fmt}
\end{table}

所有执行舍入的浮点操作都可以使用{\em rm}域来选择舍入模式,{\em rm}域的编码显示在表~\ref{rm}中。
% All floating-point operations that perform rounding can select the
% rounding mode using the {\em rm} field with the encoding shown in
% Table~\ref{rm}.

浮点最小数和最大数指令FMIN.S和FMAX.S分别把{\em rs1}和{\em rs2}中的较小者或较大者写到{\em rd}。
仅对于这些指令的目的而言,值$-0.0$被认为小于值$+0.0$。如果两个输入都是NaN,结果是规范的NaN。
如果只有一个操作数是NaN,结果是那个非NaN的操作数。发信号的NaN输入会设置无效操作异常标志,即使当结果不是NaN时也是如此。
% Floating-point minimum-number and maximum-number instructions FMIN.S and
% FMAX.S write, respectively, the smaller or larger of {\em rs1} and {\em rs2}
% to {\em rd}.  For the purposes of these instructions only, the value $-0.0$ is
% considered to be less than the value $+0.0$.  If both inputs are NaNs, the
% result is the canonical NaN.  If only one operand is a NaN, the result is the
% non-NaN operand.  Signaling NaN inputs set the invalid operation exception flag,
% even when the result is not NaN.

\begin{commentary}
  注意,在F扩展的2.2版本中,FMIN.S和FMAX.S指令被修正为实现所提出的IEEE 754-201x的mininumNumber和maximumNumber操作,
  而不是IEEE 754-2008的minNum和maxNum操作。这些操作的区别在于它们对发信号的NaN的处理。
% Note that in version 2.2 of the F extension, the FMIN.S and FMAX.S
% instructions were amended to implement the proposed IEEE 754-201x
% minimumNumber and maximumNumber operations, rather than the IEEE 754-2008
% minNum and maxNum operations.  These operations differ in their handling of
% signaling NaNs.
\end{commentary}

\vspace{-0.2in}
\begin{center}
\begin{tabular}{R@{}F@{}R@{}R@{}F@{}R@{}O}
\\
\instbitrange{31}{27} &
\instbitrange{26}{25} &
\instbitrange{24}{20} &
\instbitrange{19}{15} &
\instbitrange{14}{12} &
\instbitrange{11}{7} &
\instbitrange{6}{0} \\
\hline
\multicolumn{1}{|c|}{funct5} &
\multicolumn{1}{c|}{fmt} &
\multicolumn{1}{c|}{rs2} &
\multicolumn{1}{c|}{rs1} &
\multicolumn{1}{c|}{rm} &
\multicolumn{1}{c|}{rd} &
\multicolumn{1}{c|}{opcode} \\
\hline
5 & 2 & 5 & 5 & 3 & 5 & 7 \\
FADD/FSUB & S & src2 & src1 & RM  & dest & OP-FP  \\
FMUL/FDIV & S & src2 & src1 & RM  & dest & OP-FP  \\
FSQRT     & S & 0    & src  & RM  & dest & OP-FP  \\
FMIN-MAX  & S & src2 & src1 & MIN/MAX & dest & OP-FP  \\
\end{tabular}
\end{center}

浮点融合乘加指令需要一个新的标志指令格式。R4类型指令指定三个源寄存器({\em rs1}、{\em rs2}和{\em rs3})和一个目的寄存器({\em rd})。
这个格式只被浮点融合乘加指令使用。
% Floating-point fused multiply-add instructions require a new standard
% instruction format.  R4-type instructions specify three source
% registers ({\em rs1}, {\em rs2}, and {\em rs3}) and a destination
% register ({\em rd}).  This format is only used by the floating-point
% fused multiply-add instructions.

FMADD.S将{\em rs1}和{\em rs2}中的值相乘,加上{\em rs3}中的值,并把最终结果写到{\em rd}。FMADD.S计算{\em (rs1$\times$rs2)+rs3}。
% FMADD.S multiplies the values in {\em
% rs1} and {\em rs2}, adds the value in {\em rs3}, and writes the final
% result to {\em rd}.  FMADD.S computes {\em (rs1$\times$rs2)+rs3}.

FMSUB.S将{\em rs1}和{\em rs2}中的值相乘,减去{\em rs3}中的值,并把最终结果写到{\em rd}。FMSUB.S计算{\em (rs1$\times$rs2)-rs3}。
% FMSUB.S multiplies the values in {\em rs1} and {\em rs2}, subtracts
% the value in {\em rs3}, and writes the final result to {\em rd}.
% FMSUB.S computes {\em (rs1$\times$rs2)-rs3}.

FNMSUB.S将{\em rs1}和{\em rs2}中的值相乘,取乘积的相反数,加上{\em rs3}中的值,并把最终结果写到{\em rd}。FNMSUB.S计算{\em -(rs1$\times$rs2)+rs3}。
% FNMSUB.S multiplies the
% values in {\em rs1} and {\em rs2}, negates the product, adds the value
% in {\em rs3}, and writes the final result to {\em rd}. FNMSUB.S
% computes {\em -(rs1$\times$rs2)+rs3}.

FNMADD.S将{\em rs1}和{\em rs2}中的值相乘,取乘积的相反数,减去{\em rs2}中的值,并把最终结果写到{\em rd}。FNMADD.S计算{\em -(rs1$\times$rs2)-rs3}。
% FNMADD.S multiplies the values
% in {\em rs1} and {\em rs2}, negates the product, subtracts the value
% in {\em rs3}, and writes the final result to {\em rd}. FNMADD.S
% computes {\em -(rs1$\times$rs2)-rs3}.

\begin{commentary}
  FNMSUB和FNMADD指令的命名是反直觉的,是由于MIPS-IV中对应指令的命名。
  MIPS指令被定义为对总和的取负,而不像RISC-V的指令做的那样只对乘积取负,所以当时的命名策略更合理。
  这两个定义对于有符号的零的结果是有区别的。RISC-V的定义符合x86和ARM融合乘加指令的行为,但与x86和ARM相比,
  RISC-V FNMSUB和FNMADD指令的名字被不幸地交换了。
% The FNMSUB and FNMADD instructions are counterintuitively named, owing to the
% naming of the corresponding instructions in MIPS-IV.  The MIPS instructions
% were defined to negate the sum, rather than negating the product as the
% RISC-V instructions do, so the naming scheme was more rational at the time.
% The two definitions differ with respect to signed-zero results.  The RISC-V
% definition matches the behavior of the x86 and ARM fused multiply-add
% instructions, but unfortunately the RISC-V FNMSUB and FNMADD instruction
% names are swapped compared to x86 and ARM.
\end{commentary}

\vspace{-0.2in}
\begin{center}
\begin{tabular}{R@{}F@{}R@{}R@{}F@{}R@{}O}
\\
\instbitrange{31}{27} &
\instbitrange{26}{25} &
\instbitrange{24}{20} &
\instbitrange{19}{15} &
\instbitrange{14}{12} &
\instbitrange{11}{7} &
\instbitrange{6}{0} \\
\hline
\multicolumn{1}{|c|}{rs3} &
\multicolumn{1}{c|}{fmt} &
\multicolumn{1}{c|}{rs2} &
\multicolumn{1}{c|}{rs1} &
\multicolumn{1}{c|}{rm} &
\multicolumn{1}{c|}{rd} &
\multicolumn{1}{c|}{opcode} \\
\hline
5 & 2 & 5 & 5 & 3 & 5 & 7 \\
src3 & S & src2 & src1 & RM  & dest & F[N]MADD/F[N]MSUB  \\
\end{tabular}
\end{center}

\begin{commentary}
  融合乘加(FMA)指令会消耗32位指令编码空间的一大部分。
  考虑过某些替代方案来限制FMA只使用动态舍入模式,但是静态舍入模式在利用了缺少乘积舍入的代码中是有用的。
  另一个备选方案将使用rd来提供rs3,但是这在一些常见的序列中将需要额外的移动指令。
  当前的设计仍然使32位编码空间的大部分保持开放,同时避免让FMA是非正交的。
%  The fused multiply-add (FMA) instructions consume a large part of the
%  32-bit instruction encoding space.  Some alternatives considered were
%  to restrict FMA to only use dynamic rounding modes, but static
%  rounding modes are useful in code that exploits the lack of product
%  rounding.  Another alternative would have been to use rd to provide
%  rs3, but this would require additional move instructions in some
%  common sequences.  The current design still leaves a large portion of
%  the 32-bit encoding space open while avoiding having FMA be
%  non-orthogonal.
\end{commentary}

当被乘数是$\infty$和零时,融合乘加指令必须设置无效操作异常标志,即使加数是静默的NaN时也需如此。
% The fused multiply-add instructions must set the invalid operation exception flag
% when the multiplicands are $\infty$ and zero, even when the addend is a quiet
% NaN.
\begin{commentary}
  IEEE 754-2008标准允许(但是不必须)为\mbox{$\infty\times 0\ +$ qNaN}操作产生无效异常。
% The IEEE 754-2008 standard permits, but does not require, raising the
% invalid exception for the operation \mbox{$\infty\times 0\ +$ qNaN}.
\end{commentary}

\section{单精度浮点转换和移动\mbox{指令}}
% \section{Single-Precision Floating-Point Conversion and Move \mbox{Instructions}}

浮点到整数转换指令和整数到浮点转换指令被编码在OP-FP主操作码空间中。
FCVT.W.S或FCVT.L.S将一个浮点寄存器{\em rs1}中的浮点数分别转化为一个有符号的32位或64位整数,并将其放入整数寄存器{\em rd}中。
FCVT.S.W或FCVT.S.L分别把整数寄存器{\em rs1}中的一个32位或64位有符号整数转化为一个浮点数,并把它放入浮点寄存器{\em rd}中。
FCVT.WU.S、FCVT.LU.S、FCVT.S.WU和FCVT.S.LU的变体转化或转换为无符号整数值。
对于大于32的XLEN,FCVT.W[U].S把32位结果符号扩展到目的寄存器的宽度。FCVT.L[U].S和FCVT.S.L[U]是RV64独有的指令。
如果舍入的结果不能以目的格式表示,它将被裁剪为最接近的值,并且设置无效标志。
表~\ref{tab:int_conv}给出了FCVT.{\em int}.S的有效输入的范围和无效输入的行为。
% Floating-point-to-integer and integer-to-floating-point conversion
% instructions are encoded in the OP-FP major opcode space.
% FCVT.W.S or FCVT.L.S converts a floating-point number
% in floating-point register {\em rs1} to a signed 32-bit or 64-bit
% integer, respectively, in integer register {\em rd}.  FCVT.S.W
% or FCVT.S.L converts a 32-bit or 64-bit signed integer,
% respectively, in integer register {\em rs1} into a floating-point
% number in floating-point register {\em rd}. FCVT.WU.S,
% FCVT.LU.S, FCVT.S.WU, and FCVT.S.LU variants
% convert to or from unsigned integer values.
% For XLEN$>32$, FCVT.W[U].S sign-extends the 32-bit result to the
% destination register width.
% FCVT.L[U].S and FCVT.S.L[U] are RV64-only instructions.
% If the rounded result is not representable in the destination format,
% it is clipped to the nearest value and the invalid flag is set.
% Table~\ref{tab:int_conv} gives the range of valid inputs for FCVT.{\em int}.S
% and the behavior for invalid inputs.

\begin{table}[htp]
\begin{small}
\begin{center}
\begin{tabular}{|l|r|r|r|r|}
\hline
 & FCVT.W.S & FCVT.WU.S & FCVT.L.S & FCVT.LU.S \\
\hline
最小有效输入(舍入后) & $-2^{31}$ & 0 & $-2^{63}$ & 0 \\
最大有效输入(舍入后)& $2^{31}-1$ & $2^{32}-1$ & $2^{63}-1$ & $2^{64}-1$ \\
\hline
对于超出范围的负数输入的输出 & $-2^{31}$ & 0 & $-2^{63}$ & 0 \\
对于$-\infty$的输出& $-2^{31}$ & 0 & $-2^{63}$ & 0 \\
对于超出范围的正数输入的输出 & $2^{31}-1$ & $2^{32}-1$ & $2^{63}-1$ & $2^{64}-1$ \\
对于$+\infty$或NaN的输出 & $2^{31}-1$ & $2^{32}-1$ & $2^{63}-1$ & $2^{64}-1$ \\
\hline
\end{tabular}
\end{center}
\end{small}
\caption{浮点到整数转换的域和对于无效输入的行为。 
% Domains of float-to-integer conversions and behavior for invalid inputs.
}
\label{tab:int_conv}
\end{table}

所有的浮点到整数转换指令和整数到浮点转换指令都根据{\em rm}域进行舍入。
浮点寄存器可以使用FCVT.S.W {\em rd}, {\tt x0}被初始化为浮点正零,它将永远不会设置任何异常标志。
% All floating-point to integer and integer to floating-point conversion
% instructions round according to the {\em rm} field.  A floating-point register
% can be initialized to floating-point positive zero using FCVT.S.W {\em rd},
% {\tt x0}, which will never set any exception flags.

如果舍入结果与操作数的值不同,并且没有设置无效异常标志,那么所有的浮点转换指令都会设置不精确异常标志。
% All floating-point conversion instructions set the Inexact exception flag if
% the rounded result differs from the operand value and the Invalid exception
% flag is not set.

\vspace{-0.2in}
\begin{center}
\begin{tabular}{R@{}F@{}R@{}R@{}F@{}R@{}O}
\\
\instbitrange{31}{27} &
\instbitrange{26}{25} &
\instbitrange{24}{20} &
\instbitrange{19}{15} &
\instbitrange{14}{12} &
\instbitrange{11}{7} &
\instbitrange{6}{0} \\
\hline
\multicolumn{1}{|c|}{funct5} &
\multicolumn{1}{c|}{fmt} &
\multicolumn{1}{c|}{rs2} &
\multicolumn{1}{c|}{rs1} &
\multicolumn{1}{c|}{rm} &
\multicolumn{1}{c|}{rd} &
\multicolumn{1}{c|}{opcode} \\
\hline
5 & 2 & 5 & 5 & 3 & 5 & 7 \\
FCVT.{\em int}.{\em fmt} & S & W[U]/L[U] & src & RM  & dest & OP-FP  \\
FCVT.{\em fmt}.{\em int} & S & W[U]/L[U] & src & RM  & dest & OP-FP  \\
\end{tabular}
\end{center}

浮点到浮点的符号注入指令,FSGNJ.S、FSGNJN.S和FSGNJX.S,产生的结果是取{\em rs1}的除了符号位的所有位。
对于FSGNJ,结果的符号位是{\em rs2}的符号位;对于FSGNJN,结果的符号位是{\em rs2}的符号位取反;
而对于FSGNJX,该符号位是{\em rs1}和{\em rs2}的符号位的XOR结果。符号注入指令既不设置浮点异常标志,它们也不会将NaN规范化。
注意,FSGNJ.S {\em rx, ry, ry}把{\em ry}移动到{\em rx}(汇编器伪指令FMV.S {\em rx, ry});
FSGNJN.S {\em rx, ry, ry}把{\em ry}的相反数移动到{\em rx}(汇编器伪指令FNEG.S {\em rx, ry});
而FSGNJX.S {\em rx, ry, ry}把{\em ry}的绝对值移动到{\em rx}(汇编器伪指令FABS.S {\em rx, ry})。
% Floating-point to floating-point sign-injection instructions, FSGNJ.S,
% FSGNJN.S, and FSGNJX.S, produce a result that takes all bits except
% the sign bit from {\em rs1}.  For FSGNJ, the result's sign bit is {\em
%   rs2}'s sign bit; for FSGNJN, the result's sign bit is the opposite
% of {\em rs2}'s sign bit; and for FSGNJX, the sign bit is the XOR of
% the sign bits of {\em rs1} and {\em rs2}.  Sign-injection instructions
% do not set floating-point exception flags, nor do they canonicalize
% NaNs.  Note, FSGNJ.S {\em rx, ry,
%   ry} moves {\em ry} to {\em rx} (assembler pseudoinstruction FMV.S {\em rx,
%   ry}); FSGNJN.S {\em rx, ry, ry} moves the negation of {\em ry} to
% {\em rx} (assembler pseudoinstruction FNEG.S {\em rx, ry}); and FSGNJX.S {\em rx,
%   ry, ry} moves the absolute value of {\em ry} to {\em rx} (assembler
% pseudoinstruction FABS.S {\em rx, ry}).

\vspace{-0.2in}
\begin{center}
\begin{tabular}{R@{}F@{}R@{}R@{}F@{}R@{}O}
\\
\instbitrange{31}{27} &
\instbitrange{26}{25} &
\instbitrange{24}{20} &
\instbitrange{19}{15} &
\instbitrange{14}{12} &
\instbitrange{11}{7} &
\instbitrange{6}{0} \\
\hline
\multicolumn{1}{|c|}{funct5} &
\multicolumn{1}{c|}{fmt} &
\multicolumn{1}{c|}{rs2} &
\multicolumn{1}{c|}{rs1} &
\multicolumn{1}{c|}{rm} &
\multicolumn{1}{c|}{rd} &
\multicolumn{1}{c|}{opcode} \\
\hline
5 & 2 & 5 & 5 & 3 & 5 & 7 \\
FSGNJ & S & src2 & src1 & J[N]/JX & dest & OP-FP  \\
\end{tabular}
\end{center}

\begin{commentary}
  符号注入指令提供了浮点MV、ABS和NEG,也支持了少量其它的操作,包括IEEE copySign操作和超越数学函数库中的符号操作。
  尽管MV、ABS和NEG只需要一个寄存器操作数,而FSGNJ指令需要两个,但大多数微架构将不太可能添加优化,
  来从读取这些相对不频繁的指令的寄存器数目的减少中受益。甚至在这种情况中,微架构也可以为FSGNJ指令做简单地检测,
  当两个源寄存器是相同的时,只读取一份拷贝。
% The sign-injection instructions
% provide floating-point MV, ABS, and NEG,
% as well as supporting a few other operations, including the IEEE copySign
% operation and sign manipulation in transcendental math function
% libraries.  Although MV, ABS, and NEG only need a single register
% operand, whereas FSGNJ instructions need two, it is unlikely most
% microarchitectures would add optimizations to benefit from the reduced
% number of register reads for these relatively infrequent instructions.
% Even in this case, a microarchitecture can simply detect when both
% source registers are the same for FSGNJ instructions and only read a
% single copy.
\end{commentary}

提供了在浮点寄存器和整数寄存器之间移动位式样的指令。
FMV.X.W把浮点寄存器{\em rs1}中的以IEEE 754-2008编码表示的单精度值移动到整数寄存器{\em rd}的低32位。
在转移中这些位不会被修改,并且特别地,非规范的NaN的有效载荷也被保留。
对于RV64,目的寄存器的高32位被填充为浮点数的符号位的拷贝。
% Instructions are provided to move bit patterns between the
% floating-point and integer registers.  FMV.X.W moves the
% single-precision value in floating-point register {\em rs1}
% represented in IEEE 754-2008 encoding to the lower 32 bits of integer
% register {\em rd}.  The bits are not
% modified in the transfer, and in particular, the payloads of
% non-canonical NaNs are preserved.
% For RV64, the higher 32 bits of the destination
% register are filled with copies of the floating-point number's sign
% bit.

FMV.W.X把整数寄存器{\em rs1}的低32位中的以IEEE 754-2008标准编码的单精度值移动到浮点寄存器{\em rd}。
在转移中这些位不会被修改,并且特别地,非规范的NaN的有效载荷被保留。
% FMV.W.X moves the single-precision value encoded in IEEE
% 754-2008 standard encoding from the lower 32 bits of integer register
% {\em rs1} to the floating-point register {\em rd}.  The bits are not
% modified in the transfer, and in particular, the payloads of
% non-canonical NaNs are preserved.

\begin{commentary}
  FMV.W.X和FMV.X.W指令之前被称作FMV.S.X和FMV.X.S。
  W的使用更符合它们作为单纯移动32位而不对其进行解释的指令的语义。
  这在定义了NaN装箱之后变得更加清晰。为了避免干扰现有的代码,W版本和S版本都将被工具支持。
% The FMV.W.X and FMV.X.W instructions were previously called FMV.S.X
% and FMV.X.S.  The use of W is more consistent with their semantics as
% an instruction that moves 32 bits without interpreting them.  This
% became clearer after defining NaN-boxing.  To avoid disturbing
% existing code, both the W and S versions will be supported by tools.
\end{commentary}

\vspace{-0.2in}
\begin{center}
\begin{tabular}{R@{}F@{}R@{}R@{}F@{}R@{}O}
\\
\instbitrange{31}{27} &
\instbitrange{26}{25} &
\instbitrange{24}{20} &
\instbitrange{19}{15} &
\instbitrange{14}{12} &
\instbitrange{11}{7} &
\instbitrange{6}{0} \\
\hline
\multicolumn{1}{|c|}{funct5} &
\multicolumn{1}{c|}{fmt} &
\multicolumn{1}{c|}{rs2} &
\multicolumn{1}{c|}{rs1} &
\multicolumn{1}{c|}{rm} &
\multicolumn{1}{c|}{rd} &
\multicolumn{1}{c|}{opcode} \\
\hline
5 & 2 & 5 & 5 & 3 & 5 & 7 \\
FMV.X.W & S & 0    & src  & 000  & dest & OP-FP  \\
FMV.W.X & S & 0    & src  & 000  & dest & OP-FP  \\
\end{tabular}
\end{center}

\begin{commentary}
  基础浮点ISA被如此定义,是为了允许实现在寄存器中采用浮点格式的内部重新编码,以简化对亚正常值的处理,
  并可能减少功能单元的延迟。为此,F扩展避免在浮点寄存器中,通过定义直接读写整数寄存器文件的转化和比较操作来表示整数值。
  这也去除了许多常见的需要在整数寄存器和浮点寄存器之间显式移动的情况,为常见的混合格式代码序列减少了指令计数和关键路径。
% The base floating-point ISA was defined so as to allow implementations
% to employ an internal recoding of the floating-point format in
% registers to simplify handling of subnormal values and possibly to
% reduce functional unit latency.  To this end, the F extension avoids
% representing integer values in the floating-point registers by
% defining conversion and comparison operations that read and write the
% integer register file directly.  This also removes many of the common
% cases where explicit moves between integer and floating-point
% registers are required, reducing instruction count and critical paths
% for common mixed-format code sequences.
\end{commentary}

\section{单精度浮点比较指令}
% \section{Single-Precision Floating-Point Compare Instructions}

浮点比较指令(FEQ.S、FLT.S、FLE.S)在浮点寄存器之间执行特定的比较($\mbox{\em rs1}
= \mbox{\em rs2}$, $\mbox{\em rs1} < \mbox{\em rs2}$, $\mbox{\em rs1} \leq
\mbox{\em rs2}$),并且如果条件满足,向整数寄存器{\em rd}写入1,否则写入0。
% Floating-point compare instructions (FEQ.S, FLT.S, FLE.S) perform the
% specified comparison between floating-point registers ($\mbox{\em rs1}
% = \mbox{\em rs2}$, $\mbox{\em rs1} < \mbox{\em rs2}$, $\mbox{\em rs1} \leq
% \mbox{\em rs2}$) writing 1 to the integer register {\em rd} if the condition
% holds, and 0 otherwise.

FLT.S和FLE.S执行IEEE 754-2008标准所提及的{\em 信号}比较:即,如果某个输入是NaN,它们设置无效操作异常标志。
FEQ.S实施{\em 静默}比较:它只在某个输入是发信号的NaN时设置无效操作异常标志。对于所有这三个指令,如果有操作数是NaN,那么结果就是0。
% FLT.S and FLE.S perform what the IEEE 754-2008 standard refers to as {\em
% signaling} comparisons: that is, they set the invalid operation exception flag
% if either input is NaN.  FEQ.S performs a {\em quiet} comparison: it only
% sets the invalid operation exception flag if either input is a signaling NaN.
% For all three instructions,
% the result is 0 if either operand is NaN.

\vspace{-0.2in}
\begin{center}
\begin{tabular}{S@{}F@{}R@{}R@{}F@{}R@{}O}
\\
\instbitrange{31}{27} &
\instbitrange{26}{25} &
\instbitrange{24}{20} &
\instbitrange{19}{15} &
\instbitrange{14}{12} &
\instbitrange{11}{7} &
\instbitrange{6}{0} \\
\hline
\multicolumn{1}{|c|}{funct5} &
\multicolumn{1}{c|}{fmt} &
\multicolumn{1}{c|}{rs2} &
\multicolumn{1}{c|}{rs1} &
\multicolumn{1}{c|}{rm} &
\multicolumn{1}{c|}{rd} &
\multicolumn{1}{c|}{opcode} \\
\hline
5 & 2 & 5 & 5 & 3 & 5 & 7 \\
FCMP & S & src2 & src1 & EQ/LT/LE & dest & OP-FP  \\
\end{tabular}
\end{center}

\begin{commentary}
  F扩展提供了一个$\leq$比较,而基础ISA提供了一个$\geq$分支比较。因为$\leq$可以从$\geq$中合成,反之亦然,
  所以对于这种不一致性没有性能上的影响,但是在ISA中,它仍然是一种不幸的不协调。
% The F extension provides a $\leq$ comparison, whereas the base ISAs provide
% a $\geq$ branch comparison.  Because $\leq$ can be synthesized from $\geq$ and
% vice-versa, there is no performance implication to this inconsistency, but it
% is nevertheless an unfortunate incongruity in the ISA.
\end{commentary}

\section{单精度浮点分类指令}
% \section{Single-Precision Floating-Point Classify Instruction}

FCLASS.S指令检测浮点寄存器{\em rs1}中的值,并向整数寄存器{\em rd}写入一个10位的掩码,它表示该浮点数的类别。
掩码的格式被描述在表~\ref{tab:fclass}中。如果某个属性为真,那么将设置{\em rd}中的对应位并清除其它位。{\em rd}中的所有其它位被清除。
注意在{\em rd}中将恰好只有一位会被设置。FCLASS.S不设置浮点异常标志。
% The FCLASS.S instruction examines the value in floating-point register {\em
% rs1} and writes to integer register {\em rd} a 10-bit mask that indicates
% the class of the floating-point number.  The format of the mask is
% described in Table~\ref{tab:fclass}.  The corresponding bit in {\em rd} will
% be set if the property is true and clear otherwise.  All other bits in
% {\em rd} are cleared.  Note that exactly one bit in {\em rd} will be set.
% FCLASS.S does not set the floating-point exception flags.

\vspace{-0.2in}
\begin{center}
\begin{tabular}{S@{}F@{}R@{}R@{}F@{}R@{}O}
\\
\instbitrange{31}{27} &
\instbitrange{26}{25} &
\instbitrange{24}{20} &
\instbitrange{19}{15} &
\instbitrange{14}{12} &
\instbitrange{11}{7} &
\instbitrange{6}{0} \\
\hline
\multicolumn{1}{|c|}{funct5} &
\multicolumn{1}{c|}{fmt} &
\multicolumn{1}{c|}{rs2} &
\multicolumn{1}{c|}{rs1} &
\multicolumn{1}{c|}{rm} &
\multicolumn{1}{c|}{rd} &
\multicolumn{1}{c|}{opcode} \\
\hline
5 & 2 & 5 & 5 & 3 & 5 & 7 \\
FCLASS & S & 0 & src & 001 & dest & OP-FP  \\
\end{tabular}
\end{center}

\begin{table}[htp]
\begin{small}
\begin{center}
\begin{tabular}{|c|l|}
\hline
{\em rd} bit &
Meaning \\
\hline
0 & {\em rs1} 是 $-\infty$. \\
1 & {\em rs1} 是一个负的正常的数  %is a negative normal number.
                                \\
2 & {\em rs1} 是一个负的亚正常的数 % is a negative subnormal number. 
                                  \\
3 & {\em rs1} 是 $-0$. \\
4 & {\em rs1} 是 $+0$. \\
5 & {\em rs1} 是一个负的亚正常的数  % is a positive subnormal number. 
                                  \\
6 & {\em rs1} 是一个负的正常的数  % is a positive normal number. 
                                      \\
7 & {\em rs1} 是 $+\infty$. \\
8 & {\em rs1} 是发信号的NaN  % is a signaling NaN. 
                                \\
9 & {\em rs1} 是沉默的NaN  %  is a quiet NaN. 
                                    \\
\hline
\end{tabular}
\end{center}
\end{small}
\caption{FCLASS指令的结果的格式。
% Format of result of FCLASS instruction.
}
\label{tab:fclass}
\end{table}

\chapter{用于双精度浮点的“D”标准扩展(2.2版本)}
% \chapter{``D'' Standard Extension for Double-Precision Floating-Point,
% Version 2.2}

这章描述了标准双精度浮点指令集扩展,该扩展被命名为“D”,并添加了兼容IEEE 754-2008算数标准的双精度浮点运算指令。
D扩展依赖于基础单精度指令子集F。
% This chapter describes the standard double-precision floating-point
% instruction-set extension, which is named ``D'' and adds
% double-precision floating-point computational instructions compliant
% with the IEEE 754-2008 arithmetic standard.  The D extension depends on
% the base single-precision instruction subset F.

\section{D寄存器状态}
% \section{D Register State}

D扩展把32位浮点寄存器{\tt f0} - {\tt f31}拓宽到64位(在图~\ref{fprs}中FLEN=64)。
{\tt f}寄存器现在既可以持有32位浮点值,也可以持有64位浮点值,正如下面在~\ref{nanboxing}节中描述的那样。
% The D extension widens the 32 floating-point registers, {\tt f0}--{\tt
%   f31}, to 64 bits (FLEN=64 in Figure~\ref{fprs}).  The {\tt f}
% registers can now hold either 32-bit or 64-bit floating-point values
% as described below in Section~\ref{nanboxing}.

\begin{commentary}
  根据F扩展、D扩展和Q扩展被支持的情况,FLEN可以是32、64或者128。可以至多支持四个不同的浮点精度,包括H、F、D和Q。
% FLEN can be 32, 64, or 128 depending on which of the F, D, and Q
% extensions are supported.  There can be up to four different
% floating-point precisions supported, including H, F, D, and Q.
\end{commentary}

\section{较窄值的NaN装箱}
% \section{NaN Boxing of Narrower Values}
\label{nanboxing}

当支持多个浮点精度时,在一个FLEN位NaN值的低$n$位中表示较窄的$n$位类型($n<$FLEN)的有效值,这个过程术语叫做NaN装箱。
一个有效的NaN装箱的值的高位必须全是1。因此,当被视为任意更宽的m位值时(\mbox{$n < m \leq$ FLEN}),
有效的NaN装箱的$n$位值表现为负的静默NaN(qNaN)。
任何把较窄的结果写到一个{\tt f}寄存器的操作都必须把最高的FLEN$-n$位全写成1,以产生一个合法的NaN装箱的值。
% When multiple floating-point precisions are supported, then valid
% values of narrower $n$-bit types, \mbox{$n<$ FLEN}, are represented in
% the lower $n$ bits of an FLEN-bit NaN value, in a process termed
% NaN-boxing.  The upper bits of a valid NaN-boxed value must be all 1s.
% Valid NaN-boxed $n$-bit values therefore appear as negative quiet NaNs
% (qNaNs) when viewed as any wider $m$-bit value, \mbox{$n < m \leq$
%   FLEN}.  Any operation that writes a narrower result to an {\tt f}
% register must write all 1s to the uppermost FLEN$-n$ bits to yield a
% legal NaN-boxed value.

\begin{commentary}
  软件可能不知道存储在一个浮点寄存器中的数据的当前类型,但是必须能够保存和恢复寄存器的值,因此不得不定义使用较宽的操作来转移较窄的值的结果。
  一个常见的情况是用于由调用者保存的寄存器,但是对于包括varargs、用户级线程库、虚拟机迁移、和调试在内的特征,标准约定也是值得的。
% Software might not know the current type of data stored in a
% floating-point register but has to be able to save and restore the
% register values, hence the result of using wider operations to
% transfer narrower values has to be defined.  A common case is for
% callee-saved registers, but a standard convention is also desirable for
% features including varargs, user-level threading libraries, virtual
% machine migration, and debugging.
\end{commentary}

浮点$n$位转移操作把以IEEE标准格式保持的外部值移进和移出{\tt f}寄存器,并包含浮点加载和存储(FL$n$/FS$n$)和浮点移动指令(FMV.$n$.X/FMV.X.$n$)。
把一个较窄的$n$位(\mbox{$n<$ FLEN})转移进f寄存器将创造一个有效的NaN装箱的值。
把一个较窄的$n$位转移出浮点寄存器时,将转移该寄存器的低$n$位,而忽略高FLEN-$n$位。
% Floating-point $n$-bit transfer operations move external values held
% in IEEE standard formats into and out of the {\tt f} registers, and
% comprise floating-point loads and stores (FL$n$/FS$n$) and
% floating-point move instructions (FMV.$n$.X/FMV.X.$n$).  A narrower
% $n$-bit transfer, \mbox{$n<$ FLEN}, into the {\tt f} registers will create a
% valid NaN-boxed value.  A narrower $n$-bit transfer out of
% the floating-point registers will transfer the lower $n$ bits of the
% register ignoring the upper FLEN$-n$ bits.

除了在之前段落中描述的转移操作,所有其它的关于操作较窄$n$位的浮点操作(\mbox{$n<$ FLEN})都将检查输入操作数是否被正确地NaN装箱,
或者说,所有的FLEN$-n$位是否都是1。如果的确如此,输入的最低$n$个有效位被作为输入值使用,否则输入值被视为一个$n$位的规范NaN。
% Apart from transfer operations described in the previous paragraph,
% all other floating-point operations on narrower $n$-bit operations,
% \mbox{$n<$ FLEN}, check if the input operands are correctly NaN-boxed,
% i.e., all upper FLEN$-n$ bits are 1.  If so, the $n$ least-significant
% bits of the input are used as the input value, otherwise the input
% value is treated as an $n$-bit canonical NaN.

\begin{commentary}
  这个文档的较早的版本没有定义把较窄或较宽操作数的结果送进操作的行为,除非要求较宽的保存和恢复将保留较窄操作数的值。
  新的定义移除了这个与实现有关的行为,但仍然采纳了浮点单元的非重新编码的实现和重新编码的实现。
  如果没有正确地使用值,新的定义也帮助抓取由传播NaN引起的软件错误。
% Earlier versions of this document did not define the behavior of
% feeding the results of narrower or wider operands into an operation,
% except to require that wider saves and restores would preserve the
% value of a narrower operand.  The new definition removes this
% implementation-specific behavior, while still accommodating both
% non-recoded and recoded implementations of the floating-point unit.
% The new definition also helps catch software errors by propagating
% NaNs if values are used incorrectly.

非重新编码的实现在每个浮点指令的输入和输出上把操作数按IEEE标准格式解包和打包。
对一个非重新编码的实现的NaN装箱开销主要在于,检查较窄操作的高位是否表示了一个合法的NaN装箱的值,以及把结果的高位全写成1。
% Non-recoded implementations unpack and pack the operands to IEEE
% standard format on the input and output of every floating-point
% operation.  The NaN-boxing cost to a non-recoded implementation is
% primarily in checking if the upper bits of a narrower operation
% represent a legal NaN-boxed value, and in writing all 1s to the upper
% bits of a result.

重新编码的实现使用一个更加方便的内部格式来表示浮点值,它添加了一个指数位来允许所有的值的标准化保持。
重新编码的实现的开销主要在于,为了追踪内部类型和符号位所需要的额外的标签工作,
但是这可以通过在指数域中内部地重新编码NaN来完成,而不用添加新的状态位。
用于把值转移进出重新编码格式的管道需要一些小的改动,但是数据路径和延迟开销是很小的。
在任何情况中,对于宽操作数,重新编码的过程都必须处理输入亚正常值的移位,而提取NaN装箱的值是一个与标准化相似的过程
,除了要跳过领头的1位而不是0位以外,从而允许共享数据路径的muxing。
% Recoded implementations use a more convenient internal format to
% represent floating-point values, with an added exponent bit to allow
% all values to be held normalized.  The cost to the recoded
% implementation is primarily the extra tagging needed to track the
% internal types and sign bits, but this can be done without adding new
% state bits by recoding NaNs internally in the exponent field.  Small
% modifications are needed to the pipelines used to transfer values in
% and out of the recoded format, but the datapath and latency costs are
% minimal.  The recoding process has to handle shifting of input
% subnormal values for wide operands in any case, and extracting the
% NaN-boxed value is a similar process to normalization except for
% skipping over leading-1 bits instead of skipping over leading-0 bits,
% allowing the datapath muxing to be shared.
\end{commentary}

\section{双精度加载和存储指令}
% \section{Double-Precision Load and Store Instructions}
\label{fld_fsd}

FLD指令从内存加载一个双精度浮点值到浮点寄存器{\em rd}中。FSD把浮点寄存器中的一个双精度值存储到内存中。
% The FLD instruction loads a double-precision floating-point value from
% memory into floating-point register {\em rd}.  FSD stores a double-precision
% value from the floating-point registers to memory.
\begin{commentary}
  该双精度值可以是一个NaN装箱的单精度值。
% The double-precision value may be a NaN-boxed single-precision value.
\end{commentary}

\vspace{-0.2in}
\begin{center}
\begin{tabular}{M@{}R@{}F@{}R@{}O}
\\
\instbitrange{31}{20} &
\instbitrange{19}{15} &
\instbitrange{14}{12} &
\instbitrange{11}{7} &
\instbitrange{6}{0} \\
\hline
\multicolumn{1}{|c|}{imm[11:0]} &
\multicolumn{1}{c|}{rs1} &
\multicolumn{1}{c|}{width} &
\multicolumn{1}{c|}{rd} &
\multicolumn{1}{c|}{opcode} \\
\hline
12 & 5 & 3 & 5 & 7 \\
offset[11:0] & base & D & dest & LOAD-FP \\
\end{tabular}
\end{center}

\vspace{-0.2in}
\begin{center}
\begin{tabular}{O@{}R@{}R@{}F@{}R@{}O}
\\
\instbitrange{31}{25} &
\instbitrange{24}{20} &
\instbitrange{19}{15} &
\instbitrange{14}{12} &
\instbitrange{11}{7} &
\instbitrange{6}{0} \\
\hline
\multicolumn{1}{|c|}{imm[11:5]} &
\multicolumn{1}{c|}{rs2} &
\multicolumn{1}{c|}{rs1} &
\multicolumn{1}{c|}{width} &
\multicolumn{1}{c|}{imm[4:0]} &
\multicolumn{1}{c|}{opcode} \\
\hline
7 & 5 & 5 & 3 & 5 & 7 \\
offset[11:5] & src & base & D & offset[4:0] & STORE-FP \\
\end{tabular}
\end{center}

只有当有效的地址被自然对齐,并且XLEN$\geq$64时,FLD和FSD才保证原子性执行。
% FLD and FSD are only guaranteed to execute atomically if the effective address
% is naturally aligned and XLEN$\geq$64.

FLD和FSD不修改正在被转移的位;特别地,非规范的NaN的有效载荷被保留。
% FLD and FSD do not modify the bits being transferred; in particular, the
% payloads of non-canonical NaNs are preserved.

\section{双精度浮点运算指令}
% \section{Double-Precision Floating-Point Computational Instructions}

双精度浮点运算指令的定义与它们对应的单精度指令的定义类似,但是在双精度操作数上操作,并产生双精度的结果。
% The double-precision floating-point computational instructions are
% defined analogously to their single-precision counterparts, but operate on
% double-precision operands and produce double-precision results.
\vspace{-0.2in}
\begin{center}
\begin{tabular}{R@{}F@{}R@{}R@{}F@{}R@{}O}
\\
\instbitrange{31}{27} &
\instbitrange{26}{25} &
\instbitrange{24}{20} &
\instbitrange{19}{15} &
\instbitrange{14}{12} &
\instbitrange{11}{7} &
\instbitrange{6}{0} \\
\hline
\multicolumn{1}{|c|}{funct5} &
\multicolumn{1}{c|}{fmt} &
\multicolumn{1}{c|}{rs2} &
\multicolumn{1}{c|}{rs1} &
\multicolumn{1}{c|}{rm} &
\multicolumn{1}{c|}{rd} &
\multicolumn{1}{c|}{opcode} \\
\hline
5 & 2 & 5 & 5 & 3 & 5 & 7 \\
FADD/FSUB & D & src2 & src1 & RM  & dest & OP-FP  \\
FMUL/FDIV & D & src2 & src1 & RM  & dest & OP-FP  \\
FMIN-MAX  & D & src2 & src1 & MIN/MAX & dest & OP-FP  \\
FSQRT     & D & 0    & src  & RM  & dest & OP-FP  \\
\end{tabular}
\end{center}

\vspace{-0.2in}
\begin{center}
\begin{tabular}{R@{}F@{}R@{}R@{}F@{}R@{}O}
\\
\instbitrange{31}{27} &
\instbitrange{26}{25} &
\instbitrange{24}{20} &
\instbitrange{19}{15} &
\instbitrange{14}{12} &
\instbitrange{11}{7} &
\instbitrange{6}{0} \\
\hline
\multicolumn{1}{|c|}{rs3} &
\multicolumn{1}{c|}{fmt} &
\multicolumn{1}{c|}{rs2} &
\multicolumn{1}{c|}{rs1} &
\multicolumn{1}{c|}{rm} &
\multicolumn{1}{c|}{rd} &
\multicolumn{1}{c|}{opcode} \\
\hline
5 & 2 & 5 & 5 & 3 & 5 & 7 \\
src3 & D & src2 & src1 & RM  & dest & F[N]MADD/F[N]MSUB  \\
\end{tabular}
\end{center}

\section{双精度浮点转换和移动指令}
% \section{Double-Precision Floating-Point Conversion and Move Instructions}

浮点到整数转换指令和整数到浮点转换指令被编码在OP-FP主操作码空间中。
FCVT.W.D或FCVT.L.D把浮点寄存器{\em rs1}中的双精度浮点数分别转化为一个有符号的32位或64位整数,并将其放入整数寄存器{\em rd}中。
FCVT.D.W或FCVT.D.L分别把整数寄存器{\em rs1}中的32位或64位有符号整数转换为一个双精度浮点数,并将其放入浮点寄存器{\em rd}中。
FCVT.WU.D,FCVT.LU.D,FCVT.D.WU和FCVT.D.LU变体转化为无符号整数值、或从无符号整数值转化。
对于RV64,FCVT.W[U].D把32位结果进行符号扩展。FCVT.L[U].D和FCVT.D.L[U]是RV64独有的指令。
FCVT.{\em int}.D的有效输入范围和无效输入行为与FCVT.{\em int}.S相同。
% Floating-point-to-integer and integer-to-floating-point conversion
% instructions are encoded in the OP-FP major opcode space.
% FCVT.W.D or FCVT.L.D converts a double-precision floating-point number
% in floating-point register {\em rs1} to a signed 32-bit or 64-bit
% integer, respectively, in integer register {\em rd}.  FCVT.D.W
% or FCVT.D.L converts a 32-bit or 64-bit signed integer,
% respectively, in integer register {\em rs1} into a
% double-precision floating-point
% number in floating-point register {\em rd}. FCVT.WU.D,
% FCVT.LU.D, FCVT.D.WU, and FCVT.D.LU variants
% convert to or from unsigned integer values.
% For RV64, FCVT.W[U].D sign-extends the 32-bit result.
% FCVT.L[U].D and FCVT.D.L[U] are RV64-only instructions.
% The range of valid inputs for FCVT.{\em int}.D and
% the behavior for invalid inputs are the same as for FCVT.{\em int}.S.

所有的浮点到整数转换指令和整数到浮点转换指令都根据{\em rm}域进行舍入。
注意FCVT.D.W[U]总是产生确切的结果,而不会被舍入模式影响。
% All floating-point to integer and integer to floating-point conversion
% instructions round according to the {\em rm} field.  Note FCVT.D.W[U] always
% produces an exact result and is unaffected by rounding mode.

\vspace{-0.2in}
\begin{center}
\begin{tabular}{R@{}F@{}R@{}R@{}F@{}R@{}O}
\\
\instbitrange{31}{27} &
\instbitrange{26}{25} &
\instbitrange{24}{20} &
\instbitrange{19}{15} &
\instbitrange{14}{12} &
\instbitrange{11}{7} &
\instbitrange{6}{0} \\
\hline
\multicolumn{1}{|c|}{funct5} &
\multicolumn{1}{c|}{fmt} &
\multicolumn{1}{c|}{rs2} &
\multicolumn{1}{c|}{rs1} &
\multicolumn{1}{c|}{rm} &
\multicolumn{1}{c|}{rd} &
\multicolumn{1}{c|}{opcode} \\
\hline
5 & 2 & 5 & 5 & 3 & 5 & 7 \\
FCVT.{\em int}.D & D & W[U]/L[U] & src & RM  & dest & OP-FP  \\
FCVT.D.{\em int} & D & W[U]/L[U] & src & RM  & dest & OP-FP  \\
\end{tabular}
\end{center}

双精度到单精度的转化指令FCVT.S.D和单精度到双精度的转化指令FCVT.D.S被编码在OP-FP主操作码空间中,
并且源寄存器和目的寄存器都是浮点寄存器。{\em rs2}域编码了源寄存器的数据类型,而{\em fmt}域编码了目的寄存器的数据类型。
FCVT.S.D根据RM域进行舍入,FCVT.D.S将永远不会舍入。
% The double-precision to single-precision and single-precision to
% double-precision conversion instructions, FCVT.S.D and FCVT.D.S, are
% encoded in the OP-FP major opcode space and both the source and
% destination are floating-point registers.  The {\em rs2} field
% encodes the datatype of the source, and the {\em fmt} field encodes
% the datatype of the destination.  FCVT.S.D rounds according to the
% RM field; FCVT.D.S will never round.

\vspace{-0.2in}
\begin{center}
\begin{tabular}{R@{}F@{}R@{}R@{}F@{}R@{}O}
\\
\instbitrange{31}{27} &
\instbitrange{26}{25} &
\instbitrange{24}{20} &
\instbitrange{19}{15} &
\instbitrange{14}{12} &
\instbitrange{11}{7} &
\instbitrange{6}{0} \\
\hline
\multicolumn{1}{|c|}{funct5} &
\multicolumn{1}{c|}{fmt} &
\multicolumn{1}{c|}{rs2} &
\multicolumn{1}{c|}{rs1} &
\multicolumn{1}{c|}{rm} &
\multicolumn{1}{c|}{rd} &
\multicolumn{1}{c|}{opcode} \\
\hline
5 & 2 & 5 & 5 & 3 & 5 & 7 \\
FCVT.S.D & S & D & src & RM  & dest & OP-FP  \\
FCVT.D.S & D & S & src & RM  & dest & OP-FP  \\
\end{tabular}
\end{center}

浮点到浮点的符号注入指令,FSGNJ.D、FSGNJN.D和FSGNJX.D的定义与对应的单精度符号注入指令的定义类似。
% Floating-point to floating-point sign-injection instructions, FSGNJ.D,
% FSGNJN.D, and FSGNJX.D are defined analogously to the single-precision
% sign-injection instruction.

\vspace{-0.2in}
\begin{center}
\begin{tabular}{R@{}F@{}R@{}R@{}F@{}R@{}O}
\\
\instbitrange{31}{27} &
\instbitrange{26}{25} &
\instbitrange{24}{20} &
\instbitrange{19}{15} &
\instbitrange{14}{12} &
\instbitrange{11}{7} &
\instbitrange{6}{0} \\
\hline
\multicolumn{1}{|c|}{funct5} &
\multicolumn{1}{c|}{fmt} &
\multicolumn{1}{c|}{rs2} &
\multicolumn{1}{c|}{rs1} &
\multicolumn{1}{c|}{rm} &
\multicolumn{1}{c|}{rd} &
\multicolumn{1}{c|}{opcode} \\
\hline
5 & 2 & 5 & 5 & 3 & 5 & 7 \\
FSGNJ & D & src2 & src1 & J[N]/JX & dest & OP-FP  \\
\end{tabular}
\end{center}

只有当XLEN$\geq$64时,提供在浮点和整数寄存器之间按位式样移动的指令。
FMV.X.D把浮点寄存器{\em rs1}中的双精度值移动到一个以IEEE 754-2008标准编码表示的整数寄存器{\em rd}中。
FMV.D.X从整数寄存器{\em rs1}中把以IEEE 754-2008标准编码编码的双精度值移动到浮点寄存器{\em rd}中。
% For XLEN$\geq$64 only, instructions are provided to move bit patterns
% between the floating-point and integer registers.  FMV.X.D moves the
% double-precision value in floating-point register {\em rs1} to a
% representation in IEEE 754-2008 standard encoding in integer register
% {\em rd}.  FMV.D.X moves the double-precision value encoded in IEEE
% 754-2008 standard encoding from the integer register {\em rs1} to the
% floating-point register {\em rd}.

FMV.X.D和FMV.D.X不修改正在被转移的位;特别地,非规范的NaN的有效载荷被保留。
% FMV.X.D and FMV.D.X do not modify the bits being transferred; in particular, the
% payloads of non-canonical NaNs are preserved.

\vspace{-0.2in}
\begin{center}
\begin{tabular}{R@{}F@{}R@{}R@{}F@{}R@{}O}
\\
\instbitrange{31}{27} &
\instbitrange{26}{25} &
\instbitrange{24}{20} &
\instbitrange{19}{15} &
\instbitrange{14}{12} &
\instbitrange{11}{7} &
\instbitrange{6}{0} \\
\hline
\multicolumn{1}{|c|}{funct5} &
\multicolumn{1}{c|}{fmt} &
\multicolumn{1}{c|}{rs2} &
\multicolumn{1}{c|}{rs1} &
\multicolumn{1}{c|}{rm} &
\multicolumn{1}{c|}{rd} &
\multicolumn{1}{c|}{opcode} \\
\hline
5 & 2 & 5 & 5 & 3 & 5 & 7 \\
FMV.X.D & D & 0    & src  & 000  & dest & OP-FP  \\
FMV.D.X & D & 0    & src  & 000  & dest & OP-FP  \\
\end{tabular}
\end{center}

\begin{commentary}
  RISC-V ISA的早期版本有额外的指令来允许RV32系统在一个64位浮点寄存器的高位部分和低位部分与一个整数寄存器之间进行转移。
  然而这将成为仅有的部分寄存器写入指令,并将增加实现在重新编码浮点或寄存器重命名时的复杂性,因为需要一个管道“读-修改-写”序列。
  如果要遵循这个式样,为RV32和RV64增加到处理四精度也将需要额外的指令。
  ISA被定义为,通过把转化和比较的结果写入合适的寄存器文件,来减少整数到浮点寄存器的显式移动的数目,
  因此我们希望这些指令的收益能够比其它的ISA更低。
  % Early versions of the RISC-V ISA had additional instructions to
  % allow RV32 systems to transfer between the upper and lower portions
  % of a 64-bit floating-point register and an integer register.
  % However, these would be the only instructions with partial register
  % writes and would add complexity in implementations with recoded
  % floating-point or register renaming, requiring a pipeline read-modify-write
  % sequence.  Scaling up to handling quad-precision for RV32 and RV64
  % would also require additional instructions if they were to follow
  % this pattern.  The ISA was defined to reduce the number of explicit
  % int-float register moves, by having conversions and comparisons
  % write results to the appropriate register file, so we expect the
  % benefit of these instructions to be lower than for other ISAs.

  我们注意到,对于实现了64位浮点单元(包括融合的乘加支持)和64位浮点加载与存储的系统来说,
  从32位整数移动到64位整数的数据路径的外围硬件开销较低,
  而支持32位宽地址空间和指针的软件ABI可以被用于避免静态数据的增长和动态内存的拥塞。
  % We note that for systems that implement a 64-bit floating-point unit
  % including fused multiply-add support and 64-bit floating-point loads
  % and stores, the marginal hardware cost of moving from a 32-bit to
  % a 64-bit integer datapath is low, and a software ABI supporting 32-bit
  % wide address-space and pointers can be used to avoid growth of
  % static data and dynamic memory traffic.
\end{commentary}

\section{双精度浮点比较指令}
% \section{Double-Precision Floating-Point Compare Instructions}

双精度浮点比较指令的定义与它们对应的单精度指令的定义类似,但是在双精度操作数上操作。
% The double-precision floating-point compare instructions are
% defined analogously to their single-precision counterparts, but operate on
% double-precision operands.

\vspace{-0.2in}
\begin{center}
\begin{tabular}{S@{}F@{}R@{}R@{}F@{}R@{}O}
\\
\instbitrange{31}{27} &
\instbitrange{26}{25} &
\instbitrange{24}{20} &
\instbitrange{19}{15} &
\instbitrange{14}{12} &
\instbitrange{11}{7} &
\instbitrange{6}{0} \\
\hline
\multicolumn{1}{|c|}{funct5} &
\multicolumn{1}{c|}{fmt} &
\multicolumn{1}{c|}{rs2} &
\multicolumn{1}{c|}{rs1} &
\multicolumn{1}{c|}{rm} &
\multicolumn{1}{c|}{rd} &
\multicolumn{1}{c|}{opcode} \\
\hline
5 & 2 & 5 & 5 & 3 & 5 & 7 \\
FCMP & D & src2 & src1 & EQ/LT/LE & dest & OP-FP  \\
\end{tabular}
\end{center}

\section{双精度浮点分类指令}
% \section{Double-Precision Floating-Point Classify Instruction}

双精度浮点分类指令,FCLASS.D,其定义与对应的单精度指令的定义类似,但是在双精度操作数上操作。
% The double-precision floating-point classify instruction, FCLASS.D, is
% defined analogously to its single-precision counterpart, but operates on
% double-precision operands.

\vspace{-0.2in}
\begin{center}
\begin{tabular}{S@{}F@{}R@{}R@{}F@{}R@{}O}
\\
\instbitrange{31}{27} &
\instbitrange{26}{25} &
\instbitrange{24}{20} &
\instbitrange{19}{15} &
\instbitrange{14}{12} &
\instbitrange{11}{7} &
\instbitrange{6}{0} \\
\hline
\multicolumn{1}{|c|}{funct5} &
\multicolumn{1}{c|}{fmt} &
\multicolumn{1}{c|}{rs2} &
\multicolumn{1}{c|}{rs1} &
\multicolumn{1}{c|}{rm} &
\multicolumn{1}{c|}{rd} &
\multicolumn{1}{c|}{opcode} \\
\hline
5 & 2 & 5 & 5 & 3 & 5 & 7 \\
FCLASS & D & 0 & src & 001 & dest & OP-FP  \\
\end{tabular}
\end{center}

\chapter{用于四精度浮点的“Q”标准扩展(2.2版本)}
% \chapter{``Q'' Standard Extension for Quad-Precision Floating-Point,
  % Version 2.2}

这章描述了用于128位四精度二进制浮点指令的、兼容IEEE 754-2008算数标准的Q标准扩展。
四精度二进制浮点指令集扩展被命名为“Q”;它依赖于双精度浮点扩展D。
浮点寄存器现在被扩展为保持单精度、双精度、或者四精度的浮点值(FLEN=128)。
在~\ref{nanboxing}节中描述的NaN装箱策略现在被递归地扩展,以允许单精度值被NaN装箱在一个双精度值中,
而该双精度值自己被NaN装箱在一个四精度值中。
% This chapter describes the Q standard extension for 128-bit quad-precision binary
% floating-point instructions compliant with the IEEE 754-2008
% arithmetic standard. The quad-precision binary
% floating-point instruction-set extension is named ``Q''; it depends
% on the double-precision floating-point extension D.
% The floating-point registers are now extended to hold either
% a single, double, or quad-precision floating-point value (FLEN=128).
% The NaN-boxing scheme described in Section~\ref{nanboxing} is now
% extended recursively to allow a single-precision value to be NaN-boxed
% inside a double-precision value which is itself NaN-boxed inside a
% quad-precision value.

\section{四精度加载和存储指令}
% \section{Quad-Precision Load and Store Instructions}

添加了LOAD-FP和STORE-FP指令的新的128位变体,使用funct3宽度域的新值进行编码。
% New 128-bit variants of LOAD-FP and STORE-FP instructions are added,
% encoded with a new value for the funct3 width field.

\vspace{-0.2in}
\begin{center}
\begin{tabular}{M@{}R@{}F@{}R@{}O}
\\
\instbitrange{31}{20} &
\instbitrange{19}{15} &
\instbitrange{14}{12} &
\instbitrange{11}{7} &
\instbitrange{6}{0} \\
\hline
\multicolumn{1}{|c|}{imm[11:0]} &
\multicolumn{1}{c|}{rs1} &
\multicolumn{1}{c|}{width} &
\multicolumn{1}{c|}{rd} &
\multicolumn{1}{c|}{opcode} \\
\hline
12 & 5 & 3 & 5 & 7 \\
offset[11:0] & base & Q & dest & LOAD-FP \\
\end{tabular}
\end{center}

\vspace{-0.2in}
\begin{center}
\begin{tabular}{O@{}R@{}R@{}F@{}R@{}O}
\\
\instbitrange{31}{25} &
\instbitrange{24}{20} &
\instbitrange{19}{15} &
\instbitrange{14}{12} &
\instbitrange{11}{7} &
\instbitrange{6}{0} \\
\hline
\multicolumn{1}{|c|}{imm[11:5]} &
\multicolumn{1}{c|}{rs2} &
\multicolumn{1}{c|}{rs1} &
\multicolumn{1}{c|}{width} &
\multicolumn{1}{c|}{imm[4:0]} &
\multicolumn{1}{c|}{opcode} \\
\hline
7 & 5 & 5 & 3 & 5 & 7 \\
offset[11:5] & src & base & Q & offset[4:0] & STORE-FP \\
\end{tabular}
\end{center}

只有有效的地址被自然对齐并且XLEN=128时,才保证FLQ和FSQ原子性地执行。
% FLQ and FSQ are only guaranteed to execute atomically if the effective address
% is naturally aligned and XLEN=128.

FLQ和FSQ不会修改正在被转移的位;特别地,非规范的NaN的有效载荷被保留。
% FLQ and FSQ do not modify the bits being transferred; in particular, the
% payloads of non-canonical NaNs are preserved.

\section{四精度运算指令}
% \section{Quad-Precision Computational Instructions}

为大多数指令的格式域添加了一个新的被支持的格式,如表~\ref{tab:fpextfmt}中显示的那样。
% A new supported format is added to the format field of most
% instructions, as shown in Table~\ref{tab:fpextfmt}.

\begin{table}[htp]
\begin{center}
\begin{tabular}{|c|c|l|}
\hline
{\em fmt} 域 &
Mnemonic &
Meaning \\
\hline
00 & S & 32位单精度\\
01 & D & 64位双精度\\
10 & H & 16位半精度\\
11 & Q & 128位四精度\\
\hline
\end{tabular}
\end{center}
\caption{格式域编码。}
\label{tab:fpextfmt}
\end{table}

四精度浮点运算指令的定义与他们对应的双精度指令的定义类似,但是在四精度操作数上操作,并产生四精度的结果。
% The quad-precision floating-point computational instructions are
% defined analogously to their double-precision counterparts, but operate on
% quad-precision operands and produce quad-precision results.

\vspace{-0.2in}
\begin{center}
\begin{tabular}{R@{}F@{}R@{}R@{}F@{}R@{}O}
\\
\instbitrange{31}{27} &
\instbitrange{26}{25} &
\instbitrange{24}{20} &
\instbitrange{19}{15} &
\instbitrange{14}{12} &
\instbitrange{11}{7} &
\instbitrange{6}{0} \\
\hline
\multicolumn{1}{|c|}{funct5} &
\multicolumn{1}{c|}{fmt} &
\multicolumn{1}{c|}{rs2} &
\multicolumn{1}{c|}{rs1} &
\multicolumn{1}{c|}{rm} &
\multicolumn{1}{c|}{rd} &
\multicolumn{1}{c|}{opcode} \\
\hline
5 & 2 & 5 & 5 & 3 & 5 & 7 \\
FADD/FSUB & Q & src2 & src1 & RM  & dest & OP-FP  \\
FMUL/FDIV & Q & src2 & src1 & RM  & dest & OP-FP  \\
FMIN-MAX  & Q & src2 & src1 & MIN/MAX & dest & OP-FP  \\
FSQRT     & Q & 0    & src  & RM  & dest & OP-FP  \\
\end{tabular}
\end{center}

\vspace{-0.2in}
\begin{center}
\begin{tabular}{R@{}F@{}R@{}R@{}F@{}R@{}O}
\\
\instbitrange{31}{27} &
\instbitrange{26}{25} &
\instbitrange{24}{20} &
\instbitrange{19}{15} &
\instbitrange{14}{12} &
\instbitrange{11}{7} &
\instbitrange{6}{0} \\
\hline
\multicolumn{1}{|c|}{rs3} &
\multicolumn{1}{c|}{fmt} &
\multicolumn{1}{c|}{rs2} &
\multicolumn{1}{c|}{rs1} &
\multicolumn{1}{c|}{rm} &
\multicolumn{1}{c|}{rd} &
\multicolumn{1}{c|}{opcode} \\
\hline
5 & 2 & 5 & 5 & 3 & 5 & 7 \\
src3 & Q & src2 & src1 & RM  & dest & F[N]MADD/F[N]MSUB  \\
\end{tabular}
\end{center}

\section{四精度转换和移动指令}
% \section{Quad-Precision Conversion and Move Instructions}

添加了新的浮点到整数转化指令和整数到浮点转化指令。这些指令的定义与双精度到整数转化指令和整数到双精度转化指令的定义类似。
FCVT.W.Q或者FCVT.L.Q分别把一个四精度浮点数转化为一个有符号的32位或64位整数。
FCVT.Q.W或FCVT.Q.L分别把一个32位或64位有符号整数转化为一个四精度浮点整数。
FCVT.WU.Q、FCVT.LU.Q、FCVT.Q.WU和FCVT.Q.LU变体转化为无符号整数值、或从无符号整数值转化。
FCVT.L[U].Q和FCVT.Q.L[U]是RV64独有的指令。注意FCVT.Q.L[U]总是产生确切的结果,而不会被舍入模式影响。
% New floating-point-to-integer and integer-to-floating-point conversion
% instructions are added.  These instructions are defined analogously to the
% double-precision-to-integer and integer-to-double-precision conversion
% instructions.  FCVT.W.Q or FCVT.L.Q converts a quad-precision floating-point
% number to a signed 32-bit or 64-bit integer, respectively.  FCVT.Q.W or
% FCVT.Q.L converts a 32-bit or 64-bit signed integer, respectively, into a
% quad-precision floating-point number. FCVT.WU.Q, FCVT.LU.Q, FCVT.Q.WU, and
% FCVT.Q.LU variants convert to or from unsigned integer values.  FCVT.L[U].Q and
% FCVT.Q.L[U] are RV64-only instructions.  Note FCVT.Q.L[U] always
% produces an exact result and is unaffected by rounding mode.

\vspace{-0.2in}
\begin{center}
\begin{tabular}{R@{}F@{}R@{}R@{}F@{}R@{}O}
\\
\instbitrange{31}{27} &
\instbitrange{26}{25} &
\instbitrange{24}{20} &
\instbitrange{19}{15} &
\instbitrange{14}{12} &
\instbitrange{11}{7} &
\instbitrange{6}{0} \\
\hline
\multicolumn{1}{|c|}{funct5} &
\multicolumn{1}{c|}{fmt} &
\multicolumn{1}{c|}{rs2} &
\multicolumn{1}{c|}{rs1} &
\multicolumn{1}{c|}{rm} &
\multicolumn{1}{c|}{rd} &
\multicolumn{1}{c|}{opcode} \\
\hline
5 & 2 & 5 & 5 & 3 & 5 & 7 \\
FCVT.{\em int}.Q & Q & W[U]/L[U] & src & RM  & dest & OP-FP  \\
FCVT.Q.{\em int} & Q & W[U]/L[U] & src & RM  & dest & OP-FP  \\
\end{tabular}
\end{center}

添加了新的浮点到浮点转化指令。这些指令的定义与双精度浮点到浮点转化指令的定义类似。
FCVT.S.Q把一个四精度浮点数转化为一个单精度浮点数,FCVT.Q.S与之相反。
FCVT.D.Q把一个四精度浮点数转化为一个双精度浮点数,FCVT.Q.D与之相反。
% New floating-point-to-floating-point conversion instructions are added.  These
% instructions are defined analogously to the double-precision floating-point-to-floating-point
% conversion instructions.  FCVT.S.Q or FCVT.Q.S converts a quad-precision
% floating-point number to a single-precision floating-point number, or
% vice-versa, respectively.  FCVT.D.Q or FCVT.Q.D converts a quad-precision
% floating-point number to a double-precision floating-point number, or
% vice-versa, respectively.

\vspace{-0.2in}
\begin{center}
\begin{tabular}{R@{}F@{}R@{}R@{}F@{}R@{}O}
\\
\instbitrange{31}{27} &
\instbitrange{26}{25} &
\instbitrange{24}{20} &
\instbitrange{19}{15} &
\instbitrange{14}{12} &
\instbitrange{11}{7} &
\instbitrange{6}{0} \\
\hline
\multicolumn{1}{|c|}{funct5} &
\multicolumn{1}{c|}{fmt} &
\multicolumn{1}{c|}{rs2} &
\multicolumn{1}{c|}{rs1} &
\multicolumn{1}{c|}{rm} &
\multicolumn{1}{c|}{rd} &
\multicolumn{1}{c|}{opcode} \\
\hline
5 & 2 & 5 & 5 & 3 & 5 & 7 \\
FCVT.S.Q & S & Q & src & RM  & dest & OP-FP  \\
FCVT.Q.S & Q & S & src & RM  & dest & OP-FP  \\
FCVT.D.Q & D & Q & src & RM  & dest & OP-FP  \\
FCVT.Q.D & Q & D & src & RM  & dest & OP-FP  \\
\end{tabular}
\end{center}

浮点到浮点的符号注入指令,FSGNJ.Q、FSGNJN.Q和FSGNJX.Q的定义与双精度符号注入指令的定义类似。
% Floating-point to floating-point sign-injection instructions, FSGNJ.Q,
% FSGNJN.Q, and FSGNJX.Q are defined analogously to the double-precision
% sign-injection instruction.

\vspace{-0.2in}
\begin{center}
\begin{tabular}{R@{}F@{}R@{}R@{}F@{}R@{}O}
\\
\instbitrange{31}{27} &
\instbitrange{26}{25} &
\instbitrange{24}{20} &
\instbitrange{19}{15} &
\instbitrange{14}{12} &
\instbitrange{11}{7} &
\instbitrange{6}{0} \\
\hline
\multicolumn{1}{|c|}{funct5} &
\multicolumn{1}{c|}{fmt} &
\multicolumn{1}{c|}{rs2} &
\multicolumn{1}{c|}{rs1} &
\multicolumn{1}{c|}{rm} &
\multicolumn{1}{c|}{rd} &
\multicolumn{1}{c|}{opcode} \\
\hline
5 & 2 & 5 & 5 & 3 & 5 & 7 \\
FSGNJ & Q & src2 & src1 & J[N]/JX & dest & OP-FP  \\
\end{tabular}
\end{center}

在RV32或RV64中不提供FMV.X.Q和FMV.Q.X指令,所以四精度位式样必须通过内存被移动到整数寄存器。
% FMV.X.Q and FMV.Q.X instructions are not provided in RV32 or RV64, so
% quad-precision bit patterns must be moved to the integer registers via
% memory.

\begin{commentary}
  RV128将在Q扩展中支持FMV.X.Q和FMV.Q。
% RV128 will support FMV.X.Q and FMV.Q.X in the Q extension.
\end{commentary}

\section{四精度浮点比较指令}
% \section{Quad-Precision Floating-Point Compare Instructions}

四精度浮点比较指令的定义与它们对应的双精度指令的定义类似,但是在四精度操作数上操作。
% The quad-precision floating-point compare instructions are
% defined analogously to their double-precision counterparts, but operate on
% quad-precision operands.

\vspace{-0.2in}
\begin{center}
\begin{tabular}{S@{}F@{}R@{}R@{}F@{}R@{}O}
\\
\instbitrange{31}{27} &
\instbitrange{26}{25} &
\instbitrange{24}{20} &
\instbitrange{19}{15} &
\instbitrange{14}{12} &
\instbitrange{11}{7} &
\instbitrange{6}{0} \\
\hline
\multicolumn{1}{|c|}{funct5} &
\multicolumn{1}{c|}{fmt} &
\multicolumn{1}{c|}{rs2} &
\multicolumn{1}{c|}{rs1} &
\multicolumn{1}{c|}{rm} &
\multicolumn{1}{c|}{rd} &
\multicolumn{1}{c|}{opcode} \\
\hline
5 & 2 & 5 & 5 & 3 & 5 & 7 \\
FCMP & Q & src2 & src1 & EQ/LT/LE & dest & OP-FP  \\
\end{tabular}
\end{center}

\section{四精度浮点分类指令}
% \section{Quad-Precision Floating-Point Classify Instruction}

四精度浮点分类指令,FCLASS.Q,其定义与它对应的双精度指令类似,但是在四精度操作数上操作。
% The quad-precision floating-point classify instruction, FCLASS.Q, is
% defined analogously to its double-precision counterpart, but operates on
% quad-precision operands.

\vspace{-0.2in}
\begin{center}
\begin{tabular}{S@{}F@{}R@{}R@{}F@{}R@{}O}
\\
\instbitrange{31}{27} &
\instbitrange{26}{25} &
\instbitrange{24}{20} &
\instbitrange{19}{15} &
\instbitrange{14}{12} &
\instbitrange{11}{7} &
\instbitrange{6}{0} \\
\hline
\multicolumn{1}{|c|}{funct5} &
\multicolumn{1}{c|}{fmt} &
\multicolumn{1}{c|}{rs2} &
\multicolumn{1}{c|}{rs1} &
\multicolumn{1}{c|}{rm} &
\multicolumn{1}{c|}{rd} &
\multicolumn{1}{c|}{opcode} \\
\hline
5 & 2 & 5 & 5 & 3 & 5 & 7 \\
FCLASS & Q & 0 & src & 001 & dest & OP-FP  \\
\end{tabular}
\end{center}

\chapter{用于半精度浮点的“Zfh”和“Zfhmin”标准扩展(1.0版本}
% \chapter{``Zfh'' and ``Zfhmin'' Standard Extensions for Half-Precision Floating-Point,
  % Version 1.0}

本章描述了用于16位半精度二元浮点指令的Zfh标准扩展,兼容IEEE 754-2008算术标准。
Zfh扩展依赖于单精度浮点扩展,F。
在第14章描述的NaN装箱策略被扩展为,允许在一个单精度值中对一个半精度值NaN装箱(当D扩展或Q扩展存在时,其可以在双精度或四精度值中递归地NaN装箱)。
% This chapter describes the Zfh standard extension for 16-bit half-precision
% binary floating-point instructions compliant with the IEEE 754-2008 arithmetic
% standard.
% The Zfh extension depends on the single-precision floating-point extension, F.
% The NaN-boxing scheme described in Section~\ref{nanboxing} is extended to
% allow a half-precision value to be NaN-boxed inside a single-precision value
% (which may be recursively NaN-boxed inside a double- or quad-precision value
% when the D or Q extension is present).

\begin{commentary}
  这个扩展主要提供了消费半精度操作数并产生半精度结果的指令。然而,在计算半精度数据中使用更高的中间精度也是很常见的。
  尽管这个扩展提供了足以实现该样式的显式转换指令,未来的扩展也可能用额外的指令进一步加速此类运算,
  那些额外的指令会隐式地加宽其操作数——例如,半$\times$半$+$单$\rightarrow$单——或者隐式地收窄其结果——例如,半$+$单$\rightarrow$半。
% This extension primarily provides instructions that consume half-precision
% operands and produce half-precision results.
% However, it is also common to compute on half-precision data using higher
% intermediate precision.
% Although this extension provides explicit conversion instructions that suffice
% to implement that pattern, future extensions might further accelerate such
% computation with additional instructions that implicitly widen their
% operands---e.g., half$\times$half$+$single$\rightarrow$single---or implicitly
% narrow their results---e.g., half$+$single$\rightarrow$half.
\end{commentary}

\section{半精度加载和存储指令}
% \section{Half-Precision Load and Store Instructions}

添加了LOAD-FP和STORE-FP指令的新的16位变体,并为funct3宽度域编码了新的值。
% New 16-bit variants of LOAD-FP and STORE-FP instructions are added,
% encoded with a new value for the funct3 width field.

\vspace{-0.2in}
\begin{center}
\begin{tabular}{M@{}R@{}F@{}R@{}O}
\\
\instbitrange{31}{20} &
\instbitrange{19}{15} &
\instbitrange{14}{12} &
\instbitrange{11}{7} &
\instbitrange{6}{0} \\
\hline
\multicolumn{1}{|c|}{imm[11:0]} &
\multicolumn{1}{c|}{rs1} &
\multicolumn{1}{c|}{width} &
\multicolumn{1}{c|}{rd} &
\multicolumn{1}{c|}{opcode} \\
\hline
12 & 5 & 3 & 5 & 7 \\
offset[11:0] & base & H & dest & LOAD-FP \\
\end{tabular}
\end{center}

\vspace{-0.2in}
\begin{center}
\begin{tabular}{O@{}R@{}R@{}F@{}R@{}O}
\\
\instbitrange{31}{25} &
\instbitrange{24}{20} &
\instbitrange{19}{15} &
\instbitrange{14}{12} &
\instbitrange{11}{7} &
\instbitrange{6}{0} \\
\hline
\multicolumn{1}{|c|}{imm[11:5]} &
\multicolumn{1}{c|}{rs2} &
\multicolumn{1}{c|}{rs1} &
\multicolumn{1}{c|}{width} &
\multicolumn{1}{c|}{imm[4:0]} &
\multicolumn{1}{c|}{opcode} \\
\hline
7 & 5 & 5 & 3 & 5 & 7 \\
offset[11:5] & src & base & H & offset[4:0] & STORE-FP \\
\end{tabular}
\end{center}

只有当有效地址被自然对齐时,才保证FLH和FSH原子地执行。
% FLH and FSH are only guaranteed to execute atomically if the effective address
% is naturally aligned.

FLH和FSH不改变正在被传输的位;特别地,保留了非典型NaN的有效载荷。
FLH 将写入{\em rd}的结果进行NaN装箱,而FSH忽略{\em rs2}中除低16位之外的所有位。
% FLH and FSH do not modify the bits being transferred; in particular, the
% payloads of non-canonical NaNs are preserved.
% FLH NaN-boxes the result written to {\em rd}, whereas FSH ignores all but
% the lower 16 bits in {\em rs2}.

\section{半精度运算指令}
% \section{Half-Precision Computational Instructions}

向大多数指令的format域添加了一个新的支持格式,如表~\ref{tab:fpextfmth}所示。
% A new supported format is added to the format field of most
% instructions, as shown in Table~\ref{tab:fpextfmth}.

\begin{table}[htp]
\begin{center}
\begin{tabular}{|c|c|l|}
\hline
{\em fmt}域 &
助记符 &
含义 \\
\hline
00 & S & 32位单精度\\
01 & D & 64位双精度\\
10 & H & 16位半精度\\
11 & Q & 128位四精度\\
\hline
\end{tabular}
\end{center}
\caption{格式域编码。}
\label{tab:fpextfmth}
\end{table}

半精度浮点运算指令的定义类似于它们对应的单精度指令,但是在半精度操作数上操作,并产生半精度结果。
% The half-precision floating-point computational instructions are
% defined analogously to their single-precision counterparts, but operate on
% half-precision operands and produce half-precision results.

\vspace{-0.2in}
\begin{center}
\begin{tabular}{R@{}F@{}R@{}R@{}F@{}R@{}O}
\\
\instbitrange{31}{27} &
\instbitrange{26}{25} &
\instbitrange{24}{20} &
\instbitrange{19}{15} &
\instbitrange{14}{12} &
\instbitrange{11}{7} &
\instbitrange{6}{0} \\
\hline
\multicolumn{1}{|c|}{funct5} &
\multicolumn{1}{c|}{fmt} &
\multicolumn{1}{c|}{rs2} &
\multicolumn{1}{c|}{rs1} &
\multicolumn{1}{c|}{rm} &
\multicolumn{1}{c|}{rd} &
\multicolumn{1}{c|}{opcode} \\
\hline
5 & 2 & 5 & 5 & 3 & 5 & 7 \\
FADD/FSUB & H & src2 & src1 & RM  & dest & OP-FP  \\
FMUL/FDIV & H & src2 & src1 & RM  & dest & OP-FP  \\
FMIN-MAX  & H & src2 & src1 & MIN/MAX & dest & OP-FP  \\
FSQRT     & H & 0    & src  & RM  & dest & OP-FP  \\
\end{tabular}
\end{center}

\vspace{-0.2in}
\begin{center}
\begin{tabular}{R@{}F@{}R@{}R@{}F@{}R@{}O}
\\
\instbitrange{31}{27} &
\instbitrange{26}{25} &
\instbitrange{24}{20} &
\instbitrange{19}{15} &
\instbitrange{14}{12} &
\instbitrange{11}{7} &
\instbitrange{6}{0} \\
\hline
\multicolumn{1}{|c|}{rs3} &
\multicolumn{1}{c|}{fmt} &
\multicolumn{1}{c|}{rs2} &
\multicolumn{1}{c|}{rs1} &
\multicolumn{1}{c|}{rm} &
\multicolumn{1}{c|}{rd} &
\multicolumn{1}{c|}{opcode} \\
\hline
5 & 2 & 5 & 5 & 3 & 5 & 7 \\
src3 & H & src2 & src1 & RM  & dest & F[N]MADD/F[N]MSUB  \\
\end{tabular}
\end{center}

\section{半精度转换和移动指令}
% \section{Half-Precision Conversion and Move Instructions}

添加了新的浮点到整数转换指令和整数到浮点转换指令。这些指令的定义与单精度到整数转换指令及整数到单精度转换指令类似。
FCVT.W.H或FCVT.L.H分别把一个半精度浮点数转换为一个有符号的32位或64位整数。
FCVT.H.W或FCVT.H.L分别把一个32位或64位有符号整数转换为一个半精度浮点数。
FCVT.WU.H、FCVT.LU.H、FCVT.H.WU和FCVT.H.LU的变体与无符号整数值相互转换。
FCVT.L[U].H和FCVT.H.L[U]是只在RV64中使用的指令。
% New floating-point-to-integer and integer-to-floating-point conversion
% instructions are added.  These instructions are defined analogously to the
% single-precision-to-integer and integer-to-single-precision conversion
% instructions.  FCVT.W.H or FCVT.L.H converts a half-precision floating-point
% number to a signed 32-bit or 64-bit integer, respectively.  FCVT.H.W or
% FCVT.H.L converts a 32-bit or 64-bit signed integer, respectively, into a
% half-precision floating-point number. FCVT.WU.H, FCVT.LU.H, FCVT.H.WU, and
% FCVT.H.LU variants convert to or from unsigned integer values.  FCVT.L[U].H and
% FCVT.H.L[U] are RV64-only instructions.

\vspace{-0.2in}
\begin{center}
\begin{tabular}{R@{}F@{}R@{}R@{}F@{}R@{}O}
\\
\instbitrange{31}{27} &
\instbitrange{26}{25} &
\instbitrange{24}{20} &
\instbitrange{19}{15} &
\instbitrange{14}{12} &
\instbitrange{11}{7} &
\instbitrange{6}{0} \\
\hline
\multicolumn{1}{|c|}{funct5} &
\multicolumn{1}{c|}{fmt} &
\multicolumn{1}{c|}{rs2} &
\multicolumn{1}{c|}{rs1} &
\multicolumn{1}{c|}{rm} &
\multicolumn{1}{c|}{rd} &
\multicolumn{1}{c|}{opcode} \\
\hline
5 & 2 & 5 & 5 & 3 & 5 & 7 \\
FCVT.{\em int}.H & H & W[U]/L[U] & src & RM  & dest & OP-FP  \\
FCVT.H.{\em int} & H & W[U]/L[U] & src & RM  & dest & OP-FP  \\
\end{tabular}
\end{center}

添加了新的浮点到浮点转换指令。这些指令的定义与双精度浮点到浮点转换指令类似。
FCVT.S.H或FCVT.H.S分别把一个半精度浮点数转换为一个单精度浮点数、以及相反的操作。
如果存在D扩展,FCVT.D.H或FCVT.H.D分别把一个半精度浮点数转换为一个双精度浮点数、以及相反的操作。
如果存在Q扩展,FCVT.Q.H或FCVT.H.Q分别把一个半精度浮点数转换为一个四精度浮点数、以及相反的操作。
% New floating-point-to-floating-point conversion instructions are added.  These
% instructions are defined analogously to the double-precision
% floating-point-to-floating-point conversion instructions.
% FCVT.S.H or FCVT.H.S converts a half-precision floating-point number to
% a single-precision floating-point number, or vice-versa, respectively.
% If the D extension is present, FCVT.D.H or FCVT.H.D converts a half-precision
% floating-point number to a double-precision floating-point number, or
% vice-versa, respectively.
% If the Q extension is present, FCVT.Q.H or FCVT.H.Q converts a half-precision
% floating-point number to a quad-precision floating-point number, or
% vice-versa, respectively.

\vspace{-0.2in}
\begin{center}
\begin{tabular}{R@{}F@{}R@{}R@{}F@{}R@{}O}
\\
\instbitrange{31}{27} &
\instbitrange{26}{25} &
\instbitrange{24}{20} &
\instbitrange{19}{15} &
\instbitrange{14}{12} &
\instbitrange{11}{7} &
\instbitrange{6}{0} \\
\hline
\multicolumn{1}{|c|}{funct5} &
\multicolumn{1}{c|}{fmt} &
\multicolumn{1}{c|}{rs2} &
\multicolumn{1}{c|}{rs1} &
\multicolumn{1}{c|}{rm} &
\multicolumn{1}{c|}{rd} &
\multicolumn{1}{c|}{opcode} \\
\hline
5 & 2 & 5 & 5 & 3 & 5 & 7 \\
FCVT.S.H & S & H & src & RM  & dest & OP-FP  \\
FCVT.H.S & H & S & src & RM  & dest & OP-FP  \\
FCVT.D.H & D & H & src & RM  & dest & OP-FP  \\
FCVT.H.D & H & D & src & RM  & dest & OP-FP  \\
FCVT.Q.H & Q & H & src & RM  & dest & OP-FP  \\
FCVT.H.Q & H & Q & src & RM  & dest & OP-FP  \\
\end{tabular}
\end{center}

浮点到浮点的符号注入指令,FSGNJ.H、FSGNJN.H和FSGNJX.H的定义与单精度符号注入指令类似。
% Floating-point to floating-point sign-injection instructions, FSGNJ.H,
% FSGNJN.H, and FSGNJX.H are defined analogously to the single-precision
% sign-injection instruction.

\vspace{-0.2in}
\begin{center}
\begin{tabular}{R@{}F@{}R@{}R@{}F@{}R@{}O}
\\
\instbitrange{31}{27} &
\instbitrange{26}{25} &
\instbitrange{24}{20} &
\instbitrange{19}{15} &
\instbitrange{14}{12} &
\instbitrange{11}{7} &
\instbitrange{6}{0} \\
\hline
\multicolumn{1}{|c|}{funct5} &
\multicolumn{1}{c|}{fmt} &
\multicolumn{1}{c|}{rs2} &
\multicolumn{1}{c|}{rs1} &
\multicolumn{1}{c|}{rm} &
\multicolumn{1}{c|}{rd} &
\multicolumn{1}{c|}{opcode} \\
\hline
5 & 2 & 5 & 5 & 3 & 5 & 7 \\
FSGNJ & H & src2 & src1 & J[N]/JX & dest & OP-FP  \\
\end{tabular}
\end{center}

提供了在浮点和整数寄存器之间移动位式样的指令。
FMV.X.H把浮点寄存器{\em rs1}中的半精度值按照IEEE 754-2008标准编码的一种表示形式移动到整数寄存器{\em rd}中,
并把浮点数的符号位复制填充到高XLEN-16位。
% Instructions are provided to move bit patterns between the floating-point and
% integer registers.
% FMV.X.H moves the half-precision value in floating-point register {\em rs1} to
% a representation in IEEE 754-2008 standard encoding in integer register {\em
% rd}, filling the upper XLEN-16 bits with copies of the floating-point number's
% sign bit.

FMV.H.X把按照IEEE 754-2008标准编码的半精度值从整数寄存器{\em rs1}的低16位移动到浮点寄存器{\em rd},并把结果NaN装箱。
% FMV.H.X moves the half-precision value encoded in IEEE 754-2008 standard
% encoding from the lower 16 bits of integer register {\em rs1} to the
% floating-point register {\em rd}, NaN-boxing the result.

FMV.X.H和FMV.H.X不改变正在被传输的位;特别地,非典型NaN的有效载荷被保留。
% FMV.X.H and FMV.H.X do not modify the bits being transferred; in particular,
% the payloads of non-canonical NaNs are preserved.

\vspace{-0.2in}
\begin{center}
\begin{tabular}{R@{}F@{}R@{}R@{}F@{}R@{}O}
\\
\instbitrange{31}{27} &
\instbitrange{26}{25} &
\instbitrange{24}{20} &
\instbitrange{19}{15} &
\instbitrange{14}{12} &
\instbitrange{11}{7} &
\instbitrange{6}{0} \\
\hline
\multicolumn{1}{|c|}{funct5} &
\multicolumn{1}{c|}{fmt} &
\multicolumn{1}{c|}{rs2} &
\multicolumn{1}{c|}{rs1} &
\multicolumn{1}{c|}{rm} &
\multicolumn{1}{c|}{rd} &
\multicolumn{1}{c|}{opcode} \\
\hline
5 & 2 & 5 & 5 & 3 & 5 & 7 \\
FMV.X.H & H & 0    & src  & 000  & dest & OP-FP  \\
FMV.H.X & H & 0    & src  & 000  & dest & OP-FP  \\
\end{tabular}
\end{center}

\section{半精度浮点比较指令}
% \section{Half-Precision Floating-Point Compare Instructions}

半精度浮点比较指令的定义与其对应的单精度版本类似,但是在半精度操作数上操作。
% The half-precision floating-point compare instructions are
% defined analogously to their single-precision counterparts, but operate on
% half-precision operands.

\vspace{-0.2in}
\begin{center}
\begin{tabular}{S@{}F@{}R@{}R@{}F@{}R@{}O}
\\
\instbitrange{31}{27} &
\instbitrange{26}{25} &
\instbitrange{24}{20} &
\instbitrange{19}{15} &
\instbitrange{14}{12} &
\instbitrange{11}{7} &
\instbitrange{6}{0} \\
\hline
\multicolumn{1}{|c|}{funct5} &
\multicolumn{1}{c|}{fmt} &
\multicolumn{1}{c|}{rs2} &
\multicolumn{1}{c|}{rs1} &
\multicolumn{1}{c|}{rm} &
\multicolumn{1}{c|}{rd} &
\multicolumn{1}{c|}{opcode} \\
\hline
5 & 2 & 5 & 5 & 3 & 5 & 7 \\
FCMP & H & src2 & src1 & EQ/LT/LE & dest & OP-FP  \\
\end{tabular}
\end{center}

\section{半精度浮点分类指令}
% \section{Half-Precision Floating-Point Classify Instruction}

半精度浮点分类指令,FCLASS.H,其定义与其对应的单精度版本类似,但是在半精度操作数上操作。
% The half-precision floating-point classify instruction, FCLASS.H, is
% defined analogously to its single-precision counterpart, but operates on
% half-precision operands.

\vspace{-0.2in}
\begin{center}
\begin{tabular}{S@{}F@{}R@{}R@{}F@{}R@{}O}
\\
\instbitrange{31}{27} &
\instbitrange{26}{25} &
\instbitrange{24}{20} &
\instbitrange{19}{15} &
\instbitrange{14}{12} &
\instbitrange{11}{7} &
\instbitrange{6}{0} \\
\hline
\multicolumn{1}{|c|}{funct5} &
\multicolumn{1}{c|}{fmt} &
\multicolumn{1}{c|}{rs2} &
\multicolumn{1}{c|}{rs1} &
\multicolumn{1}{c|}{rm} &
\multicolumn{1}{c|}{rd} &
\multicolumn{1}{c|}{opcode} \\
\hline
5 & 2 & 5 & 5 & 3 & 5 & 7 \\
FCLASS & H & 0 & src & 001 & dest & OP-FP  \\
\end{tabular}
\end{center}

\section{用于最小半精度浮点支持的“Zfhmin”标准扩展}
% \section{``Zfhmin'' Standard Extension for Minimal Half-Precision Floating-Point Support}

本节描述了Zfhmin标准扩展,它为16位半精度二进制浮点指令提供了最低限度的支持。
Zfhmin扩展是Zfh扩展的一个子集,仅由数据传输与转换指令组成。像Zfh一样,Zfhmin扩展依赖于单精度浮点扩展,F。
预想的Zfhmin软件主要使用半精度格式存储,而以更高精度执行大多数运算。
% This section describes the Zfhmin standard extension, which provides minimal
% support for 16-bit half-precision binary floating-point instructions.
% The Zfhmin extension is a subset of the Zfh extension, consisting only
% of data transfer and conversion instructions.
% Like Zfh, the Zfhmin extension depends on the single-precision floating-point
% extension, F.
% The expectation is that Zfhmin software primarily uses the half-precision
% format for storage, performing most computation in higher precision.

Zfhmin扩展包括下列来自Zfh的指令:FLH、FSH、FMV.X.H、FMV.H.X、FCVT.S.H和FCVT.H.S。
如果D扩展存在,也包括FCVT.D.H和FCVT.H.D指令。如果Q扩展存在,还额外包括了FCVT.Q.H和FCVT.H.Q指令。
% The Zfhmin extension includes the following instructions from the Zfh
% extension: FLH, FSH, FMV.X.H, FMV.H.X, FCVT.S.H, and FCVT.H.S.
% If the D extension is present, the FCVT.D.H and FCVT.H.D instructions are
% also included.
% If the Q extension is present, the FCVT.Q.H and FCVT.H.Q instructions are
% additionally included.

\begin{commentary}
  Zfhmin不包括FSGNJ.H指令,因为将半精度值在浮点寄存器之间移动,使用FSGNJ.S指令代替就足够了。
% Zfhmin does not include the FSGNJ.H instruction, because it suffices to
% instead use the FSGNJ.S instruction to move half-precision values between
% floating-point registers.
\end{commentary}

\begin{commentary}
  半精度加法、减法、乘法、除法,和平方根操作可以通过把半精度操作数转化为单精度、
  使用单精度算术执行操作再转化回单精度~\cite{roux:hal-01091186}来忠实地模拟。
  对于RNE和RMM舍入模式,使用这种方法执行半精度融合乘加,在某些输入时会产生1-ulp的误差。
% Half-precision addition, subtraction, multiplication, division, and
% square-root operations can be faithfully emulated by converting the
% half-precision operands to single-precision, performing the operation
% using single-precision arithmetic, then converting back to
% half-precision~\cite{roux:hal-01091186}.
% Performing half-precision fused multiply-addition using this method incurs
% a 1-ulp error on some inputs for the RNE and RMM rounding modes.

把8位或16位整数转化为半精度,可以通过先转化为单精度、再转化到半精度来模拟。
从32位整数的转换可以用首先转化为双精度来模拟。如果D扩展不存在,而且在RNE或RMM模式下1-ulp的误差是可容忍的,
那么32位整数也可以先被转化为单精度。同样的标注也适用于没有Q扩展时的从64位整数的转换。
% Conversion from 8- or 16-bit integers to half-precision can be emulated by
% first converting to single-precision, then converting to half-precision.
% Conversion from 32-bit integer can be emulated by first converting to
% double-precision.
% If the D extension is not present and a 1-ulp error under RNE or RMM is
% tolerable, 32-bit integers can be first converted to single-precision instead.
% The same remark applies to conversions from 64-bit integers without the Q
% extension.
\end{commentary}

\chapter{RVWMO内存一致性模型(2.0版本)}
% \chapter{RVWMO Memory Consistency Model, Version 2.0}
\label{ch:memorymodel}

这章定义了RISC-V内存一致性模型。内存一致性模型是一组规则的集合,它指定了可以被内存的加载所返回的值。
RISC-V使用一个叫做“RVWMO”(RISC-V弱内存次序)的内存模型,它被设计来为架构提供构建高性能可扩展设计的灵活性,
并同时支持易处理的编程模型。
% This chapter defines the RISC-V memory consistency model.
% A memory consistency model is a set of rules specifying the values that can be returned by loads of memory.
% RISC-V uses a memory model called ``RVWMO'' (RISC-V Weak Memory Ordering) which is designed to provide flexibility for architects to build high-performance scalable designs while simultaneously supporting a tractable programming model.

在RVWMO下,从同一硬件线程的其它内存指令的视角来看,运行在单一硬件线程的代码看似有序地执行,但是从另一个硬件线程的内存指令,可能观察到第一个硬件线程的内存指令正在以一种不同的次序被执行。
因此,多线程代码可能需要显式的同步,来保证来自不同硬件线程的内存指令之间的次序。基础RISC-V ISA出于这个目的,提供了一个FENCE指令,它描述在~\ref{sec:fence}节中,同时原子扩展“A”额外定义了“加载-保留”/“存储-条件”和原子性“读-修改-写”指令。
% Under RVWMO, code running on a single hart appears to execute in order from the perspective of other memory instructions in the same hart, but memory instructions from another hart may observe the memory instructions from the first hart being executed in a different order.
% Therefore, multithreaded code may require explicit synchronization to guarantee ordering between memory instructions from different harts.
% The base RISC-V ISA provides a FENCE instruction for this purpose, described in Section~\ref{sec:fence}, while the atomics extension ``A'' additionally defines load-reserved/store-conditional and atomic read-modify-write instructions.

用于未对齐原子性的标准ISA扩展“Zam”(第~\ref{sec:zam}章)和用于全存储排序的标准ISA扩展“Ztso”(第~\ref{sec:ztso}章)为RVWMO增加了特定于那些扩展的额外的规则。
% The standard ISA extension for misaligned atomics ``Zam'' (Chapter~\ref{sec:zam}) and the standard ISA extension for total store ordering ``Ztso'' (Chapter~\ref{sec:ztso}) augment RVWMO with additional rules specific to those extensions.

这个规范的附录提供了公理化的和操作规范化的内存一致性以及补充说明材料。
% The appendices to this specification provide both axiomatic and operational formalizations of the memory consistency model as well as additional explanatory material.

\begin{commentary}
  这章定义了用于规则的主内存操作的内存模型。使用I/O内存、指令获取、FENCE.I、页表游走和SFENCE.VMA的内存模型交互还没有被规范化。上述中的一些或全部可能在这个规范未来的修订中被规范化。RV128基础ISA和未来的ISA扩展(例如“V”向量扩展和“J”JIT扩展)也将需要被并入未来的修订中。
  % This chapter defines the memory model for regular main memory operations.  The interaction of the memory model with I/O memory, instruction fetches, FENCE.I, page table walks, and SFENCE.VMA is not (yet) formalized.  Some or all of the above may be formalized in a future revision of this specification.  The RV128 base ISA and future ISA extensions such as the ``V'' vector and ``J'' JIT extensions will need to be incorporated into a future revision as well.

  支持不同宽度同时进行重叠的内存访问的内存一致性模型仍然是学术研究的一个积极区域,并且还没有被完全理解。关于不同尺寸的内存访问如何在RVWMO下交互的细节是我们当前能做到的最好的,但是当新问题被揭露后,它们将不得不再修订。
  % Memory consistency models supporting overlapping memory accesses of different widths simultaneously remain an active area of academic research and are not yet fully understood.  The specifics of how memory accesses of different sizes interact under RVWMO are specified to the best of our current abilities, but they are subject to revision should new issues be uncovered.
\end{commentary}

\section{RVWMO内存模型的定义}
% \section{Definition of the RVWMO Memory Model}
\label{sec:rvwmo}

{\em 全局内存次序},即所有硬件线程所产生的内存操作的总体次序,据此定义了RVWMO内存模型。总的来说,一个多线程程序有许多种不同可能的执行,而每种执行有其自己相应的全局内存次序。
% The RVWMO memory model is defined in terms of the {\em global memory order}, a total ordering of the memory operations produced by all harts.
% In general, a multithreaded program has many different possible executions, with each execution having its own corresponding global memory order.

全局内存次序定义在由内存指令生成的原始的加载和存储操作之上。然后它将受到本章余下部分定义的约束的限制。任何满足所有内存模型约束的执行都是合法的执行(至少在内存模型所关注的方面如此)。
% The global memory order is defined over the primitive load and store operations generated by memory instructions.
% It is then subject to the constraints defined in the rest of this chapter.
% Any execution satisfying all of the memory model constraints is a legal execution (as far as the memory model is concerned).

\subsection*{内存模型原语}
% \subsection*{Memory Model Primitives}
\label{sec:rvwmo:primitives}
在内存操作上的{\em 程序次序}反映了生成每个加载和存储的指令在硬件线程的动态指令流中的次序;即,简单有序处理器将执行的该硬件线程的指令的次序。
% The {\em program order} over memory operations reflects the order in which the instructions that generate each load and store are logically laid out in that hart's dynamic instruction stream; i.e., the order in which a simple in-order processor would execute the instructions of that hart.

内存访问指令造成了内存操作。{\em 内存操作}可以是一个{\em 加载操作}、{\em 存储操作},或者是二者同时。
所有的内存操作都是“单拷贝原子”的:它们可以永远不会被观察到处于一种部分完成的状态。
% Memory-accessing instructions give rise to {\em memory operations}.
% A memory operation can be either a {\em load operation}, a {\em store operation}, or both simultaneously.
% All memory operations are single-copy atomic: they can never be observed in a partially complete state.

在RV32GC和RV64GC的指令之中,每个对齐的内存指令都确切造成一次内存操作和两个异常。
首先,一次不成功的SC指令不会造成任何内存操作。第二,如果XLEN$<$64,就像~\ref{fld_fsd}节中所陈述的和下面澄清的那样,那么FLD和FSD指令每次可以造成多个内存操作。一个对齐的AMO指令造成单次内存操作,它同时是一个加载操作和一个存储操作。
% Among instructions in RV32GC and RV64GC, each aligned memory instruction gives rise to exactly one memory operation, with two exceptions.
% First, an unsuccessful SC instruction does not give rise to any memory operations.
% Second, FLD and FSD instructions may each give rise to multiple memory operations if XLEN$<$64, as stated in Section~\ref{fld_fsd} and clarified below.
% An aligned AMO gives rise to a single memory operation that is both a load operation and a store operation simultaneously.

\begin{commentary}
  RV128基础指令集中的指令和诸如V(向量)和P(SMID)的未来ISA扩展中的指令可能造成多个内存操作。然而对于这些扩展的内存模型还没有被规范化。
  % Instructions in the RV128 base instruction set and in future ISA extensions such as V (vector) and P (SIMD) may give rise to multiple memory operations.  However, the memory model for these extensions has not yet been formalized.
\end{commentary}

未对齐的加载或存储指令可能被分解为任意粒度的组件内存操作的集合。对于XLEN$<$64的FLD或FSD指令也可能被分解为任意粒度的组件内存操作的集合。
通过这样的指令生成的内存操作并不按照相互间的程序次序被排序,但是它们可以根据“在程序次序中位于先前或后续指令所生成的内存操作”来正常地排序。原子扩展“A”完全不需要执行环境提供未对齐的原子指令;
然而,如果通过“Zam”扩展支持未对齐的原子指令,那么LR、SC和AMO可以按照未对齐原子指令的原子性公理的约束而被分解,该约束定义在第~\ref{sec:zam}章中。
% A misaligned load or store instruction may be decomposed into a set of component memory operations of any granularity.
% An FLD or FSD instruction for which XLEN$<$64 may also be decomposed into a set of component memory operations of any granularity.
% The memory operations generated by such instructions are not ordered with respect to each other in program order, but they are ordered normally with respect to the memory operations generated by preceding and subsequent instructions in program order.
% The atomics extension ``A'' does not require execution environments to support misaligned atomic instructions at all; however, if misaligned atomics are supported via the ``Zam'' extension, LRs, SCs, and AMOs may be decomposed subject to the constraints of the atomicity axiom for misaligned atomics, which is defined in Chapter~\ref{sec:zam}.

\begin{commentary}
  将未对齐内存操作分解下到字节粒度有利于在原本不支持未对齐访问的实现上进行模拟。例如,这种实现可能简单地逐个迭代未对齐的访问的字节。
  % The decomposition of misaligned memory operations down to byte granularity facilitates emulation on implementations that do not natively support misaligned accesses.
  % Such implementations might, for example, simply iterate over the bytes of a misaligned access one by one.
\end{commentary}

如果在程序次序中,LR先于SC,并且在它们之间没有其它的LR或SC指令,那么LR指令和SC指令被称作成对的;对应的内存操作也被称为成对的(除了在SC失败的情况中,那里没有产生存储操作)。决定一个SC是否必定成功、可能成功、或者必定失败的条件的完整列表定义在~\ref{sec:lrsc}节中。
% An LR instruction and an SC instruction are said to be {\em paired} if the LR precedes the SC in program order and if there are no other LR or SC instructions in between; the corresponding memory operations are said to be paired as well (except in case of a failed SC, where no store operation is generated).
% The complete list of conditions determining whether an SC must succeed, may succeed, or must fail is defined in Section~\ref{sec:lrsc}.

加载和存储操作也可以从下列集合中携带一个或多个次序注释:“acquire-RCpc”、“acquire-RCsc”、“release-RCpc”和“release-RCsc”。
一个设置了{\em aq}的AMO或LR指令具有“acquire-RCsc”注释。
一个设置了{\em rl}的AMO或SC指令具有“releaseRCsc”注释。
同时设置了{\em aq}和{\em rl}的AMO、LR或SC指令也同时有“acquire-RCsc”和“release-RCsc”注释。
% Load and store operations may also carry one or more ordering annotations from the following set: ``acquire-RCpc'', ``acquire-RCsc'', ``release-RCpc'', and ``release-RCsc''.
% An AMO or LR instruction with {\em aq} set has an ``acquire-RCsc'' annotation.
% An AMO or SC instruction with {\em rl} set has a ``release-RCsc'' annotation.
% An AMO, LR, or SC instruction with both {\em aq} and {\em rl} set has both ``acquire-RCsc'' and ``release-RCsc'' annotations.

为了方便,我们使用术语“acquire注释”来指代一个acquire-RCpc注释或者一个acquire-RCsc注释。
类似地,用“release注释”指代一个release-RCpc注释或者一个release-RCsc注释。
用“RCpc注释”指代一个acquire-RCpc注释或者一个releaseRCpc注释。
用“RCsc注释”指代一个acquire-RCsc注释或者一个release-RCsc注释。
% For convenience, we use the term ``acquire annotation'' to refer to an acquire-RCpc annotation or an acquire-RCsc annotation.
% Likewise, a ``release annotation'' refers to a release-RCpc annotation or a release-RCsc annotation.
% An ``RCpc annotation'' refers to an acquire-RCpc annotation or a release-RCpc annotation.
% An ``RCsc annotation'' refers to an acquire-RCsc annotation or a release-RCsc annotation.

\begin{commentary}
  在内存模型文献中,术语“RCpc”代表带有与处理器一致的同步操作的释放一致性,而术语“RCsc”代表带有顺序一致的同步操作的释放一致性~\cite{Gharachorloo90memoryconsistency}。
  % In the memory model literature, the term ``RCpc'' stands for release consistency with processor-consistent synchronization operations, and the term ``RCsc'' stands for release consistency with sequentially consistent synchronization operations~\cite{Gharachorloo90memoryconsistency}.

  虽然在文献中对于acquire注释和release注释有许多不同的定义,在RVWMO的上下文中,这些术语由保留的程序次序规则\ref{ppo:acquire}-\ref{ppo:rcsc}简洁而完整地定义。
  % While there are many different definitions for acquire and release annotations in the literature, in the context of RVWMO these terms are concisely and completely defined by Preserved Program Order rules \ref{ppo:acquire}--\ref{ppo:rcsc}.

  “RCpc”注释目前只被用在各标准扩展“Ztso”(第~\ref{sec:ztso}章)被隐式地分配给每个内存访问时。甚至,尽管ISA目前既不包括原生的“加载-获取”或者“存储-释放”指令,也因此不包括其中的RCpc变量,RVWMO模型本身被设计为向前兼容的,可以在未来的扩展中把上面的任何或者所有的潜在的附件兼容进ISA中。
  % ``RCpc'' annotations are currently only used when implicitly assigned to every memory access per the standard extension ``Ztso'' (Chapter~\ref{sec:ztso}).  Furthermore, although the ISA does not currently contain native load-acquire or store-release instructions, nor RCpc variants thereof, the RVWMO model itself is designed to be forwards-compatible with the potential addition of any or all of the above into the ISA in a future extension.
\end{commentary}

\subsection*{句法依赖}
% \subsection*{Syntactic Dependencies}
\label{sec:memorymodel:dependencies}
RVWMO内存模型的定义部分依赖于句法依赖的概念,后者定义如下。
% The definition of the RVWMO memory model depends in part on the notion of a syntactic dependency, defined as follows.

在定义的依赖的上下文中,“寄存器”或者指代一个完整的通用目的寄存器,或者指代CSR的某些部分,或者指代一个完整的CSR。通过CSR追踪的依赖的粒度特定于每个CSR,并在~\ref{sec:csr-granularity}节中定义。
% In the context of defining dependencies, a ``register'' refers either to an entire general-purpose register, some portion of a CSR, or an entire CSR.  The granularity at which dependencies are tracked through CSRs is specific to each CSR and is defined in Section~\ref{sec:csr-granularity}.

句法依赖的定义依据于指令的源寄存器、指令的目的寄存器,和指令从它们的{\em 源寄存器}到{\em 目的寄存器}携带依赖的方式。本节提供了一个所有这些术语的通用的定义;然而,~\ref{sec:source-dest-regs}节提供了每个指令的详细信息的一个完整的列表。
% Syntactic dependencies are defined in terms of instructions' {\em source registers}, instructions' {\em destination registers}, and the way instructions {\em carry a dependency} from their source registers to their destination registers.
% This section provides a general definition of all of these terms; however, Section~\ref{sec:source-dest-regs} provides a complete listing of the specifics for each instruction.

总体上,对于一个指令$i$,如果满足任意下列条件,{\em 源寄存器}是寄存器$r$,而不是{\tt x0}:
% In general, a register $r$ other than {\tt x0} is a {\em source register} for an instruction $i$ if any of the following hold:
\begin{itemize}
  \item 在$i$的操作码中,{\em rs1}、{\em rs2}或者{\em rs3}被设置为$r$  % In the opcode of $i$, {\em rs1}, {\em rs2}, or {\em rs3} is set to $r$
  \item $i$是一个CSR指令,并且在$i$的操作码中,{\em csr}被设置为$r$(除非$i$是CSRRW或CSRRWI,并且{\em rd}被设置为{\em x0})  % $i$ is a CSR instruction, and in the opcode of $i$, {\em csr} is set to $r$, unless $i$ is CSRRW or CSRRWI and {\em rd} is set to {\tt x0}
  \item $r$是一个CSR,且对于$i$,$r$是一个隐式的源寄存器,就像~\ref{sec:source-dest-regs}节中定义的那样  % $r$ is a CSR and an implicit source register for $i$, as defined in Section~\ref{sec:source-dest-regs}
  \item 对于$i$,$r$是一个作为另一个源寄存器的别名的CSR % $r$ is a CSR that aliases with another source register for $i$
\end{itemize}
内存指令也进一步指定了哪个源寄存器是{\em 地址源寄存器},以及哪个是{\em 数据源寄存器}。
% Memory instructions also further specify which source registers are {\em address source registers} and which are {\em data source registers}.

总体上,对于一个指令$i$,如果满足任意下列条件,{\em 目的寄存器}是寄存器$r$,而不是{\em x0}:
% In general, a register $r$ other than {\tt x0} is a {\em destination register} for an instruction $i$ if any of the following hold:
\begin{itemize}
  \item 在i的操作码中,rd被设置为r  % In the opcode of $i$, {\em rd} is set to $r$
  \item $i$是一个CSR指令,且在$i$的操作码中,{\em csr}被设置为$r$(除非$i$是CSRRS或CSRRC,并且{\em rs1}被设置为{\em x0};或者$i$是CSRRI或CSRRCI,并且uimm[4:0]被设置为0)  % $i$ is a CSR instruction, and in the opcode of $i$, {\em csr} is set to $r$, unless $i$ is CSRRS or CSRRC and {\em rs1} is set to {\tt x0} or $i$ is CSRRSI or CSRRCI and uimm[4:0] is set to zero.
  \item $r$是一个CSR,并且对于$i$,$r$是一个隐式的目的寄存器,正如~\ref{sec:source-dest-regs}节中定义的那样  % $r$ is a CSR and an implicit destination register for $i$, as defined in Section~\ref{sec:source-dest-regs}
  \item 对于$i$,$r$是一个作为另一个目的寄存器的别名的CSR  % $r$ is a CSR that aliases with another destination register for $i$
\end{itemize}

大多数非内存指令携带有从它们的每个源寄存器到它们的每个目的寄存器的依赖。然而,对于这个规则是有例外的;见~\ref{sec:source-dest-regs}节。
% Most non-memory instructions {\em carry a dependency} from each of their source registers to each of their destination registers.
% However, there are exceptions to this rule; see Section~\ref{sec:source-dest-regs}

如果满足下列之一,通过$i$的目的寄存器$s$和$j$的源寄存器$r$,指令$j$有一个关于指令$i$的{\em 句法依赖}:
% Instruction $j$ has a {\em syntactic dependency} on instruction $i$ via destination register $s$ of $i$ and source register $r$ of $j$ if either of the following hold:
\begin{itemize}
  \item $s$与$r$相同,并且按程序次序,在i和j之间没有指令把$r$作为目的寄存器  % $s$ is the same as $r$, and no instruction program-ordered between $i$ and $j$ has $r$ as a destination register
  \item 按程序次序,在$i$和$j$之间有指令$m$,使得满足所有下列条件: % There is an instruction $m$ program-ordered between $i$ and $j$ such that all of the following hold:
    \begin{enumerate}
      \item 通过目的寄存器$q$和源寄存器$r$,$j$有一个关于$m$的句法依赖  % $j$ has a syntactic dependency on $m$ via destination register $q$ and source register $r$
      \item 通过目的寄存器$s$和源寄存器$p$,$m$有一个关于$i$的句法依赖   % $m$ has a syntactic dependency on $i$ via destination register $s$ and source register $p$
      \item $m$携带有从$p$到$q$的依赖  % $m$ carries a dependency from $p$ to $q$
    \end{enumerate}
\end{itemize}

最后,在下面的定义中,$a$和$b$是两个内存操作,而$i$和$j$是分别生成$a$和$b$的指令。
% Finally, in the definitions that follow, let $a$ and $b$ be two memory operations, and let $i$ and $j$ be the instructions that generate $a$ and $b$, respectively.

$b$有一个关于a的{\em 句法地址依赖},如果$r$是$j$的一个地址源寄存器,而$j$通过源寄存器$r$有一个关于$i$的句法依赖
% $b$ has a {\em syntactic address dependency} on $a$ if $r$ is an address source register for $j$ and $j$ has a syntactic dependency on $i$ via source register $r$

$b$有一个关于$a$的{\em 句法数据依赖},如果$b$是一个存储操作,$r$是$j$的一个数据源寄存器,且$j$通过源寄存器$r$有一个关于$i$的句法依赖
$b$ has a {\em syntactic data dependency} on $a$ if $b$ is a store operation, $r$ is a data source register for $j$, and $j$ has a syntactic dependency on $i$ via source register $r$

$b$有一个关于$a$的{\em 句法控制依赖},如果按程序次序,在$i$和$j$之间有一个指令$m$,使$m$是一个分支、或者间接跳转,并且$m$有一个关于$i$的句法依赖。
% $b$ has a {\em syntactic control dependency} on $a$ if there is an instruction $m$ program-ordered between $i$ and $j$ such that $m$ is a branch or indirect jump and $m$ has a syntactic dependency on $i$.

\begin{commentary}
  总的来说,非AMO加载指令没有数据源寄存器,而无条件非AMO存储指令没有目的寄存器。然而,一个成功的SC指令被认为在{\em rd}中指定了寄存器作为目的寄存器,并因此对于一条指令,可能有一个关于程序次序中先于它的成功SC指令的句法依赖。
  % Generally speaking, non-AMO load instructions do not have data source registers, and unconditional non-AMO store instructions do not have destination registers.  However, a successful SC instruction is considered to have the register specified in {\em rd} as a destination register, and hence it is possible for an instruction to have a syntactic dependency on a successful SC instruction that precedes it in program order.
\end{commentary}

\subsection*{保留的程序次序}
% \subsection*{Preserved Program Order}
对于任意给定的程序执行,全局内存次序遵循着各个硬件线程的内存次序中的一部分(但不是所有)。必须被全局内存次序所遵循的程序次序子集被称为{\em 保留的程序次序}。
% The global memory order for any given execution of a program respects some but not all of each hart's program order.
% The subset of program order that must be respected by the global memory order is known as {\em preserved program order}.

% \newcommand{\ppost}{$b$ is a store, and $a$ and $b$ access overlapping memory addresses}
% \newcommand{\ppofence}{There is a FENCE instruction that orders $a$ before $b$}
% \newcommand{\ppoacquire}{$a$ has an acquire annotation}
% \newcommand{\pporelease}{$b$ has a release annotation}
% \newcommand{\pporcsc}{$a$ and $b$ both have RCsc annotations}
% \newcommand{\ppoamoforward}{$a$ is generated by an AMO or SC instruction, $b$ is a load, and $b$ returns a value written by $a$}
% \newcommand{\ppoaddr}{$b$ has a syntactic address dependency on $a$}
% \newcommand{\ppodata}{$b$ has a syntactic data dependency on $a$}
% \newcommand{\ppoctrl}{$b$ is a store, and $b$ has a syntactic control dependency on $a$}
% \newcommand{\ppopair}{$a$ is paired with $b$}
% \newcommand{\ppordw}{$a$ and $b$ are loads, $x$ is a byte read by both $a$ and $b$, there is no store to $x$ between $a$ and $b$ in program order, and $a$ and $b$ return values for $x$ written by different memory operations}
% \newcommand{\ppoaddrdatarfi}{$b$ is a load, and there exists some store $m$ between $a$ and $b$ in program order such that $m$ has an address or data dependency on $a$, and $b$ returns a value written by $m$}
% \newcommand{\ppoaddrpo}{$b$ is a store, and there exists some instruction $m$ between $a$ and $b$ in program order such that $m$ has an address dependency on $a$}
%\newcommand{\ppoctrlcfence}{$a$ and $b$ are loads, $b$ has a syntactic control dependency on $a$, and there exists a {\tt fence.i} between the branch used to form the control dependency and $b$ in program order}
%\newcommand{\ppoaddrpocfence}{$a$ is a load, there exists an instruction $m$ which has a syntactic address dependency on $a$, and there exists a {\tt fence.i} between $m$ and $b$ in program order}

保留的程序次序的完整定义如下(且注意AMO是同时进行加载和存储):在保留的程序次序中,内存操作$a$先于内存操作$b$(并因此在全局内存次序中也是如此),如果以程序次序$a$先于$b$,$a$和$b$都访问常规的主内存(而不是I/O区域),并且满足任何下列:
% The complete definition of preserved program order is as follows (and note that AMOs are simultaneously both loads and stores):
% memory operation $a$ precedes memory operation $b$ in preserved program order (and hence also in the global memory order) if $a$ precedes $b$ in program order, $a$ and $b$ both access regular main memory (rather than I/O regions), and any of the following hold:

\begin{itemize}
  \item 重叠地址次序: % Overlapping-Address Orderings:
    \begin{enumerate}
      \item\label{ppo:->st} \ppost
      \item\label{ppo:rdw} \ppordw
      \item\label{ppo:amoforward} \ppoamoforward
    \end{enumerate}
  \item 显示同步  % Explicit Synchronization
    \begin{enumerate}[resume]
      \item\label{ppo:fence} \ppofence
      \item\label{ppo:acquire} \ppoacquire
      \item\label{ppo:release} \pporelease
      \item\label{ppo:rcsc} \pporcsc
      \item\label{ppo:pair} \ppopair
    \end{enumerate}
  \item 句法依赖 %  Syntactic Dependencies
    \begin{enumerate}[resume]
      \item\label{ppo:addr} \ppoaddr
      \item\label{ppo:data} \ppodata
      \item\label{ppo:ctrl} \ppoctrl
    \end{enumerate}
  \item 流水线依赖  % Pipeline Dependencies
    \begin{enumerate}[resume]
      \item\label{ppo:addrdatarfi} \ppoaddrdatarfi
      \item\label{ppo:addrpo} \ppoaddrpo
      %\item\label{ppo:ctrlcfence} \ppoctrlcfence
      %\item\label{ppo:addrpocfence} \ppoaddrpocfence
    \end{enumerate}
\end{itemize}

\subsection*{内存模型公理}
% \subsection*{Memory Model Axioms}

只有当存在一个符合保留的程序次序并且满足{\em 加载值公理}、{\em 原子性公理}和{\em 进度公理}的全局内存次序时,RISC-V程序的执行遵循RVWMO内存一致性模型。
% An execution of a RISC-V program obeys the RVWMO memory consistency model only if there exists a global memory order conforming to preserved program order and satisfying the {\em load value axiom}, the {\em atomicity axiom}, and the {\em progress axiom}.

% \newcommand{\loadvalueaxiom}{
%   每个加载$i$的各个位所返回的值,由下列存储中在全局内存次序中最近的那个写到该位:
%   % Each byte of each load $i$ returns the value written to that byte by the store that is the latest in global memory order among the following stores:
%   \begin{enumerate}
%     \item 写该位,并且在全局内存次序中先于$i$的存储  % Stores that write that byte and that precede $i$ in the global memory order
%     \item 写该位,并且在程序次序中先于$i$的存储  % Stores that write that byte and that precede $i$ in program order
%   \end{enumerate}
% }

% \newcommand{\atomicityaxiom}{If $r$ and $w$ are paired load and store operations generated by aligned LR and SC instructions in a hart $h$, $s$ is a store to byte $x$, and $r$ returns a value written by $s$, then $s$ must precede $w$ in the global memory order, and there can be no store from a hart other than $h$ to byte $x$ following $s$ and preceding $w$ in the global memory order.}

% \newcommand{\progressaxiom}{No memory operation may be preceded in the global memory order by an infinite sequence of other memory operations.}

\paragraph{加载值公理}
\label{rvwmo:ax:load}
\loadvalueaxiom

\paragraph{原子性公理}
\label{rvwmo:ax:atom}
\atomicityaxiom

\begin{commentary}
  \nameref{rvwmo:ax:atom}理论上支持不同宽度的LR/SC对,以及不匹配的地址,因为允许实现在这种情况中使SC操作成功。然而,在实际中,我们希望这种式样是稀有的,并且不鼓励使用它们。
  % The \nameref{rvwmo:ax:atom} theoretically supports LR/SC pairs of different widths and to mismatched addresses, since implementations are permitted to allow SC operations to succeed in such cases.  However, in practice, we expect such patterns to be rare, and their use is discouraged.
\end{commentary}

\paragraph{进度公理}
\label{rvwmo:ax:prog}
\progressaxiom

\section{CSR依赖跟踪粒度}
% \section{CSR Dependency Tracking Granularity}
\label{sec:csr-granularity}

\begin{table}[h!]
  \centering
  \begin{tabular}{|l|l|l|}
    \hline
    名称 & 作为独立单元被追踪的部分 & 别称  \\
    \hline
    {\tt fflags} & Bits 4, 3, 2, 1, 0 & {\tt fcsr}  \\
    \hline
    {\tt frm} & CSR整体 & {\tt fcsr} \\
    \hline
    {\tt fcsr} & 位 7-5, 4, 3, 2, 1, 0 & {\tt fflags}, {\tt frm} \\
    \hline
  \end{tabular}
  \caption{通过CSR跟踪的句法依赖的粒度
    % Granularities at which syntactic dependencies are tracked through CSRs
    }
\end{table}

注意:没有列出只读的CSR,因为它们不参与句法依赖的定义。
% Note: read-only CSRs are not listed, as they do not participate in the definition of syntactic dependencies.

\section{源寄存器和目的寄存器列表}
% \section{Source and Destination Register Listings}
\label{sec:source-dest-regs}

这节提供了每个指令的源寄存器和目的寄存器的具体列表。这些列表被用于定义~\ref{sec:memorymodel:dependencies}节中的句法依赖。
% This section provides a concrete listing of the source and destination registers for each instruction.
% These listings are used in the definition of syntactic dependencies in Section~\ref{sec:memorymodel:dependencies}.

术语“累积CSR”被用于描述一种CSR,它同时是一个源寄存器和一个目的寄存器,但是只携带一个从它自己到它自己的依赖。
% The term ``accumulating CSR'' is used to describe a CSR that is both a source and a destination register, but which carries a dependency only from itself to itself.

除非另有说明,指令携带的依赖是:从“源寄存器”列中的各个源寄存器到“目的寄存器”列中的各个目的寄存器的依赖、从“源寄存器”列中的各个源寄存器到“累积CSR”列中的各个CSR的依赖,以及从“累积CSR”列中的各个CSR到其自身的依赖。
% Instructions carry a dependency from each source register in the ``Source Registers'' column to each destination register in the ``Destination Registers'' column, from each source register in the ``Source Registers'' column to each CSR in the ``Accumulating CSRs'' column, and from each CSR in the ``Accumulating CSRs'' column to itself, except where annotated otherwise.

要点:
Key:

$^A$ 地址源寄存器 % Address source register

$^D$ 数据源寄存器 % Data source register

$^\dagger$ 指令没有携带从任何源寄存器到任何目的寄存器的依赖  % The instruction does not carry a dependency from any source register to any destination register

$^\ddagger$ 指令按照指定携带了从源寄存器到目的寄存器的依赖 % The instruction carries dependencies from source register(s) to destination register(s) as specified

\begin{longtable}{p{3cm}|p{3cm}|p{2cm}|p{4cm}|p{4cm}}
  \multicolumn{4}{l}{\bf RV32I 基础整数指令集} \\
  \cline{2-4}
   & 源    & 目的 & 累积 \\
   & 寄存器 & 寄存器   & CSR \\
  \cline{2-4}
   LUI &  & {\em rd} &   & \\
   \cline{2-4}
   AUIPC &  & {\em rd} &   & \\
   \cline{2-4}
   JAL &  & {\em rd} &  & \\
   \cline{2-4}
   JALR$^\dagger$ & {\em rs1} & {\em rd} &  & \\
   \cline{2-4}
   BEQ & {\em rs1}, {\em rs2} &  &   & \\
   \cline{2-4}
   BNE & {\em rs1}, {\em rs2} &  &   & \\
   \cline{2-4}
   BLT & {\em rs1}, {\em rs2} &  &   & \\
   \cline{2-4}
   BGE & {\em rs1}, {\em rs2} &  &   & \\
   \cline{2-4}
   BLTU & {\em rs1}, {\em rs2} &  &   & \\
   \cline{2-4}
   BGEU & {\em rs1}, {\em rs2} &  &   & \\
   \cline{2-4}
   LB$^\dagger$ & {\em rs1}$^A$ & {\em rd} &   & \\
   \cline{2-4}
   LH$^\dagger$ & {\em rs1}$^A$ & {\em rd} &   & \\
   \cline{2-4}
   LW$^\dagger$ & {\em rs1}$^A$ & {\em rd} &   & \\
   \cline{2-4}
   LBU$^\dagger$ & {\em rs1}$^A$ & {\em rd} &   & \\
   \cline{2-4}
   LHU$^\dagger$ & {\em rs1}$^A$ & {\em rd} &   & \\
   \cline{2-4}
   SB & {\em rs1}$^A$, {\em rs2}$^D$ &  &   & \\
   \cline{2-4}
   SH & {\em rs1}$^A$, {\em rs2}$^D$ &  &   & \\
   \cline{2-4}
   SW & {\em rs1}$^A$, {\em rs2}$^D$ &  &   & \\
   \cline{2-4}
   ADDI & {\em rs1} & {\em rd} &   & \\
   \cline{2-4}
   SLTI & {\em rs1} & {\em rd} &   & \\
   \cline{2-4}
   SLTIU & {\em rs1} & {\em rd} &   & \\
   \cline{2-4}
   XORI & {\em rs1} & {\em rd} &   & \\
   \cline{2-4}
   ORI & {\em rs1} & {\em rd} &   & \\
   \cline{2-4}
   ANDI & {\em rs1} & {\em rd} &   & \\
   \cline{2-4}
   SLLI & {\em rs1} & {\em rd} &   & \\
   \cline{2-4}
   SRLI & {\em rs1} & {\em rd} &   & \\
   \cline{2-4}
   SRAI & {\em rs1} & {\em rd} &   & \\
   \cline{2-4}
   ADD & {\em rs1}, {\em rs2} & {\em rd} &   & \\
   \cline{2-4}
   SUB & {\em rs1}, {\em rs2} & {\em rd} &   & \\
   \cline{2-4}
   SLL & {\em rs1}, {\em rs2} & {\em rd} &   & \\
   \cline{2-4}
   SLT & {\em rs1}, {\em rs2} & {\em rd} &   & \\
   \cline{2-4}
   SLTU & {\em rs1}, {\em rs2} & {\em rd} &   & \\
   \cline{2-4}
   XOR & {\em rs1}, {\em rs2} & {\em rd} &   & \\
   \cline{2-4}
   SRL & {\em rs1}, {\em rs2} & {\em rd} &   & \\
   \cline{2-4}
   SRA & {\em rs1}, {\em rs2} & {\em rd} &   & \\
   \cline{2-4}
   OR & {\em rs1}, {\em rs2} & {\em rd} &   & \\
   \cline{2-4}
   AND & {\em rs1}, {\em rs2} & {\em rd} &   & \\
   \cline{2-4}
   FENCE &  &  &   & \\
   \cline{2-4}
   FENCE.I &  &  &   & \\
   \cline{2-4}
   ECALL &  &  &   & \\
   \cline{2-4}
   EBREAK &  &  &   & \\
   \cline{2-4}
\end{longtable}

\begin{tabular}{p{3cm}|p{3cm}|p{2cm}|p{4cm}|p{4cm}}
  \multicolumn{4}{l}{\bf RV32I 基础整数指令集(续)} \\
  \cline{2-4}
    & 源    & 目的 & 累积 \\
    & 寄存器 & 寄存器   & CSR \\
  \cline{2-4}
   CSRRW$^\ddagger$ & {\em rs1}, {\em csr}$^*$ & {\em rd}, {\em csr} & & $^*$除非{\em rd}={\tt x0}  \\
   \cline{2-4}
   CSRRS$^\ddagger$ & {\em rs1}, {\em csr} & {\em rd}$^*$, {\em csr} & & $^*$除非{\em rs1}={\tt x0}  \\
   \cline{2-4}
   CSRRC$^\ddagger$ & {\em rs1}, {\em csr} & {\em rd}$^*$, {\em csr} & & $^*$除非{\em rs1}={\tt x0}  \\
   \cline{2-4}
   \multicolumn{1}{c}{} & \multicolumn{3}{l}{$\ddagger$carries a dependency from {\em rs1} to {\em csr} and from {\em csr} to {\em rd}}
\end{tabular}

\begin{tabular}{p{3cm}|p{3cm}|p{2cm}|p{4cm}|p{4cm}}
  \multicolumn{4}{l}{\bf RV32I 基础整数指令集(续)} \\
  \cline{2-4}
    & 源    & 目的 & 累积 \\
    & 寄存器 & 寄存器   & CSR \\
   \cline{2-4}
   CSRRWI$^\ddagger$ & {\em csr}$^*$ & {\em rd}, {\em csr} & & $^*$unless {\em rd}={\tt x0} \\
   \cline{2-4}
   CSRRSI$^\ddagger$ & {\em csr} & {\em rd}, {\em csr}$^*$ & & $^*$unless uimm[4:0]=0  \\
   \cline{2-4}
   CSRRCI$^\ddagger$ & {\em csr} & {\em rd}, {\em csr}$^*$ & & $^*$unless uimm[4:0]=0  \\
   \cline{2-4}
   \multicolumn{1}{c}{} & \multicolumn{3}{l}{$\ddagger$carries a dependency from {\em csr} to {\em rd}}
\end{tabular}

\begin{tabular}{p{3cm}|p{3cm}|p{2cm}|p{4cm}|p{4cm}}
  \multicolumn{4}{l}{\bf RV64I 基础整数指令集} \\
  \cline{2-4}
    & 源    & 目的 & 累积 \\
    & 寄存器 & 寄存器   & CSR \\
  \cline{2-4}
   LWU$^\dagger$ & {\em rs1}$^A$ & {\em rd} &   & \\
   \cline{2-4}
   LD$^\dagger$ & {\em rs1}$^A$ & {\em rd} &   & \\
   \cline{2-4}
   SD & {\em rs1}$^A$, {\em rs2}$^D$ &  &   & \\
   \cline{2-4}
   SLLI & {\em rs1} & {\em rd} &   & \\
   \cline{2-4}
   SRLI & {\em rs1} & {\em rd} &   & \\
   \cline{2-4}
   SRAI & {\em rs1} & {\em rd} &   & \\
   \cline{2-4}
   ADDIW & {\em rs1} & {\em rd} &   & \\
   \cline{2-4}
   SLLIW & {\em rs1} & {\em rd} &   & \\
   \cline{2-4}
   SRLIW & {\em rs1} & {\em rd} &   & \\
   \cline{2-4}
   SRAIW & {\em rs1} & {\em rd} &   & \\
   \cline{2-4}
   ADDW & {\em rs1}, {\em rs2} & {\em rd} &   & \\
   \cline{2-4}
   SUBW & {\em rs1}, {\em rs2} & {\em rd} &   & \\
   \cline{2-4}
   SLLW & {\em rs1}, {\em rs2} & {\em rd} &   & \\
   \cline{2-4}
   SRLW & {\em rs1}, {\em rs2} & {\em rd} &   & \\
   \cline{2-4}
   SRAW & {\em rs1}, {\em rs2} & {\em rd} &   & \\
   \cline{2-4}
\end{tabular}

\begin{tabular}{p{3cm}|p{3cm}|p{2cm}|p{4cm}|p{4cm}}
  \multicolumn{4}{l}{\bf RV32M 标准扩展} \\
  \cline{2-4}
    & 源    & 目的 & 累积 \\
    & 寄存器 & 寄存器   & CSR \\
  \cline{2-4}
   MUL & {\em rs1}, {\em rs2} & {\em rd} &   & \\
   \cline{2-4}
   MULH & {\em rs1}, {\em rs2} & {\em rd} &   & \\
   \cline{2-4}
   MULHSU & {\em rs1}, {\em rs2} & {\em rd} &   & \\
   \cline{2-4}
   MULHU & {\em rs1}, {\em rs2} & {\em rd} &   & \\
   \cline{2-4}
   DIV & {\em rs1}, {\em rs2} & {\em rd} &   & \\
   \cline{2-4}
   DIVU & {\em rs1}, {\em rs2} & {\em rd} &   & \\
   \cline{2-4}
   REM & {\em rs1}, {\em rs2} & {\em rd} &   & \\
   \cline{2-4}
   REMU & {\em rs1}, {\em rs2} & {\em rd} &   & \\
   \cline{2-4}
\end{tabular}

\begin{tabular}{p{3cm}|p{3cm}|p{2cm}|p{4cm}|p{4cm}}
  \multicolumn{4}{l}{\bf RV64M 标准扩展} \\
  \cline{2-4}
    & 源    & 目的 & 累积 \\
    & 寄存器 & 寄存器   & CSR \\
  \cline{2-4}
   MULW & {\em rs1}, {\em rs2} & {\em rd} &   & \\
   \cline{2-4}
   DIVW & {\em rs1}, {\em rs2} & {\em rd} &   & \\
   \cline{2-4}
   DIVUW & {\em rs1}, {\em rs2} & {\em rd} &   & \\
   \cline{2-4}
   REMW & {\em rs1}, {\em rs2} & {\em rd} &   & \\
   \cline{2-4}
   REMUW & {\em rs1}, {\em rs2} & {\em rd} &   & \\
   \cline{2-4}
\end{tabular}

\begin{tabular}{p{3cm}|p{3cm}|p{2cm}|p{4cm}|p{4cm}}
  \multicolumn{4}{l}{\bf RV32A 标准扩展} \\
  \cline{2-4}
    & 源    & 目的 & 累积 \\
    & 寄存器 & 寄存器   & CSR \\
  \cline{2-4}
   LR.W$^\dagger$ & {\em rs1}$^A$ & {\em rd} &   & \\
   \cline{2-4}
   SC.W$^\dagger$ & {\em rs1}$^A$, {\em rs2}$^D$ & {\em rd}$^*$ & & $^*$如果成功  \\
   \cline{2-4}
   AMOSWAP.W$^\dagger$ & {\em rs1}$^A$, {\em rs2}$^D$ & {\em rd} &   & \\
   \cline{2-4}
   AMOADD.W$^\dagger$ & {\em rs1}$^A$, {\em rs2}$^D$ & {\em rd} &   & \\
   \cline{2-4}
   AMOXOR.W$^\dagger$ & {\em rs1}$^A$, {\em rs2}$^D$ & {\em rd} &   & \\
   \cline{2-4}
   AMOAND.W$^\dagger$ & {\em rs1}$^A$, {\em rs2}$^D$ & {\em rd} &   & \\
   \cline{2-4}
   AMOOR.W$^\dagger$ & {\em rs1}$^A$, {\em rs2}$^D$ & {\em rd} &   & \\
   \cline{2-4}
   AMOMIN.W$^\dagger$ & {\em rs1}$^A$, {\em rs2}$^D$ & {\em rd} &   & \\
   \cline{2-4}
   AMOMAX.W$^\dagger$ & {\em rs1}$^A$, {\em rs2}$^D$ & {\em rd} &   & \\
   \cline{2-4}
   AMOMINU.W$^\dagger$ & {\em rs1}$^A$, {\em rs2}$^D$ & {\em rd} &   & \\
   \cline{2-4}
   AMOMAXU.W$^\dagger$ & {\em rs1}$^A$, {\em rs2}$^D$ & {\em rd} &   & \\
   \cline{2-4}
\end{tabular}

\begin{tabular}{p{3cm}|p{3cm}|p{2cm}|p{4cm}|p{4cm}}
  \multicolumn{4}{l}{\bf RV64A 标准扩展} \\
  \cline{2-4}
    & 源    & 目的 & 累积 \\
    & 寄存器 & 寄存器   & CSR \\
  \cline{2-4}
   LR.D$^\dagger$ & {\em rs1}$^A$ & {\em rd} &   & \\
   \cline{2-4}
   SC.D$^\dagger$ & {\em rs1}$^A$, {\em rs2}$^D$ & {\em rd}$^*$ & & $^*$如果成功  \\
   \cline{2-4}
   AMOSWAP.D$^\dagger$ & {\em rs1}$^A$, {\em rs2}$^D$ & {\em rd} &   & \\
   \cline{2-4}
   AMOADD.D$^\dagger$ & {\em rs1}$^A$, {\em rs2}$^D$ & {\em rd} &   & \\
   \cline{2-4}
   AMOXOR.D$^\dagger$ & {\em rs1}$^A$, {\em rs2}$^D$ & {\em rd} &   & \\
   \cline{2-4}
   AMOAND.D$^\dagger$ & {\em rs1}$^A$, {\em rs2}$^D$ & {\em rd} &   & \\
   \cline{2-4}
   AMOOR.D$^\dagger$ & {\em rs1}$^A$, {\em rs2}$^D$ & {\em rd} &   & \\
   \cline{2-4}
   AMOMIN.D$^\dagger$ & {\em rs1}$^A$, {\em rs2}$^D$ & {\em rd} &   & \\
   \cline{2-4}
   AMOMAX.D$^\dagger$ & {\em rs1}$^A$, {\em rs2}$^D$ & {\em rd} &   & \\
   \cline{2-4}
   AMOMINU.D$^\dagger$ & {\em rs1}$^A$, {\em rs2}$^D$ & {\em rd} &   & \\
   \cline{2-4}
   AMOMAXU.D$^\dagger$ & {\em rs1}$^A$, {\em rs2}$^D$ & {\em rd} &   & \\
   \cline{2-4}
\end{tabular}

\begin{tabular}{p{3cm}|p{3cm}|p{2cm}|p{4cm}|p{4cm}}
  \multicolumn{4}{l}{\bf RV32F 标准扩展} \\
  \cline{2-4}
    & 源    & 目的 & 累积 \\
    & 寄存器 & 寄存器   & CSR \\
  \cline{2-4}
   FLW$^\dagger$ & {\em rs1}$^A$ & {\em rd} & & \\
   \cline{2-4}
   FSW & {\em rs1}$^A$, {\em rs2}$^D$ &  &   & \\
   \cline{2-4}
   FMADD.S & {\em rs1}, {\em rs2},  {\em rs3}, frm$^*$ & {\em rd} & NV, OF, UF, NX & $^*$如果rm=111 \\
   \cline{2-4}
   FMSUB.S & {\em rs1}, {\em rs2},  {\em rs3}, frm$^*$ & {\em rd} & NV, OF, UF, NX & $^*$如果rm=111  \\
   \cline{2-4}
   FNMSUB.S & {\em rs1}, {\em rs2},  {\em rs3}, frm$^*$ & {\em rd} & NV, OF, UF, NX & $^*$如果rm=111  \\
   \cline{2-4}
   FNMADD.S & {\em rs1}, {\em rs2},  {\em rs3}, frm$^*$ & {\em rd} & NV, OF, UF, NX & $^*$如果rm=111  \\
   \cline{2-4}
   FADD.S & {\em rs1}, {\em rs2}, frm$^*$ & {\em rd} & NV, OF, NX & $^*$如果rm=111  \\
   \cline{2-4}
   FSUB.S & {\em rs1}, {\em rs2}, frm$^*$ & {\em rd} & NV, OF, NX & $^*$如果rm=111  \\
   \cline{2-4}
   FMUL.S & {\em rs1}, {\em rs2}, frm$^*$ & {\em rd} & NV, OF, UF, NX & $^*$如果rm=111  \\
   \cline{2-4}
   FDIV.S & {\em rs1}, {\em rs2}, frm$^*$ & {\em rd} & NV, DZ, OF, UF, NX & $^*$如果rm=111  \\
   \cline{2-4}
   FSQRT.S & {\em rs1}, frm$^*$ & {\em rd} & NV, NX & $^*$如果rm=111  \\
   \cline{2-4}
   FSGNJ.S & {\em rs1}, {\em rs2} & {\em rd} &   & \\
   \cline{2-4}
   FSGNJN.S & {\em rs1}, {\em rs2} & {\em rd} &   & \\
   \cline{2-4}
   FSGNJX.S & {\em rs1}, {\em rs2} & {\em rd} &   & \\
   \cline{2-4}
   FMIN.S & {\em rs1}, {\em rs2} & {\em rd} & NV &   \\
   \cline{2-4}
   FMAX.S & {\em rs1}, {\em rs2} & {\em rd} & NV &   \\
   \cline{2-4}
   FCVT.W.S & {\em rs1}, frm$^*$ & {\em rd} & NV, NX & $^*$如果rm=111  \\
   \cline{2-4}
   FCVT.WU.S & {\em rs1}, frm$^*$ & {\em rd} & NV, NX & $^*$如果rm=111  \\
   \cline{2-4}
   FMV.X.W & {\em rs1} & {\em rd} &   & \\
   \cline{2-4}
   FEQ.S & {\em rs1}, {\em rs2} & {\em rd} & NV &   \\
   \cline{2-4}
   FLT.S & {\em rs1}, {\em rs2} & {\em rd} & NV &   \\
   \cline{2-4}
   FLE.S & {\em rs1}, {\em rs2} & {\em rd} & NV &   \\
   \cline{2-4}
   FCLASS.S & {\em rs1} & {\em rd} &   & \\
   \cline{2-4}
   FCVT.S.W & {\em rs1}, frm$^*$ & {\em rd} & NX & $^*$如果rm=111  \\
   \cline{2-4}
   FCVT.S.WU & {\em rs1}, frm$^*$ & {\em rd} & NX & $^*$如果rm=111  \\
   \cline{2-4}
   FMV.W.X & {\em rs1} & {\em rd} &   & \\
   \cline{2-4}
\end{tabular}

\begin{tabular}{p{3cm}|p{3cm}|p{2cm}|p{4cm}|p{4cm}}
  \multicolumn{4}{l}{\bf RV64F 标准扩展 }\\
  \cline{2-4}
    & 源    & 目的 & 累积 \\
    & 寄存器 & 寄存器   & CSR \\
  \cline{2-4}
   FCVT.L.S & {\em rs1}, frm$^*$ & {\em rd} & NV, NX & $^*$如果rm=111  \\
   \cline{2-4}
   FCVT.LU.S & {\em rs1}, frm$^*$ & {\em rd} & NV, NX & $^*$如果rm=111  \\
   \cline{2-4}
   FCVT.S.L & {\em rs1}, frm$^*$ & {\em rd} & NX & $^*$如果rm=111  \\
   \cline{2-4}
   FCVT.S.LU & {\em rs1}, frm$^*$ & {\em rd} & NX & $^*$如果rm=111   \\
   \cline{2-4}
\end{tabular}

\begin{tabular}{p{3cm}|p{3cm}|p{2cm}|p{4cm}|p{4cm}}
  \multicolumn{4}{l}{\bf RV32D 标准扩展} \\
  \cline{2-4}
    & 源    & 目的 & 累积 \\
    & 寄存器 & 寄存器   & CSR \\
  \cline{2-4}
   FLD$^\dagger$ & {\em rs1}$^A$ & {\em rd} &   & \\
   \cline{2-4}
   FSD & {\em rs1}$^A$, {\em rs2}$^D$ &  &   & \\
   \cline{2-4}
   FMADD.D & {\em rs1}, {\em rs2},  {\em rs3}, frm$^*$ & {\em rd} & NV, OF, UF, NX & $^*$如果rm=111  \\
   \cline{2-4}
   FMSUB.D & {\em rs1}, {\em rs2},  {\em rs3}, frm$^*$ & {\em rd} & NV, OF, UF, NX & $^*$如果rm=111  \\
   \cline{2-4}
   FNMSUB.D & {\em rs1}, {\em rs2},  {\em rs3}, frm$^*$ & {\em rd} & NV, OF, UF, NX & $^*$如果rm=111  \\
   \cline{2-4}
   FNMADD.D & {\em rs1}, {\em rs2},  {\em rs3}, frm$^*$ & {\em rd} & NV, OF, UF, NX & $^*$如果rm=111  \\
   \cline{2-4}
   FADD.D & {\em rs1}, {\em rs2}, frm$^*$ & {\em rd} & NV, OF, NX & $^*$如果rm=111  \\
   \cline{2-4}
   FSUB.D & {\em rs1}, {\em rs2}, frm$^*$ & {\em rd} & NV, OF, NX & $^*$如果rm=111  \\
   \cline{2-4}
   FMUL.D & {\em rs1}, {\em rs2}, frm$^*$ & {\em rd} & NV, OF, UF, NX & $^*$如果rm=111  \\
   \cline{2-4}
   FDIV.D & {\em rs1}, {\em rs2}, frm$^*$ & {\em rd} & NV, DZ, OF, UF, NX & $^*$如果rm=111  \\
   \cline{2-4}
   FSQRT.D & {\em rs1}, frm$^*$ & {\em rd} & NV, NX & $^*$如果rm=111  \\
   \cline{2-4}
   FSGNJ.D & {\em rs1}, {\em rs2} & {\em rd} &   & \\
   \cline{2-4}
   FSGNJN.D & {\em rs1}, {\em rs2} & {\em rd} &   & \\
   \cline{2-4}
   FSGNJX.D & {\em rs1}, {\em rs2} & {\em rd} &   & \\
   \cline{2-4}
   FMIN.D & {\em rs1}, {\em rs2} & {\em rd} & NV &   \\
   \cline{2-4}
   FMAX.D & {\em rs1}, {\em rs2} & {\em rd} & NV &   \\
   \cline{2-4}
   FCVT.S.D & {\em rs1}, frm$^*$ & {\em rd} & NV, OF, UF, NX & $^*$如果rm=111  \\
   \cline{2-4}
   FCVT.D.S & {\em rs1} & {\em rd} & NV &   \\
   \cline{2-4}
   FEQ.D & {\em rs1}, {\em rs2} & {\em rd} & NV &   \\
   \cline{2-4}
   FLT.D & {\em rs1}, {\em rs2} & {\em rd} & NV &   \\
   \cline{2-4}
   FLE.D & {\em rs1}, {\em rs2} & {\em rd} & NV &   \\
   \cline{2-4}
   FCLASS.D & {\em rs1} & {\em rd} &   & \\
   \cline{2-4}
   FCVT.W.D & {\em rs1}, frm$^*$ & {\em rd} & NV, NX & $^*$如果rm=111  \\
   \cline{2-4}
   FCVT.WU.D & {\em rs1}, frm$^*$ & {\em rd} & NV, NX & $^*$如果rm=111  \\
   \cline{2-4}
   FCVT.D.W & {\em rs1} & {\em rd} &  & \\
   \cline{2-4}
   FCVT.D.WU & {\em rs1} & {\em rd} &  & \\
   \cline{2-4}
\end{tabular}

\begin{tabular}{p{3cm}|p{3cm}|p{2cm}|p{4cm}|p{4cm}}
  \multicolumn{4}{l}{\bf RV64D 标准扩展} \\
  \cline{2-4}
    & 源    & 目的 & 累积 \\
    & 寄存器 & 寄存器   & CSR \\
  \cline{2-4}
   FCVT.L.D & {\em rs1}, frm$^*$ & {\em rd} & NV, NX & $^*$如果rm=111  \\
   \cline{2-4}
   FCVT.LU.D & {\em rs1}, frm$^*$ & {\em rd} & NV, NX & $^*$如果rm=111  \\
   \cline{2-4}
   FMV.X.D & {\em rs1} & {\em rd} &   & \\
   \cline{2-4}
   FCVT.D.L & {\em rs1}, frm$^*$ & {\em rd} & NX & $^*$如果rm=111  \\
   \cline{2-4}
   FCVT.D.LU & {\em rs1}, frm$^*$ & {\em rd} & NX & $^*$如果rm=111  \\
   \cline{2-4}
   FMV.D.X & {\em rs1} & {\em rd} &   & \\
   \cline{2-4}
\end{tabular}


\chapter{用于压缩指令的“C”标准扩展(2.0版本)}
% \chapter{``C'' Standard Extension for Compressed Instructions, Version
% 2.0}
\label{compressed}

这章描述了RISC-V标准压缩指令集扩展,命名为“C”,它通过为常见的操作添加短16位指令编码,减少了静态和动态的代码尺寸。
C扩展可以被添加到任何基础ISA(RV32、RV64、RV128),而我们使用通用术语“RVC”来涵盖所有这些加入了C扩展的ISA。
通常,程序中的50\%~60\%的RISC-V指令可以被RVC指令替代,从而减少了25\%~30\%的代码尺寸。
% This chapter describes the RISC-V
% standard compressed instruction-set extension, named ``C'', which
% reduces static and dynamic code size by adding short 16-bit
% instruction encodings for common operations.  The C extension can be
% added to any of the base ISAs (RV32, RV64, RV128), and we use the
% generic term ``RVC'' to cover any of these.  Typically, 50\%--60\% of
% the RISC-V instructions in a program can be replaced with RVC
% instructions, resulting in a 25\%--30\% code-size reduction.

\section{概览}
% \section{Overview}

RVC使用了一个简单的压缩策略,它提供常见32位RISC-V指令的较短的16位版本,当:
% RVC uses a simple compression scheme that offers shorter 16-bit
% versions of common 32-bit RISC-V instructions when:
\begin{tightlist}
	\item 立即数或地址偏移量较小,或者  % the immediate or address offset is small, or
	\item 其中一个寄存器是零寄存器(x0)、ABI链接寄存器(x1),或者ABI栈指针(x2),或者
      % one of the registers is the zero register ({\tt x0}), the
      % ABI link register ({\tt x1}), or the ABI stack pointer ({\tt
      % x2}), or
	\item 目的寄存器和第一个源寄存器完全相同,或者 
      % the destination register and the first source register are
      % identical, or   
	\item 使用的寄存器是8个最流行的寄存器。%  the registers used are the 8 most popular ones. 
\end{tightlist}

C扩展与所有其它的标准指令扩展相兼容。C扩展允许16位指令与32位指令自由混合,其中32位指令现在可以从任何16位边界开始,
也就是说,IALIGN=16。除了C扩展外,没有指令可以引发指令地址未对齐异常。
% The C extension is compatible with all other standard instruction
% extensions.  The C extension allows 16-bit instructions to be freely
% intermixed with 32-bit instructions, with the latter now able to start
% on any 16-bit boundary, i.e., IALIGN=16.  With the addition of the C
% extension, no instructions can raise instruction-address-misaligned
% exceptions.

\begin{commentary}
在原始的32位指令上移除32位对齐限制,可以显著提高代码密度。
% Removing the 32-bit alignment constraint on the original 32-bit
% instructions allows significantly greater code density.
\end{commentary}

压缩的指令编码大多在RV32C、RV64C和RV128C之间通用,但是如表~\ref{rvcopcodemap}中显示的那样,
根据基础ISA,也有少量操作码被用于不同的目的。
例如,较宽的地址空间RV64C和RV128C变体需要额外的操作码来压缩64位整数值的加载和存储,
而RV32C使用相同的操作码来加载和存储单精度浮点值。
类似地,RV128C需要额外的操作码来捕获128位整数值的加载和存储,
然而这些相同的操作码在RV32C和RV64C中被用于双精度浮点值的加载和存储。
如果C扩展被实现了,不论相关的标准浮点扩展(F和/或D)是否也被实现,都必须提供合适的压缩的浮点加载和存储指令。
此外,RV32C包括了一个压缩的跳转和链接指令,以压缩较短范围的子例程调用,而在RV64C和RV128C中,相同的操作码被用于压缩ADDIW。
% The compressed instruction encodings are mostly common across RV32C,
% RV64C, and RV128C, but as shown in Table~\ref{rvcopcodemap}, a few
% opcodes are used for different purposes depending on base ISA.
% For example, the wider address-space RV64C and RV128C variants require
% additional opcodes to compress loads and stores of 64-bit integer
% values, while RV32C uses the same opcodes to compress loads and stores
% of single-precision floating-point values.  Similarly, RV128C requires
% additional opcodes to capture loads and stores of 128-bit integer
% values, while these same opcodes are used for loads and stores of
% double-precision floating-point values in RV32C and RV64C.  If the C
% extension is implemented, the appropriate compressed floating-point
% load and store instructions must be provided whenever the relevant
% standard floating-point extension (F and/or D) is also implemented.
% In addition, RV32C includes a compressed jump and link instruction to
% compress short-range subroutine calls, where the same opcode is used
% to compress ADDIW for RV64C and RV128C.

\begin{commentary}
双精度加载和存储是静态和动态指令的一个重要部分,因此有必要将其包含在RV32C和RV64C的编码中。
% Double-precision loads and stores are a significant fraction of static
% and dynamic instructions, hence the motivation to include them in the
% RV32C and RV64C encoding.

尽管对于当前支持ABI的编译的基准,单精度加载和存储不是静态或动态压缩的一个重要来源,
但是对于只提供硬件单精度浮点单元、并且有只支持单精度浮点数的ABI的微控制器来说,
在衡量基准中,单精度加载和存储的使用至少与双精度加载和存储频率相同。因此,在RV32C中,有必要为这些操作提供压缩的支持。
% Although single-precision loads and stores are not a significant
% source of static or dynamic compression for benchmarks compiled for
% the currently supported ABIs, for microcontrollers that only provide
% hardware single-precision floating-point units and have an ABI that
% only supports single-precision floating-point numbers, the
% single-precision loads and stores will be used at least as frequently
% as double-precision loads and stores in the measured benchmarks.
% Hence, the motivation to provide compressed support for these in
% RV32C.

对于微控制器,较短范围的子例程调用更可能出现在小型二进制代码中,因此有必要在RV32C中包括这些。
% Short-range subroutine calls are more likely in small binaries for
% microcontrollers, hence the motivation to include these in RV32C.

尽管在不同基础ISA下,为了不同的目的重用操作码,会增加文档的复杂性,
然而即使是对于支持多个基础ISA的设计,对其实现的复杂性的影响也很小。
压缩的浮点加载和存储变体使用与较宽的整数加载和存储相同的指令格式,带有相同的寄存器修饰符。
% Although reusing opcodes for different purposes for different base
% ISAs adds some complexity to documentation, the impact on
% implementation complexity is small even for designs that support
% multiple base ISAs.  The compressed floating-point load
% and store variants use the same instruction format with the same
% register specifiers as the wider integer loads and stores.
\end{commentary}

RVC的设计有一个约束,每个RVC指令扩展到某个基础ISA(RV32I/E、RV64I或RV128I)
或者现有的F和D标准扩展中的一个单独的32位指令。采用这个约束有两个主要的好处:
% RVC was designed under the constraint that each RVC instruction
% expands into a single 32-bit instruction in either the base ISA
% (RV32I/E, RV64I, or RV128I) or the F and D standard extensions where
% present.  Adopting this constraint has two main benefits:

\begin{tightlist}
\item 硬件设计可以在解码期间简单地扩展RVC指令,简化了验证并最小化了对现存微架构的修改。
% Hardware designs can simply expand RVC instructions during
%   decode, simplifying verification and minimizing modifications to
%   existing microarchitectures.
\item 编译器可以感知不到RVC扩展,而把代码压缩留给汇编器和链接器,即使一个能感知到压缩的编译器通常将能够产生更好的结果。
% Compilers can be unaware of the RVC extension and leave code
%   compression to the assembler and linker, although a
%   compression-aware compiler will generally be able to produce better
%   results.
\end{tightlist}

\begin{commentary}
我们感觉,在C和基础IFD指令之间的简单一对一映射的所减少的多重复杂度远远超出了稍微更密集的编码的潜在收益,
这种编码或者添加了额外的只能在C扩展中支持的指令,或者允许在一个C指令中进行多重IFD指令的编码。
% We felt the multiple complexity reductions of a simple one-one mapping
% between C and base IFD instructions far outweighed the potential gains
% of a slightly denser encoding that added additional instructions only
% supported in the C extension, or that allowed encoding of multiple IFD
% instructions in one C instruction.
\end{commentary}

注意,C扩展并没有被设计为一个独立的ISA,意味着它必须随着一个基础ISA使用,这一点很重要。
% It is important to note that the C extension is not designed to be a
% stand-alone ISA, and is meant to be used alongside a base ISA.

\begin{commentary}
可变长度的指令集长期被用来改进代码密度。例如,IBM Stretch~\cite{stretch},开发于1950年代晚期,
有一个带有32位和64位指令的ISA,那里某些32位指令是完整的64位指令的压缩版本。
Stretch也采用了限制寄存器集的概念,这些寄存器在某些较短的指令格式中是可编址的,
而短分支指令只能引用一个索引寄存器。稍后的IBM 360架构~\cite{ibm360}支持了一个简单的可变长度指令编码,包括16位、32位或48位指令格式。
% Variable-length instruction sets have long been used to improve code
% density.  For example, the IBM Stretch~\cite{stretch}, developed in
% the late 1950s, had an ISA with 32-bit and 64-bit instructions, where
% some of the 32-bit instructions were compressed versions of the full
% 64-bit instructions.  Stretch also employed the concept of limiting
% the set of registers that were addressable in some of the shorter
% instruction formats, with short branch instructions that could only
% refer to one of the index registers.  The later IBM 360
% architecture~\cite{ibm360} supported a simple variable-length
% instruction encoding with 16-bit, 32-bit, or 48-bit instruction
% formats.

在1963年,CDC介绍了Cray设计的CDC 6600~\cite{cdc6600},一个RISC架构的前身,
它引入了带有两种长度(15位和30位)指令的富寄存器的加载-存储架构。
稍后的Cray-1设计使用了非常相似的指令格式,带有16位和32位指令长度。
% In 1963, CDC introduced the Cray-designed CDC 6600~\cite{cdc6600}, a
% precursor to RISC architectures, that introduced a register-rich
% load-store architecture with instructions of two lengths, 15-bits and
% 30-bits.  The later Cray-1 design used a very similar instruction
% format, with 16-bit and 32-bit instruction lengths.

在1980年代的最初的RISC ISA,都将性能放在第一位,代码尺寸放在第二位,这对于工作站环境是合理的,
但是对于嵌入式环境则不然。因此,ARM和MIPS随后都推出了提供更小代码尺寸的ISA版本,
通过提供备选的16位宽指令集来代替标准的32位宽指令。压缩的RISC ISA相对于它们的起点,减少了大约25~30\%的代码尺寸,
生成的代码显著\emph{小于}80x86。这个结果让一些人感到惊讶,因为他们的直觉是,
可变长度的CISC ISA应当比只提供了16位和32位格式的RISC ISA更小。
% The initial RISC ISAs from the 1980s all picked performance over code
% size, which was reasonable for a workstation environment, but not for
% embedded systems. Hence, both ARM and MIPS subsequently made versions
% of the ISAs that offered smaller code size by offering an alternative
% 16-bit wide instruction set instead of the standard 32-bit wide
% instructions.  The compressed RISC ISAs reduced code size relative to
% their starting points by about 25--30\%, yielding code that was
% significantly \emph{smaller} than 80x86.  This result surprised some,
% as their intuition was that the variable-length CISC ISA should be
% smaller than RISC ISAs that offered only 16-bit and 32-bit formats.

由于原始的RISC ISA没有留出足够的操作码空间来自由地包括这些计划之外的压缩指令,
它们转而作为完整的新的ISA进行开发。这意味着编译器需要不同的代码生成器用于独立的压缩ISA。
第一代压缩RISC ISA扩展(例如,ARM Thumb和MIPS16)只使用了固定的16位指令尺寸,这很好地减少了静态代码尺寸,
但是引起了动态指令数目的增长,这导致了与原始的定宽32位指令尺寸相比更低的性能。
这引起了第二代压缩RISC ISA设计的发展,使用混合的16位和32位指令长度(例如,ARM Thumb2、microMIPS、PowerPC VLE),
因此性能与纯32位指令相似,但是显著节省了代码尺寸。不幸的是,这些不同代际的压缩ISA是互相不兼容的,
也与原始的未压缩的ISA不兼容,导致了文档、实现和软件工具支持中的明显的复杂性。
% Since the original RISC ISAs did not leave sufficient opcode space
% free to include these unplanned compressed instructions, they were
% instead developed as complete new ISAs.  This meant compilers needed
% different code generators for the separate compressed ISAs.  The first
% compressed RISC ISA extensions (e.g., ARM Thumb and MIPS16) used only
% a fixed 16-bit instruction size, which gave good reductions in static
% code size but caused an increase in dynamic instruction count, which
% led to lower performance compared to the original fixed-width 32-bit
% instruction size.  This led to the development of a second generation
% of compressed RISC ISA designs with mixed 16-bit and 32-bit
% instruction lengths (e.g., ARM Thumb2, microMIPS, PowerPC VLE), so
% that performance was similar to pure 32-bit instructions but with
% significant code size savings.  Unfortunately, these different
% generations of compressed ISAs are incompatible with each other and
% with the original uncompressed ISA, leading to significant complexity
% in documentation, implementations, and software tools support.

在常见的使用64位的ISA中,只有PowerPC和microMIPS目前支持压缩指令格式。
奇怪的是,大多数流行的64位移动平台(ARM v8)的64位ISA并没有包括压缩指令格式,
而静态代码尺寸和动态指令获取带宽对它们来说是很重要的指标。尽管静态代码尺寸在较大系统中不是主要关心的问题,
但是指令获取带宽可能成为运行商业工作负载的服务器(它们经常含有大量的指令工作集)中的主要瓶颈。
% Of the commonly used 64-bit ISAs, only PowerPC and microMIPS currently
% supports a compressed instruction format.  It is surprising that the
% most popular 64-bit ISA for mobile platforms (ARM v8) does not include
% a compressed instruction format given that static code size and
% dynamic instruction fetch bandwidth are important metrics.  Although
% static code size is not a major concern in larger systems, instruction
% fetch bandwidth can be a major bottleneck in servers running
% commercial workloads, which often have a large instruction working
% set.

得益于25年的事后观察,RISC-V从一开始就被设计为支持压缩指令的,为RVC留出了足够的操作码空间,
来(与许多其它的扩展一起)作为一个简单的扩展被添加到基础ISA之上。
RVC的哲学是为嵌入式应用减少代码尺寸,并为所有应用提升性能和能源效率以减少指令缓存的缺失。
Waterman显示RVC获取的指令位减少了25\%~30\%,这减少了20\%~25\%的指令缓存缺失,
或者说,与将指令缓存尺寸翻倍的性能影响大致相同~\cite{waterman-ms}。
% Benefiting from 25 years of hindsight, RISC-V was designed to support
% compressed instructions from the outset, leaving enough opcode
% space for RVC to be added as a simple extension on top of the base ISA
% (along with many other extensions).  The philosophy of RVC is to
% reduce code size for embedded applications \emph{and} to improve
% performance and energy-efficiency for all applications due to fewer
% misses in the instruction cache. Waterman shows that RVC fetches
% 25\%-30\% fewer instruction bits, which reduces instruction cache
% misses by 20\%-25\%, or roughly the same performance impact as
% doubling the instruction cache size~\cite{waterman-ms}.
\end{commentary}

\section{压缩指令格式}
% \section{Compressed Instruction Formats}

表~\ref{rvc-formats}显示了九个压缩指令格式。CR、CI和CSS可以任意使用32个RVI寄存器,
但是CIW、CL、CS、CA和CB被限制为只能使用其中的8个。表~\ref{registers}列出了这些常用的寄存器,它们对应于寄存器{\tt x8}到{\tt x15}。
注意,使用栈指针作为基地址寄存器的加载和存储指令有各自独立的版本,因为保存到栈和从栈中恢复是如此普遍,
以至于它们要使用CI和CSS格式,以允许访问所有的32个数据寄存器。对于ADDI4SPN指令,CIW支持一个8位的立即数。
% Table~\ref{rvc-formats} shows the nine compressed instruction
% formats. CR, CI, and CSS can use any of the 32 RVI registers, but CIW,
% CL, CS, CA, and CB are limited to just 8 of them. Table~\ref{registers}
% lists these popular registers, which correspond to registers {\tt x8}
% to {\tt x15}.  Note that there is a
% separate version of load and store instructions that use the stack
% pointer as the base address register, since saving to and restoring
% from the stack are so prevalent, and that they use the CI and CSS
% formats to allow access to all 32 data registers. CIW supplies an
% 8-bit immediate for the ADDI4SPN instruction.

\begin{commentary}
RISC-V ABI被更改为把频繁使用的寄存器映射到寄存器{\tt x8}-{\tt x15}。
这简化了解压缩的解码器,因为它有一组连续的自然对齐的寄存器号,并且也与RV32E基础ISA兼容,后者只有16个整数寄存器。
% The RISC-V ABI was changed to make the frequently used registers map
% to registers {\tt x8}--{\tt x15}.  This simplifies the decompression
% decoder by having a contiguous naturally aligned set of register
% numbers, and is also compatible with the RV32E base ISA,
% which only has 16 integer registers.
\end{commentary}

基于压缩寄存器的浮点加载和存储也分别使用CL和CS格式,带有八个映射到{\tt f8}到{\tt f15}的寄存器。
% Compressed register-based floating-point loads and stores also use the
% CL and CS formats respectively, with the eight registers mapping to
% {\tt f8} to {\tt f15}.

\begin{commentary}
标准RISC-V调用约定把最频繁使用的浮点寄存器映射到寄存器{\tt f8}到{\tt f15},这将允许使用与整数寄存器号相同的寄存器进行解压缩解码。
% The standard RISC-V calling convention maps the most frequently used
% floating-point registers to registers {\tt f8} to {\tt f15}, which
% allows the same register decompression decoding as for integer
% register numbers.
\end{commentary}

在所有的指令中,格式都被设计为,将两个寄存器源修饰符位保持在相同位置,而目的寄存器域可以移动。
当存在完整的5位目的寄存器修饰符时,它位于与32位RISC-V编码中的相同位置。
如果立即数是被符号扩展的,符号扩展总是从位12开始。正如在基础规范中的那样,立即数域已经被加扰,以减少必需的立即数mux的数目。
% The formats were designed to keep bits for the two register source
% specifiers in the same place in all instructions, while the
% destination register field can move.  When the full 5-bit destination
% register specifier is present, it is in the same place as in the
% 32-bit RISC-V encoding.  Where immediates are
% sign-extended, the sign-extension is always from bit 12.  Immediate
% fields have been scrambled, as in the base specification, to reduce
% the number of immediate muxes required.

\begin{commentary}
在指令格式中,立即数域是加扰的,而不是按顺序的,这样在每个指令中,可以使尽可能多的位位于相同位置,因此简化了实现。
% The immediate fields are scrambled in the instruction formats instead
% of in sequential order so that as many bits as possible are in the
% same position in every instruction, thereby simplifying
% implementations.
\end{commentary}

对于许多RVC指令,值为零的立即数是不允许的,且{\tt x0}并非是一个有效的5位寄存器修饰符。
这些限制为其它的需要更少的操作数位的指令释放了编码空间。
% For many RVC instructions, zero-valued immediates are disallowed and
% {\tt x0} is not a valid 5-bit register specifier.  These restrictions
% free up encoding space for other instructions requiring fewer operand
% bits.

\newcommand{\rdprime}{rd\,$'$}
\newcommand{\rsoneprime}{rs1\,$'$}
\newcommand{\rstwoprime}{rs2\,$'$}

\begin{table}[h]
{
\begin{small}
\begin{center}
\begin{tabular}{c c p{0in}p{0.05in}p{0.05in}p{0.05in}p{0.05in}p{0.05in}p{0.05in}p{0.05in}p{0.05in}p{0.05in}p{0.05in}p{0.05in}p{0.05in}p{0.05in}p{0.05in}p{0.05in}p{0.05in}}
& & & & & & & & & \\
格式  &  含义                  &
\instbit{15} &
\instbit{14} &
\instbit{13} &
\multicolumn{1}{c}{\instbit{12}} &
\instbit{11} &
\instbit{10} &
\instbit{9} &
\instbit{8} &
\instbit{7} &
\instbit{6} &
\multicolumn{1}{r}{\instbit{5}} &
\instbit{4} &
\instbit{3} &
\instbit{2} &
\instbit{1} &
\instbit{0} \\
\cline{3-18}

CR & 寄存器 &
\multicolumn{4}{|c|}{funct4} &
\multicolumn{5}{c|}{rd/rs1} &
\multicolumn{5}{c|}{rs2} &
\multicolumn{2}{c|}{op} \\
\cline{3-18}

CI & 立即数 &
\multicolumn{3}{|c|}{funct3} &
\multicolumn{1}{c|}{imm} &
\multicolumn{5}{c|}{rd/rs1} &
\multicolumn{5}{c|}{imm} &
\multicolumn{2}{c|}{op} \\
\cline{3-18}

CSS & 栈相关存储 &
\multicolumn{3}{|c|}{funct3} &
\multicolumn{6}{c|}{imm} &
\multicolumn{5}{c|}{rs2} &
\multicolumn{2}{c|}{op} \\
\cline{3-18}

CIW & 宽立即数 &
\multicolumn{3}{|c|}{funct3} &
\multicolumn{8}{c|}{imm} &
\multicolumn{3}{c|}{\rdprime} &
\multicolumn{2}{c|}{op} \\
\cline{3-18}

CL & 加载 &
\multicolumn{3}{|c|}{funct3} &
\multicolumn{3}{c|}{imm} &
\multicolumn{3}{c|}{\rsoneprime} &
\multicolumn{2}{c|}{imm} &
\multicolumn{3}{c|}{\rdprime} &
\multicolumn{2}{c|}{op} \\
\cline{3-18}

CS & 存储 &
\multicolumn{3}{|c|}{funct3} &
\multicolumn{3}{c|}{imm} &
\multicolumn{3}{c|}{\rsoneprime} &
\multicolumn{2}{c|}{imm} &
\multicolumn{3}{c|}{\rstwoprime} &
\multicolumn{2}{c|}{op} \\
\cline{3-18}

CA & 算术 &
\multicolumn{6}{|c|}{funct6} &
\multicolumn{3}{c|}{\rdprime/\rsoneprime} &
\multicolumn{2}{c|}{funct2} &
\multicolumn{3}{c|}{\rstwoprime} &
\multicolumn{2}{c|}{op} \\
\cline{3-18}

CB & 分支/算术 &
\multicolumn{3}{|c|}{funct3} &
\multicolumn{3}{c|}{offset} &
\multicolumn{3}{c|}{\rdprime/\rsoneprime} &
\multicolumn{5}{c|}{offset} &
\multicolumn{2}{c|}{op} \\
\cline{3-18}

CJ & 跳转 &
\multicolumn{3}{|c|}{funct3} &
\multicolumn{11}{c|}{jump target} &
\multicolumn{2}{c|}{op} \\
\cline{3-18}

\end{tabular}
\end{center}
\end{small}
}
\caption{压缩的16位RVC指令格式。}
\label{rvc-formats}
\end{table}


\begin{table}[H]
{
\begin{center}
\begin{tabular}{l|c|c|c|c|c|c|c|c|}
\cline{2-9}
RVC寄存器编号 & 000 & 001 & 010 & 011 & 100 & 101 & 110 & 111
\\ \cline{2-9}
整数寄存器编号 & {\tt x8} & {\tt x9} & {\tt x10} & {\tt x11} & {\tt x12} & {\tt x13} & {\tt x14}  & {\tt x15} \\ \cline{2-9}
整数寄存器ABI名    & {\tt s0}  &  {\tt s1} &  {\tt a0} &  {\tt a1} &  {\tt a2} &  {\tt a3} & {\tt a4}  & {\tt a5} \\ \cline{2-9}
浮点寄存器编号 & {\tt f8} & {\tt f9} & {\tt f10} & {\tt f11} & {\tt f12} & {\tt f13} & {\tt f14}  & {\tt f15} \\ \cline{2-9}
浮点寄存器ABI名    & {\tt fs0}  &  {\tt fs1} &  {\tt fa0} &  {\tt fa1} &  {\tt fa2} &  {\tt fa3} & {\tt fa4}  & {\tt fa5} \\ \cline{2-9}
\end{tabular}
\end{center}
}
\caption{通过CIW、CL、CS、CA和CB格式的3位的{\em \rsoneprime}、{\em \rstwoprime}和{\em \rdprime}域指定的寄存器。
% Registers specified by the three-bit {\em \rsoneprime}, {\em \rstwoprime}, and {\em \rdprime} fields of the CIW, CL, CS, CA, and CB formats.
}
\label{registers}
\end{table}

\section{加载和存储指令}
% \section{Load and Store Instructions}

为了增加16位指令的访问范围,数据转移指令使用零扩展的立即数,它按照数据的字节尺寸进行缩放:字$\times$4,双字$\times$8,四字$\times$16。
% To increase the reach of 16-bit instructions, data-transfer
% instructions use zero-extended immediates that are scaled by the size
% of the data in bytes: $\times$4 for words, $\times$8 for double words,
% and $\times$16 for quad words.

RVC提供了加载和存储的两个变体。一个使用ABI栈指针{\tt x2}作为基础地址,而可以把任意数据寄存器作为目标。
另一个可以引用8个基础地址寄存器中的一个和8个数据寄存器中的一个。
% RVC provides two variants of loads and stores.  One uses the ABI stack
% pointer, {\tt x2}, as the base address and can target any data register.  The
% other can reference one of 8 base address registers and one of 8 data
% registers.

\subsection*{基于栈指针的加载和存储}
% \subsection*{Stack-Pointer-Based Loads and Stores}

\begin{center}
\begin{tabular}{S@{}W@{}T@{}T@{}Y}
\\
\instbitrange{15}{13} &
\multicolumn{1}{c}{\instbit{12}} &
\instbitrange{11}{7} &
\instbitrange{6}{2} &
\instbitrange{1}{0} \\
\hline
\multicolumn{1}{|c|}{funct3} &
\multicolumn{1}{c|}{imm} &
\multicolumn{1}{c|}{rd} &
\multicolumn{1}{c|}{imm} &
\multicolumn{1}{c|}{op} \\
\hline
3 & 1 & 5 & 5 & 2 \\
C.LWSP & offset[5] & dest$\neq$0 & offset[4:2$\vert$7:6] & C2 \\
C.LDSP & offset[5] & dest$\neq$0 & offset[4:3$\vert$8:6] & C2 \\
C.LQSP & offset[5] & dest$\neq$0 & offset[4$\vert$9:6] & C2 \\
C.FLWSP& offset[5] & dest        & offset[4:2$\vert$7:6] & C2 \\
C.FLDSP& offset[5] & dest        & offset[4:3$\vert$8:6] & C2 \\
\end{tabular}
\end{center}
这些指令使用CI格式。
% These instructions use the CI format.

C.LWSP从内存把一个32位的值加载到寄存器{\em rd}中。它通过将{\em 零}扩展的偏移量扩大4倍,加到栈指针{\tt x2}上,
来计算出有效地址。它扩展到{\tt lw rd, offset(x2)}。C.LWSP只有在$\textit{rd}{\neq}\texttt{x0}$时有效;$\textit{rd}{=}\texttt{x0}$的代码点被保留。
% C.LWSP loads a 32-bit value from memory into register {\em rd}.  It computes
% an effective address by adding the {\em zero}-extended offset, scaled by 4, to
% the stack pointer, {\tt x2}.  It expands to {\tt lw rd, offset(x2)}.
% C.LWSP is only valid when $\textit{rd}{\neq}\texttt{x0}$;
% the code points with $\textit{rd}{=}\texttt{x0}$ are reserved.

C.LDSP是一个RV64C/RV128C独有的指令,它从内存加载一个64位的值到寄存器{\em rd}中。它通过将零扩展的偏移量扩大8倍,
加到栈指针{\tt x2}上,来计算出有效地址。它扩展到{\tt ld rd, offset(x2)}。C.LDSP只有在$\textit{rd}{\neq}\texttt{x0}$时有效;$\textit{rd}{=}\texttt{x0}$的代码点被保留。
% C.LDSP is an RV64C/RV128C-only instruction that loads a 64-bit value from memory into
% register {\em rd}.  It computes its effective address by adding the
% zero-extended offset, scaled by 8, to the stack pointer, {\tt x2}.
% It expands to {\tt ld rd, offset(x2)}.
% C.LDSP is only valid when $\textit{rd}{\neq}\texttt{x0}$;
% the code points with $\textit{rd}{=}\texttt{x0}$ are reserved.

C.LQSP是一个RV128C独有的指令,它从内存加载一个128位的值到寄存器{\em rd}中。
它通过将零扩展的偏移量扩大16倍,加到栈指针{\tt x2}上,来计算出有效地址。它扩展到{\tt lq rd, offset(x2)}。
C.LQSP只在$\textit{rd}{\neq}\texttt{x0}$时有效;$\textit{rd}{=}\texttt{x0}$的代码点被保留。
% C.LQSP is an RV128C-only instruction that loads a 128-bit value from memory
% into register {\em rd}.  It computes its effective address by adding the
% zero-extended offset, scaled by 16, to the stack pointer, {\tt x2}.
% It expands to {\tt lq rd, offset(x2)}.
% C.LQSP is only valid when $\textit{rd}{\neq}\texttt{x0}$;
% the code points with $\textit{rd}{=}\texttt{x0}$ are reserved.

C.FLWSP是一个RV32FC独有的指令,它从内存加载一个单精度浮点值到浮点寄存器{\em rd}中。
它通过将{\em 零}扩展的偏移量扩大4倍,加到栈指针{\tt x2}上,来计算出有效地址。它扩展到{\tt flw rd, offset(x2)}。
% C.FLWSP is an RV32FC-only instruction that loads a single-precision
% floating-point value from memory into floating-point register {\em rd}. It
% computes its effective address by adding the {\em zero}-extended offset,
% scaled by 4, to the stack pointer, {\tt x2}.  It expands to {\tt flw rd,
% offset(x2)}.

C.FLDSP是一个RV32DC/RV64DC独有的指令,它从内存加载一个双精度浮点值到浮点寄存器{\em rd}中。
它通过将{\em 零}扩展的偏移量扩大8倍,加到栈指针{\tt x2}上,来计算出有效地址。它扩展到{\tt fld rd, offset(x2)}。
% C.FLDSP is an RV32DC/RV64DC-only instruction that loads a double-precision
% floating-point value from memory into floating-point register {\em rd}. It
% computes its effective address by adding the {\em zero}-extended offset,
% scaled by 8, to the stack pointer, {\tt x2}.  It expands to {\tt fld rd,
% offset(x2)}.

\begin{center}
\begin{tabular}{S@{}M@{}T@{}Y}
\\
\instbitrange{15}{13} &
\instbitrange{12}{7} &
\instbitrange{6}{2} &
\instbitrange{1}{0} \\
\hline
\multicolumn{1}{|c|}{funct3} &
\multicolumn{1}{c|}{imm} &
\multicolumn{1}{c|}{rs2} &
\multicolumn{1}{c|}{op} \\
\hline
3 & 6 & 5 & 2 \\
C.SWSP & offset[5:2$\vert$7:6] & src & C2 \\
C.SDSP & offset[5:3$\vert$8:6] & src & C2 \\
C.SQSP & offset[5:4$\vert$9:6] & src & C2 \\
C.FSWSP& offset[5:2$\vert$7:6] & src & C2 \\
C.FSDSP& offset[5:3$\vert$8:6] & src & C2 \\
\end{tabular}
\end{center}

这些指令使用CSS格式。
% These instructions use the CSS format.

C.SWSP把一个32位的值存储到寄存器{\em rs2}中。它通过将{\em 零}扩展的偏移量扩大4倍,加到栈指针{\tt x2}上,
来计算出有效地址。它扩展到{\tt sw rs2, offset(x2)}。
% C.SWSP stores a 32-bit value in register {\em rs2} to memory.  It computes
% an effective address by adding the {\em zero}-extended offset, scaled by 4, to
% the stack pointer, {\tt x2}.
% It expands to {\tt sw rs2, offset(x2)}.

C.SDSP是一个RV64C/RV128独有的指令,它把寄存器{\em rs2}中的一个64位的值存储到内存。它通过将{\em 零}扩展的偏移量扩大8倍,
加到栈指针{\tt x2}上,来计算出有效地址。它扩展到{\tt sd rs2, offset(x2)}。
% C.SDSP is an RV64C/RV128C-only instruction that stores a 64-bit value in register
% {\em rs2} to memory.  It computes an effective address by adding the {\em
% zero}-extended offset, scaled by 8, to the stack pointer, {\tt x2}.
% It expands to {\tt sd rs2, offset(x2)}.

C.SQSP是一个RV128C独有的指令,它把寄存器{\em rs2}中的一个128位的值存储到内存。它通过将{\em 零}扩展的偏移量扩大16倍,
加到栈指针{\tt x2}上,来计算出有效地址。它扩展到{\tt sq rs2, offset(x2)}。
% C.SQSP is an RV128C-only instruction that stores a 128-bit value in register
% {\em rs2} to memory.  It computes an effective address by adding the {\em
% zero}-extended offset, scaled by 16, to the stack pointer, {\tt x2}.
% It expands to {\tt sq rs2, offset(x2)}.

C.FSWSP是一个RV32FC独有的指令,它把浮点寄存器{\em rs2}中的一个单精度浮点值存储到内存。
它通过将{\em 零}扩展的扩大4倍,加到栈指针{\tt x2}上,来计算出有效地址。它扩展到{\tt fsw rs2, offset(x2)}。
% C.FSWSP is an RV32FC-only instruction that stores a single-precision
% floating-point value in floating-point register {\em rs2} to memory.  It
% computes an effective address by adding the {\em zero}-extended offset, scaled
% by 4, to the stack pointer, {\tt x2}.  It expands to {\tt fsw rs2,
% offset(x2)}.

C.FSDSP是一个RV32DC/RV64DC独有的指令,它把浮点寄存器{\em rs2}中的一个双精度浮点数存储到内存。
它通过将{\em 零}扩展的偏移量扩大8倍,加到栈指针{\tt x2}上,来计算出有效地址。它扩展到{\tt fsd rs2, offset(x2)}。
% C.FSDSP is an RV32DC/RV64DC-only instruction that stores a double-precision
% floating-point value in floating-point register {\em rs2} to memory.  It
% computes an effective address by adding the {\em zero}-extended offset, scaled
% by 8, to the stack pointer, {\tt x2}.  It expands to {\tt fsd rs2,
% offset(x2)}.

\begin{commentary}
在函数的入口/出口处的寄存器保存/恢复代码代表了静态代码尺寸的一大部分。
在RVC中,基于栈指针的压缩的加载和存储可以有效地减少两倍的保存/恢复静态代码尺寸,同时通过减少动态指令带宽来提升性能。
% Register save/restore code at function entry/exit represents a
% significant portion of static code size.  The stack-pointer-based
% compressed loads and stores in RVC are effective at reducing the
% save/restore static code size by a factor of 2 while improving
% performance by reducing dynamic instruction bandwidth.

为了进一步减少保存/恢复代码尺寸,在其它ISA中使用的一个常见的机制是多重加载和多重存储指令。
我们考虑过为RISC-V采用这些指令,但是注意到这些指令的如下缺点:
% A common mechanism used in other ISAs to further reduce
% save/restore code size is load-multiple and store-multiple
% instructions.  We considered adopting these for RISC-V but noted the
% following drawbacks to these instructions:
\begin{itemize}
\item 这些指令让处理器的实现复杂化。 %  These instructions complicate processor implementations.
\item 对于虚拟内存系统,有些数据访问可能驻留在物理内存中,而有些不能,需要为部分执行的指令使用一种新的重启机制。 
      % For virtual memory systems, some data accesses could be
      % resident in physical memory and some could not, which requires a
      % new restart mechanism for partially executed instructions.
\item 不像其余的RVC指令,没有等价于多重加载和多重存储的IFD。 
      % Unlike the rest of the RVC instructions, there is no IFD
      % equivalent to Load Multiple and Store Multiple.
\item 不像其余的RVC指令,编译器将不得不注意这些指令,来生成指令和按次序分配寄存器,
      以最大化它们被保存和存储的机会——因为它们要按顺序次序被保存和存储。
      % Unlike the rest of the RVC instructions, the compiler would
      % have to be aware of these instructions to both generate the
      % instructions and to allocate registers in an order to maximize
      % the chances of the them being saved and stored, since they would
      % be saved and restored in sequential order.
\item 简单微架构的实现将限制如何围绕加载多重和存储多重指令调度其它指令,导致潜在的性能损失。
      % Simple microarchitectural implementations will constrain how
      % other instructions can be scheduled around the load and store
      % multiple instructions, leading to a potential performance loss.
\item 期望的顺序寄存器分配可能与为CIW、CL、CS、CA和CB格式选择的特征寄存器冲突。
      The desire for sequential register allocation might conflict with
      the featured registers selected for the CIW, CL, CS, CA, and CB formats.
\end{itemize}
此外,在软件中,通过~\cite{waterman-phd}的5.6节中描述的一种技术,使用子例程调用公共的序言和结语代码,替换序言和结语代码,可以实现大多数收益。
% Furthermore, much of the gains can be realized in software by replacing
% prologue and epilogue code with subroutine calls to common
% prologue and epilogue code, a technique described in
% Section 5.6 of~\cite{waterman-phd}.

虽然合理的架构可能得出不同的结论,我们决定忽略加载多重和存储多重,代之使用纯软件的方法,
调用保持/恢复millicode例程以获得最大程度的代码尺寸减少。
% While reasonable architects might come to different conclusions, we
% decided to omit load and store multiple and instead use the
% software-only approach of calling save/restore millicode routines to
% attain the greatest code size reduction.
\end{commentary}

\subsection*{基于寄存器的加载和存储}
% \subsection*{Register-Based Loads and Stores}

\begin{center}
\begin{tabular}{S@{}S@{}S@{}Y@{}S@{}Y}
\\
\instbitrange{15}{13} &
\instbitrange{12}{10} &
\instbitrange{9}{7} &
\instbitrange{6}{5} &
\instbitrange{4}{2} &
\instbitrange{1}{0} \\
\hline
\multicolumn{1}{|c|}{funct3} &
\multicolumn{1}{c|}{imm} &
\multicolumn{1}{c|}{\rsoneprime} &
\multicolumn{1}{c|}{imm} &
\multicolumn{1}{c|}{\rdprime} &
\multicolumn{1}{c|}{op} \\
\hline
3 & 3 & 3 & 2 & 3 & 2 \\
C.LW & offset[5:3] & base & offset[2$\vert$6] & dest & C0 \\
C.LD & offset[5:3] & base & offset[7:6] & dest & C0 \\
C.LQ & offset[5$\vert$4$\vert$8] & base & offset[7:6] & dest & C0 \\
C.FLW& offset[5:3] & base & offset[2$\vert$6] & dest & C0 \\
C.FLD& offset[5:3] & base & offset[7:6] & dest & C0 \\
\end{tabular}
\end{center}

这些指令使用CL格式。
% These instructions use the CL format.

C.LW从内存加载一个32位的值到寄存器{\em \rdprime}中。它通过将{\em 零}扩展的偏移量扩大4倍,加到寄存器{\em \rsoneprime}中的基地址上,
来计算出有效地址。它扩展到{\tt lw \rdprime, offset(\rsoneprime)}。
% C.LW loads a 32-bit value from memory into register {\em \rdprime}.  It computes
% an effective address by adding the {\em zero}-extended offset, scaled by 4, to
% the base address in register {\em \rsoneprime}.
% It expands to {\tt lw \rdprime, offset(\rsoneprime)}.

C.LD是一个RV64C/RV128C独有的指令,它从内存加载一个64位的值到寄存器{\em \rdprime}中。它通过将{\em 零}扩展的偏移量扩大8倍,
加到寄存器{\em \rsoneprime}中的基地址上,来计算出有效地址。它扩展到{\tt ld \rdprime, offset(\rsoneprime)}。
% C.LD is an RV64C/RV128C-only instruction that loads a 64-bit value from memory into
% register {\em \rdprime}.  It computes an effective address by adding the {\em
% zero}-extended offset, scaled by 8, to the base address in register {\em
% \rsoneprime}.
% It expands to {\tt ld \rdprime, offset(\rsoneprime)}.

C.LQ是一个RV128C独有的指令,它从内存加载一个128位的值到寄存器{\em \rdprime}中。它通过将{\em 零}扩展的偏移量扩大16倍,
加到寄存器{\em \rsoneprime}中的基地址上,来计算出有效地址。它扩展到{\tt lq \rdprime, offset(\rsoneprime)}。
% C.LQ is an RV128C-only instruction that loads a 128-bit value from memory into
% register {\em \rdprime}.  It computes an effective address by adding the {\em
% zero}-extended offset, scaled by 16, to the base address in register {\em
% \rsoneprime}.
% It expands to {\tt lq \rdprime, offset(\rsoneprime)}.

C.FLW是一个RV32FC独有的指令,它从内存加载一个单精度浮点值到浮点寄存器{\em \rdprime}中。
它通过将{\em 零}扩展的偏移量扩大4倍,加到寄存器{\em \rsoneprime}中的基地址上,来计算出有效地址。它扩展到{\tt flw
\rdprime, offset(\rsoneprime)}。
% C.FLW is an RV32FC-only instruction that loads a single-precision
% floating-point value from memory into floating-point register {\em \rdprime}.  It
% computes an effective address by adding the {\em zero}-extended offset, scaled
% by 4, to the base address in register {\em \rsoneprime}.  It expands to {\tt flw
% \rdprime, offset(\rsoneprime)}.

C.FLD是一个RV32DC/RV64DC独有的指令,它从内存加载一个双精度浮点值到浮点寄存器{\em \rdprime}中。
它通过将{\em 零}扩展的偏移量扩大8倍,加到寄存器{\em \rsoneprime}中的基地址上,来计算出有效地址。它扩展到{\tt fld
\rdprime, offset(\rsoneprime)}。
% C.FLD is an RV32DC/RV64DC-only instruction that loads a double-precision
% floating-point value from memory into floating-point register {\em \rdprime}.  It
% computes an effective address by adding the {\em zero}-extended offset, scaled
% by 8, to the base address in register {\em \rsoneprime}.  It expands to {\tt fld
% \rdprime, offset(\rsoneprime)}.

\begin{center}
\begin{tabular}{S@{}S@{}S@{}Y@{}S@{}Y}
\\
\instbitrange{15}{13} &
\instbitrange{12}{10} &
\instbitrange{9}{7} &
\instbitrange{6}{5} &
\instbitrange{4}{2} &
\instbitrange{1}{0} \\
\hline
\multicolumn{1}{|c|}{funct3} &
\multicolumn{1}{c|}{imm} &
\multicolumn{1}{c|}{\rsoneprime} &
\multicolumn{1}{c|}{imm} &
\multicolumn{1}{c|}{\rstwoprime} &
\multicolumn{1}{c|}{op} \\
\hline
3 & 3 & 3 & 2 & 3 & 2 \\
C.SW & offset[5:3] & base & offset[2$\vert$6] & src & C0 \\
C.SD & offset[5:3] & base & offset[7:6] & src & C0 \\
C.SQ & offset[5$\vert$4$\vert$8] & base & offset[7:6] & src & C0 \\
C.FSW& offset[5:3] & base & offset[2$\vert$6] & src & C0 \\
C.FSD& offset[5:3] & base & offset[7:6] & src & C0 \\
\end{tabular}
\end{center}
这些指令使用CS格式。
% These instructions use the CS format.

C.SW把寄存器{\em \rstwoprime}中的一个32位的值存储到内存。它通过将{\em 零}扩展的偏移量扩大4倍,加到寄存器{\em \rsoneprime}中的基地址上,
来计算出有效地址。它扩展到{\tt sw \rstwoprime, offset(\rsoneprime)}。
% C.SW stores a 32-bit value in register {\em \rstwoprime} to memory.  It computes an
% effective address by adding the {\em zero}-extended offset, scaled by 4, to
% the base address in register {\em \rsoneprime}.
% It expands to {\tt sw \rstwoprime, offset(\rsoneprime)}.

C.SD是一个RV64C/RV128C独有的指令,它把寄存器{\em \rstwoprime}中的一个64位的值存储到内存。
它通过将{\em 零}扩展的偏移量扩大8倍,加到寄存器{\em \rsoneprime}中的基地址上,来计算出有效地址。它扩展到{\tt sd \rstwoprime, offset(\rsoneprime)}。
% C.SD is an RV64C/RV128C-only instruction that stores a 64-bit value in
% register {\em \rstwoprime} to memory.  It computes an effective address by adding
% the {\em zero}-extended offset, scaled by 8, to the base address in register
% {\em \rsoneprime}.
% It expands to {\tt sd \rstwoprime, offset(\rsoneprime)}.

C.SQ是一个RV128C独有的指令,它把寄存器{\em \rstwoprime}中的一个128位的值存储到内存。它通过将{\em 零}扩展的偏移量扩大16倍,
加到寄存器{\em \rsoneprime}中的基地址上,来计算出有效地址。它扩展到{\tt sq \rstwoprime, offset(\rsoneprime)}。
% C.SQ is an RV128C-only instruction that stores a 128-bit value in register
% {\em \rstwoprime} to memory.  It computes an effective address by adding the {\em
% zero}-extended offset, scaled by 16, to the base address in register {\em
% \rsoneprime}.
% It expands to {\tt sq \rstwoprime, offset(\rsoneprime)}.

C.FSW是一个RV32FC独有的指令,它把浮点寄存器{\em \rstwoprime}中的一个单精度浮点值存储到内存。
它通过将{\em 零}扩展的偏移量扩大4倍,加到寄存器{\em \rsoneprime}中的基地址上,来计算出有效地址。它扩展到{\tt fsw
\rstwoprime, offset(\rsoneprime)}。
% C.FSW is an RV32FC-only instruction that stores a single-precision
% floating-point value in floating-point register {\em \rstwoprime} to memory.  It
% computes an effective address by adding the {\em zero}-extended offset, scaled
% by 4, to the base address in register {\em \rsoneprime}.  It expands to {\tt fsw
% \rstwoprime, offset(\rsoneprime)}.

C.FSD是一个RV32DC/RV64DC独有的指令,它把浮点寄存器{\em \rstwoprime}中的一个双精度浮点值存储到内存。
它通过将{\em 零}扩展的偏移量扩大8倍,加到寄存器{\em \rsoneprime}中的基地址上,来计算出有效地址。它扩展到{\tt fsd
\rstwoprime, offset(\rsoneprime)}。
% C.FSD is an RV32DC/RV64DC-only instruction that stores a double-precision
% floating-point value in floating-point register {\em \rstwoprime} to memory.  It
% computes an effective address by adding the {\em zero}-extended offset, scaled
% by 8, to the base address in register {\em \rsoneprime}.  It expands to {\tt fsd
% \rstwoprime, offset(\rsoneprime)}.

\section{控制转移指令}
% \section{Control Transfer Instructions}

RVC提供了无条件跳转指令和条件分支指令。因为带有基础RVI指令,所有的RVC控制转移指令的偏移量都是2字节的倍数。
% RVC provides unconditional jump instructions and conditional branch
% instructions. As with base RVI instructions, the offsets of all RVC
% control transfer instructions are in multiples of 2 bytes.

\begin{center}
\begin{tabular}{S@{}L@{}Y}
\\
\instbitrange{15}{13} &
\instbitrange{12}{2} &
\instbitrange{1}{0} \\
\hline
\multicolumn{1}{|c|}{funct3} &
\multicolumn{1}{c|}{imm} &
\multicolumn{1}{c|}{op} \\
\hline
3 & 11 & 2 \\
C.J & offset[11$\vert$4$\vert$9:8$\vert$10$\vert$6$\vert$7$\vert$3:1$\vert$5] & C1 \\
C.JAL & offset[11$\vert$4$\vert$9:8$\vert$10$\vert$6$\vert$7$\vert$3:1$\vert$5] & C1 \\
\end{tabular}
\end{center}
这些指令使用CJ格式。
% These instructions use the CJ format.

C.J实施无条件控制转移。偏移量被符号扩展,并被加到{\tt pc}以形成跳转的目标地址。C.J因此可以有$\pm$\wunits{2}{KiB}的目标范围。C.J扩展到{\tt jal x0, offset}。
% C.J performs an unconditional control transfer.  The offset is sign-extended and
% added to the {\tt pc} to form the jump target address.  C.J can therefore target
% a $\pm$\wunits{2}{KiB} range.  C.J expands to {\tt jal x0, offset}.

C.JAL是一个RV32C独有的指令,它实施与C.J相同的操作,但是额外地把跳转({\tt pc}+2)之后的指令的地址写到链接寄存器{\tt x1}。C.JAL扩展到{\tt jal x1, offset}。
% C.JAL is an RV32C-only instruction that performs the same operation as C.J,
% but additionally writes the address of the instruction following the jump
% ({\tt pc}+2) to the link register, {\tt x1}.  C.JAL expands to {\tt jal x1,
% offset}.

\begin{center}
\begin{tabular}{E@{}T@{}T@{}Y}
\\
\instbitrange{15}{12} &
\instbitrange{11}{7} &
\instbitrange{6}{2} &
\instbitrange{1}{0} \\
\hline
\multicolumn{1}{|c|}{funct4} &
\multicolumn{1}{c|}{rs1} &
\multicolumn{1}{c|}{rs2} &
\multicolumn{1}{c|}{op} \\
\hline
4 & 5 & 5 & 2 \\
C.JR & src$\neq$0 & 0 & C2 \\
C.JALR & src$\neq$0 & 0 & C2 \\
\end{tabular}
\end{center}
这些指令使用CR格式。
% These instructions use the CR format.

C.JR(跳转寄存器)实施到寄存器{\em rs1}中的地址的无条件控制转移。C.JR扩展到{\tt jalr x0, 0(rs1)}。C.JR只有在$\textit{rs1}{\neq}\texttt{x0}$时有效;$\textit{rs1}{=}\texttt{x0}$的代码点被保留。
% C.JR (jump register) performs an unconditional control transfer to
% the address in register {\em rs1}.  C.JR expands to {\tt jalr x0, 0(rs1)}.
% C.JR is only valid when $\textit{rs1}{\neq}\texttt{x0}$; the code point
% with $\textit{rs1}{=}\texttt{x0}$ is reserved.

C.JALR(跳转和链接寄存器)实施与C.JR相同的操作,但是额外把跳转({\tt pc}+2)之后的指令的地址写到链接寄存器{\tt x1}。
C.JALR扩展到{\tt jalr x1, 0(rs1)}。C.JALR只有在$\textit{rs1}{\neq}\texttt{x0}$时有效;$\textit{rs1}{=}\texttt{x0}$的代码点对应于C.EBREAK指令。
% C.JALR (jump and link register) performs the same operation as C.JR,
% but additionally writes the address of the instruction following the
% jump ({\tt pc}+2) to the link register, {\tt x1}.  C.JALR expands to
% {\tt jalr x1, 0(rs1)}.
% C.JALR is only valid when $\textit{rs1}{\neq}\texttt{x0}$; the code point
% with $\textit{rs1}{=}\texttt{x0}$ corresponds to the C.EBREAK instruction.

\begin{commentary}
严格地讲,C.JALR并不确切地扩展到某个基础RVI指令,因为加到{\tt pc}形成链接地址的值是2,而不像基础ISA中那样是4,
但是同时支持2字节和4字节的偏移量对于基础微架构只是一个非常微小的改变。
% Strictly speaking, C.JALR does not expand exactly to a base RVI
% instruction as the value added to the {\tt pc} to form the link address is 2
% rather than 4 as in the base ISA, but supporting both offsets of 2 and
% 4 bytes is only a very minor change to the base microarchitecture.
\end{commentary}

\begin{center}
\begin{tabular}{S@{}S@{}S@{}T@{}Y}
\\
\instbitrange{15}{13} &
\instbitrange{12}{10} &
\instbitrange{9}{7} &
\instbitrange{6}{2} &
\instbitrange{1}{0} \\
\hline
\multicolumn{1}{|c|}{funct3} &
\multicolumn{1}{c|}{imm} &
\multicolumn{1}{c|}{\rsoneprime} &
\multicolumn{1}{c|}{imm} &
\multicolumn{1}{c|}{op} \\
\hline
3 & 3 & 3 & 5 & 2 \\
C.BEQZ & offset[8$\vert$4:3] & src & offset[7:6$\vert$2:1$\vert$5] & C1 \\
C.BNEZ & offset[8$\vert$4:3] & src & offset[7:6$\vert$2:1$\vert$5] & C1 \\
\end{tabular}
\end{center}
这些指令使用CB格式。
% These instructions use the CB format.

C.BEQZ实施条件控制转移。偏移量被符号扩展,并被加到{\tt pc}以形成分支目标地址。它因此可以有$\pm$\wunits{256}{B}的目标范围。
如果寄存器{\em \rsoneprime}中的值是零,C.BEQZ采取其分支。它扩展到{\tt beq \rsoneprime, x0, offset}。
% C.BEQZ performs conditional control transfers.  The offset is sign-extended
% and added to the {\tt pc} to form the branch target address.  It can
% therefore target a $\pm$\wunits{256}{B} range.  C.BEQZ takes the branch if the
% value in register {\em \rsoneprime} is zero.  It expands to {\tt beq \rsoneprime, x0,
% offset}.

C.BNEZ的定义类似,但是它采取其分支,是在{\em \rsoneprime}包含一个非零的值时。它扩展到{\tt bne \rsoneprime, x0, offset}。
% C.BNEZ is defined analogously, but it takes the branch if {\em \rsoneprime} contains
% a nonzero value.  It expands to {\tt bne \rsoneprime, x0, offset}.

\section{整数运算指令}
% \section{Integer Computational Instructions}

RVC提供了用于整数运算和常量生成的一些指令。
% RVC provides several instructions for integer arithmetic and constant generation.

\subsection*{整数常量-生成指令}
% \subsection*{Integer Constant-Generation Instructions}

这两个常量生成指令都使用CI指令格式,并且能够把任何整数寄存器作为目标。
% The two constant-generation instructions both use the CI instruction
% format and can target any integer register.

\vspace{-0.4in}
\begin{center}
\begin{tabular}{S@{}W@{}T@{}T@{}Y}
\\
\instbitrange{15}{13} &
\multicolumn{1}{c}{\instbit{12}} &
\instbitrange{11}{7} &
\instbitrange{6}{2} &
\instbitrange{1}{0} \\
\hline
\multicolumn{1}{|c|}{funct3} &
\multicolumn{1}{c|}{imm[5]} &
\multicolumn{1}{c|}{rd} &
\multicolumn{1}{c|}{imm[4:0]} &
\multicolumn{1}{c|}{op} \\
\hline
3 & 1 & 5 & 5 & 2 \\
C.LI     & imm[5] & dest$\neq$0 & imm[4:0] & C1 \\
C.LUI    & nzimm[17] & $\textrm{dest}{\neq}{\left\{0,2\right\}}$ & nzimm[16:12] & C1 \\
\end{tabular}
\end{center}
C.LI把符号扩展的6位立即数{\em imm}加载进寄存器{\em rd}中。C.LI扩展到{\tt addi rd, x0, imm}。
C.LI只有当{\em rd}$\neq${\tt x0}时有效;带有{\em rd}={\tt x0}的代码点编码了HINT。
% C.LI loads the sign-extended 6-bit immediate, {\em imm}, into
% register {\em rd}.
% C.LI expands into {\tt addi rd, x0, imm}.
% C.LI is only valid when {\em rd}$\neq${\tt x0};
% the code points with {\em rd}={\tt x0} encode HINTs.

C.LUI把非零的6位立即数域加载到目的寄存器的位17 - 12,清除底部的12位,并把位17符号扩展到目的寄存器的所有更高位。
C.LUI扩展到{\tt lui rd, nzimm}。C.LUI只有当$\textit{rd}{\neq}{\left\{\texttt{x0},\texttt{x2}\right\}}$,并且当立即数不等于零时有效。
{\em nzimm}=0的代码点被保留;余下的{\em rd}={\tt x0}的代码点是HINT;余下的{\em rd}={\tt x2}的代码点对应于C.ADDI16SP指令。
% C.LUI loads the non-zero 6-bit immediate field into bits 17--12 of the
% destination register, clears the bottom 12 bits, and sign-extends bit
% 17 into all higher bits of the destination.
% C.LUI expands into {\tt lui rd, nzimm}.
% C.LUI is only valid when
% $\textit{rd}{\neq}{\left\{\texttt{x0},\texttt{x2}\right\}}$,
% and when the immediate is not equal to zero.
% The code points with {\em nzimm}=0 are reserved; the remaining code points
% with {\em rd}={\tt x0} are HINTs; and the remaining code points with
% {\em rd}={\tt x2} correspond to the C.ADDI16SP instruction.

\subsection*{整数寄存器-立即数操作}
% \subsection*{Integer Register-Immediate Operations}

这些整数寄存器-立即数操作以CI格式编码,并实施在整数寄存器和6位立即数上的操作。
% These integer register-immediate operations are encoded in the CI
% format and perform operations on an integer register and
% a 6-bit immediate.

\vspace{-0.4in}
\begin{center}
\begin{tabular}{S@{}W@{}T@{}T@{}Y}
\\
\instbitrange{15}{13} &
\multicolumn{1}{c}{\instbit{12}} &
\instbitrange{11}{7} &
\instbitrange{6}{2} &
\instbitrange{1}{0} \\
\hline
\multicolumn{1}{|c|}{funct3} &
\multicolumn{1}{c|}{imm[5]} &
\multicolumn{1}{c|}{rd/rs1} &
\multicolumn{1}{c|}{imm[4:0]} &
\multicolumn{1}{c|}{op} \\
\hline
3 & 1 & 5 & 5 & 2 \\
C.ADDI     & nzimm[5] & dest$\neq$0 & nzimm[4:0] & C1 \\
C.ADDIW    & imm[5]   & dest$\neq$0 & imm[4:0] & C1 \\
C.ADDI16SP & nzimm[9] & 2 & nzimm[4$\vert$6$\vert$8:7$\vert$5] & C1 \\
\end{tabular}
\end{center}

C.ADDI把非零的符号扩展的6位立即数加到寄存器{\em rd}中的值,然后把结果写到{\em rd}。
C.ADDI扩展到{\tt addi rd, rd, nzimm}。C.ADDI只有当{\tt addi rd, rd, nzimm}$\neq${\tt x0}且{\em nzimm}$\neq$0时有效。
{\em rd}={\tt x0}的代码点编码了C.NOP指令;余下的{\em nzimm}=0的代码点编码了HINT。
% C.ADDI adds the non-zero sign-extended 6-bit immediate to the value in
% register {\em rd} then writes the result to {\em rd}.  C.ADDI expands
% into {\tt addi rd, rd, nzimm}.
% C.ADDI is only valid when {\em rd}$\neq${\tt x0} and {\em nzimm}$\neq$0.
% The code points with {\em rd}={\tt x0} encode the C.NOP instruction;
% the remaining code points with {\em nzimm}=0 encode HINTs.

C.ADDIW是一个RV64C/RV128C独有的指令,它执行相同的计算,但是产生一个32位的结果,然后把结果符号扩展到64位。
C.ADDIW扩展到{\tt addiw rd, rd, imm}。对于C.ADDIW,立即数可以是零,这对应于{\tt
sext.w rd}。
C.ADDIW只有在{\em rd}$\neq${\tt x0}时有效;{\em rd}={\tt x0}的代码点被保留。
% C.ADDIW is an RV64C/RV128C-only instruction that performs the same
% computation but produces a 32-bit result, then sign-extends result to
% 64 bits.  C.ADDIW expands into {\tt addiw rd, rd, imm}.  The
% immediate can be zero for C.ADDIW, where this corresponds to {\tt
% sext.w rd}.  C.ADDIW is only valid when {\em rd}$\neq${\tt x0};
% the code points with {\em rd}={\tt x0} are reserved.

C.ADDI16SP与C.LUI共享操作码,但是有一个目的域{\tt x2}。C.ADDI16SP把非零的符号扩展的6位立即数加到栈指针中的值({\tt sp}={\tt x2}),
那里立即数被缩放来代表范围(-512, 496)中的16的倍数。C.ADDI16SP被用于调整过程序言和结语中的栈指针。
它扩展到{\tt addi x2, x2, nzimm}。C.ADDI16SP只有当{\em nzimm}$\neq$0时有效;{\em nzimm}=0的代码点被保留。
% C.ADDI16SP shares the opcode with C.LUI, but has a destination field
% of {\tt x2}. C.ADDI16SP adds the non-zero sign-extended 6-bit immediate to
% the value in the stack pointer ({\tt sp}={\tt x2}), where the
% immediate is scaled to represent multiples of 16 in the range
% (-512,496). C.ADDI16SP is used to adjust the stack pointer in procedure
% prologues and epilogues.  It expands into {\tt addi x2, x2, nzimm}.
% C.ADDI16SP is only valid when {\em nzimm}$\neq$0;
% the code point with {\em nzimm}=0 is reserved.

\begin{commentary}
在标准的RISC-V调用约定中,栈指针{\tt sp}总是16位对齐的。
% In the standard RISC-V calling convention, the stack pointer {\tt sp}
% is always 16-byte aligned.
\end{commentary}

\begin{center}
\begin{tabular}{@{}S@{}K@{}S@{}Y}
\\
\instbitrange{15}{13} &
\instbitrange{12}{5} &
\instbitrange{4}{2} &
\instbitrange{1}{0} \\
\hline
\multicolumn{1}{|c|}{funct3} &
\multicolumn{1}{c|}{imm} &
\multicolumn{1}{c|}{\rdprime} &
\multicolumn{1}{c|}{op} \\
\hline
3 & 8 & 3 & 2 \\
C.ADDI4SPN & nzuimm[5:4$\vert$9:6$\vert$2$\vert$3] & dest & C0 \\
\end{tabular}
\end{center}

C.ADDI4SPN是一个CIW格式的指令,它把一个{\em 零}扩展的非零立即数,扩大4倍,加到栈指针{\tt x2}上,并把结果写到{\tt \rdprime}。
这个指令被用于生成指向栈分配变量的指针,且扩展到{\tt addi \rdprime, x2, nzuimm}。C.ADDI4SPN只在{\em nzuimm}$\neq$0时生效;{\em nzuimm}=0的代码点被保留。
% C.ADDI4SPN is a CIW-format instruction that adds a {\em zero}-extended
% non-zero immediate, scaled by 4, to the stack pointer, {\tt x2}, and
% writes the result to {\tt \rdprime}.  This instruction is used
% to generate pointers to stack-allocated variables, and expands to
% {\tt addi \rdprime, x2, nzuimm}.
% C.ADDI4SPN is only valid when {\em nzuimm}$\neq$0;
% the code points with {\em nzuimm}=0 are reserved.

\vspace{-0.4in}
\begin{center}
\begin{tabular}{S@{}W@{}T@{}T@{}Y}
\\
\instbitrange{15}{13} &
\multicolumn{1}{c}{\instbit{12}} &
\instbitrange{11}{7} &
\instbitrange{6}{2} &
\instbitrange{1}{0} \\
\hline
\multicolumn{1}{|c|}{funct3} &
\multicolumn{1}{c|}{shamt[5]} &
\multicolumn{1}{c|}{rd/rs1} &
\multicolumn{1}{c|}{shamt[4:0]} &
\multicolumn{1}{c|}{op} \\
\hline
3 & 1 & 5 & 5 & 2 \\
C.SLLI  & shamt[5] & dest$\neq$0 & shamt[4:0] & C2 \\
\end{tabular}
\end{center}

SLLI是一个CI格式的指令,它对寄存器{\em rd}中的值执行逻辑左移,然后把结果写到{\em rd}。
移位的数目被编码在{\em shamt}域之中。对于RV128C,移位数目零被用于编码64的移位。
C.SLLI扩展到{\tt slli rd, rd, shamt};但{\tt shamt=0}的RV128C除外,它扩展到{\tt slli rd, rd, 64}。
% C.SLLI is a CI-format instruction that performs a logical left shift
% of the value in register {\em rd} then writes the result to {\em rd}.
% The shift amount is encoded in the {\em shamt} field.
% For RV128C, a shift amount of zero is used to encode a shift of 64.
% C.SLLI expands into {\tt slli rd, rd, shamt}, except for
% RV128C with {\tt shamt=0}, which expands to {\tt slli rd, rd, 64}.

对于RV32C,{\em shamt[5]}必须是零;{\em shamt[5]}=1的代码点被指定用于自定义扩展。
对于RV32C和RV64C,移位的数目必须是非零的;{\em shamt}=0的代码点是HINT。
对于所有的基础ISA,除了RV32C中那些{\em shamt[5]}=1的之外,{\em rd}={\tt x0}的代码点都是HINT。
% For RV32C, {\em shamt[5]} must be zero; the code points with {\em shamt[5]}=1
% are designated for custom extensions.  For RV32C and RV64C, the shift
% amount must be non-zero; the code points with {\em shamt}=0 are HINTs.  For
% all base ISAs, the code points with {\em rd}={\tt x0} are HINTs, except those
% with {\em shamt[5]}=1 in RV32C.

\vspace{-0.4in}
\begin{center}
\begin{tabular}{S@{}W@{}Y@{}S@{}T@{}Y}
\\
\instbitrange{15}{13} &
\multicolumn{1}{c}{\instbit{12}} &
\instbitrange{11}{10} &
\instbitrange{9}{7} &
\instbitrange{6}{2} &
\instbitrange{1}{0} \\
\hline
\multicolumn{1}{|c|}{funct3} &
\multicolumn{1}{c|}{shamt[5]} &
\multicolumn{1}{|c|}{funct2} &
\multicolumn{1}{c|}{\rdprime/\rsoneprime} &
\multicolumn{1}{c|}{shamt[4:0]} &
\multicolumn{1}{c|}{op} \\
\hline
3 & 1 & 2 & 3 & 5 & 2 \\
C.SRLI  & shamt[5] & C.SRLI & dest & shamt[4:0] & C1 \\
C.SRAI  & shamt[5] & C.SRAI & dest & shamt[4:0] & C1 \\
\end{tabular}
\end{center}

C.SRLI是一个CB格式的指令,它对寄存器{\em \rdprime}中的值实施逻辑右移,然后把结果写到{\em \rdprime}。
移位的数目被编码在{\em shamt}域之中。对于RV128C,移位数目零被用于编码64的移位。
甚至,对于RV128C,移位数目被符号扩展,并因此合法的移位数目是1 - 13,64,和96 - 127。
C.SRLI扩展到{\tt srli \rdprime, \rdprime, shamt};但{\tt shamt=0}的RV128C除外,它扩展到{\tt srli \rdprime, \rdprime, 64}。
% C.SRLI is a CB-format instruction that performs a logical right shift
% of the value in register {\em \rdprime} then writes the result to {\em \rdprime}.
% The shift amount is encoded in the {\em shamt} field.
% For RV128C, a shift amount of zero is used to encode a shift of 64.
% Furthermore, the shift amount is sign-extended
% for RV128C, and so the legal shift amounts are 1--31, 64, and 96--127.
% C.SRLI expands into {\tt srli \rdprime, \rdprime, shamt},
% except for RV128C with {\tt shamt=0}, which expands to
% {\tt srli \rdprime, \rdprime, 64}.

对于RV32C,{\em shamt[5]}必须是零;{\em shamt[5]}=1的代码点被指定用于自定义扩展。对于RV32C和RV64C,移位数目必须是非零的;{\em shamt}=0的代码点是HINT。
% For RV32C, {\em shamt[5]} must be zero; the code points with {\em shamt[5]}=1
% are designated for custom extensions.  For RV32C and RV64C, the shift
% amount must be non-zero; the code points with {\em shamt}=0 are HINTs.

C.SRAI的定义与C.SRLI类似,但是执行的是算数右移。C.SRAI扩展到{\tt srai \rdprime, \rdprime, shamt}。
% C.SRAI is defined analogously to C.SRLI, but instead performs an arithmetic
% right shift.
% C.SRAI expands to {\tt srai \rdprime, \rdprime, shamt}.

\begin{commentary}
左移通常比右移更加频繁,因为左移被频繁用于放缩地址的值。右移因此被赋予较少的编码空间,并被放置在一个编码象限中,
那里所有其它的立即数都是被符号扩展的。对于RV128,作出该决策是为了让6位移位数目立即数也被符号扩展。
除了减少解码的复杂度,我们相信96 -127的右移数目比64 - 95的数目更加有用,以允许提取位于128位地址指针的高部分中的标签。
我们注意到RV128C将不会与RV32C和RV64C被冻结在同一点,以允许评估128位地址空间代码的常见用法。
% Left shifts are usually more frequent than right shifts, as left
% shifts are frequently used to scale address values.  Right shifts have
% therefore been granted less encoding space and are placed in an
% encoding quadrant where all other immediates are sign-extended.  For
% RV128, the decision was made to have the 6-bit shift-amount immediate
% also be sign-extended.  Apart from reducing the decode complexity, we
% believe right-shift amounts of 96--127 will be more useful than 64--95,
% to allow extraction of tags located in the high portions of 128-bit
% address pointers.  We note that RV128C will not be frozen at the same
% point as RV32C and RV64C, to allow evaluation of typical usage of
% 128-bit address-space codes.
\end{commentary}

\begin{center}
\begin{tabular}{S@{}W@{}Y@{}S@{}T@{}Y}
\\
\instbitrange{15}{13} &
\multicolumn{1}{c}{\instbit{12}} &
\instbitrange{11}{10} &
\instbitrange{9}{7} &
\instbitrange{6}{2} &
\instbitrange{1}{0} \\
\hline
\multicolumn{1}{|c|}{funct3} &
\multicolumn{1}{c|}{imm[5]} &
\multicolumn{1}{|c|}{funct2} &
\multicolumn{1}{c|}{\rdprime/\rsoneprime} &
\multicolumn{1}{c|}{imm[4:0]} &
\multicolumn{1}{c|}{op} \\
\hline
3 & 1 & 2 & 3 & 5 & 2 \\
C.ANDI  & imm[5] & C.ANDI & dest & imm[4:0] & C1 \\
\end{tabular}
\end{center}

C.ANDI是一个CB格式的指令,它计算寄存器{\em \rdprime}中的值与符号扩展的6位立即数的按位AND,然后把结果写到{\em \rdprime}。
C.ANDI扩展到{\tt andi \rdprime, \rdprime, imm}。
% C.ANDI is a CB-format instruction that computes the bitwise AND of
% the value in register {\em \rdprime} and the sign-extended 6-bit immediate,
% then writes the result to {\em \rdprime}.
% C.ANDI expands to {\tt andi \rdprime, \rdprime, imm}.


\subsection*{整数寄存器-寄存器操作}
\subsection*{Integer Register-Register Operations}
\vspace{-0.4in}
\begin{center}
\begin{tabular}{E@{}T@{}T@{}Y}
\\
\instbitrange{15}{12} &
\instbitrange{11}{7} &
\instbitrange{6}{2} &
\instbitrange{1}{0} \\
\hline
\multicolumn{1}{|c|}{funct4} &
\multicolumn{1}{c|}{rd/rs1} &
\multicolumn{1}{c|}{rs2} &
\multicolumn{1}{c|}{op} \\
\hline
4 & 5 & 5 & 2 \\
C.MV & dest$\neq$0 & src$\neq$0 & C2 \\
C.ADD & dest$\neq$0 & src$\neq$0 & C2 \\
\end{tabular}
\end{center}
这些指令使用CR格式。
% These instructions use the CR format.

C.MV把寄存器{\em rs2}中的值复制到寄存器{\em rd}中。C.MV扩展到{\tt add rd, x0, rs2}。
C.MV只在rs2=x0时有效;$\textit{rs2}{=}\texttt{x0}$的代码点对应于C.JR指令。
$\textit{rs2}{\neq}\texttt{x0}$和$\textit{rd}{=}\texttt{x0}$的代码点是HINT。
% C.MV copies the value in register {\em rs2} into register {\em rd}.  C.MV
% expands into {\tt add rd, x0, rs2}.
% C.MV is only valid when $\textit{rs2}{\neq}\texttt{x0}$; the code points
% with $\textit{rs2}{=}\texttt{x0}$ correspond to the C.JR instruction.
% The code points with $\textit{rs2}{\neq}\texttt{x0}$ and
% $\textit{rd}{=}\texttt{x0}$ are HINTs.

\begin{commentary}
C.MV扩展到与典型的MV伪指令(其使用ADDI)不同的指令。
专门处理MV的实现,例如,使用寄存器重命名的硬件,可能会发现把C.MV扩展到MV而不是ADD会更加方便,只需轻微的额外的硬件开销。
% C.MV expands to a different instruction than the canonical MV
% pseudoinstruction, which instead uses ADDI. Implementations that handle MV
% specially, e.g. using register-renaming hardware, may find it more convenient
% to expand C.MV to MV instead of ADD, at slight additional hardware cost.
\end{commentary}

C.ADD把寄存器{\em rd}和{\em rs2}中的值相加,并把结果写到寄存器{\em rd}。
C.ADD扩展到{\tt add rd, rd, rs2}。C.ADD只在$\textit{rs2}{\neq}\texttt{x0}$时有效;r$\textit{rs2}{=}\texttt{x0}$的代码点对应于C.JALR和C.EBREAK指令。
$\textit{rs2}{\neq}\texttt{x0}$0和$\textit{rd}{=}\texttt{x0}$的代码点是HINT。
% C.ADD adds the values in registers {\em rd} and {\em rs2} and writes the
% result to register {\em rd}.  C.ADD expands into {\tt add rd, rd, rs2}.
% C.ADD is only valid when $\textit{rs2}{\neq}\texttt{x0}$; the code points
% with $\textit{rs2}{=}\texttt{x0}$ correspond to the C.JALR and C.EBREAK instructions.
% The code points with $\textit{rs2}{\neq}\texttt{x0}$ and
% $\textit{rd}{=}\texttt{x0}$ are HINTs.

\vspace{-0.4in}
\begin{center}
\begin{tabular}{M@{}S@{}Y@{}S@{}Y}
\\
\instbitrange{15}{10} &
\instbitrange{9}{7} &
\instbitrange{6}{5} &
\instbitrange{4}{2} &
\instbitrange{1}{0} \\
\hline
\multicolumn{1}{|c|}{funct6} &
\multicolumn{1}{c|}{\rdprime/\rsoneprime} &
\multicolumn{1}{c|}{funct2} &
\multicolumn{1}{c|}{\rstwoprime} &
\multicolumn{1}{c|}{op} \\
\hline
6 & 3 & 2 & 3 & 2 \\
C.AND  & dest & C.AND  & src & C1 \\
C.OR   & dest & C.OR   & src & C1 \\
C.XOR  & dest & C.XOR  & src & C1 \\
C.SUB & dest & C.SUB & src & C1 \\
C.ADDW & dest & C.ADDW & src & C1 \\
C.SUBW & dest & C.SUBW & src & C1 \\
\end{tabular}
\end{center}

这些指令使用CA格式。
% These instructions use the CA format.

C.AND 计算寄存器{\em \rdprime}和{\em \rstwoprime}中的值的按位AND,然后把结果写到寄存器{\em \rdprime}。C.AND扩展到{\tt and \rdprime, \rdprime, \rstwoprime}。
% C.AND computes the bitwise AND of the values in registers {\em \rdprime}
% and {\em \rstwoprime}, then writes the result to register {\em \rdprime}.
% C.AND expands into {\tt and \rdprime, \rdprime, \rstwoprime}.

C.OR 计算寄存器{\em \rdprime}和{\em \rstwoprime}中的值的按位OR,然后把结果写到寄存器{\em \rdprime}。C.OR扩展到{\tt or \rdprime, \rdprime, \rstwoprime}。
% C.OR computes the bitwise OR of the values in registers {\em \rdprime}
% and {\em \rstwoprime}, then writes the result to register {\em \rdprime}.
% C.OR expands into {\tt or \rdprime, \rdprime, \rstwoprime}.

C.XOR计算寄存器{\em \rdprime}和{\em \rstwoprime}中的值的按位XOR,然后把结果写到寄存器{\em \rdprime}。C.XOR扩展到{\tt xor \rdprime, \rdprime, \rstwoprime}。
% C.XOR computes the bitwise XOR of the values in registers {\em \rdprime}
% and {\em \rstwoprime}, then writes the result to register {\em \rdprime}.
% C.XOR expands into {\tt xor \rdprime, \rdprime, \rstwoprime}.

C.SUB从寄存器{\em \rdprime}中的值中减去寄存器{\em \rstwoprime}中的值,然后把结果写到寄存器{\em \rdprime}。C.SUB扩展到{\tt sub \rdprime, \rdprime, \rstwoprime}。
% C.SUB subtracts the value in register {\em \rstwoprime} from the value in
% register {\em \rdprime}, then writes the result to register {\em \rdprime}.
% C.SUB expands into {\tt sub \rdprime, \rdprime, \rstwoprime}.

C.ADDW是一个RV64C/4V128C独有的指令,它把寄存器{\em \rdprime}和{\em \rstwoprime}中的值相加,然后在把结果写到寄存器{\em \rdprime}之前,对和的低32位进行符号扩展。C.ADDW扩展到{\tt addw \rdprime, \rdprime, \rstwoprime}。
% C.ADDW is an RV64C/RV128C-only instruction that adds the values in
% registers {\em \rdprime} and {\em \rstwoprime}, then sign-extends the lower
% 32 bits of the sum before writing the result to register {\em \rdprime}.
% C.ADDW expands into {\tt addw \rdprime, \rdprime, \rstwoprime}.

C.SUBW是一个RV64C/RV128独有的指令,它从寄存器{\em \rdprime}中的值中减去寄存器{\em \rstwoprime}中的值,然后在把结果写到寄存器{\em \rdprime}之前,对差的低32位进行符号扩展。C.SUBW扩展到{\tt subw \rdprime, \rdprime, \rstwoprime}。
% C.SUBW is an RV64C/RV128C-only instruction that subtracts the value in
% register {\em \rstwoprime} from the value in register {\em \rdprime}, then
% sign-extends the lower 32 bits of the difference before writing the result
% to register {\em \rdprime}. C.SUBW expands into {\tt subw \rdprime, \rdprime, \rstwoprime}.

\begin{commentary}
这组的六个指令虽然不会各自提供(对资源的)大量节省,但是不会占据太多的编码空间,而且易于实现;
并且作为一个组,在静态和动态压缩方面提供了值得改进的地方。
% This group of six instructions do not provide large savings
% individually, but do not occupy much encoding space and are
% straightforward to implement, and as a group provide a worthwhile
% improvement in static and dynamic compression.
\end{commentary}

\subsection*{Defined Illegal Instruction}
\vspace{-0.4in}
\begin{center}
\begin{tabular}{SW@{}T@{}T@{}Y}
\\
\instbitrange{15}{13} &
\multicolumn{1}{c}{\instbit{12}} &
\instbitrange{11}{7} &
\instbitrange{6}{2} &
\instbitrange{1}{0} \\
\hline
\multicolumn{1}{|c|}{0} &
\multicolumn{1}{c|}{0} &
\multicolumn{1}{c|}{0} &
\multicolumn{1}{c|}{0} &
\multicolumn{1}{c|}{0} \\
\hline
3 & 1 & 5 & 5 & 2 \\
0 & 0 & 0 & 0 & 0 \\
\end{tabular}
\end{center}

所有位都是零的16位指令被永久性地保留为一个非法指令。
% A 16-bit instruction with all bits zero is permanently reserved as an
% illegal instruction.
\begin{commentary}
我们把全零指令保留为非法指令,以帮助对尝试执行内存空间中以零结尾的、或者不存在的部分进行陷入。
在任何非标准的扩展中,全零的值都应当被重新定义。
类似地,我们保留了所有位都设为1的指令(对应于RISC-V可变长度编码策略中的非常长的指令)作为非法指令,
以捕获另一些常见的在不存在的内存区域中的值。
% We reserve all-zero instructions to be illegal instructions to help
% trap attempts to execute zero-ed or non-existent portions of the
% memory space.  The all-zero value should not be redefined in any
% non-standard extension.  Similarly, we reserve instructions with all
% bits set to 1 (corresponding to very long instructions in the RISC-V
% variable-length encoding scheme) as illegal to capture another common
% value seen in non-existent memory regions.
\end{commentary}

\subsection*{NOP指令}
% \subsection*{NOP Instruction}
\vspace{-0.4in}
\begin{center}
\begin{tabular}{SW@{}T@{}T@{}Y}
\\
\instbitrange{15}{13} &
\multicolumn{1}{c}{\instbit{12}} &
\instbitrange{11}{7} &
\instbitrange{6}{2} &
\instbitrange{1}{0} \\
\hline
\multicolumn{1}{|c|}{funct3} &
\multicolumn{1}{c|}{imm[5]} &
\multicolumn{1}{c|}{rd/rs1} &
\multicolumn{1}{c|}{imm[4:0]} &
\multicolumn{1}{c|}{op} \\
\hline
3 & 1 & 5 & 5 & 2 \\
C.NOP & 0 & 0 & 0 & C1 \\
\end{tabular}
\end{center}

C.NOP是一个CI格式的指令,除了提升{\tt pc}和增加任何适用的性能计数器,它不改变任何用户可见的状态。
C.NOP扩展到{\tt nop}。C.NOP只有在{\em imm}=0时有效;{\em imm}$\neq$0的代码点编码了HINT。
% C.NOP is a CI-format instruction that does not change any user-visible state,
% except for advancing the {\tt pc} and incrementing any applicable performance
% counters.  C.NOP expands to {\tt nop}.  C.NOP is only valid when {\em imm}=0;
% the code points with {\em imm}$\neq$0 encode HINTs.

\subsection*{断点指令}
% \subsection*{Breakpoint Instruction}
\vspace{-0.4in}
\begin{center}
\begin{tabular}{E@{}U@{}Y}
\\
\instbitrange{15}{12} &
\instbitrange{11}{2} &
\instbitrange{1}{0} \\
\hline
\multicolumn{1}{|c|}{funct4} &
\multicolumn{1}{c|}{0} &
\multicolumn{1}{c|}{op} \\
\hline
4 & 10 & 2 \\
C.EBREAK & 0 & C2 \\
\end{tabular}
\end{center}

调试器可以使用C.EBREAK指令,它扩展到{\tt ebreak},以造成控制被转移回调试环境。
C.EBREAK与C.ADD共享操作码,但是{\em rd}和{\em rs2}都是零,因此也可以使用CR格式。
% Debuggers can use the C.EBREAK instruction, which expands to {\tt ebreak},
% to cause control to be transferred back to the debugging environment.
% C.EBREAK shares the opcode with the C.ADD instruction, but with {\em
%   rd} and {\em rs2} both zero, thus can also use the CR format.

\section{C指令在LR/SC序列中的使用}
% \section{Usage of C Instructions in LR/SC Sequences}

在支持C扩展的实现上,在受限的LR/SC序列内部允许I指令的压缩形式,就像~\ref{sec:lrscseq}节中描述的那样,也允许在受限的LR/SC序列中使用。
% On implementations that support the C extension, compressed forms of the
% I instructions permitted inside constrained LR/SC sequences, as described in
% Section~\ref{sec:lrscseq}, are also permitted inside constrained LR/SC
% sequences.

\begin{commentary}
这意味着,任何声称同时支持A扩展和C扩展的实现都必须确保:包含有效C指令的LR/SC序列将最终完成。
% The implication is that any implementation that claims to support both
% the A and C extensions must ensure that LR/SC sequences containing
% valid C instructions will eventually complete.
\end{commentary}

\section{“提示”指令}
% \section{HINT Instructions}
\label{sec:rvc-hints}

RVC编码空间的一部分被保留用于微架构HINT。像在RV32I基础ISA中的HINT(见~\ref{sec:rv32i-hints}节),
这些指令除了增加{\tt pc}和任何适用的性能计数器,不修改任何架构状态。在实现上,HINT作为no-op执行而忽略它们。
% A portion of the RVC encoding space is reserved for microarchitectural HINTs.
% Like the HINTs in the RV32I base ISA (see Section~\ref{sec:rv32i-hints}),
% these instructions do not modify any architectural state, except for advancing
% the {\tt pc} and any applicable performance counters.  HINTs are
% executed as no-ops on implementations that ignore them.

RVC HINT被编码为不修改架构状态的运算指令,或者是因为{\em rd}={\tt x0}(例如,\mbox{C.ADD {\em x0}, {\em t0}}),
或者是因为{\em rd}被它自己的拷贝所覆写(例如,\mbox{C.ADDI {\em t0}, 0})。
% RVC HINTs are encoded as computational instructions that do not modify the
% architectural state, either because {\em rd}={\tt x0}
% (e.g. \mbox{C.ADD {\em x0}, {\em t0}}), or because {\em rd} is overwritten
% with a copy of itself (e.g. \mbox{C.ADDI {\em t0}, 0}).

\begin{commentary}
选择这样的HINT编码,使得简单的实现可以忽略全部的HINT,代替为把HINT作为一个恰好不会改变架构状态的常规运算指令来执行。
% This HINT encoding has been chosen so that simple implementations can ignore
% HINTs altogether, and instead execute a HINT as a regular computational
% instruction that happens not to mutate the architectural state.
\end{commentary}

没有必要把RVC HINT扩展到它们对应的RVI HINT。例如,\mbox{C.ADD {\em x0}, {\em a0}}可能不会编码为与\mbox{ADD {\em x0}, {\em x0}, {\em a0}}相同的HINT。
% RVC HINTs do not necessarily expand to their RVI HINT counterparts.  For
% example, \mbox{C.ADD {\em x0}, {\em a0}} might not encode the same HINT
% as \mbox{ADD {\em x0}, {\em x0}, {\em a0}}.

\begin{commentary}
不需要把RVC HINT扩展到RVI HINT的主要原因是,HINT不可能以与底层运算指令相同的方式被压缩。
并且,解耦RVC和RVI HINT的映射可以使稀缺的RVC HINT空间被分配给最常用的HINT,特别地,分配给适用于宏操作融合的HINT。
% The primary reason to not require an RVC HINT to expand to an RVI HINT
% is that HINTs are unlikely to be compressible in the same manner as
% the underlying computational instruction.  Also, decoupling the RVC
% and RVI HINT mappings allows the scarce RVC HINT space to be allocated
% to the most popular HINTs, and in particular, to HINTs that are
% amenable to macro-op fusion.
\end{commentary}

表~\ref{tab:rvc-hints}列出了所有的RVC HINT代码点。对于RV32C,78\%的HINT空间被保留给标准HINT。
余下的HINT空间被指定给自定义HINT:永远不会有标准HINT将被定义在这个子空间中
% Table~\ref{tab:rvc-hints} lists all RVC HINT code points.  For RV32C, 78\% of
% the HINT space is reserved for standard HINTs.
% The remainder of the HINT space is designated for custom HINTs: no standard
% HINTs will ever be defined in this subspace.

\begin{table}[hbt]
\centering
\begin{tabular}{|l|l|r|l|}
  \hline
  指令                     & 约束                                        & 代码点        & 用途 \\ \hline \hline
  C.NOP                   & {\em nzimm}$\neq$0                          & 63          & \multirow{6}{*}{\em 保留供未来标准使用} \\ \cline{1-3}
  C.ADDI                  & {\em rd}$\neq${\tt x0}, {\em nzimm}=0       & 31          & \\ \cline{1-3}
  C.LI                    & {\em rd}={\tt x0}                           & 64          & \\ \cline{1-3}
  C.LUI                   & {\em rd}={\tt x0}, {\em nzimm}$\neq$0       & 63          & \\ \cline{1-3}
  C.MV                    & {\em rd}={\tt x0}, {\em rs2}$\neq${\tt x0}  & 31          & \\ \cline{1-3}
  C.ADD                   & {\em rd}={\tt x0}, {\em rs2}$\neq${\tt x0}, {\em rs2}$\neq${\tt x2}--{\tt x5} & 27   & \\ \hline
  \multirow{4}{*}{C.ADD}  & \multirow{4}{*}{{\em rd}={\tt x0}, {\em rs2}={\tt x2}--{\tt x5}}
                                                                        & \multirow{4}{*}{$4$}
                                                                                      & ({\em rs2}={\tt x2}) C.NTL.P1 \\
                          &                                             &             & ({\em rs2}={\tt x3}) C.NTL.PALL \\
                          &                                             &             & ({\em rs2}={\tt x4}) C.NTL.S1 \\
                          &                                             &             & ({\em rs2}={\tt x5}) C.NTL.ALL \\ \hline
  \multirow{2}{*}{C.SLLI} & \multirow{2}{*}{{\em rd}={\tt x0}, {\em nzimm}$\neq$0} & 31 (RV32)   & \multirow{6}{*}{\em 指定为自定义使用} \\
                          &                                             & 63 (RV64/128) & \\ \cline{1-3}
  C.SLLI64                & {\em rd}={\tt x0}                           & 1           & \\ \cline{1-3}
  C.SLLI64                & {\em rd}$\neq${\tt x0}, RV32 和 RV64 独有  & 31          & \\ \cline{1-3}
  C.SRLI64                & RV32 和 RV64 独有                          & 8           & \\ \cline{1-3}
  C.SRAI64                & RV32 和 RV64 独有                          & 8           & \\ \hline
\end{tabular}
\caption{RVC HINT指令。}
\label{tab:rvc-hints}
\end{table}

\clearpage

\section{RVC指令集列表}
% \section{RVC Instruction Set Listings}

表~\ref{rvcopcodemap}显示了RVC主操作码的映射。表的每一行对应于编码空间的一个象限。最后一个象限,其设置了两个最小有效位,
对应于宽度超过16位的指令,包括那些在基础ISA中的指令。一些指令只对特定的操作数有效;
当无效的时候,它们被标记为{\em RES}来表示该操作码被保留用于未来的标准扩展;
或者标记为{\em Custom}来表示该操作码被指定用于自定义扩展;或者标记为{\em HINT}来表示该操作码被保留用于微架构提示(见~\ref{sec:rvc-hints}节)。
% Table~\ref{rvcopcodemap} shows a map of the major opcodes for RVC.
% Each row of the table corresponds to one quadrant of the encoding
% space.  The last quadrant, which has the two
% least-significant bits set, corresponds to instructions wider
% than 16 bits, including those in the base ISAs.  Several instructions
% are only valid for certain operands; when invalid, they are marked
% either {\em RES} to indicate that the opcode is reserved for future
% standard extensions; {\em Custom} to indicate that the opcode is designated
% for custom extensions; or {\em HINT} to indicate that the opcode
% is reserved for microarchitectural hints (see Section~\ref{sec:rvc-hints}).

\vspace{0.1in}
\definecolor{gray}{RGB}{180,180,180}
\begin{table*}[htbp]
\begin{center}
{\footnotesize
\setlength{\tabcolsep}{4pt}
\begin{tabular}{|r|c|c|c|c|c|c|c|c|l}
  \cline{1-9}
  inst[15:13] & \multirow{2}{*}{000}& \multirow{2}{*}{001}& \multirow{2}{*}{010}& \multirow{2}{*}{011}& \multirow{2}{*}{100}& \multirow{2}{*}{101}& \multirow{2}{*}{110}& \multirow{2}{*}{111}\\ \cline{1-1}
  inst[1:0] & & & & & & & & \\ \cline{1-9}
    \multirow{3}{*}{00} & \multirow{3}{*}{ADDI4SPN} & FLD   & \multirow{3}{*}{LW}   & FLW                           & \multirow{3}{*}{\em Reserved}  & FSD                & \multirow{3}{*}{SW}   & FSW                   & RV32  \\
                        &                           & FLD   &                       & LD                            &                                & FSD                &                       & SD                    & RV64  \\
                        &                           & LQ    &                       & LD                            &                                & SQ                 &                       & SD                    & RV128 \\ \hline
    \multirow{3}{*}{01} & \multirow{3}{*}{ADDI}     & JAL   & \multirow{3}{*}{LI}   & \multirow{3}{*}{LUI/ADDI16SP} & \multirow{3}{*}{MISC-ALU}      & \multirow{3}{*}{J} & \multirow{3}{*}{BEQZ} & \multirow{3}{*}{BNEZ} & RV32  \\
                        &                           & ADDIW &                       &                               &                                &                    &                       &                       & RV64  \\
                        &                           & ADDIW &                       &                               &                                &                    &                       &                       & RV128 \\ \hline
    \multirow{3}{*}{10} & \multirow{3}{*}{SLLI}     & FLDSP & \multirow{3}{*}{LWSP} & FLWSP                         & \multirow{3}{*}{J[AL]R/MV/ADD} & FSDSP              & \multirow{3}{*}{SWSP} & FSWSP                 & RV32  \\
                        &                           & FLDSP &                       & LDSP                          &                                & FSDSP              &                       & SDSP                  & RV64  \\
                        &                           & LQSP  &                       & LDSP                          &                                & SQSP               &                       & SDSP                  & RV128 \\ \cline{1-9}
    \cellcolor{gray} 11  & \multicolumn{8}{c|}{\cellcolor{gray} $>$16b} \\ \cline{1-9}
 \end{tabular}
}
\end{center}
\vspace{-0.15in}
\caption{RVC操作码映射}
\label{rvcopcodemap}
\end{table*}


表~\ref{rvc-instr-table0} - \ref{rvc-instr-table2} 列出了RVC指令。
% Tables~\ref{rvc-instr-table0}--\ref{rvc-instr-table2} list the RVC instructions.

\begin{table}[h]
\begin{small}
\begin{center}
\begin{tabular}{p{0in}p{0.05in}p{0.05in}p{0.05in}p{0.05in}p{0.05in}p{0.05in}p{0.05in}p{0.05in}p{0.05in}p{0.05in}p{0.05in}p{0.05in}p{0.05in}p{0.05in}p{0.05in}p{0.05in}l}
& & & & & & & & & & \\
                      &
\instbit{15} &
\instbit{14} &
\instbit{13} &
\multicolumn{1}{c}{\instbit{12}} &
\instbit{11} &
\instbit{10} &
\instbit{9} &
\instbit{8} &
\instbit{7} &
\instbit{6} &
\multicolumn{1}{c}{\instbit{5}} &
\instbit{4} &
\instbit{3} &
\instbit{2} &
\instbit{1} &
\instbit{0} \\
\cline{2-17}


&
\multicolumn{3}{|c|}{000} &
\multicolumn{8}{c|}{0} &
\multicolumn{3}{c|}{0} &
\multicolumn{2}{c|}{00} & {\em 非法指令} \\
\cline{2-17}

&
\multicolumn{3}{|c|}{000} &
\multicolumn{8}{c|}{nzuimm[5:4$\vert$9:6$\vert$2$\vert$3]} &
\multicolumn{3}{c|}{\rdprime} &
\multicolumn{2}{c|}{00} & C.ADDI4SPN {\em \tiny (RES, nzuimm=0)} \\
\whline{2-17}

&
\multicolumn{3}{|c|}{001} &
\multicolumn{3}{c|}{uimm[5:3]} &
\multicolumn{3}{c|}{\rsoneprime} &
\multicolumn{2}{c|}{uimm[7:6]} &
\multicolumn{3}{c|}{\rdprime} &
\multicolumn{2}{c|}{00} & C.FLD {\em \tiny (RV32/64)}\\
\cline{2-17}

&
\multicolumn{3}{|c|}{001} &
\multicolumn{3}{c|}{uimm[5:4$\vert$8]} &
\multicolumn{3}{c|}{\rsoneprime} &
\multicolumn{2}{c|}{uimm[7:6]} &
\multicolumn{3}{c|}{\rdprime} &
\multicolumn{2}{c|}{00} & C.LQ {\em \tiny (RV128)}\\
\whline{2-17}

&
\multicolumn{3}{|c|}{010} &
\multicolumn{3}{c|}{uimm[5:3]} &
\multicolumn{3}{c|}{\rsoneprime} &
\multicolumn{2}{c|}{uimm[2$\vert$6]} &
\multicolumn{3}{c|}{\rdprime} &
\multicolumn{2}{c|}{00} & C.LW \\
\whline{2-17}

&
\multicolumn{3}{|c|}{011} &
\multicolumn{3}{c|}{uimm[5:3]} &
\multicolumn{3}{c|}{\rsoneprime} &
\multicolumn{2}{c|}{uimm[2$\vert$6]} &
\multicolumn{3}{c|}{\rdprime} &
\multicolumn{2}{c|}{00} & C.FLW {\em \tiny (RV32)} \\
\cline{2-17}

&
\multicolumn{3}{|c|}{011} &
\multicolumn{3}{c|}{uimm[5:3]} &
\multicolumn{3}{c|}{\rsoneprime} &
\multicolumn{2}{c|}{uimm[7:6]} &
\multicolumn{3}{c|}{\rdprime} &
\multicolumn{2}{c|}{00} & C.LD {\em \tiny (RV64/128)}\\
\whline{2-17}

&
\multicolumn{3}{|c|}{100} &
\multicolumn{11}{c|}{---} &
\multicolumn{2}{c|}{00} & {\em 保留} \\
\whline{2-17}

&
\multicolumn{3}{|c|}{101} &
\multicolumn{3}{c|}{uimm[5:3]} &
\multicolumn{3}{c|}{\rsoneprime} &
\multicolumn{2}{c|}{uimm[7:6]} &
\multicolumn{3}{c|}{\rstwoprime} &
\multicolumn{2}{c|}{00} & C.FSD {\em \tiny (RV32/64)}\\
\cline{2-17}

&
\multicolumn{3}{|c|}{101} &
\multicolumn{3}{c|}{uimm[5:4$\vert$8]} &
\multicolumn{3}{c|}{\rsoneprime} &
\multicolumn{2}{c|}{uimm[7:6]} &
\multicolumn{3}{c|}{\rstwoprime} &
\multicolumn{2}{c|}{00} & C.SQ {\em \tiny (RV128)}\\
\whline{2-17}

&
\multicolumn{3}{|c|}{110} &
\multicolumn{3}{c|}{uimm[5:3]} &
\multicolumn{3}{c|}{\rsoneprime} &
\multicolumn{2}{c|}{uimm[2$\vert$6]} &
\multicolumn{3}{c|}{\rstwoprime} &
\multicolumn{2}{c|}{00} & C.SW \\
\whline{2-17}

&
\multicolumn{3}{|c|}{111} &
\multicolumn{3}{c|}{uimm[5:3]} &
\multicolumn{3}{c|}{\rsoneprime} &
\multicolumn{2}{c|}{uimm[2$\vert$6]} &
\multicolumn{3}{c|}{\rstwoprime} &
\multicolumn{2}{c|}{00} & C.FSW {\em \tiny (RV32)} \\
\cline{2-17}

&
\multicolumn{3}{|c|}{111} &
\multicolumn{3}{c|}{uimm[5:3]} &
\multicolumn{3}{c|}{\rsoneprime} &
\multicolumn{2}{c|}{uimm[7:6]} &
\multicolumn{3}{c|}{\rstwoprime} &
\multicolumn{2}{c|}{00} & C.SD {\em \tiny (RV64/128)}\\
\cline{2-17}

\end{tabular}
\end{center}
\end{small}
\caption{RVC的指令列表,第0象限。}
\label{rvc-instr-table0}
\end{table}

\begin{table}[h]
\begin{small}
\begin{center}
\begin{tabular}{p{0in}p{0.05in}p{0.05in}p{0.05in}p{0.05in}p{0.05in}p{0.05in}p{0.05in}p{0.05in}p{0.05in}p{0.05in}p{0.05in}p{0.05in}p{0.05in}p{0.05in}p{0.05in}p{0.05in}l}
& & & & & & & & & & \\
                      &
\instbit{15} &
\instbit{14} &
\instbit{13} &
\multicolumn{1}{c}{\instbit{12}} &
\instbit{11} &
\instbit{10} &
\instbit{9} &
\instbit{8} &
\instbit{7} &
\instbit{6} &
\multicolumn{1}{c}{\instbit{5}} &
\instbit{4} &
\instbit{3} &
\instbit{2} &
\instbit{1} &
\instbit{0} \\
\cline{2-17}

&
\multicolumn{3}{|c|}{000} &
\multicolumn{1}{c|}{nzimm[5]} &
\multicolumn{5}{c|}{0} &
\multicolumn{5}{c|}{nzimm[4:0]} &
\multicolumn{2}{c|}{01} & C.NOP {\em \tiny (HINT, nzimm$\neq$0) } \\
\cline{2-17}

&
\multicolumn{3}{|c|}{000} &
\multicolumn{1}{c|}{nzimm[5]} &
\multicolumn{5}{c|}{rs1/rd$\neq$0} &
\multicolumn{5}{c|}{nzimm[4:0]} &
\multicolumn{2}{c|}{01} & C.ADDI {\em \tiny (HINT, nzimm=0)} \\
\whline{2-17}

&
\multicolumn{3}{|c|}{001} &
\multicolumn{11}{c|}{imm[11$\vert$4$\vert$9:8$\vert$10$\vert$6$\vert$7$\vert$3:1$\vert$5]} &
\multicolumn{2}{c|}{01} & C.JAL {\em \tiny (RV32)} \\
\cline{2-17}

&
\multicolumn{3}{|c|}{001} &
\multicolumn{1}{c|}{imm[5]} &
\multicolumn{5}{c|}{rs1/rd$\neq$0} &
\multicolumn{5}{c|}{imm[4:0]} &
\multicolumn{2}{c|}{01} & C.ADDIW {\em \tiny (RV64/128; RES, rd=0)} \\
\whline{2-17}

&
\multicolumn{3}{|c|}{010} &
\multicolumn{1}{c|}{imm[5]} &
\multicolumn{5}{c|}{rd$\neq$0} &
\multicolumn{5}{c|}{imm[4:0]} &
\multicolumn{2}{c|}{01} & C.LI {\em \tiny (HINT, rd=0)} \\
\whline{2-17}

&
\multicolumn{3}{|c|}{011} &
\multicolumn{1}{c|}{nzimm[9]} &
\multicolumn{5}{c|}{2} &
\multicolumn{5}{c|}{nzimm[4$\vert$6$\vert$8:7$\vert$5]} &
\multicolumn{2}{c|}{01} & C.ADDI16SP {\em \tiny (RES, nzimm=0)} \\
\cline{2-17}

&
\multicolumn{3}{|c|}{011} &
\multicolumn{1}{c|}{nzimm[17]} &
\multicolumn{5}{c|}{rd$\neq$$\{0,2\}$} &
\multicolumn{5}{c|}{nzimm[16:12]} &
\multicolumn{2}{c|}{01} & C.LUI {\em \tiny (RES, nzimm=0; HINT, rd=0)}\\
\whline{2-17}

&
\multicolumn{3}{|c|}{100} &
\multicolumn{1}{c|}{nzuimm[5]} &
\multicolumn{2}{c|}{00} &
\multicolumn{3}{c|}{\rsoneprime/\rdprime} &
\multicolumn{5}{c|}{nzuimm[4:0]} &
\multicolumn{2}{c|}{01} & C.SRLI {\em \tiny (RV32 Custom, nzuimm[5]=1)} \\
\cline{2-17}

&
\multicolumn{3}{|c|}{100} &
\multicolumn{1}{c|}{0} &
\multicolumn{2}{c|}{00} &
\multicolumn{3}{c|}{\rsoneprime/\rdprime} &
\multicolumn{5}{c|}{0} &
\multicolumn{2}{c|}{01} & C.SRLI64 {\em \tiny (RV128; RV32/64 HINT)} \\
\cline{2-17}

&
\multicolumn{3}{|c|}{100} &
\multicolumn{1}{c|}{nzuimm[5]} &
\multicolumn{2}{c|}{01} &
\multicolumn{3}{c|}{\rsoneprime/\rdprime} &
\multicolumn{5}{c|}{nzuimm[4:0]} &
\multicolumn{2}{c|}{01} & C.SRAI {\em \tiny (RV32 Custom, nzuimm[5]=1)} \\
\cline{2-17}

&
\multicolumn{3}{|c|}{100} &
\multicolumn{1}{c|}{0} &
\multicolumn{2}{c|}{01} &
\multicolumn{3}{c|}{\rsoneprime/\rdprime} &
\multicolumn{5}{c|}{0} &
\multicolumn{2}{c|}{01} & C.SRAI64 {\em \tiny (RV128; RV32/64 HINT)} \\
\cline{2-17}

&
\multicolumn{3}{|c|}{100} &
\multicolumn{1}{c|}{imm[5]} &
\multicolumn{2}{c|}{10} &
\multicolumn{3}{c|}{\rsoneprime/\rdprime} &
\multicolumn{5}{c|}{imm[4:0]} &
\multicolumn{2}{c|}{01} & C.ANDI \\
\cline{2-17}

&
\multicolumn{3}{|c|}{100} &
\multicolumn{1}{c|}{0} &
\multicolumn{2}{c|}{11} &
\multicolumn{3}{c|}{\rsoneprime/\rdprime} &
\multicolumn{2}{c|}{00} &
\multicolumn{3}{c|}{\rstwoprime} &
\multicolumn{2}{c|}{01} & C.SUB \\
\cline{2-17}

&
\multicolumn{3}{|c|}{100} &
\multicolumn{1}{c|}{0} &
\multicolumn{2}{c|}{11} &
\multicolumn{3}{c|}{\rsoneprime/\rdprime} &
\multicolumn{2}{c|}{01} &
\multicolumn{3}{c|}{\rstwoprime} &
\multicolumn{2}{c|}{01} & C.XOR \\
\cline{2-17}

&
\multicolumn{3}{|c|}{100} &
\multicolumn{1}{c|}{0} &
\multicolumn{2}{c|}{11} &
\multicolumn{3}{c|}{\rsoneprime/\rdprime} &
\multicolumn{2}{c|}{10} &
\multicolumn{3}{c|}{\rstwoprime} &
\multicolumn{2}{c|}{01} & C.OR \\
\cline{2-17}

&
\multicolumn{3}{|c|}{100} &
\multicolumn{1}{c|}{0} &
\multicolumn{2}{c|}{11} &
\multicolumn{3}{c|}{\rsoneprime/\rdprime} &
\multicolumn{2}{c|}{11} &
\multicolumn{3}{c|}{\rstwoprime} &
\multicolumn{2}{c|}{01} & C.AND \\
\cline{2-17}

&
\multicolumn{3}{|c|}{100} &
\multicolumn{1}{c|}{1} &
\multicolumn{2}{c|}{11} &
\multicolumn{3}{c|}{\rsoneprime/\rdprime} &
\multicolumn{2}{c|}{00} &
\multicolumn{3}{c|}{\rstwoprime} &
\multicolumn{2}{c|}{01} & C.SUBW {\em \tiny (RV64/128; RV32 RES)} \\
\cline{2-17}

&
\multicolumn{3}{|c|}{100} &
\multicolumn{1}{c|}{1} &
\multicolumn{2}{c|}{11} &
\multicolumn{3}{c|}{\rsoneprime/\rdprime} &
\multicolumn{2}{c|}{01} &
\multicolumn{3}{c|}{\rstwoprime} &
\multicolumn{2}{c|}{01} & C.ADDW {\em \tiny (RV64/128; RV32 RES)} \\
\cline{2-17}

&
\multicolumn{3}{|c|}{100} &
\multicolumn{1}{c|}{1} &
\multicolumn{2}{c|}{11} &
\multicolumn{3}{c|}{---} &
\multicolumn{2}{c|}{10} &
\multicolumn{3}{c|}{---} &
\multicolumn{2}{c|}{01} & {\em 保留} \\
\cline{2-17}

&
\multicolumn{3}{|c|}{100} &
\multicolumn{1}{c|}{1} &
\multicolumn{2}{c|}{11} &
\multicolumn{3}{c|}{---} &
\multicolumn{2}{c|}{11} &
\multicolumn{3}{c|}{---} &
\multicolumn{2}{c|}{01} & {\em 保留} \\
\whline{2-17}

&
\multicolumn{3}{|c|}{101} &
\multicolumn{11}{c|}{imm[11$\vert$4$\vert$9:8$\vert$10$\vert$6$\vert$7$\vert$3:1$\vert$5]} &
\multicolumn{2}{c|}{01} & C.J \\
\whline{2-17}

&
\multicolumn{3}{|c|}{110} &
\multicolumn{3}{c|}{imm[8$\vert$4:3]} &
\multicolumn{3}{c|}{\rsoneprime} &
\multicolumn{5}{c|}{imm[7:6$\vert$2:1$\vert$5]} &
\multicolumn{2}{c|}{01} & C.BEQZ \\
\whline{2-17}

&
\multicolumn{3}{|c|}{111} &
\multicolumn{3}{c|}{imm[8$\vert$4:3]} &
\multicolumn{3}{c|}{\rsoneprime} &
\multicolumn{5}{c|}{imm[7:6$\vert$2:1$\vert$5]} &
\multicolumn{2}{c|}{01} & C.BNEZ \\
\cline{2-17}
  

\end{tabular}
\end{center}
\end{small}
\caption{RVC的指令列表,第1象限。}
\label{rvc-instr-table1}
\end{table}

\begin{table}[h]
\begin{small}
\begin{center}
\begin{tabular}{p{0in}p{0.05in}p{0.05in}p{0.05in}p{0.05in}p{0.05in}p{0.05in}p{0.05in}p{0.05in}p{0.05in}p{0.05in}p{0.05in}p{0.05in}p{0.05in}p{0.05in}p{0.05in}p{0.05in}l}
& & & & & & & & & & \\
                      &
\instbit{15} &
\instbit{14} &
\instbit{13} &
\multicolumn{1}{c}{\instbit{12}} &
\instbit{11} &
\instbit{10} &
\instbit{9} &
\instbit{8} &
\instbit{7} &
\instbit{6} &
\multicolumn{1}{c}{\instbit{5}} &
\instbit{4} &
\instbit{3} &
\instbit{2} &
\instbit{1} &
\instbit{0} \\
\cline{2-17}

&
\multicolumn{3}{|c|}{000} &
\multicolumn{1}{c|}{nzuimm[5]} &
\multicolumn{5}{c|}{rs1/rd$\neq$0} &
\multicolumn{5}{c|}{nzuimm[4:0]} &
\multicolumn{2}{c|}{10} & C.SLLI {\em \tiny (HINT, rd=0; RV32 Custom, nzuimm[5]=1)} \\
\cline{2-17}

&
\multicolumn{3}{|c|}{000} &
\multicolumn{1}{c|}{0} &
\multicolumn{5}{c|}{rs1/rd$\neq$0} &
\multicolumn{5}{c|}{0} &
\multicolumn{2}{c|}{10} & C.SLLI64 {\em \tiny (RV128; RV32/64 HINT; HINT, rd=0)} \\
\whline{2-17}

&
\multicolumn{3}{|c|}{001} &
\multicolumn{1}{c|}{uimm[5]} &
\multicolumn{5}{c|}{rd} &
\multicolumn{5}{c|}{uimm[4:3$\vert$8:6]} &
\multicolumn{2}{c|}{10} & C.FLDSP {\em \tiny (RV32/64)} \\
\cline{2-17}

&
\multicolumn{3}{|c|}{001} &
\multicolumn{1}{c|}{uimm[5]} &
\multicolumn{5}{c|}{rd$\neq$0} &
\multicolumn{5}{c|}{uimm[4$\vert$9:6]} &
\multicolumn{2}{c|}{10} & C.LQSP {\em \tiny (RV128; RES, rd=0)} \\
\whline{2-17}

&
\multicolumn{3}{|c|}{010} &
\multicolumn{1}{c|}{uimm[5]} &
\multicolumn{5}{c|}{rd$\neq$0} &
\multicolumn{5}{c|}{uimm[4:2$\vert$7:6]} &
\multicolumn{2}{c|}{10} & C.LWSP {\em \tiny (RES, rd=0)} \\
\whline{2-17}

&
\multicolumn{3}{|c|}{011} &
\multicolumn{1}{c|}{uimm[5]} &
\multicolumn{5}{c|}{rd} &
\multicolumn{5}{c|}{uimm[4:2$\vert$7:6]} &
\multicolumn{2}{c|}{10} & C.FLWSP {\em \tiny (RV32)} \\
\cline{2-17}

&
\multicolumn{3}{|c|}{011} &
\multicolumn{1}{c|}{uimm[5]} &
\multicolumn{5}{c|}{rd$\neq$0} &
\multicolumn{5}{c|}{uimm[4:3$\vert$8:6]} &
\multicolumn{2}{c|}{10} & C.LDSP {\em \tiny (RV64/128; RES, rd=0)} \\
\whline{2-17}

&
\multicolumn{3}{|c|}{100} &
\multicolumn{1}{c|}{0} &
\multicolumn{5}{c|}{rs1$\neq$0} &
\multicolumn{5}{c|}{0} &
\multicolumn{2}{c|}{10} & C.JR {\em \tiny (RES, rs1=0)}\\
\cline{2-17}

&
\multicolumn{3}{|c|}{100} &
\multicolumn{1}{c|}{0} &
\multicolumn{5}{c|}{rd$\neq$0} &
\multicolumn{5}{c|}{rs2$\neq$0} &
\multicolumn{2}{c|}{10} & C.MV {\em \tiny (HINT, rd=0)}\\
\cline{2-17}

&
\multicolumn{3}{|c|}{100} &
\multicolumn{1}{c|}{1} &
\multicolumn{5}{c|}{0} &
\multicolumn{5}{c|}{0} &
\multicolumn{2}{c|}{10} & C.EBREAK \\
\cline{2-17}

&
\multicolumn{3}{|c|}{100} &
\multicolumn{1}{c|}{1} &
\multicolumn{5}{c|}{rs1$\neq$0} &
\multicolumn{5}{c|}{0} &
\multicolumn{2}{c|}{10} & C.JALR \\
\cline{2-17}

&
\multicolumn{3}{|c|}{100} &
\multicolumn{1}{c|}{1} &
\multicolumn{5}{c|}{rs1/rd$\neq$0} &
\multicolumn{5}{c|}{rs2$\neq$0} &
\multicolumn{2}{c|}{10} & C.ADD {\em \tiny (HINT, rd=0)} \\
\whline{2-17}

&
\multicolumn{3}{|c|}{101} &
\multicolumn{6}{c|}{uimm[5:3$\vert$8:6]} &
\multicolumn{5}{c|}{rs2} &
\multicolumn{2}{c|}{10} & C.FSDSP {\em \tiny (RV32/64)}\\
\cline{2-17}

&
\multicolumn{3}{|c|}{101} &
\multicolumn{6}{c|}{uimm[5:4$\vert$9:6]} &
\multicolumn{5}{c|}{rs2} &
\multicolumn{2}{c|}{10} & C.SQSP {\em \tiny (RV128)}\\
\whline{2-17}

&
\multicolumn{3}{|c|}{110} &
\multicolumn{6}{c|}{uimm[5:2$\vert$7:6]} &
\multicolumn{5}{c|}{rs2} &
\multicolumn{2}{c|}{10} & C.SWSP \\
\whline{2-17}

&
\multicolumn{3}{|c|}{111} &
\multicolumn{6}{c|}{uimm[5:2$\vert$7:6]} &
\multicolumn{5}{c|}{rs2} &
\multicolumn{2}{c|}{10} & C.FSWSP {\em \tiny (RV32)} \\
\cline{2-17}

&
\multicolumn{3}{|c|}{111} &
\multicolumn{6}{c|}{uimm[5:3$\vert$8:6]} &
\multicolumn{5}{c|}{rs2} &
\multicolumn{2}{c|}{10} & C.SDSP {\em \tiny (RV64/128)}\\
\cline{2-17}

\end{tabular}
\end{center}
\end{small}
\caption{RVC的指令列表,第2象限。}
\label{rvc-instr-table2}
\end{table}


\chapter{用于位操作的“B”标准扩展(0.0版本)}
% \chapter{``B'' Standard Extension for Bit Manipulation, Version 0.0}
\label{sec:bits}

这章是为未来的提供位操作指令的标准扩展占位的,包括了插入、提取和测试位域的指令,和用于旋转、漏斗型移位、位与字节置换的指令。
% This chapter is a placeholder for a future standard extension to
% provide bit manipulation instructions, including instructions to
% insert, extract, and test bit fields, and for rotations, funnel
% shifts, and bit and byte permutations.

\begin{commentary}
尽管位操作指令在一些应用领域(特别是当处理外部打包的数据结构时)非常有效,我们仍然把它们排除在基础ISA之外,
因为它们不是在所有领域中都有用的,并且为了支持所有所需的操作数,会增加额外的复杂度或指令格式。
% Although bit manipulation instructions are very effective in some
% application domains, particularly when dealing with externally packed
% data structures, we excluded them from the base ISAs as they are not
% useful in all domains and can add additional complexity or instruction
% formats to supply all needed operands.

我们预计B扩展将成为基础30位指令空间中编码的一个棕色地带,等待着重新的发掘。
% We anticipate the B extension will be a brownfield encoding within the
% base 30-bit instruction space.
\end{commentary}


\chapter{用于动态翻译语言的“J”标准扩展(0.0版本)}
% \chapter{``J'' Standard Extension for Dynamically Translated Languages, Version 0.0}
\label{sec:j}

这章是为未来的支持动态翻译语言的标准扩展占位的。
% This chapter is a placeholder for a future standard extension to
% support dynamically translated languages.

\begin{commentary}
  许多流行的语言通常通过动态翻译实现,包括Java和Javascript。这些语言可以从对动态检查和垃圾回收的额外的ISA支持中获益。
  % Many popular languages are usually implemented via dynamic
  % translation, including Java and Javascript. These languages can
  % benefit from additional ISA support for dynamic checks and garbage
  % collection.
\end{commentary}


\chapter{用于打包SIMD指令的“P”标准扩展(0.2版本)}
% \chapter{``P'' Standard Extension for Packed-SIMD Instructions,
%   Version 0.2}
\label{sec:packedsimd}

\begin{commentary}
  第五次RISC-V研讨会的讨论表示,希望放弃这个用于浮点寄存器的打包SIMD的提案,转而支持在V扩展上对大型浮点SIMD操作的标准化。
  然而,在小型RISC-V实现的整数寄存器中,还存在对打包SIMD定点操作的使用兴趣。一个任务组正在为了定义新的P扩展而工作着。
  % Discussions at the 5th RISC-V workshop indicated a desire to drop
  % this packed-SIMD proposal for floating-point registers in favor of
  % standardizing on the V extension for large floating-point SIMD
  % operations.  However, there was interest in packed-SIMD fixed-point
  % operations for use in the integer registers of small RISC-V
  % implementations. A task group is working to define the new P
  % extension.
\end{commentary}


\chapter{用于向量操作的“V”标准扩展(0.7版本)}
\label{sec:vector}
% \chapter{``V'' Standard Extension for Vector Operations, Version 0.7}

当前的工作组草案被寄放在了{\tt https://github.com/riscv/riscv-v-spec}。
% The current working group draft is hosted at {\tt
%   https://github.com/riscv/riscv-v-spec}.

\begin{commentary}
  基础向量扩展意图在32位指令编码空间中为数据并行执行提供通用的支持,而后续的向量扩展会针对特定领域支持更丰富的功能性。
% The base vector extension is intended to provide general support for
% data-parallel execution within the 32-bit instruction encoding space,
% with later vector extensions supporting richer functionality for
% certain domains.
\end{commentary}
  



\chapter{用于非对齐原子的“Zam”标准扩展(0.1版本)}
% \chapter{``Zam'' Standard Extension for Misaligned Atomics, v0.1}
\label{sec:zam}

本章定义了“Zam”扩展,它通过对未对齐的原子内存操作(AMO)进行标准化支持,扩展了“A”扩展。在实现了“Zam”的平台,
未对齐的AMO只需要针对相同地址和相同尺寸的其它访问(包括非原子的加载和存储)进行原子化地执行。
更确切地说,实现了“Zam”的执行环境服从下列公理:
% This chapter defines the ``Zam'' extension, which extends the ``A'' extension by standardizing support for misaligned atomic memory operations (AMOs).
% On platforms implementing ``Zam'', misaligned AMOs need only execute atomically with respect
% to other accesses (including non-atomic loads and stores) to the same address and of the same size.
% More precisely, execution environments implementing ``Zam'' are subject to the following axiom:

\newcommand{\misalignedatomicityaxiom}{
  如果$r$和$w$是来自硬件线程$h$的成对的未对齐的加载和存储指令,它们具有相同的地址和相同的尺寸,那么以全局内存次序,在$r$和$w$生成的内存操作之间,不会再有某个存储指令$s$生成的存储操作——其中$s$满足:来自除了$h$之外的硬件线程,并且与$r$和$w$具有相同地址和相同尺寸。并且,以全局内存次序,在$r$或者$w$生成的两个内存操作之间,不会再有某个加载指令$l$生成的加载操作——其中$l$满足:来自除了$h$之外的硬件线程,并且与$r$和$w$具有相同地址和相同尺寸。
  % If $r$ and $w$ are paired misaligned load and store instructions from a hart $h$ with the same address and of the same size, then there can be no store instruction $s$ from a hart other than $h$ with the same address and of the same size as $r$ and $w$ such that a store operation generated by $s$ lies in between memory operations generated by $r$ and $w$ in the global memory order.  Furthermore, there can be no load instruction $l$ from a hart other than $h$ with the same address and of the same size as $r$ and $w$ such that a load operation generated by $l$ lies between two memory operations generated by $r$ or by $w$ in the global memory order.
  }

\vspace{-0.2in}
\paragraph{未对齐原子的原子性公理  
% Atomicity Axiom for misaligned atomics
}
\label{rvwmo:ax:misaligned}
\misalignedatomicityaxiom

原子性的这个受限制的形式试图在应用的需求(其需要支持未对齐原子)与实现的能力(确实提供必要程度的原子性)之间进行平衡。
% This restricted form of atomicity is intended to balance the needs of applications which require support for misaligned atomics and the ability of the implementation to actually provide the necessary degree of atomicity.

在“Zam”下,对齐的指令继续按照它们在RVWMO下的行为正常工作。
% Aligned instructions under ``Zam'' continue to behave as they normally do under RVWMO.

\begin{commentary}
  “Zam”的意图是,它可以用两种方式实现:
  % The intention of ``Zam'' is that it can be implemented in one of two ways:
  \begin{enumerate}
    \item 对于相关的地址和尺寸的原子未对齐访问,在原生支持该访问的硬件上(例如,对于在单一缓存行中进行的未对齐访问):通过简单地遵循与对齐的AMO所应用的相同的规则来实现。
    % On hardware that natively supports atomic misaligned accesses to the address and size in question (e.g., for misaligned accesses within a single cache line): by simply following the same rules that would be applied for aligned AMOs.
    \item 对于相关的地址和尺寸的未对齐访问,在缺少对其原生支持的硬件上:通过用该地址和尺寸陷入所有指令(包括加载指令),并在一个mutex(它是一个给定了内存地址和访问尺寸的函数)中执行它们(通过任意数目的内存操作)。可以通过把它们分割为独立的加载和存储指令来模拟AMO,但是所有保留的程序次序规则(例如,传入和传出的语法依赖项)的行为必须像AMO仍然是一个单一的内存操作时一样。
    % On hardware that does not natively support misaligned accesses to the address and size in question: by trapping on all instructions (including loads) with that address and size and executing them (via any number of memory operations) inside a mutex that is a function of the given memory address and access size.  AMOs may be emulated by splitting them into separate load and store operations, but all preserved program order rules (e.g., incoming and outgoing syntactic dependencies) must behave as if the AMO is still a single memory operation.
  \end{enumerate}
\end{commentary}

\chapter{用于在整数寄存器中使用浮点的“Zfinx”、“Zdinx”、“Zhinx”、“Zhinxmin”标准扩展(1.0版本)}
% \chapter{``Zfinx'', ``Zdinx'', ``Zhinx'', ``Zhinxmin'': Standard Extensions for Floating-Point in Integer Registers, Version 1.0}
\label{sec:zfinx}

本章定义了“Zfinx”扩展(发音为“z-f-in-x”),为单精度浮点指令提供了与标准浮点F扩展类似的指令,但是操作在{\tt x}寄存器而不是{\tt f}寄存器上。
本章还定义了“Zdinx”、“Zhinx”和“Zhinxmin”扩展,为其他浮点精度提供了类似的指令。
% This chapter defines the ``Zfinx'' extension (pronounced ``z-f-in-x'')
% that provides instructions similar to those in the standard
% floating-point F extension for single-precision floating-point
% instructions but which operate on the {\tt x} registers instead of the
% {\tt f} registers.  This chapter also defines the ``Zdinx'',
% ``Zhinx'', and ``Zhinxmin'' extensions that provide similar
% instructions for other floating-point precisions.

\begin{commentary}
F扩展使用单独的{\tt f}寄存器进行浮点计算,以减少寄存器压力并为宽超标量简化寄存器文件端口的提供。
然而,额外的\wunits{128}{B}架构状态增加了最小实现成本。
通过消除{\tt f}寄存器,Zfinx扩展极大地减少了支持带有浮点指令集的简单RISC-V实现的成本。Zfinx也降低了上下文切换成本。
% The F extension uses separate {\tt f} registers for floating-point
% computation, to reduce register pressure and simplify the provision of
% register-file ports for wide superscalars.
% However, the additional \wunits{128}{B} of architectural state increases the
% minimal implementation cost.
% By eliminating the {\tt f} registers, the Zfinx extension substantially
% reduces the cost of simple RISC-V implementations with floating-point
% instruction-set support.
% Zfinx also reduces context-switch cost.

一般来说,假定F扩展存在的软件与假定Zfinx扩展存在的软件是不兼容的,反之亦然。
% In general, software that assumes the presence of the F extension
% is incompatible with software that assumes the presence of the Zfinx
% extension, and vice versa.
\end{commentary}

{\em 除了}转移指令FLW、FSW、FMV.W.X、FMV.X.W、C.FLW[SP]和C.FSW[SP]外,Zfinx扩展添加了F扩展所增加的所有指令。
% The Zfinx extension adds all of the instructions that the F extension
% adds, {\em except} for the transfer instructions FLW, FSW, FMV.W.X,
% FMV.X.W, C.FLW[SP], and C.FSW[SP].

\begin{commentary}
Zfinx软件使用整数的加载与存储在内存之间传递浮点值。在寄存器之间的传输既可以使用整数算术指令,也可以使用浮点符号注入指令。
% Zfinx software uses integer loads and stores to transfer floating-point values
% from and to memory.
% Transfers between registers use either integer arithmetic or floating-point
% sign-injection instructions.
\end{commentary}

这些F扩展指令的Zfinx变体具有相同的语义,只是当这样一条指令将访问{\tt f}寄存器时,改为访问具有相同编号的{\tt x}寄存器。
% The Zfinx variants of these F-extension instructions have the same semantics,
% except that whenever such an instruction would have accessed an {\tt f}
% register, it instead accesses the {\tt x} register with the same number.

\section{处理更窄的值}
% \section{Processing of Narrower Values}

宽度\mbox{{\em w} $<$ XLEN bits}位的浮点操作数占据{\tt x}寄存器的位\mbox{{\em w}-1:0}。对于{\em w}位的浮点操作会忽略位\mbox{XLEN-1:{\em w}}。
% Floating-point operands of width \mbox{{\em w} $<$ XLEN bits} occupy bits
% \mbox{{\em w}-1:0} of an {\tt x} register.
% Floating-point operations on {\em w}-bit operands ignore operand bits
% \mbox{XLEN-1:{\em w}}.

产生\mbox{{\em w} $<$ XLEN-bit}位结果的浮点操作用位\mbox{{\em w}-1}(符号位)填充位\mbox{XLEN-1:{\em w}}。
% Floating-point operations that produce \mbox{{\em w} $<$ XLEN-bit} results
% fill bits \mbox{XLEN-1:{\em w}} with copies of bit \mbox{{\em w}-1} (the
% sign bit).

\begin{commentary}
{\tt f}寄存器中采用的NaN装箱策略的设计足够支持重新编码的浮点格式。
不过,由于浮点操作数和整数操作数占据相同的寄存器,重新编码对于Zfinx不太实用。因此,减少了对NaN装箱的需求。
% The NaN-boxing scheme employed in the {\tt f} registers was designed to
% efficiently support recoded floating-point formats.
% Recoding is less practical for Zfinx, though, since the same registers
% hold both floating-point and integer operands.
% Hence, the need for NaN boxing is diminished.

当符号扩展的32位浮点数保存在RV64的{\tt x}寄存器时,匹配现有的RV64调用约定,这需要所有的32位类型在传递到x寄存器、
或从{\tt x}寄存器返回时,都被符号扩展。为了使架构更加规范,我们在RV32和RV64中都把这个式样扩展到了16位浮点数。
% Sign-extending 32-bit floating-point numbers when held in RV64 {\tt x}
% registers matches the existing RV64 calling conventions, which require all
% 32-bit types to be sign-extended when passed or returned in {\tt x} registers.
% To keep the architecture more regular, we extend this pattern to 16-bit
% floating-point numbers in both RV32 and RV64.
\end{commentary}

\section{Zdinx}
% \section{Zdinx}

Zdinx扩展提供了类似于双精度浮点的指令。Zdinx扩展需要Zfinx扩展。
% The Zdinx extension provides analogous double-precision floating-point
% instructions.
% The Zdinx extension requires the Zfinx extension.

{\em 除了}转移指令FLD、FSD、FMV.D.X、C.FLD[SP]和C.FSD[SP],Zdinx扩展添加了D扩展增加的所有指令。
% The Zdinx extension adds all of the instructions that the D extension
% adds, {\em except} for the transfer instructions FLD, FSD, FMV.D.X,
% FMV.X.D, C.FLD[SP], and C.FSD[SP].

这些D扩展指令的Zdinx变体具有相同的语义,只是当这样一条指令将访问{\tt f}寄存器时,改为访问具有相同编号的{\tt x}寄存器。
% The Zdinx variants of these D-extension instructions have the same semantics,
% except that whenever such an instruction would have accessed an {\tt f}
% register, it instead accesses the {\tt x} register with the same number.

\section{处理更宽的值}
% \section{Processing of Wider Values}

RV32Zdinx中的双精度操作数保存在对齐的{\tt x}寄存器对中,也就是说,寄存器数量必须是偶数。为双宽度浮点操作数使用未对齐的(奇数数量的)寄存器是{\em 保留的}。
% Double-precision operands in RV32Zdinx
% are held in aligned {\tt x}-register pairs, i.e.,
% register numbers must be even.
% Use of misaligned (odd-numbered) registers for double-width floating-point
% operands is {\em reserved}.

不论字节序如何,较低编号的寄存器都保存低序位,而较高编号的寄存器保存高序位:例如,
RV32Zdinx中,双精度操作数的位31:0可能保存在寄存器{\tt x14}中,同时该操作数的位63:32保存在{\tt x15}中。
% Regardless of endianness, the lower-numbered register holds the low-order
% bits, and the higher-numbered register holds the high-order bits: e.g., bits
% 31:0 of a double-precision operand in RV32Zdinx might be held in register
% {\tt x14}, with bits 63:32 of that operand held in {\tt x15}.

当一个双宽度浮点结果写入{\tt x0}时,整个写入是无效的:例如,对于RV32Zdinx,向{\tt x0}写入一个双精度结果并不会导致{\tt x1}被写入。
% When a double-width floating-point result is written to {\tt x0}, the entire
% write takes no effect: e.g., for RV32Zdinx, writing a double-precision result
% to {\tt x0} does not cause {\tt x1} to be written.

当{\tt x0}被用作一个双宽度浮点操作数时,完整的操作数是零——即,{\tt x1}是未访问的。
% When {\tt x0} is used as a double-width floating-point operand, the entire
% operand is zero---i.e., {\tt x1} is not accessed.

\begin{commentary}
没有提供按对加载和按对存储的指令,所以在RV32Zdinx中从内存或向内存转移双精度操作数需要两次加载或存储。
然而,寄存器的移动只需要一个单独的FSGNJ.D指令。
% Load-pair and store-pair instructions are not provided, so transferring
% double-precision operands in RV32Zdinx from or to memory requires
% two loads or stores.
% Register moves need only a single FSGNJ.D instruction, however.
\end{commentary}

\section{Zhinx}

Zhinx扩展提供了类似于半精度浮点的指令。Zhinx扩展需要Zfinx扩展。
% The Zhinx extension provides analogous half-precision floating-point
% instructions.
% The Zhinx extension requires the Zfinx extension.

{\em 除了}转移指令FLH、FSH、FMV.H.X和FMV.X.H外,Zhinx扩展添加了所有Zfh扩展增加的指令。
% The Zhinx extension adds all of the instructions that the Zfh extension
% adds, {\em except} for the transfer instructions FLH, FSH, FMV.H.X,
% and FMV.X.H.

这些Zfh扩展指令的Zhinx变体具有相同的语义,只是在这样一条指令将访问{\tt f}寄存器时,改为访问具有相同编号的{\tt x}寄存器。
% The Zhinx variants of these Zfh-extension instructions have the same semantics,
% except that whenever such an instruction would have accessed an {\tt f}
% register, it instead accesses the {\tt x} register with the same number.

\section{Zhinxmin}

Zhinxmin扩展提供了对操作在{\tt x}寄存器上的16位半精度浮点指令的最小限度的支持。Zhinxmin扩展需要Zfinx扩展。
% The Zhinxmin extension provides minimal support for 16-bit half-precision
% floating-point instructions that operate on the {\tt x} registers.
% The Zhinxmin extension requires the Zfinx extension.

Zhinxmin扩展包含来自Zhinx扩展的下列指令:FCVT.S.H和FCVT.H.S。如果Zdinx扩展存在,也包含了FCVT.D.H和FCVT.H.D指令。
% The Zhinxmin extension includes the following instructions from the Zhinx
% extension: FCVT.S.H and FCVT.H.S.
% If the Zdinx extension is present, the FCVT.D.H and FCVT.H.D instructions are
% also included.

\begin{commentary}
在未来,可能定义与RV32Zdinx类似的RV64Zqinx四精度扩展。RV32Zqinx扩展也可能被定义,但是需要四寄存器组(译者注:指每组包含共同工作的四个寄存器,原文quad-register groups)。
% In the future, an RV64Zqinx quad-precision extension could be defined analogously
% to RV32Zdinx.
% An RV32Zqinx extension could also be defined but would require
% quad-register groups.
\end{commentary}

\section{特权架构的影响}
% \section{Privileged Architecture Implications}

在第二卷定义的标准特权架构中,如果实现了Zfinx扩展,{\tt mstatus}域FS被硬布线为零,
并且FS不再影响浮点指令或{\tt fcsr}访问的陷入行为。
% In the standard privileged architecture defined in Volume II, the
% {\tt mstatus} field FS is hardwired to 0 if the Zfinx extension is
% implemented, and FS no longer affects the trapping behavior of
% floating-point instructions or {\tt fcsr} accesses.

当实现了Zfinx扩展时,{\tt misa}位F、D和Q被硬布线为零。
% The {\tt misa} bits F, D, and Q are hardwired to 0 when the Zfinx
% extension is implemented.

\begin{commentary}
    未来的发现机制可能被用于探查Zfinx、Zhinx和Zdinx扩展的存在性。
% A future discoverability mechanism might be used to probe the existence
% of the Zfinx, Zhinx, and Zdinx extensions.
\end{commentary}

\chapter{用于全存储排序的“Ztso”标准扩展(0.1版本)}
% \chapter{``Ztso'' Standard Extension for Total Store Ordering, v0.1}
\label{sec:ztso}

本章定义了用于RISC-V全存储排序(RVTSO)内存一致性模型的“Ztso扩展”。RVTSO被定义为与RVWMO(定义在~\ref{sec:rvwmo}章中)有差异的部分。
% This chapter defines the ``Ztso'' extension for the RISC-V Total Store Ordering (RVTSO) memory consistency model.
% RVTSO is defined as a delta from RVWMO, which is defined in Chapter~\ref{sec:rvwmo}.

\begin{commentary}
  Ztso扩展旨在帮助原本为带有TSO模型的架构(例如x86或某些版本的SPARC)写的代码进行移植,而这两种代码都默认使用TSO。它也支持那些固有地提供RVTSO行为并希望将此事实暴露给软件的实现。
  % The Ztso extension is meant to facilitate the porting of code originally written for architectures with TSO memory models, such as x86 or some versions of SPARC.
  % It also supports implementations which inherently provide RVTSO behavior and want to expose that fact to software.
\end{commentary}

RVTSO对RVWMO做了如下调整:
% RVTSO makes the following adjustments to RVWMO:

\begin{itemize}
  \item 所有的加载操作的行为如同它们有acquire-RCpc注释一样  % All load operations behave as if they have an acquire-RCpc annotation
  \item 所有的存储操作的行为如同它们有release-RCpc注释一样  % All store operations behave as if they have a release-RCpc annotation.
  \item 所有的AMO行为如同它们同时有acquire-RCsc和release-RCsc注释一样  % All AMOs behave as if they have both acquire-RCsc and release-RCsc annotations.
\end{itemize}

\begin{commentary}
  这些规则让除了\ref{ppo:fence}-\ref{ppo:rcsc}之外的所有PPO规则变得多余。它们也让任何没有同时设置了PW和SR的非I/O屏障变得多余。
  最终,它们也暗示了,不会有内存操作将被重新排序而在任何方向上超越一个AMO。
  % These rules render all PPO rules except \ref{ppo:fence}--\ref{ppo:rcsc} redundant.
  % They also make redundant any non-I/O fences that do not have both PW and SR set.
  % Finally, they also imply that no memory operation will be reordered past an AMO in either direction.
  
  在RVTSO的上下文中,就像是RVWMO的情况那样,通过PPO规则\ref{ppo:acquire}-\ref{ppo:rcsc}简明而完整地定义了存储次序。在这两个内存模型中,都是\nameref{rvwmo:ax:load}使得硬件线程能够将一个来自它的存储缓冲区的值发送到一个后续的(以程序次序)加载——那就是说,存储可以在它们对其它硬件线程可见之前,被本地发送。
  % In the context of RVTSO, as is the case for RVWMO, the storage ordering annotations are concisely and completely defined by PPO rules \ref{ppo:acquire}--\ref{ppo:rcsc}. In both of these memory models, it is the \nameref{rvwmo:ax:load} that allows a hart to forward a value from its store buffer to a subsequent (in program order) load---that is to say that stores can be forwarded locally before they are visible to other harts.
\end{commentary}

此外,如果实现了Ztso扩展,则V扩展和Zve系列扩展中的向量内存指令在指令级遵循RVTSO。Ztso扩展不强化指令内部元素访问的顺序。
% Additionally, if the Ztso extension is implemented, then vector memory
% instructions in the V extension and Zve family of extensions follow RVTSO at
% the instruction level.
% The Ztso extension does not strengthen the ordering of intra-instruction
% element accesses.

尽管事实上Ztso没有向ISA添加新的指令,假定RVTSO写出的代码也将不能正确运行在不支持Ztso的实现上。编译的二进制只能运行在Ztso下,这应当通过二进制中的一个标志被表示出来,使得没有实现Ztso的平台可以简单地拒绝运行它们。
% In spite of the fact that Ztso adds no new instructions to the ISA, code written assuming RVTSO will not run correctly on implementations not supporting Ztso.
% Binaries compiled to run only under Ztso should indicate as such via a flag in the binary, so that platforms which do not implement Ztso can simply refuse to run them.

\chapter{RV32/64G指令集列表}
% \chapter{RV32/64G Instruction Set Listings}

RISC-V项目的一个目标是,将它作为一个稳定的软件开发目标来使用。
为了这个目的,我们定义了一个基础ISA(RV32I或RV64I)加上选择的标准扩展(IMAFD、Zicsr、Zifencei)的组合,
作为一个“通用目的”的ISA,而且,我们使用缩写G来表示指令集扩展的IMAFDZicsr Zifencei组合。
本章展示了为RV32G和RV64G列出的操作码映射和指令集。
% One goal of the RISC-V project is that it be used as a stable software
% development target.  For this purpose, we define a combination of a
% base ISA (RV32I or RV64I) plus selected standard extensions (IMAFD, Zicsr, Zifencei) as
% a ``general-purpose'' ISA, and we use the abbreviation G for the IMAFDZicsr\_Zifencei
% combination of instruction-set extensions.    This chapter presents
% opcode maps and instruction-set listings for RV32G and RV64G.

\vspace{0.1in}
\definecolor{gray}{RGB}{180,180,180}
\begin{table*}[htbp]
\begin{center}
{\footnotesize
\setlength{\tabcolsep}{4pt}
\begin{tabular}{|r|c|c|c|c|c|c|c|c|}
  \hline
  inst[4:2] & 000    & 001      & 010            & 011      & 100    & 101            & 110                  & \cellcolor{gray}111 \\ \cline{1-1}
  inst[6:5] &        &          &                &          &        &                &                      & \cellcolor{gray}($>32b$)  \\ \hline
         00 & LOAD   & LOAD-FP  & {\em custom-0} & MISC-MEM & OP-IMM & AUIPC          & OP-IMM-32            & \cellcolor{gray} $48b$\\ \hline
         01 & STORE  & STORE-FP & {\em custom-1} & AMO      & OP     & LUI            & OP-32                & \cellcolor{gray} $64b$ \\ \hline
         10 & MADD   & MSUB     & NMSUB          & NMADD    & OP-FP  & OP-V           & {\em custom-2/rv128} & \cellcolor{gray} $48b$\\ \hline
         11 & BRANCH & JALR     & {\em 保留} & JAL      & SYSTEM & {\em 保留} & {\em custom-3/rv128} & \cellcolor{gray} $\geq80b$\\ \hline

 \end{tabular}
}
\end{center}
\vspace{-0.15in}
\caption{RISCV-V基础操作码映射,instinst[1:0]=11
% RISC-V base opcode map, inst[1:0]=11
}
\label{opcodemap}
\end{table*}


表~\ref{opcodemap}显示了RVG的主要操作码的映射。设置了3或更低位的主要操作码被保留用于超过32位的指令长度。
标记为{\em 保留的}的操作码应当被避免用于自定义的指令集扩展,因为它们可能会用在未来的标准扩展中。
标记为{\em 自定义-0}和{\em 自定义-1}的主操作码应当被避免用于未来的标准扩展,并且建议用于具有基础32位指令格式的自定义指令集扩展。
标记为{\em 自定义-2/rv128}和{\em 自定义-3/rv128}的操作码被保留,以供RV128未来使用,
但如果不是RV128,那么它们将避免被用于标准扩展,并因此也可以被用于RV32和RV64中的自定义指令集扩展。
% Table~\ref{opcodemap} shows a map of the major opcodes for RVG.  Major
% opcodes with 3 or more lower bits set are reserved for instruction
% lengths greater than 32 bits.  Opcodes marked as {\em reserved} should
% be avoided for custom instruction-set extensions as they might be used
% by future standard extensions.  Major opcodes marked as {\em custom-0}
% and {\em custom-1} will be avoided by future standard extensions and
% are recommended for use by custom instruction-set extensions within
% the base 32-bit instruction format.  The opcodes marked {\em
%   custom-2/rv128} and {\em custom-3/rv128} are reserved for future use
% by RV128, but will otherwise be avoided for standard extensions and so
% can also be used for custom instruction-set extensions in RV32 and
% RV64.

我们相信RV32G和RV64G为大量通用目的计算提供了简单但完整的指令集。
可以添加可选的在~\ref{compressed}章中描述的压缩指令集(形成RV32GC和RV64GC)来改善性能、代码尺寸和能量效率,尽管这会带来额外的硬件复杂度。
% We believe RV32G and RV64G provide simple but complete instruction
% sets for a broad range of general-purpose computing.  The optional
% compressed instruction set described in Chapter~\ref{compressed} can
% be added (forming RV32GC and RV64GC) to improve performance, code
% size, and energy efficiency, though with some additional hardware
% complexity.

随着我们的脚步超越了IMAFDC,走进进一步的指令集扩展,添加的指令更加倾向于特定领域,
而只对严格的某类应用(例如,多媒体应用或者安全应用)提供收益。不像大多数商业化的ISA,
RISC-V ISA的设计将基础ISA和广泛可用的标准扩展与这些更特定化的额外部分清晰地分离开。
第~\ref{extensions}章对于向RISC-V ISA添加扩展的方法进行了更加广泛的讨论。
% As we move beyond IMAFDC into further instruction-set extensions, the
% added instructions tend to be more domain-specific and only provide
% benefits to a restricted class of applications, e.g., for multimedia
% or security.  Unlike most commercial ISAs, the RISC-V ISA design
% clearly separates the base ISA and broadly applicable standard
% extensions from these more specialized additions.
% Chapter~\ref{extensions} has a more extensive discussion of ways to
% add extensions to the RISC-V ISA.


\newpage

\begin{table}[p]
\begin{small}
\begin{center}
\begin{tabular}{p{0in}p{0.4in}p{0.05in}p{0.05in}p{0.05in}p{0.05in}p{0.4in}p{0.6in}p{0.4in}p{0.6in}p{0.7in}l}
& & & & & & & & & & \\
                      &
\multicolumn{1}{l}{\instbit{31}} &
\multicolumn{1}{r}{\instbit{27}} &
\instbit{26} &
\instbit{25} &
\multicolumn{1}{l}{\instbit{24}} &
\multicolumn{1}{r}{\instbit{20}} &
\instbitrange{19}{15} &
\instbitrange{14}{12} &
\instbitrange{11}{7} &
\instbitrange{6}{0} \\
\cline{2-11}


&
\multicolumn{4}{|c|}{funct7} &
\multicolumn{2}{c|}{rs2} &
\multicolumn{1}{c|}{rs1} &
\multicolumn{1}{c|}{funct3} &
\multicolumn{1}{c|}{rd} &
\multicolumn{1}{c|}{opcode} & R-type \\
\cline{2-11}


&
\multicolumn{6}{|c|}{imm[11:0]} &
\multicolumn{1}{c|}{rs1} &
\multicolumn{1}{c|}{funct3} &
\multicolumn{1}{c|}{rd} &
\multicolumn{1}{c|}{opcode} & I-type \\
\cline{2-11}


&
\multicolumn{4}{|c|}{imm[11:5]} &
\multicolumn{2}{c|}{rs2} &
\multicolumn{1}{c|}{rs1} &
\multicolumn{1}{c|}{funct3} &
\multicolumn{1}{c|}{imm[4:0]} &
\multicolumn{1}{c|}{opcode} & S-type \\
\cline{2-11}


&
\multicolumn{4}{|c|}{imm[12$\vert$10:5]} &
\multicolumn{2}{c|}{rs2} &
\multicolumn{1}{c|}{rs1} &
\multicolumn{1}{c|}{funct3} &
\multicolumn{1}{c|}{imm[4:1$\vert$11]} &
\multicolumn{1}{c|}{opcode} & B-type \\
\cline{2-11}


&
\multicolumn{8}{|c|}{imm[31:12]} &
\multicolumn{1}{c|}{rd} &
\multicolumn{1}{c|}{opcode} & U-type \\
\cline{2-11}


&
\multicolumn{8}{|c|}{imm[20$\vert$10:1$\vert$11$\vert$19:12]} &
\multicolumn{1}{c|}{rd} &
\multicolumn{1}{c|}{opcode} & J-type \\
\cline{2-11}


&
\multicolumn{10}{c}{} & \\
&
\multicolumn{10}{c}{\bf RV32I基础指令集} & \\
\cline{2-11}


&
\multicolumn{8}{|c|}{imm[31:12]} &
\multicolumn{1}{c|}{rd} &
\multicolumn{1}{c|}{0110111} & LUI \\
\cline{2-11}


&
\multicolumn{8}{|c|}{imm[31:12]} &
\multicolumn{1}{c|}{rd} &
\multicolumn{1}{c|}{0010111} & AUIPC \\
\cline{2-11}


&
\multicolumn{8}{|c|}{imm[20$\vert$10:1$\vert$11$\vert$19:12]} &
\multicolumn{1}{c|}{rd} &
\multicolumn{1}{c|}{1101111} & JAL \\
\cline{2-11}


&
\multicolumn{6}{|c|}{imm[11:0]} &
\multicolumn{1}{c|}{rs1} &
\multicolumn{1}{c|}{000} &
\multicolumn{1}{c|}{rd} &
\multicolumn{1}{c|}{1100111} & JALR \\
\cline{2-11}


&
\multicolumn{4}{|c|}{imm[12$\vert$10:5]} &
\multicolumn{2}{c|}{rs2} &
\multicolumn{1}{c|}{rs1} &
\multicolumn{1}{c|}{000} &
\multicolumn{1}{c|}{imm[4:1$\vert$11]} &
\multicolumn{1}{c|}{1100011} & BEQ \\
\cline{2-11}


&
\multicolumn{4}{|c|}{imm[12$\vert$10:5]} &
\multicolumn{2}{c|}{rs2} &
\multicolumn{1}{c|}{rs1} &
\multicolumn{1}{c|}{001} &
\multicolumn{1}{c|}{imm[4:1$\vert$11]} &
\multicolumn{1}{c|}{1100011} & BNE \\
\cline{2-11}


&
\multicolumn{4}{|c|}{imm[12$\vert$10:5]} &
\multicolumn{2}{c|}{rs2} &
\multicolumn{1}{c|}{rs1} &
\multicolumn{1}{c|}{100} &
\multicolumn{1}{c|}{imm[4:1$\vert$11]} &
\multicolumn{1}{c|}{1100011} & BLT \\
\cline{2-11}


&
\multicolumn{4}{|c|}{imm[12$\vert$10:5]} &
\multicolumn{2}{c|}{rs2} &
\multicolumn{1}{c|}{rs1} &
\multicolumn{1}{c|}{101} &
\multicolumn{1}{c|}{imm[4:1$\vert$11]} &
\multicolumn{1}{c|}{1100011} & BGE \\
\cline{2-11}


&
\multicolumn{4}{|c|}{imm[12$\vert$10:5]} &
\multicolumn{2}{c|}{rs2} &
\multicolumn{1}{c|}{rs1} &
\multicolumn{1}{c|}{110} &
\multicolumn{1}{c|}{imm[4:1$\vert$11]} &
\multicolumn{1}{c|}{1100011} & BLTU \\
\cline{2-11}


&
\multicolumn{4}{|c|}{imm[12$\vert$10:5]} &
\multicolumn{2}{c|}{rs2} &
\multicolumn{1}{c|}{rs1} &
\multicolumn{1}{c|}{111} &
\multicolumn{1}{c|}{imm[4:1$\vert$11]} &
\multicolumn{1}{c|}{1100011} & BGEU \\
\cline{2-11}


&
\multicolumn{6}{|c|}{imm[11:0]} &
\multicolumn{1}{c|}{rs1} &
\multicolumn{1}{c|}{000} &
\multicolumn{1}{c|}{rd} &
\multicolumn{1}{c|}{0000011} & LB \\
\cline{2-11}


&
\multicolumn{6}{|c|}{imm[11:0]} &
\multicolumn{1}{c|}{rs1} &
\multicolumn{1}{c|}{001} &
\multicolumn{1}{c|}{rd} &
\multicolumn{1}{c|}{0000011} & LH \\
\cline{2-11}


&
\multicolumn{6}{|c|}{imm[11:0]} &
\multicolumn{1}{c|}{rs1} &
\multicolumn{1}{c|}{010} &
\multicolumn{1}{c|}{rd} &
\multicolumn{1}{c|}{0000011} & LW \\
\cline{2-11}


&
\multicolumn{6}{|c|}{imm[11:0]} &
\multicolumn{1}{c|}{rs1} &
\multicolumn{1}{c|}{100} &
\multicolumn{1}{c|}{rd} &
\multicolumn{1}{c|}{0000011} & LBU \\
\cline{2-11}


&
\multicolumn{6}{|c|}{imm[11:0]} &
\multicolumn{1}{c|}{rs1} &
\multicolumn{1}{c|}{101} &
\multicolumn{1}{c|}{rd} &
\multicolumn{1}{c|}{0000011} & LHU \\
\cline{2-11}


&
\multicolumn{4}{|c|}{imm[11:5]} &
\multicolumn{2}{c|}{rs2} &
\multicolumn{1}{c|}{rs1} &
\multicolumn{1}{c|}{000} &
\multicolumn{1}{c|}{imm[4:0]} &
\multicolumn{1}{c|}{0100011} & SB \\
\cline{2-11}


&
\multicolumn{4}{|c|}{imm[11:5]} &
\multicolumn{2}{c|}{rs2} &
\multicolumn{1}{c|}{rs1} &
\multicolumn{1}{c|}{001} &
\multicolumn{1}{c|}{imm[4:0]} &
\multicolumn{1}{c|}{0100011} & SH \\
\cline{2-11}


&
\multicolumn{4}{|c|}{imm[11:5]} &
\multicolumn{2}{c|}{rs2} &
\multicolumn{1}{c|}{rs1} &
\multicolumn{1}{c|}{010} &
\multicolumn{1}{c|}{imm[4:0]} &
\multicolumn{1}{c|}{0100011} & SW \\
\cline{2-11}


&
\multicolumn{6}{|c|}{imm[11:0]} &
\multicolumn{1}{c|}{rs1} &
\multicolumn{1}{c|}{000} &
\multicolumn{1}{c|}{rd} &
\multicolumn{1}{c|}{0010011} & ADDI \\
\cline{2-11}


&
\multicolumn{6}{|c|}{imm[11:0]} &
\multicolumn{1}{c|}{rs1} &
\multicolumn{1}{c|}{010} &
\multicolumn{1}{c|}{rd} &
\multicolumn{1}{c|}{0010011} & SLTI \\
\cline{2-11}


&
\multicolumn{6}{|c|}{imm[11:0]} &
\multicolumn{1}{c|}{rs1} &
\multicolumn{1}{c|}{011} &
\multicolumn{1}{c|}{rd} &
\multicolumn{1}{c|}{0010011} & SLTIU \\
\cline{2-11}


&
\multicolumn{6}{|c|}{imm[11:0]} &
\multicolumn{1}{c|}{rs1} &
\multicolumn{1}{c|}{100} &
\multicolumn{1}{c|}{rd} &
\multicolumn{1}{c|}{0010011} & XORI \\
\cline{2-11}


&
\multicolumn{6}{|c|}{imm[11:0]} &
\multicolumn{1}{c|}{rs1} &
\multicolumn{1}{c|}{110} &
\multicolumn{1}{c|}{rd} &
\multicolumn{1}{c|}{0010011} & ORI \\
\cline{2-11}


&
\multicolumn{6}{|c|}{imm[11:0]} &
\multicolumn{1}{c|}{rs1} &
\multicolumn{1}{c|}{111} &
\multicolumn{1}{c|}{rd} &
\multicolumn{1}{c|}{0010011} & ANDI \\
\cline{2-11}


&
\multicolumn{4}{|c|}{0000000} &
\multicolumn{2}{c|}{shamt} &
\multicolumn{1}{c|}{rs1} &
\multicolumn{1}{c|}{001} &
\multicolumn{1}{c|}{rd} &
\multicolumn{1}{c|}{0010011} & SLLI \\
\cline{2-11}


&
\multicolumn{4}{|c|}{0000000} &
\multicolumn{2}{c|}{shamt} &
\multicolumn{1}{c|}{rs1} &
\multicolumn{1}{c|}{101} &
\multicolumn{1}{c|}{rd} &
\multicolumn{1}{c|}{0010011} & SRLI \\
\cline{2-11}


&
\multicolumn{4}{|c|}{0100000} &
\multicolumn{2}{c|}{shamt} &
\multicolumn{1}{c|}{rs1} &
\multicolumn{1}{c|}{101} &
\multicolumn{1}{c|}{rd} &
\multicolumn{1}{c|}{0010011} & SRAI \\
\cline{2-11}


&
\multicolumn{4}{|c|}{0000000} &
\multicolumn{2}{c|}{rs2} &
\multicolumn{1}{c|}{rs1} &
\multicolumn{1}{c|}{000} &
\multicolumn{1}{c|}{rd} &
\multicolumn{1}{c|}{0110011} & ADD \\
\cline{2-11}


&
\multicolumn{4}{|c|}{0100000} &
\multicolumn{2}{c|}{rs2} &
\multicolumn{1}{c|}{rs1} &
\multicolumn{1}{c|}{000} &
\multicolumn{1}{c|}{rd} &
\multicolumn{1}{c|}{0110011} & SUB \\
\cline{2-11}


&
\multicolumn{4}{|c|}{0000000} &
\multicolumn{2}{c|}{rs2} &
\multicolumn{1}{c|}{rs1} &
\multicolumn{1}{c|}{001} &
\multicolumn{1}{c|}{rd} &
\multicolumn{1}{c|}{0110011} & SLL \\
\cline{2-11}


&
\multicolumn{4}{|c|}{0000000} &
\multicolumn{2}{c|}{rs2} &
\multicolumn{1}{c|}{rs1} &
\multicolumn{1}{c|}{010} &
\multicolumn{1}{c|}{rd} &
\multicolumn{1}{c|}{0110011} & SLT \\
\cline{2-11}


&
\multicolumn{4}{|c|}{0000000} &
\multicolumn{2}{c|}{rs2} &
\multicolumn{1}{c|}{rs1} &
\multicolumn{1}{c|}{011} &
\multicolumn{1}{c|}{rd} &
\multicolumn{1}{c|}{0110011} & SLTU \\
\cline{2-11}


&
\multicolumn{4}{|c|}{0000000} &
\multicolumn{2}{c|}{rs2} &
\multicolumn{1}{c|}{rs1} &
\multicolumn{1}{c|}{100} &
\multicolumn{1}{c|}{rd} &
\multicolumn{1}{c|}{0110011} & XOR \\
\cline{2-11}


&
\multicolumn{4}{|c|}{0000000} &
\multicolumn{2}{c|}{rs2} &
\multicolumn{1}{c|}{rs1} &
\multicolumn{1}{c|}{101} &
\multicolumn{1}{c|}{rd} &
\multicolumn{1}{c|}{0110011} & SRL \\
\cline{2-11}


&
\multicolumn{4}{|c|}{0100000} &
\multicolumn{2}{c|}{rs2} &
\multicolumn{1}{c|}{rs1} &
\multicolumn{1}{c|}{101} &
\multicolumn{1}{c|}{rd} &
\multicolumn{1}{c|}{0110011} & SRA \\
\cline{2-11}


&
\multicolumn{4}{|c|}{0000000} &
\multicolumn{2}{c|}{rs2} &
\multicolumn{1}{c|}{rs1} &
\multicolumn{1}{c|}{110} &
\multicolumn{1}{c|}{rd} &
\multicolumn{1}{c|}{0110011} & OR \\
\cline{2-11}


&
\multicolumn{4}{|c|}{0000000} &
\multicolumn{2}{c|}{rs2} &
\multicolumn{1}{c|}{rs1} &
\multicolumn{1}{c|}{111} &
\multicolumn{1}{c|}{rd} &
\multicolumn{1}{c|}{0110011} & AND \\
\cline{2-11}


&
\multicolumn{2}{|c|}{fm} &
\multicolumn{3}{c|}{pred} &
\multicolumn{1}{c|}{succ} &
\multicolumn{1}{c|}{rs1} &
\multicolumn{1}{c|}{000} &
\multicolumn{1}{c|}{rd} &
\multicolumn{1}{c|}{0001111} & FENCE \\
\cline{2-11}


&
\multicolumn{2}{|c|}{1000} &
\multicolumn{3}{c|}{0011} &
\multicolumn{1}{c|}{0011} &
\multicolumn{1}{c|}{00000} &
\multicolumn{1}{c|}{000} &
\multicolumn{1}{c|}{00000} &
\multicolumn{1}{c|}{0001111} & FENCE.TSO \\
\cline{2-11}


&
\multicolumn{2}{|c|}{0000} &
\multicolumn{3}{c|}{0001} &
\multicolumn{1}{c|}{0000} &
\multicolumn{1}{c|}{00000} &
\multicolumn{1}{c|}{000} &
\multicolumn{1}{c|}{00000} &
\multicolumn{1}{c|}{0001111} & PAUSE \\
\cline{2-11}


&
\multicolumn{6}{|c|}{000000000000} &
\multicolumn{1}{c|}{00000} &
\multicolumn{1}{c|}{000} &
\multicolumn{1}{c|}{00000} &
\multicolumn{1}{c|}{1110011} & ECALL \\
\cline{2-11}


&
\multicolumn{6}{|c|}{000000000001} &
\multicolumn{1}{c|}{00000} &
\multicolumn{1}{c|}{000} &
\multicolumn{1}{c|}{00000} &
\multicolumn{1}{c|}{1110011} & EBREAK \\
\cline{2-11}


\end{tabular}
\end{center}
\end{small}

\end{table}


\newpage

\begin{table}[p]
\begin{small}
\begin{center}
\begin{tabular}{p{0in}p{0.4in}p{0.05in}p{0.05in}p{0.05in}p{0.05in}p{0.4in}p{0.6in}p{0.4in}p{0.6in}p{0.7in}l}
& & & & & & & & & & \\
                      &
\multicolumn{1}{l}{\instbit{31}} &
\multicolumn{1}{r}{\instbit{27}} &
\instbit{26} &
\instbit{25} &
\multicolumn{1}{l}{\instbit{24}} &
\multicolumn{1}{r}{\instbit{20}} &
\instbitrange{19}{15} &
\instbitrange{14}{12} &
\instbitrange{11}{7} &
\instbitrange{6}{0} \\
\cline{2-11}


&
\multicolumn{4}{|c|}{funct7} &
\multicolumn{2}{c|}{rs2} &
\multicolumn{1}{c|}{rs1} &
\multicolumn{1}{c|}{funct3} &
\multicolumn{1}{c|}{rd} &
\multicolumn{1}{c|}{opcode} & R-type \\
\cline{2-11}


&
\multicolumn{6}{|c|}{imm[11:0]} &
\multicolumn{1}{c|}{rs1} &
\multicolumn{1}{c|}{funct3} &
\multicolumn{1}{c|}{rd} &
\multicolumn{1}{c|}{opcode} & I-type \\
\cline{2-11}


&
\multicolumn{4}{|c|}{imm[11:5]} &
\multicolumn{2}{c|}{rs2} &
\multicolumn{1}{c|}{rs1} &
\multicolumn{1}{c|}{funct3} &
\multicolumn{1}{c|}{imm[4:0]} &
\multicolumn{1}{c|}{opcode} & S-type \\
\cline{2-11}


&
\multicolumn{10}{c}{} & \\
&
\multicolumn{10}{c}{\bf RV64I基础指令集(RV32I之外的部分)} & \\
\cline{2-11}


&
\multicolumn{6}{|c|}{imm[11:0]} &
\multicolumn{1}{c|}{rs1} &
\multicolumn{1}{c|}{110} &
\multicolumn{1}{c|}{rd} &
\multicolumn{1}{c|}{0000011} & LWU \\
\cline{2-11}


&
\multicolumn{6}{|c|}{imm[11:0]} &
\multicolumn{1}{c|}{rs1} &
\multicolumn{1}{c|}{011} &
\multicolumn{1}{c|}{rd} &
\multicolumn{1}{c|}{0000011} & LD \\
\cline{2-11}


&
\multicolumn{4}{|c|}{imm[11:5]} &
\multicolumn{2}{c|}{rs2} &
\multicolumn{1}{c|}{rs1} &
\multicolumn{1}{c|}{011} &
\multicolumn{1}{c|}{imm[4:0]} &
\multicolumn{1}{c|}{0100011} & SD \\
\cline{2-11}


&
\multicolumn{3}{|c|}{000000} &
\multicolumn{3}{c|}{shamt} &
\multicolumn{1}{c|}{rs1} &
\multicolumn{1}{c|}{001} &
\multicolumn{1}{c|}{rd} &
\multicolumn{1}{c|}{0010011} & SLLI \\
\cline{2-11}


&
\multicolumn{3}{|c|}{000000} &
\multicolumn{3}{c|}{shamt} &
\multicolumn{1}{c|}{rs1} &
\multicolumn{1}{c|}{101} &
\multicolumn{1}{c|}{rd} &
\multicolumn{1}{c|}{0010011} & SRLI \\
\cline{2-11}


&
\multicolumn{3}{|c|}{010000} &
\multicolumn{3}{c|}{shamt} &
\multicolumn{1}{c|}{rs1} &
\multicolumn{1}{c|}{101} &
\multicolumn{1}{c|}{rd} &
\multicolumn{1}{c|}{0010011} & SRAI \\
\cline{2-11}


&
\multicolumn{6}{|c|}{imm[11:0]} &
\multicolumn{1}{c|}{rs1} &
\multicolumn{1}{c|}{000} &
\multicolumn{1}{c|}{rd} &
\multicolumn{1}{c|}{0011011} & ADDIW \\
\cline{2-11}


&
\multicolumn{4}{|c|}{0000000} &
\multicolumn{2}{c|}{shamt} &
\multicolumn{1}{c|}{rs1} &
\multicolumn{1}{c|}{001} &
\multicolumn{1}{c|}{rd} &
\multicolumn{1}{c|}{0011011} & SLLIW \\
\cline{2-11}


&
\multicolumn{4}{|c|}{0000000} &
\multicolumn{2}{c|}{shamt} &
\multicolumn{1}{c|}{rs1} &
\multicolumn{1}{c|}{101} &
\multicolumn{1}{c|}{rd} &
\multicolumn{1}{c|}{0011011} & SRLIW \\
\cline{2-11}


&
\multicolumn{4}{|c|}{0100000} &
\multicolumn{2}{c|}{shamt} &
\multicolumn{1}{c|}{rs1} &
\multicolumn{1}{c|}{101} &
\multicolumn{1}{c|}{rd} &
\multicolumn{1}{c|}{0011011} & SRAIW \\
\cline{2-11}


&
\multicolumn{4}{|c|}{0000000} &
\multicolumn{2}{c|}{rs2} &
\multicolumn{1}{c|}{rs1} &
\multicolumn{1}{c|}{000} &
\multicolumn{1}{c|}{rd} &
\multicolumn{1}{c|}{0111011} & ADDW \\
\cline{2-11}


&
\multicolumn{4}{|c|}{0100000} &
\multicolumn{2}{c|}{rs2} &
\multicolumn{1}{c|}{rs1} &
\multicolumn{1}{c|}{000} &
\multicolumn{1}{c|}{rd} &
\multicolumn{1}{c|}{0111011} & SUBW \\
\cline{2-11}


&
\multicolumn{4}{|c|}{0000000} &
\multicolumn{2}{c|}{rs2} &
\multicolumn{1}{c|}{rs1} &
\multicolumn{1}{c|}{001} &
\multicolumn{1}{c|}{rd} &
\multicolumn{1}{c|}{0111011} & SLLW \\
\cline{2-11}


&
\multicolumn{4}{|c|}{0000000} &
\multicolumn{2}{c|}{rs2} &
\multicolumn{1}{c|}{rs1} &
\multicolumn{1}{c|}{101} &
\multicolumn{1}{c|}{rd} &
\multicolumn{1}{c|}{0111011} & SRLW \\
\cline{2-11}


&
\multicolumn{4}{|c|}{0100000} &
\multicolumn{2}{c|}{rs2} &
\multicolumn{1}{c|}{rs1} &
\multicolumn{1}{c|}{101} &
\multicolumn{1}{c|}{rd} &
\multicolumn{1}{c|}{0111011} & SRAW \\
\cline{2-11}


&
\multicolumn{10}{c}{} & \\
&
\multicolumn{10}{c}{\bf RV32/RV64 \emph{Zifencei} 标准扩展} & \\
\cline{2-11}


&
\multicolumn{6}{|c|}{imm[11:0]} &
\multicolumn{1}{c|}{rs1} &
\multicolumn{1}{c|}{001} &
\multicolumn{1}{c|}{rd} &
\multicolumn{1}{c|}{0001111} & FENCE.I \\
\cline{2-11}


&
\multicolumn{10}{c}{} & \\
&
\multicolumn{10}{c}{\bf RV32/RV64 \emph{Zicsr} 标准扩展} & \\
\cline{2-11}


&
\multicolumn{6}{|c|}{csr} &
\multicolumn{1}{c|}{rs1} &
\multicolumn{1}{c|}{001} &
\multicolumn{1}{c|}{rd} &
\multicolumn{1}{c|}{1110011} & CSRRW \\
\cline{2-11}


&
\multicolumn{6}{|c|}{csr} &
\multicolumn{1}{c|}{rs1} &
\multicolumn{1}{c|}{010} &
\multicolumn{1}{c|}{rd} &
\multicolumn{1}{c|}{1110011} & CSRRS \\
\cline{2-11}


&
\multicolumn{6}{|c|}{csr} &
\multicolumn{1}{c|}{rs1} &
\multicolumn{1}{c|}{011} &
\multicolumn{1}{c|}{rd} &
\multicolumn{1}{c|}{1110011} & CSRRC \\
\cline{2-11}


&
\multicolumn{6}{|c|}{csr} &
\multicolumn{1}{c|}{uimm} &
\multicolumn{1}{c|}{101} &
\multicolumn{1}{c|}{rd} &
\multicolumn{1}{c|}{1110011} & CSRRWI \\
\cline{2-11}


&
\multicolumn{6}{|c|}{csr} &
\multicolumn{1}{c|}{uimm} &
\multicolumn{1}{c|}{110} &
\multicolumn{1}{c|}{rd} &
\multicolumn{1}{c|}{1110011} & CSRRSI \\
\cline{2-11}


&
\multicolumn{6}{|c|}{csr} &
\multicolumn{1}{c|}{uimm} &
\multicolumn{1}{c|}{111} &
\multicolumn{1}{c|}{rd} &
\multicolumn{1}{c|}{1110011} & CSRRCI \\
\cline{2-11}


&
\multicolumn{10}{c}{} & \\
&
\multicolumn{10}{c}{\bf RV32M标准扩展 } & \\
\cline{2-11}


&
\multicolumn{4}{|c|}{0000001} &
\multicolumn{2}{c|}{rs2} &
\multicolumn{1}{c|}{rs1} &
\multicolumn{1}{c|}{000} &
\multicolumn{1}{c|}{rd} &
\multicolumn{1}{c|}{0110011} & MUL \\
\cline{2-11}


&
\multicolumn{4}{|c|}{0000001} &
\multicolumn{2}{c|}{rs2} &
\multicolumn{1}{c|}{rs1} &
\multicolumn{1}{c|}{001} &
\multicolumn{1}{c|}{rd} &
\multicolumn{1}{c|}{0110011} & MULH \\
\cline{2-11}


&
\multicolumn{4}{|c|}{0000001} &
\multicolumn{2}{c|}{rs2} &
\multicolumn{1}{c|}{rs1} &
\multicolumn{1}{c|}{010} &
\multicolumn{1}{c|}{rd} &
\multicolumn{1}{c|}{0110011} & MULHSU \\
\cline{2-11}


&
\multicolumn{4}{|c|}{0000001} &
\multicolumn{2}{c|}{rs2} &
\multicolumn{1}{c|}{rs1} &
\multicolumn{1}{c|}{011} &
\multicolumn{1}{c|}{rd} &
\multicolumn{1}{c|}{0110011} & MULHU \\
\cline{2-11}


&
\multicolumn{4}{|c|}{0000001} &
\multicolumn{2}{c|}{rs2} &
\multicolumn{1}{c|}{rs1} &
\multicolumn{1}{c|}{100} &
\multicolumn{1}{c|}{rd} &
\multicolumn{1}{c|}{0110011} & DIV \\
\cline{2-11}


&
\multicolumn{4}{|c|}{0000001} &
\multicolumn{2}{c|}{rs2} &
\multicolumn{1}{c|}{rs1} &
\multicolumn{1}{c|}{101} &
\multicolumn{1}{c|}{rd} &
\multicolumn{1}{c|}{0110011} & DIVU \\
\cline{2-11}


&
\multicolumn{4}{|c|}{0000001} &
\multicolumn{2}{c|}{rs2} &
\multicolumn{1}{c|}{rs1} &
\multicolumn{1}{c|}{110} &
\multicolumn{1}{c|}{rd} &
\multicolumn{1}{c|}{0110011} & REM \\
\cline{2-11}


&
\multicolumn{4}{|c|}{0000001} &
\multicolumn{2}{c|}{rs2} &
\multicolumn{1}{c|}{rs1} &
\multicolumn{1}{c|}{111} &
\multicolumn{1}{c|}{rd} &
\multicolumn{1}{c|}{0110011} & REMU \\
\cline{2-11}


&
\multicolumn{10}{c}{} & \\
&
\multicolumn{10}{c}{\bf RV64M标准扩展(RV32M之外的部分)} & \\
\cline{2-11}


&
\multicolumn{4}{|c|}{0000001} &
\multicolumn{2}{c|}{rs2} &
\multicolumn{1}{c|}{rs1} &
\multicolumn{1}{c|}{000} &
\multicolumn{1}{c|}{rd} &
\multicolumn{1}{c|}{0111011} & MULW \\
\cline{2-11}


&
\multicolumn{4}{|c|}{0000001} &
\multicolumn{2}{c|}{rs2} &
\multicolumn{1}{c|}{rs1} &
\multicolumn{1}{c|}{100} &
\multicolumn{1}{c|}{rd} &
\multicolumn{1}{c|}{0111011} & DIVW \\
\cline{2-11}


&
\multicolumn{4}{|c|}{0000001} &
\multicolumn{2}{c|}{rs2} &
\multicolumn{1}{c|}{rs1} &
\multicolumn{1}{c|}{101} &
\multicolumn{1}{c|}{rd} &
\multicolumn{1}{c|}{0111011} & DIVUW \\
\cline{2-11}


&
\multicolumn{4}{|c|}{0000001} &
\multicolumn{2}{c|}{rs2} &
\multicolumn{1}{c|}{rs1} &
\multicolumn{1}{c|}{110} &
\multicolumn{1}{c|}{rd} &
\multicolumn{1}{c|}{0111011} & REMW \\
\cline{2-11}


&
\multicolumn{4}{|c|}{0000001} &
\multicolumn{2}{c|}{rs2} &
\multicolumn{1}{c|}{rs1} &
\multicolumn{1}{c|}{111} &
\multicolumn{1}{c|}{rd} &
\multicolumn{1}{c|}{0111011} & REMUW \\
\cline{2-11}


\end{tabular}
\end{center}
\end{small}

\end{table}


\newpage

\begin{table}[p]
\begin{small}
\begin{center}
\begin{tabular}{p{0in}p{0.4in}p{0.05in}p{0.05in}p{0.05in}p{0.05in}p{0.4in}p{0.6in}p{0.4in}p{0.6in}p{0.7in}l}
& & & & & & & & & & \\
                      &
\multicolumn{1}{l}{\instbit{31}} &
\multicolumn{1}{r}{\instbit{27}} &
\instbit{26} &
\instbit{25} &
\multicolumn{1}{l}{\instbit{24}} &
\multicolumn{1}{r}{\instbit{20}} &
\instbitrange{19}{15} &
\instbitrange{14}{12} &
\instbitrange{11}{7} &
\instbitrange{6}{0} \\
\cline{2-11}


&
\multicolumn{4}{|c|}{funct7} &
\multicolumn{2}{c|}{rs2} &
\multicolumn{1}{c|}{rs1} &
\multicolumn{1}{c|}{funct3} &
\multicolumn{1}{c|}{rd} &
\multicolumn{1}{c|}{opcode} & R-type \\
\cline{2-11}


&
\multicolumn{10}{c}{} & \\
&
\multicolumn{10}{c}{\bf RV32A标准扩展} & \\
\cline{2-11}


&
\multicolumn{2}{|c|}{00010} &
\multicolumn{1}{c|}{aq} &
\multicolumn{1}{c|}{rl} &
\multicolumn{2}{c|}{00000} &
\multicolumn{1}{c|}{rs1} &
\multicolumn{1}{c|}{010} &
\multicolumn{1}{c|}{rd} &
\multicolumn{1}{c|}{0101111} & LR.W \\
\cline{2-11}


&
\multicolumn{2}{|c|}{00011} &
\multicolumn{1}{c|}{aq} &
\multicolumn{1}{c|}{rl} &
\multicolumn{2}{c|}{rs2} &
\multicolumn{1}{c|}{rs1} &
\multicolumn{1}{c|}{010} &
\multicolumn{1}{c|}{rd} &
\multicolumn{1}{c|}{0101111} & SC.W \\
\cline{2-11}


&
\multicolumn{2}{|c|}{00001} &
\multicolumn{1}{c|}{aq} &
\multicolumn{1}{c|}{rl} &
\multicolumn{2}{c|}{rs2} &
\multicolumn{1}{c|}{rs1} &
\multicolumn{1}{c|}{010} &
\multicolumn{1}{c|}{rd} &
\multicolumn{1}{c|}{0101111} & AMOSWAP.W \\
\cline{2-11}


&
\multicolumn{2}{|c|}{00000} &
\multicolumn{1}{c|}{aq} &
\multicolumn{1}{c|}{rl} &
\multicolumn{2}{c|}{rs2} &
\multicolumn{1}{c|}{rs1} &
\multicolumn{1}{c|}{010} &
\multicolumn{1}{c|}{rd} &
\multicolumn{1}{c|}{0101111} & AMOADD.W \\
\cline{2-11}


&
\multicolumn{2}{|c|}{00100} &
\multicolumn{1}{c|}{aq} &
\multicolumn{1}{c|}{rl} &
\multicolumn{2}{c|}{rs2} &
\multicolumn{1}{c|}{rs1} &
\multicolumn{1}{c|}{010} &
\multicolumn{1}{c|}{rd} &
\multicolumn{1}{c|}{0101111} & AMOXOR.W \\
\cline{2-11}


&
\multicolumn{2}{|c|}{01100} &
\multicolumn{1}{c|}{aq} &
\multicolumn{1}{c|}{rl} &
\multicolumn{2}{c|}{rs2} &
\multicolumn{1}{c|}{rs1} &
\multicolumn{1}{c|}{010} &
\multicolumn{1}{c|}{rd} &
\multicolumn{1}{c|}{0101111} & AMOAND.W \\
\cline{2-11}


&
\multicolumn{2}{|c|}{01000} &
\multicolumn{1}{c|}{aq} &
\multicolumn{1}{c|}{rl} &
\multicolumn{2}{c|}{rs2} &
\multicolumn{1}{c|}{rs1} &
\multicolumn{1}{c|}{010} &
\multicolumn{1}{c|}{rd} &
\multicolumn{1}{c|}{0101111} & AMOOR.W \\
\cline{2-11}


&
\multicolumn{2}{|c|}{10000} &
\multicolumn{1}{c|}{aq} &
\multicolumn{1}{c|}{rl} &
\multicolumn{2}{c|}{rs2} &
\multicolumn{1}{c|}{rs1} &
\multicolumn{1}{c|}{010} &
\multicolumn{1}{c|}{rd} &
\multicolumn{1}{c|}{0101111} & AMOMIN.W \\
\cline{2-11}


&
\multicolumn{2}{|c|}{10100} &
\multicolumn{1}{c|}{aq} &
\multicolumn{1}{c|}{rl} &
\multicolumn{2}{c|}{rs2} &
\multicolumn{1}{c|}{rs1} &
\multicolumn{1}{c|}{010} &
\multicolumn{1}{c|}{rd} &
\multicolumn{1}{c|}{0101111} & AMOMAX.W \\
\cline{2-11}


&
\multicolumn{2}{|c|}{11000} &
\multicolumn{1}{c|}{aq} &
\multicolumn{1}{c|}{rl} &
\multicolumn{2}{c|}{rs2} &
\multicolumn{1}{c|}{rs1} &
\multicolumn{1}{c|}{010} &
\multicolumn{1}{c|}{rd} &
\multicolumn{1}{c|}{0101111} & AMOMINU.W \\
\cline{2-11}


&
\multicolumn{2}{|c|}{11100} &
\multicolumn{1}{c|}{aq} &
\multicolumn{1}{c|}{rl} &
\multicolumn{2}{c|}{rs2} &
\multicolumn{1}{c|}{rs1} &
\multicolumn{1}{c|}{010} &
\multicolumn{1}{c|}{rd} &
\multicolumn{1}{c|}{0101111} & AMOMAXU.W \\
\cline{2-11}


&
\multicolumn{10}{c}{} & \\
&
\multicolumn{10}{c}{\bf RV64A标准扩展(RV32A之外的部分)} & \\
\cline{2-11}


&
\multicolumn{2}{|c|}{00010} &
\multicolumn{1}{c|}{aq} &
\multicolumn{1}{c|}{rl} &
\multicolumn{2}{c|}{00000} &
\multicolumn{1}{c|}{rs1} &
\multicolumn{1}{c|}{011} &
\multicolumn{1}{c|}{rd} &
\multicolumn{1}{c|}{0101111} & LR.D \\
\cline{2-11}


&
\multicolumn{2}{|c|}{00011} &
\multicolumn{1}{c|}{aq} &
\multicolumn{1}{c|}{rl} &
\multicolumn{2}{c|}{rs2} &
\multicolumn{1}{c|}{rs1} &
\multicolumn{1}{c|}{011} &
\multicolumn{1}{c|}{rd} &
\multicolumn{1}{c|}{0101111} & SC.D \\
\cline{2-11}


&
\multicolumn{2}{|c|}{00001} &
\multicolumn{1}{c|}{aq} &
\multicolumn{1}{c|}{rl} &
\multicolumn{2}{c|}{rs2} &
\multicolumn{1}{c|}{rs1} &
\multicolumn{1}{c|}{011} &
\multicolumn{1}{c|}{rd} &
\multicolumn{1}{c|}{0101111} & AMOSWAP.D \\
\cline{2-11}


&
\multicolumn{2}{|c|}{00000} &
\multicolumn{1}{c|}{aq} &
\multicolumn{1}{c|}{rl} &
\multicolumn{2}{c|}{rs2} &
\multicolumn{1}{c|}{rs1} &
\multicolumn{1}{c|}{011} &
\multicolumn{1}{c|}{rd} &
\multicolumn{1}{c|}{0101111} & AMOADD.D \\
\cline{2-11}


&
\multicolumn{2}{|c|}{00100} &
\multicolumn{1}{c|}{aq} &
\multicolumn{1}{c|}{rl} &
\multicolumn{2}{c|}{rs2} &
\multicolumn{1}{c|}{rs1} &
\multicolumn{1}{c|}{011} &
\multicolumn{1}{c|}{rd} &
\multicolumn{1}{c|}{0101111} & AMOXOR.D \\
\cline{2-11}


&
\multicolumn{2}{|c|}{01100} &
\multicolumn{1}{c|}{aq} &
\multicolumn{1}{c|}{rl} &
\multicolumn{2}{c|}{rs2} &
\multicolumn{1}{c|}{rs1} &
\multicolumn{1}{c|}{011} &
\multicolumn{1}{c|}{rd} &
\multicolumn{1}{c|}{0101111} & AMOAND.D \\
\cline{2-11}


&
\multicolumn{2}{|c|}{01000} &
\multicolumn{1}{c|}{aq} &
\multicolumn{1}{c|}{rl} &
\multicolumn{2}{c|}{rs2} &
\multicolumn{1}{c|}{rs1} &
\multicolumn{1}{c|}{011} &
\multicolumn{1}{c|}{rd} &
\multicolumn{1}{c|}{0101111} & AMOOR.D \\
\cline{2-11}


&
\multicolumn{2}{|c|}{10000} &
\multicolumn{1}{c|}{aq} &
\multicolumn{1}{c|}{rl} &
\multicolumn{2}{c|}{rs2} &
\multicolumn{1}{c|}{rs1} &
\multicolumn{1}{c|}{011} &
\multicolumn{1}{c|}{rd} &
\multicolumn{1}{c|}{0101111} & AMOMIN.D \\
\cline{2-11}


&
\multicolumn{2}{|c|}{10100} &
\multicolumn{1}{c|}{aq} &
\multicolumn{1}{c|}{rl} &
\multicolumn{2}{c|}{rs2} &
\multicolumn{1}{c|}{rs1} &
\multicolumn{1}{c|}{011} &
\multicolumn{1}{c|}{rd} &
\multicolumn{1}{c|}{0101111} & AMOMAX.D \\
\cline{2-11}


&
\multicolumn{2}{|c|}{11000} &
\multicolumn{1}{c|}{aq} &
\multicolumn{1}{c|}{rl} &
\multicolumn{2}{c|}{rs2} &
\multicolumn{1}{c|}{rs1} &
\multicolumn{1}{c|}{011} &
\multicolumn{1}{c|}{rd} &
\multicolumn{1}{c|}{0101111} & AMOMINU.D \\
\cline{2-11}


&
\multicolumn{2}{|c|}{11100} &
\multicolumn{1}{c|}{aq} &
\multicolumn{1}{c|}{rl} &
\multicolumn{2}{c|}{rs2} &
\multicolumn{1}{c|}{rs1} &
\multicolumn{1}{c|}{011} &
\multicolumn{1}{c|}{rd} &
\multicolumn{1}{c|}{0101111} & AMOMAXU.D \\
\cline{2-11}


\end{tabular}
\end{center}
\end{small}

\end{table}


\newpage

\begin{table}[p]
\begin{small}
\begin{center}
\begin{tabular}{p{0in}p{0.4in}p{0.05in}p{0.05in}p{0.05in}p{0.05in}p{0.4in}p{0.6in}p{0.4in}p{0.6in}p{0.7in}l}
& & & & & & & & & & \\
                      &
\multicolumn{1}{l}{\instbit{31}} &
\multicolumn{1}{r}{\instbit{27}} &
\instbit{26} &
\instbit{25} &
\multicolumn{1}{l}{\instbit{24}} &
\multicolumn{1}{r}{\instbit{20}} &
\instbitrange{19}{15} &
\instbitrange{14}{12} &
\instbitrange{11}{7} &
\instbitrange{6}{0} \\
\cline{2-11}


&
\multicolumn{4}{|c|}{funct7} &
\multicolumn{2}{c|}{rs2} &
\multicolumn{1}{c|}{rs1} &
\multicolumn{1}{c|}{funct3} &
\multicolumn{1}{c|}{rd} &
\multicolumn{1}{c|}{opcode} & R-type \\
\cline{2-11}


&
\multicolumn{2}{|c|}{rs3} &
\multicolumn{2}{c|}{funct2} &
\multicolumn{2}{c|}{rs2} &
\multicolumn{1}{c|}{rs1} &
\multicolumn{1}{c|}{funct3} &
\multicolumn{1}{c|}{rd} &
\multicolumn{1}{c|}{opcode} & R4-type \\
\cline{2-11}


&
\multicolumn{6}{|c|}{imm[11:0]} &
\multicolumn{1}{c|}{rs1} &
\multicolumn{1}{c|}{funct3} &
\multicolumn{1}{c|}{rd} &
\multicolumn{1}{c|}{opcode} & I-type \\
\cline{2-11}


&
\multicolumn{4}{|c|}{imm[11:5]} &
\multicolumn{2}{c|}{rs2} &
\multicolumn{1}{c|}{rs1} &
\multicolumn{1}{c|}{funct3} &
\multicolumn{1}{c|}{imm[4:0]} &
\multicolumn{1}{c|}{opcode} & S-type \\
\cline{2-11}


&
\multicolumn{10}{c}{} & \\
&
\multicolumn{10}{c}{\bf RV32F标准扩展} & \\
\cline{2-11}


&
\multicolumn{6}{|c|}{imm[11:0]} &
\multicolumn{1}{c|}{rs1} &
\multicolumn{1}{c|}{010} &
\multicolumn{1}{c|}{rd} &
\multicolumn{1}{c|}{0000111} & FLW \\
\cline{2-11}


&
\multicolumn{4}{|c|}{imm[11:5]} &
\multicolumn{2}{c|}{rs2} &
\multicolumn{1}{c|}{rs1} &
\multicolumn{1}{c|}{010} &
\multicolumn{1}{c|}{imm[4:0]} &
\multicolumn{1}{c|}{0100111} & FSW \\
\cline{2-11}


&
\multicolumn{2}{|c|}{rs3} &
\multicolumn{2}{c|}{00} &
\multicolumn{2}{c|}{rs2} &
\multicolumn{1}{c|}{rs1} &
\multicolumn{1}{c|}{rm} &
\multicolumn{1}{c|}{rd} &
\multicolumn{1}{c|}{1000011} & FMADD.S \\
\cline{2-11}


&
\multicolumn{2}{|c|}{rs3} &
\multicolumn{2}{c|}{00} &
\multicolumn{2}{c|}{rs2} &
\multicolumn{1}{c|}{rs1} &
\multicolumn{1}{c|}{rm} &
\multicolumn{1}{c|}{rd} &
\multicolumn{1}{c|}{1000111} & FMSUB.S \\
\cline{2-11}


&
\multicolumn{2}{|c|}{rs3} &
\multicolumn{2}{c|}{00} &
\multicolumn{2}{c|}{rs2} &
\multicolumn{1}{c|}{rs1} &
\multicolumn{1}{c|}{rm} &
\multicolumn{1}{c|}{rd} &
\multicolumn{1}{c|}{1001011} & FNMSUB.S \\
\cline{2-11}


&
\multicolumn{2}{|c|}{rs3} &
\multicolumn{2}{c|}{00} &
\multicolumn{2}{c|}{rs2} &
\multicolumn{1}{c|}{rs1} &
\multicolumn{1}{c|}{rm} &
\multicolumn{1}{c|}{rd} &
\multicolumn{1}{c|}{1001111} & FNMADD.S \\
\cline{2-11}


&
\multicolumn{4}{|c|}{0000000} &
\multicolumn{2}{c|}{rs2} &
\multicolumn{1}{c|}{rs1} &
\multicolumn{1}{c|}{rm} &
\multicolumn{1}{c|}{rd} &
\multicolumn{1}{c|}{1010011} & FADD.S \\
\cline{2-11}


&
\multicolumn{4}{|c|}{0000100} &
\multicolumn{2}{c|}{rs2} &
\multicolumn{1}{c|}{rs1} &
\multicolumn{1}{c|}{rm} &
\multicolumn{1}{c|}{rd} &
\multicolumn{1}{c|}{1010011} & FSUB.S \\
\cline{2-11}


&
\multicolumn{4}{|c|}{0001000} &
\multicolumn{2}{c|}{rs2} &
\multicolumn{1}{c|}{rs1} &
\multicolumn{1}{c|}{rm} &
\multicolumn{1}{c|}{rd} &
\multicolumn{1}{c|}{1010011} & FMUL.S \\
\cline{2-11}


&
\multicolumn{4}{|c|}{0001100} &
\multicolumn{2}{c|}{rs2} &
\multicolumn{1}{c|}{rs1} &
\multicolumn{1}{c|}{rm} &
\multicolumn{1}{c|}{rd} &
\multicolumn{1}{c|}{1010011} & FDIV.S \\
\cline{2-11}


&
\multicolumn{4}{|c|}{0101100} &
\multicolumn{2}{c|}{00000} &
\multicolumn{1}{c|}{rs1} &
\multicolumn{1}{c|}{rm} &
\multicolumn{1}{c|}{rd} &
\multicolumn{1}{c|}{1010011} & FSQRT.S \\
\cline{2-11}


&
\multicolumn{4}{|c|}{0010000} &
\multicolumn{2}{c|}{rs2} &
\multicolumn{1}{c|}{rs1} &
\multicolumn{1}{c|}{000} &
\multicolumn{1}{c|}{rd} &
\multicolumn{1}{c|}{1010011} & FSGNJ.S \\
\cline{2-11}


&
\multicolumn{4}{|c|}{0010000} &
\multicolumn{2}{c|}{rs2} &
\multicolumn{1}{c|}{rs1} &
\multicolumn{1}{c|}{001} &
\multicolumn{1}{c|}{rd} &
\multicolumn{1}{c|}{1010011} & FSGNJN.S \\
\cline{2-11}


&
\multicolumn{4}{|c|}{0010000} &
\multicolumn{2}{c|}{rs2} &
\multicolumn{1}{c|}{rs1} &
\multicolumn{1}{c|}{010} &
\multicolumn{1}{c|}{rd} &
\multicolumn{1}{c|}{1010011} & FSGNJX.S \\
\cline{2-11}


&
\multicolumn{4}{|c|}{0010100} &
\multicolumn{2}{c|}{rs2} &
\multicolumn{1}{c|}{rs1} &
\multicolumn{1}{c|}{000} &
\multicolumn{1}{c|}{rd} &
\multicolumn{1}{c|}{1010011} & FMIN.S \\
\cline{2-11}


&
\multicolumn{4}{|c|}{0010100} &
\multicolumn{2}{c|}{rs2} &
\multicolumn{1}{c|}{rs1} &
\multicolumn{1}{c|}{001} &
\multicolumn{1}{c|}{rd} &
\multicolumn{1}{c|}{1010011} & FMAX.S \\
\cline{2-11}


&
\multicolumn{4}{|c|}{1100000} &
\multicolumn{2}{c|}{00000} &
\multicolumn{1}{c|}{rs1} &
\multicolumn{1}{c|}{rm} &
\multicolumn{1}{c|}{rd} &
\multicolumn{1}{c|}{1010011} & FCVT.W.S \\
\cline{2-11}


&
\multicolumn{4}{|c|}{1100000} &
\multicolumn{2}{c|}{00001} &
\multicolumn{1}{c|}{rs1} &
\multicolumn{1}{c|}{rm} &
\multicolumn{1}{c|}{rd} &
\multicolumn{1}{c|}{1010011} & FCVT.WU.S \\
\cline{2-11}


&
\multicolumn{4}{|c|}{1110000} &
\multicolumn{2}{c|}{00000} &
\multicolumn{1}{c|}{rs1} &
\multicolumn{1}{c|}{000} &
\multicolumn{1}{c|}{rd} &
\multicolumn{1}{c|}{1010011} & FMV.X.W \\
\cline{2-11}


&
\multicolumn{4}{|c|}{1010000} &
\multicolumn{2}{c|}{rs2} &
\multicolumn{1}{c|}{rs1} &
\multicolumn{1}{c|}{010} &
\multicolumn{1}{c|}{rd} &
\multicolumn{1}{c|}{1010011} & FEQ.S \\
\cline{2-11}


&
\multicolumn{4}{|c|}{1010000} &
\multicolumn{2}{c|}{rs2} &
\multicolumn{1}{c|}{rs1} &
\multicolumn{1}{c|}{001} &
\multicolumn{1}{c|}{rd} &
\multicolumn{1}{c|}{1010011} & FLT.S \\
\cline{2-11}


&
\multicolumn{4}{|c|}{1010000} &
\multicolumn{2}{c|}{rs2} &
\multicolumn{1}{c|}{rs1} &
\multicolumn{1}{c|}{000} &
\multicolumn{1}{c|}{rd} &
\multicolumn{1}{c|}{1010011} & FLE.S \\
\cline{2-11}


&
\multicolumn{4}{|c|}{1110000} &
\multicolumn{2}{c|}{00000} &
\multicolumn{1}{c|}{rs1} &
\multicolumn{1}{c|}{001} &
\multicolumn{1}{c|}{rd} &
\multicolumn{1}{c|}{1010011} & FCLASS.S \\
\cline{2-11}


&
\multicolumn{4}{|c|}{1101000} &
\multicolumn{2}{c|}{00000} &
\multicolumn{1}{c|}{rs1} &
\multicolumn{1}{c|}{rm} &
\multicolumn{1}{c|}{rd} &
\multicolumn{1}{c|}{1010011} & FCVT.S.W \\
\cline{2-11}


&
\multicolumn{4}{|c|}{1101000} &
\multicolumn{2}{c|}{00001} &
\multicolumn{1}{c|}{rs1} &
\multicolumn{1}{c|}{rm} &
\multicolumn{1}{c|}{rd} &
\multicolumn{1}{c|}{1010011} & FCVT.S.WU \\
\cline{2-11}


&
\multicolumn{4}{|c|}{1111000} &
\multicolumn{2}{c|}{00000} &
\multicolumn{1}{c|}{rs1} &
\multicolumn{1}{c|}{000} &
\multicolumn{1}{c|}{rd} &
\multicolumn{1}{c|}{1010011} & FMV.W.X \\
\cline{2-11}


&
\multicolumn{10}{c}{} & \\
&
\multicolumn{10}{c}{\bf RV64F标准扩展(RV32F之外的部分)} & \\
\cline{2-11}


&
\multicolumn{4}{|c|}{1100000} &
\multicolumn{2}{c|}{00010} &
\multicolumn{1}{c|}{rs1} &
\multicolumn{1}{c|}{rm} &
\multicolumn{1}{c|}{rd} &
\multicolumn{1}{c|}{1010011} & FCVT.L.S \\
\cline{2-11}


&
\multicolumn{4}{|c|}{1100000} &
\multicolumn{2}{c|}{00011} &
\multicolumn{1}{c|}{rs1} &
\multicolumn{1}{c|}{rm} &
\multicolumn{1}{c|}{rd} &
\multicolumn{1}{c|}{1010011} & FCVT.LU.S \\
\cline{2-11}


&
\multicolumn{4}{|c|}{1101000} &
\multicolumn{2}{c|}{00010} &
\multicolumn{1}{c|}{rs1} &
\multicolumn{1}{c|}{rm} &
\multicolumn{1}{c|}{rd} &
\multicolumn{1}{c|}{1010011} & FCVT.S.L \\
\cline{2-11}


&
\multicolumn{4}{|c|}{1101000} &
\multicolumn{2}{c|}{00011} &
\multicolumn{1}{c|}{rs1} &
\multicolumn{1}{c|}{rm} &
\multicolumn{1}{c|}{rd} &
\multicolumn{1}{c|}{1010011} & FCVT.S.LU \\
\cline{2-11}


\end{tabular}
\end{center}
\end{small}

\end{table}


\newpage

\begin{table}[p]
\begin{small}
\begin{center}
\begin{tabular}{p{0in}p{0.4in}p{0.05in}p{0.05in}p{0.05in}p{0.05in}p{0.4in}p{0.6in}p{0.4in}p{0.6in}p{0.7in}l}
& & & & & & & & & & \\
                      &
\multicolumn{1}{l}{\instbit{31}} &
\multicolumn{1}{r}{\instbit{27}} &
\instbit{26} &
\instbit{25} &
\multicolumn{1}{l}{\instbit{24}} &
\multicolumn{1}{r}{\instbit{20}} &
\instbitrange{19}{15} &
\instbitrange{14}{12} &
\instbitrange{11}{7} &
\instbitrange{6}{0} \\
\cline{2-11}


&
\multicolumn{4}{|c|}{funct7} &
\multicolumn{2}{c|}{rs2} &
\multicolumn{1}{c|}{rs1} &
\multicolumn{1}{c|}{funct3} &
\multicolumn{1}{c|}{rd} &
\multicolumn{1}{c|}{opcode} & R-type \\
\cline{2-11}


&
\multicolumn{2}{|c|}{rs3} &
\multicolumn{2}{c|}{funct2} &
\multicolumn{2}{c|}{rs2} &
\multicolumn{1}{c|}{rs1} &
\multicolumn{1}{c|}{funct3} &
\multicolumn{1}{c|}{rd} &
\multicolumn{1}{c|}{opcode} & R4-type \\
\cline{2-11}


&
\multicolumn{6}{|c|}{imm[11:0]} &
\multicolumn{1}{c|}{rs1} &
\multicolumn{1}{c|}{funct3} &
\multicolumn{1}{c|}{rd} &
\multicolumn{1}{c|}{opcode} & I-type \\
\cline{2-11}


&
\multicolumn{4}{|c|}{imm[11:5]} &
\multicolumn{2}{c|}{rs2} &
\multicolumn{1}{c|}{rs1} &
\multicolumn{1}{c|}{funct3} &
\multicolumn{1}{c|}{imm[4:0]} &
\multicolumn{1}{c|}{opcode} & S-type \\
\cline{2-11}


&
\multicolumn{10}{c}{} & \\
&
\multicolumn{10}{c}{\bf RV32D标准扩展} & \\
\cline{2-11}


&
\multicolumn{6}{|c|}{imm[11:0]} &
\multicolumn{1}{c|}{rs1} &
\multicolumn{1}{c|}{011} &
\multicolumn{1}{c|}{rd} &
\multicolumn{1}{c|}{0000111} & FLD \\
\cline{2-11}


&
\multicolumn{4}{|c|}{imm[11:5]} &
\multicolumn{2}{c|}{rs2} &
\multicolumn{1}{c|}{rs1} &
\multicolumn{1}{c|}{011} &
\multicolumn{1}{c|}{imm[4:0]} &
\multicolumn{1}{c|}{0100111} & FSD \\
\cline{2-11}


&
\multicolumn{2}{|c|}{rs3} &
\multicolumn{2}{c|}{01} &
\multicolumn{2}{c|}{rs2} &
\multicolumn{1}{c|}{rs1} &
\multicolumn{1}{c|}{rm} &
\multicolumn{1}{c|}{rd} &
\multicolumn{1}{c|}{1000011} & FMADD.D \\
\cline{2-11}


&
\multicolumn{2}{|c|}{rs3} &
\multicolumn{2}{c|}{01} &
\multicolumn{2}{c|}{rs2} &
\multicolumn{1}{c|}{rs1} &
\multicolumn{1}{c|}{rm} &
\multicolumn{1}{c|}{rd} &
\multicolumn{1}{c|}{1000111} & FMSUB.D \\
\cline{2-11}


&
\multicolumn{2}{|c|}{rs3} &
\multicolumn{2}{c|}{01} &
\multicolumn{2}{c|}{rs2} &
\multicolumn{1}{c|}{rs1} &
\multicolumn{1}{c|}{rm} &
\multicolumn{1}{c|}{rd} &
\multicolumn{1}{c|}{1001011} & FNMSUB.D \\
\cline{2-11}


&
\multicolumn{2}{|c|}{rs3} &
\multicolumn{2}{c|}{01} &
\multicolumn{2}{c|}{rs2} &
\multicolumn{1}{c|}{rs1} &
\multicolumn{1}{c|}{rm} &
\multicolumn{1}{c|}{rd} &
\multicolumn{1}{c|}{1001111} & FNMADD.D \\
\cline{2-11}


&
\multicolumn{4}{|c|}{0000001} &
\multicolumn{2}{c|}{rs2} &
\multicolumn{1}{c|}{rs1} &
\multicolumn{1}{c|}{rm} &
\multicolumn{1}{c|}{rd} &
\multicolumn{1}{c|}{1010011} & FADD.D \\
\cline{2-11}


&
\multicolumn{4}{|c|}{0000101} &
\multicolumn{2}{c|}{rs2} &
\multicolumn{1}{c|}{rs1} &
\multicolumn{1}{c|}{rm} &
\multicolumn{1}{c|}{rd} &
\multicolumn{1}{c|}{1010011} & FSUB.D \\
\cline{2-11}


&
\multicolumn{4}{|c|}{0001001} &
\multicolumn{2}{c|}{rs2} &
\multicolumn{1}{c|}{rs1} &
\multicolumn{1}{c|}{rm} &
\multicolumn{1}{c|}{rd} &
\multicolumn{1}{c|}{1010011} & FMUL.D \\
\cline{2-11}


&
\multicolumn{4}{|c|}{0001101} &
\multicolumn{2}{c|}{rs2} &
\multicolumn{1}{c|}{rs1} &
\multicolumn{1}{c|}{rm} &
\multicolumn{1}{c|}{rd} &
\multicolumn{1}{c|}{1010011} & FDIV.D \\
\cline{2-11}


&
\multicolumn{4}{|c|}{0101101} &
\multicolumn{2}{c|}{00000} &
\multicolumn{1}{c|}{rs1} &
\multicolumn{1}{c|}{rm} &
\multicolumn{1}{c|}{rd} &
\multicolumn{1}{c|}{1010011} & FSQRT.D \\
\cline{2-11}


&
\multicolumn{4}{|c|}{0010001} &
\multicolumn{2}{c|}{rs2} &
\multicolumn{1}{c|}{rs1} &
\multicolumn{1}{c|}{000} &
\multicolumn{1}{c|}{rd} &
\multicolumn{1}{c|}{1010011} & FSGNJ.D \\
\cline{2-11}


&
\multicolumn{4}{|c|}{0010001} &
\multicolumn{2}{c|}{rs2} &
\multicolumn{1}{c|}{rs1} &
\multicolumn{1}{c|}{001} &
\multicolumn{1}{c|}{rd} &
\multicolumn{1}{c|}{1010011} & FSGNJN.D \\
\cline{2-11}


&
\multicolumn{4}{|c|}{0010001} &
\multicolumn{2}{c|}{rs2} &
\multicolumn{1}{c|}{rs1} &
\multicolumn{1}{c|}{010} &
\multicolumn{1}{c|}{rd} &
\multicolumn{1}{c|}{1010011} & FSGNJX.D \\
\cline{2-11}


&
\multicolumn{4}{|c|}{0010101} &
\multicolumn{2}{c|}{rs2} &
\multicolumn{1}{c|}{rs1} &
\multicolumn{1}{c|}{000} &
\multicolumn{1}{c|}{rd} &
\multicolumn{1}{c|}{1010011} & FMIN.D \\
\cline{2-11}


&
\multicolumn{4}{|c|}{0010101} &
\multicolumn{2}{c|}{rs2} &
\multicolumn{1}{c|}{rs1} &
\multicolumn{1}{c|}{001} &
\multicolumn{1}{c|}{rd} &
\multicolumn{1}{c|}{1010011} & FMAX.D \\
\cline{2-11}


&
\multicolumn{4}{|c|}{0100000} &
\multicolumn{2}{c|}{00001} &
\multicolumn{1}{c|}{rs1} &
\multicolumn{1}{c|}{rm} &
\multicolumn{1}{c|}{rd} &
\multicolumn{1}{c|}{1010011} & FCVT.S.D \\
\cline{2-11}


&
\multicolumn{4}{|c|}{0100001} &
\multicolumn{2}{c|}{00000} &
\multicolumn{1}{c|}{rs1} &
\multicolumn{1}{c|}{rm} &
\multicolumn{1}{c|}{rd} &
\multicolumn{1}{c|}{1010011} & FCVT.D.S \\
\cline{2-11}


&
\multicolumn{4}{|c|}{1010001} &
\multicolumn{2}{c|}{rs2} &
\multicolumn{1}{c|}{rs1} &
\multicolumn{1}{c|}{010} &
\multicolumn{1}{c|}{rd} &
\multicolumn{1}{c|}{1010011} & FEQ.D \\
\cline{2-11}


&
\multicolumn{4}{|c|}{1010001} &
\multicolumn{2}{c|}{rs2} &
\multicolumn{1}{c|}{rs1} &
\multicolumn{1}{c|}{001} &
\multicolumn{1}{c|}{rd} &
\multicolumn{1}{c|}{1010011} & FLT.D \\
\cline{2-11}


&
\multicolumn{4}{|c|}{1010001} &
\multicolumn{2}{c|}{rs2} &
\multicolumn{1}{c|}{rs1} &
\multicolumn{1}{c|}{000} &
\multicolumn{1}{c|}{rd} &
\multicolumn{1}{c|}{1010011} & FLE.D \\
\cline{2-11}


&
\multicolumn{4}{|c|}{1110001} &
\multicolumn{2}{c|}{00000} &
\multicolumn{1}{c|}{rs1} &
\multicolumn{1}{c|}{001} &
\multicolumn{1}{c|}{rd} &
\multicolumn{1}{c|}{1010011} & FCLASS.D \\
\cline{2-11}


&
\multicolumn{4}{|c|}{1100001} &
\multicolumn{2}{c|}{00000} &
\multicolumn{1}{c|}{rs1} &
\multicolumn{1}{c|}{rm} &
\multicolumn{1}{c|}{rd} &
\multicolumn{1}{c|}{1010011} & FCVT.W.D \\
\cline{2-11}


&
\multicolumn{4}{|c|}{1100001} &
\multicolumn{2}{c|}{00001} &
\multicolumn{1}{c|}{rs1} &
\multicolumn{1}{c|}{rm} &
\multicolumn{1}{c|}{rd} &
\multicolumn{1}{c|}{1010011} & FCVT.WU.D \\
\cline{2-11}


&
\multicolumn{4}{|c|}{1101001} &
\multicolumn{2}{c|}{00000} &
\multicolumn{1}{c|}{rs1} &
\multicolumn{1}{c|}{rm} &
\multicolumn{1}{c|}{rd} &
\multicolumn{1}{c|}{1010011} & FCVT.D.W \\
\cline{2-11}


&
\multicolumn{4}{|c|}{1101001} &
\multicolumn{2}{c|}{00001} &
\multicolumn{1}{c|}{rs1} &
\multicolumn{1}{c|}{rm} &
\multicolumn{1}{c|}{rd} &
\multicolumn{1}{c|}{1010011} & FCVT.D.WU \\
\cline{2-11}


&
\multicolumn{10}{c}{} & \\
&
\multicolumn{10}{c}{\bf RV64D标准扩展(RV32D之外的部分)} & \\
\cline{2-11}


&
\multicolumn{4}{|c|}{1100001} &
\multicolumn{2}{c|}{00010} &
\multicolumn{1}{c|}{rs1} &
\multicolumn{1}{c|}{rm} &
\multicolumn{1}{c|}{rd} &
\multicolumn{1}{c|}{1010011} & FCVT.L.D \\
\cline{2-11}


&
\multicolumn{4}{|c|}{1100001} &
\multicolumn{2}{c|}{00011} &
\multicolumn{1}{c|}{rs1} &
\multicolumn{1}{c|}{rm} &
\multicolumn{1}{c|}{rd} &
\multicolumn{1}{c|}{1010011} & FCVT.LU.D \\
\cline{2-11}


&
\multicolumn{4}{|c|}{1110001} &
\multicolumn{2}{c|}{00000} &
\multicolumn{1}{c|}{rs1} &
\multicolumn{1}{c|}{000} &
\multicolumn{1}{c|}{rd} &
\multicolumn{1}{c|}{1010011} & FMV.X.D \\
\cline{2-11}


&
\multicolumn{4}{|c|}{1101001} &
\multicolumn{2}{c|}{00010} &
\multicolumn{1}{c|}{rs1} &
\multicolumn{1}{c|}{rm} &
\multicolumn{1}{c|}{rd} &
\multicolumn{1}{c|}{1010011} & FCVT.D.L \\
\cline{2-11}


&
\multicolumn{4}{|c|}{1101001} &
\multicolumn{2}{c|}{00011} &
\multicolumn{1}{c|}{rs1} &
\multicolumn{1}{c|}{rm} &
\multicolumn{1}{c|}{rd} &
\multicolumn{1}{c|}{1010011} & FCVT.D.LU \\
\cline{2-11}


&
\multicolumn{4}{|c|}{1111001} &
\multicolumn{2}{c|}{00000} &
\multicolumn{1}{c|}{rs1} &
\multicolumn{1}{c|}{000} &
\multicolumn{1}{c|}{rd} &
\multicolumn{1}{c|}{1010011} & FMV.D.X \\
\cline{2-11}


\end{tabular}
\end{center}
\end{small}

\end{table}


\newpage

\begin{table}[p]
\begin{small}
\begin{center}
\begin{tabular}{p{0in}p{0.4in}p{0.05in}p{0.05in}p{0.05in}p{0.05in}p{0.4in}p{0.6in}p{0.4in}p{0.6in}p{0.7in}l}
& & & & & & & & & & \\
                      &
\multicolumn{1}{l}{\instbit{31}} &
\multicolumn{1}{r}{\instbit{27}} &
\instbit{26} &
\instbit{25} &
\multicolumn{1}{l}{\instbit{24}} &
\multicolumn{1}{r}{\instbit{20}} &
\instbitrange{19}{15} &
\instbitrange{14}{12} &
\instbitrange{11}{7} &
\instbitrange{6}{0} \\
\cline{2-11}


&
\multicolumn{4}{|c|}{funct7} &
\multicolumn{2}{c|}{rs2} &
\multicolumn{1}{c|}{rs1} &
\multicolumn{1}{c|}{funct3} &
\multicolumn{1}{c|}{rd} &
\multicolumn{1}{c|}{opcode} & R-type \\
\cline{2-11}


&
\multicolumn{2}{|c|}{rs3} &
\multicolumn{2}{c|}{funct2} &
\multicolumn{2}{c|}{rs2} &
\multicolumn{1}{c|}{rs1} &
\multicolumn{1}{c|}{funct3} &
\multicolumn{1}{c|}{rd} &
\multicolumn{1}{c|}{opcode} & R4-type \\
\cline{2-11}


&
\multicolumn{6}{|c|}{imm[11:0]} &
\multicolumn{1}{c|}{rs1} &
\multicolumn{1}{c|}{funct3} &
\multicolumn{1}{c|}{rd} &
\multicolumn{1}{c|}{opcode} & I-type \\
\cline{2-11}


&
\multicolumn{4}{|c|}{imm[11:5]} &
\multicolumn{2}{c|}{rs2} &
\multicolumn{1}{c|}{rs1} &
\multicolumn{1}{c|}{funct3} &
\multicolumn{1}{c|}{imm[4:0]} &
\multicolumn{1}{c|}{opcode} & S-type \\
\cline{2-11}


&
\multicolumn{10}{c}{} & \\
&
\multicolumn{10}{c}{\bf RV32Q标准扩展} & \\
\cline{2-11}


&
\multicolumn{6}{|c|}{imm[11:0]} &
\multicolumn{1}{c|}{rs1} &
\multicolumn{1}{c|}{100} &
\multicolumn{1}{c|}{rd} &
\multicolumn{1}{c|}{0000111} & FLQ \\
\cline{2-11}


&
\multicolumn{4}{|c|}{imm[11:5]} &
\multicolumn{2}{c|}{rs2} &
\multicolumn{1}{c|}{rs1} &
\multicolumn{1}{c|}{100} &
\multicolumn{1}{c|}{imm[4:0]} &
\multicolumn{1}{c|}{0100111} & FSQ \\
\cline{2-11}


&
\multicolumn{2}{|c|}{rs3} &
\multicolumn{2}{c|}{11} &
\multicolumn{2}{c|}{rs2} &
\multicolumn{1}{c|}{rs1} &
\multicolumn{1}{c|}{rm} &
\multicolumn{1}{c|}{rd} &
\multicolumn{1}{c|}{1000011} & FMADD.Q \\
\cline{2-11}


&
\multicolumn{2}{|c|}{rs3} &
\multicolumn{2}{c|}{11} &
\multicolumn{2}{c|}{rs2} &
\multicolumn{1}{c|}{rs1} &
\multicolumn{1}{c|}{rm} &
\multicolumn{1}{c|}{rd} &
\multicolumn{1}{c|}{1000111} & FMSUB.Q \\
\cline{2-11}


&
\multicolumn{2}{|c|}{rs3} &
\multicolumn{2}{c|}{11} &
\multicolumn{2}{c|}{rs2} &
\multicolumn{1}{c|}{rs1} &
\multicolumn{1}{c|}{rm} &
\multicolumn{1}{c|}{rd} &
\multicolumn{1}{c|}{1001011} & FNMSUB.Q \\
\cline{2-11}


&
\multicolumn{2}{|c|}{rs3} &
\multicolumn{2}{c|}{11} &
\multicolumn{2}{c|}{rs2} &
\multicolumn{1}{c|}{rs1} &
\multicolumn{1}{c|}{rm} &
\multicolumn{1}{c|}{rd} &
\multicolumn{1}{c|}{1001111} & FNMADD.Q \\
\cline{2-11}


&
\multicolumn{4}{|c|}{0000011} &
\multicolumn{2}{c|}{rs2} &
\multicolumn{1}{c|}{rs1} &
\multicolumn{1}{c|}{rm} &
\multicolumn{1}{c|}{rd} &
\multicolumn{1}{c|}{1010011} & FADD.Q \\
\cline{2-11}


&
\multicolumn{4}{|c|}{0000111} &
\multicolumn{2}{c|}{rs2} &
\multicolumn{1}{c|}{rs1} &
\multicolumn{1}{c|}{rm} &
\multicolumn{1}{c|}{rd} &
\multicolumn{1}{c|}{1010011} & FSUB.Q \\
\cline{2-11}


&
\multicolumn{4}{|c|}{0001011} &
\multicolumn{2}{c|}{rs2} &
\multicolumn{1}{c|}{rs1} &
\multicolumn{1}{c|}{rm} &
\multicolumn{1}{c|}{rd} &
\multicolumn{1}{c|}{1010011} & FMUL.Q \\
\cline{2-11}


&
\multicolumn{4}{|c|}{0001111} &
\multicolumn{2}{c|}{rs2} &
\multicolumn{1}{c|}{rs1} &
\multicolumn{1}{c|}{rm} &
\multicolumn{1}{c|}{rd} &
\multicolumn{1}{c|}{1010011} & FDIV.Q \\
\cline{2-11}


&
\multicolumn{4}{|c|}{0101111} &
\multicolumn{2}{c|}{00000} &
\multicolumn{1}{c|}{rs1} &
\multicolumn{1}{c|}{rm} &
\multicolumn{1}{c|}{rd} &
\multicolumn{1}{c|}{1010011} & FSQRT.Q \\
\cline{2-11}


&
\multicolumn{4}{|c|}{0010011} &
\multicolumn{2}{c|}{rs2} &
\multicolumn{1}{c|}{rs1} &
\multicolumn{1}{c|}{000} &
\multicolumn{1}{c|}{rd} &
\multicolumn{1}{c|}{1010011} & FSGNJ.Q \\
\cline{2-11}


&
\multicolumn{4}{|c|}{0010011} &
\multicolumn{2}{c|}{rs2} &
\multicolumn{1}{c|}{rs1} &
\multicolumn{1}{c|}{001} &
\multicolumn{1}{c|}{rd} &
\multicolumn{1}{c|}{1010011} & FSGNJN.Q \\
\cline{2-11}


&
\multicolumn{4}{|c|}{0010011} &
\multicolumn{2}{c|}{rs2} &
\multicolumn{1}{c|}{rs1} &
\multicolumn{1}{c|}{010} &
\multicolumn{1}{c|}{rd} &
\multicolumn{1}{c|}{1010011} & FSGNJX.Q \\
\cline{2-11}


&
\multicolumn{4}{|c|}{0010111} &
\multicolumn{2}{c|}{rs2} &
\multicolumn{1}{c|}{rs1} &
\multicolumn{1}{c|}{000} &
\multicolumn{1}{c|}{rd} &
\multicolumn{1}{c|}{1010011} & FMIN.Q \\
\cline{2-11}


&
\multicolumn{4}{|c|}{0010111} &
\multicolumn{2}{c|}{rs2} &
\multicolumn{1}{c|}{rs1} &
\multicolumn{1}{c|}{001} &
\multicolumn{1}{c|}{rd} &
\multicolumn{1}{c|}{1010011} & FMAX.Q \\
\cline{2-11}


&
\multicolumn{4}{|c|}{0100000} &
\multicolumn{2}{c|}{00011} &
\multicolumn{1}{c|}{rs1} &
\multicolumn{1}{c|}{rm} &
\multicolumn{1}{c|}{rd} &
\multicolumn{1}{c|}{1010011} & FCVT.S.Q \\
\cline{2-11}


&
\multicolumn{4}{|c|}{0100011} &
\multicolumn{2}{c|}{00000} &
\multicolumn{1}{c|}{rs1} &
\multicolumn{1}{c|}{rm} &
\multicolumn{1}{c|}{rd} &
\multicolumn{1}{c|}{1010011} & FCVT.Q.S \\
\cline{2-11}


&
\multicolumn{4}{|c|}{0100001} &
\multicolumn{2}{c|}{00011} &
\multicolumn{1}{c|}{rs1} &
\multicolumn{1}{c|}{rm} &
\multicolumn{1}{c|}{rd} &
\multicolumn{1}{c|}{1010011} & FCVT.D.Q \\
\cline{2-11}


&
\multicolumn{4}{|c|}{0100011} &
\multicolumn{2}{c|}{00001} &
\multicolumn{1}{c|}{rs1} &
\multicolumn{1}{c|}{rm} &
\multicolumn{1}{c|}{rd} &
\multicolumn{1}{c|}{1010011} & FCVT.Q.D \\
\cline{2-11}


&
\multicolumn{4}{|c|}{1010011} &
\multicolumn{2}{c|}{rs2} &
\multicolumn{1}{c|}{rs1} &
\multicolumn{1}{c|}{010} &
\multicolumn{1}{c|}{rd} &
\multicolumn{1}{c|}{1010011} & FEQ.Q \\
\cline{2-11}


&
\multicolumn{4}{|c|}{1010011} &
\multicolumn{2}{c|}{rs2} &
\multicolumn{1}{c|}{rs1} &
\multicolumn{1}{c|}{001} &
\multicolumn{1}{c|}{rd} &
\multicolumn{1}{c|}{1010011} & FLT.Q \\
\cline{2-11}


&
\multicolumn{4}{|c|}{1010011} &
\multicolumn{2}{c|}{rs2} &
\multicolumn{1}{c|}{rs1} &
\multicolumn{1}{c|}{000} &
\multicolumn{1}{c|}{rd} &
\multicolumn{1}{c|}{1010011} & FLE.Q \\
\cline{2-11}


&
\multicolumn{4}{|c|}{1110011} &
\multicolumn{2}{c|}{00000} &
\multicolumn{1}{c|}{rs1} &
\multicolumn{1}{c|}{001} &
\multicolumn{1}{c|}{rd} &
\multicolumn{1}{c|}{1010011} & FCLASS.Q \\
\cline{2-11}


&
\multicolumn{4}{|c|}{1100011} &
\multicolumn{2}{c|}{00000} &
\multicolumn{1}{c|}{rs1} &
\multicolumn{1}{c|}{rm} &
\multicolumn{1}{c|}{rd} &
\multicolumn{1}{c|}{1010011} & FCVT.W.Q \\
\cline{2-11}


&
\multicolumn{4}{|c|}{1100011} &
\multicolumn{2}{c|}{00001} &
\multicolumn{1}{c|}{rs1} &
\multicolumn{1}{c|}{rm} &
\multicolumn{1}{c|}{rd} &
\multicolumn{1}{c|}{1010011} & FCVT.WU.Q \\
\cline{2-11}


&
\multicolumn{4}{|c|}{1101011} &
\multicolumn{2}{c|}{00000} &
\multicolumn{1}{c|}{rs1} &
\multicolumn{1}{c|}{rm} &
\multicolumn{1}{c|}{rd} &
\multicolumn{1}{c|}{1010011} & FCVT.Q.W \\
\cline{2-11}


&
\multicolumn{4}{|c|}{1101011} &
\multicolumn{2}{c|}{00001} &
\multicolumn{1}{c|}{rs1} &
\multicolumn{1}{c|}{rm} &
\multicolumn{1}{c|}{rd} &
\multicolumn{1}{c|}{1010011} & FCVT.Q.WU \\
\cline{2-11}


&
\multicolumn{10}{c}{} & \\
&
\multicolumn{10}{c}{\bf RV64Q标准扩展(RV32Q之外的部分)} & \\
\cline{2-11}


&
\multicolumn{4}{|c|}{1100011} &
\multicolumn{2}{c|}{00010} &
\multicolumn{1}{c|}{rs1} &
\multicolumn{1}{c|}{rm} &
\multicolumn{1}{c|}{rd} &
\multicolumn{1}{c|}{1010011} & FCVT.L.Q \\
\cline{2-11}


&
\multicolumn{4}{|c|}{1100011} &
\multicolumn{2}{c|}{00011} &
\multicolumn{1}{c|}{rs1} &
\multicolumn{1}{c|}{rm} &
\multicolumn{1}{c|}{rd} &
\multicolumn{1}{c|}{1010011} & FCVT.LU.Q \\
\cline{2-11}


&
\multicolumn{4}{|c|}{1101011} &
\multicolumn{2}{c|}{00010} &
\multicolumn{1}{c|}{rs1} &
\multicolumn{1}{c|}{rm} &
\multicolumn{1}{c|}{rd} &
\multicolumn{1}{c|}{1010011} & FCVT.Q.L \\
\cline{2-11}


&
\multicolumn{4}{|c|}{1101011} &
\multicolumn{2}{c|}{00011} &
\multicolumn{1}{c|}{rs1} &
\multicolumn{1}{c|}{rm} &
\multicolumn{1}{c|}{rd} &
\multicolumn{1}{c|}{1010011} & FCVT.Q.LU \\
\cline{2-11}


\end{tabular}
\end{center}
\end{small}

\end{table}


\newpage

\begin{table}[p]
\begin{small}
\begin{center}
\begin{tabular}{p{0in}p{0.4in}p{0.05in}p{0.05in}p{0.05in}p{0.05in}p{0.4in}p{0.6in}p{0.4in}p{0.6in}p{0.7in}l}
& & & & & & & & & & \\
                      &
\multicolumn{1}{l}{\instbit{31}} &
\multicolumn{1}{r}{\instbit{27}} &
\instbit{26} &
\instbit{25} &
\multicolumn{1}{l}{\instbit{24}} &
\multicolumn{1}{r}{\instbit{20}} &
\instbitrange{19}{15} &
\instbitrange{14}{12} &
\instbitrange{11}{7} &
\instbitrange{6}{0} \\
\cline{2-11}


&
\multicolumn{4}{|c|}{funct7} &
\multicolumn{2}{c|}{rs2} &
\multicolumn{1}{c|}{rs1} &
\multicolumn{1}{c|}{funct3} &
\multicolumn{1}{c|}{rd} &
\multicolumn{1}{c|}{opcode} & R-type \\
\cline{2-11}


&
\multicolumn{2}{|c|}{rs3} &
\multicolumn{2}{c|}{funct2} &
\multicolumn{2}{c|}{rs2} &
\multicolumn{1}{c|}{rs1} &
\multicolumn{1}{c|}{funct3} &
\multicolumn{1}{c|}{rd} &
\multicolumn{1}{c|}{opcode} & R4-type \\
\cline{2-11}


&
\multicolumn{6}{|c|}{imm[11:0]} &
\multicolumn{1}{c|}{rs1} &
\multicolumn{1}{c|}{funct3} &
\multicolumn{1}{c|}{rd} &
\multicolumn{1}{c|}{opcode} & I-type \\
\cline{2-11}


&
\multicolumn{4}{|c|}{imm[11:5]} &
\multicolumn{2}{c|}{rs2} &
\multicolumn{1}{c|}{rs1} &
\multicolumn{1}{c|}{funct3} &
\multicolumn{1}{c|}{imm[4:0]} &
\multicolumn{1}{c|}{opcode} & S-type \\
\cline{2-11}


&
\multicolumn{10}{c}{} & \\
&
\multicolumn{10}{c}{\bf RV32Zfh标准扩展} & \\
\cline{2-11}


&
\multicolumn{6}{|c|}{imm[11:0]} &
\multicolumn{1}{c|}{rs1} &
\multicolumn{1}{c|}{001} &
\multicolumn{1}{c|}{rd} &
\multicolumn{1}{c|}{0000111} & FLH \\
\cline{2-11}


&
\multicolumn{4}{|c|}{imm[11:5]} &
\multicolumn{2}{c|}{rs2} &
\multicolumn{1}{c|}{rs1} &
\multicolumn{1}{c|}{001} &
\multicolumn{1}{c|}{imm[4:0]} &
\multicolumn{1}{c|}{0100111} & FSH \\
\cline{2-11}


&
\multicolumn{2}{|c|}{rs3} &
\multicolumn{2}{c|}{10} &
\multicolumn{2}{c|}{rs2} &
\multicolumn{1}{c|}{rs1} &
\multicolumn{1}{c|}{rm} &
\multicolumn{1}{c|}{rd} &
\multicolumn{1}{c|}{1000011} & FMADD.H \\
\cline{2-11}


&
\multicolumn{2}{|c|}{rs3} &
\multicolumn{2}{c|}{10} &
\multicolumn{2}{c|}{rs2} &
\multicolumn{1}{c|}{rs1} &
\multicolumn{1}{c|}{rm} &
\multicolumn{1}{c|}{rd} &
\multicolumn{1}{c|}{1000111} & FMSUB.H \\
\cline{2-11}


&
\multicolumn{2}{|c|}{rs3} &
\multicolumn{2}{c|}{10} &
\multicolumn{2}{c|}{rs2} &
\multicolumn{1}{c|}{rs1} &
\multicolumn{1}{c|}{rm} &
\multicolumn{1}{c|}{rd} &
\multicolumn{1}{c|}{1001011} & FNMSUB.H \\
\cline{2-11}


&
\multicolumn{2}{|c|}{rs3} &
\multicolumn{2}{c|}{10} &
\multicolumn{2}{c|}{rs2} &
\multicolumn{1}{c|}{rs1} &
\multicolumn{1}{c|}{rm} &
\multicolumn{1}{c|}{rd} &
\multicolumn{1}{c|}{1001111} & FNMADD.H \\
\cline{2-11}


&
\multicolumn{4}{|c|}{0000010} &
\multicolumn{2}{c|}{rs2} &
\multicolumn{1}{c|}{rs1} &
\multicolumn{1}{c|}{rm} &
\multicolumn{1}{c|}{rd} &
\multicolumn{1}{c|}{1010011} & FADD.H \\
\cline{2-11}


&
\multicolumn{4}{|c|}{0000110} &
\multicolumn{2}{c|}{rs2} &
\multicolumn{1}{c|}{rs1} &
\multicolumn{1}{c|}{rm} &
\multicolumn{1}{c|}{rd} &
\multicolumn{1}{c|}{1010011} & FSUB.H \\
\cline{2-11}


&
\multicolumn{4}{|c|}{0001010} &
\multicolumn{2}{c|}{rs2} &
\multicolumn{1}{c|}{rs1} &
\multicolumn{1}{c|}{rm} &
\multicolumn{1}{c|}{rd} &
\multicolumn{1}{c|}{1010011} & FMUL.H \\
\cline{2-11}


&
\multicolumn{4}{|c|}{0001110} &
\multicolumn{2}{c|}{rs2} &
\multicolumn{1}{c|}{rs1} &
\multicolumn{1}{c|}{rm} &
\multicolumn{1}{c|}{rd} &
\multicolumn{1}{c|}{1010011} & FDIV.H \\
\cline{2-11}


&
\multicolumn{4}{|c|}{0101110} &
\multicolumn{2}{c|}{00000} &
\multicolumn{1}{c|}{rs1} &
\multicolumn{1}{c|}{rm} &
\multicolumn{1}{c|}{rd} &
\multicolumn{1}{c|}{1010011} & FSQRT.H \\
\cline{2-11}


&
\multicolumn{4}{|c|}{0010010} &
\multicolumn{2}{c|}{rs2} &
\multicolumn{1}{c|}{rs1} &
\multicolumn{1}{c|}{000} &
\multicolumn{1}{c|}{rd} &
\multicolumn{1}{c|}{1010011} & FSGNJ.H \\
\cline{2-11}


&
\multicolumn{4}{|c|}{0010010} &
\multicolumn{2}{c|}{rs2} &
\multicolumn{1}{c|}{rs1} &
\multicolumn{1}{c|}{001} &
\multicolumn{1}{c|}{rd} &
\multicolumn{1}{c|}{1010011} & FSGNJN.H \\
\cline{2-11}


&
\multicolumn{4}{|c|}{0010010} &
\multicolumn{2}{c|}{rs2} &
\multicolumn{1}{c|}{rs1} &
\multicolumn{1}{c|}{010} &
\multicolumn{1}{c|}{rd} &
\multicolumn{1}{c|}{1010011} & FSGNJX.H \\
\cline{2-11}


&
\multicolumn{4}{|c|}{0010110} &
\multicolumn{2}{c|}{rs2} &
\multicolumn{1}{c|}{rs1} &
\multicolumn{1}{c|}{000} &
\multicolumn{1}{c|}{rd} &
\multicolumn{1}{c|}{1010011} & FMIN.H \\
\cline{2-11}


&
\multicolumn{4}{|c|}{0010110} &
\multicolumn{2}{c|}{rs2} &
\multicolumn{1}{c|}{rs1} &
\multicolumn{1}{c|}{001} &
\multicolumn{1}{c|}{rd} &
\multicolumn{1}{c|}{1010011} & FMAX.H \\
\cline{2-11}


&
\multicolumn{4}{|c|}{0100000} &
\multicolumn{2}{c|}{00010} &
\multicolumn{1}{c|}{rs1} &
\multicolumn{1}{c|}{rm} &
\multicolumn{1}{c|}{rd} &
\multicolumn{1}{c|}{1010011} & FCVT.S.H \\
\cline{2-11}


&
\multicolumn{4}{|c|}{0100010} &
\multicolumn{2}{c|}{00000} &
\multicolumn{1}{c|}{rs1} &
\multicolumn{1}{c|}{rm} &
\multicolumn{1}{c|}{rd} &
\multicolumn{1}{c|}{1010011} & FCVT.H.S \\
\cline{2-11}


&
\multicolumn{4}{|c|}{0100001} &
\multicolumn{2}{c|}{00010} &
\multicolumn{1}{c|}{rs1} &
\multicolumn{1}{c|}{rm} &
\multicolumn{1}{c|}{rd} &
\multicolumn{1}{c|}{1010011} & FCVT.D.H \\
\cline{2-11}


&
\multicolumn{4}{|c|}{0100010} &
\multicolumn{2}{c|}{00001} &
\multicolumn{1}{c|}{rs1} &
\multicolumn{1}{c|}{rm} &
\multicolumn{1}{c|}{rd} &
\multicolumn{1}{c|}{1010011} & FCVT.H.D \\
\cline{2-11}


&
\multicolumn{4}{|c|}{0100011} &
\multicolumn{2}{c|}{00010} &
\multicolumn{1}{c|}{rs1} &
\multicolumn{1}{c|}{rm} &
\multicolumn{1}{c|}{rd} &
\multicolumn{1}{c|}{1010011} & FCVT.Q.H \\
\cline{2-11}


&
\multicolumn{4}{|c|}{0100010} &
\multicolumn{2}{c|}{00011} &
\multicolumn{1}{c|}{rs1} &
\multicolumn{1}{c|}{rm} &
\multicolumn{1}{c|}{rd} &
\multicolumn{1}{c|}{1010011} & FCVT.H.Q \\
\cline{2-11}


&
\multicolumn{4}{|c|}{1010010} &
\multicolumn{2}{c|}{rs2} &
\multicolumn{1}{c|}{rs1} &
\multicolumn{1}{c|}{010} &
\multicolumn{1}{c|}{rd} &
\multicolumn{1}{c|}{1010011} & FEQ.H \\
\cline{2-11}


&
\multicolumn{4}{|c|}{1010010} &
\multicolumn{2}{c|}{rs2} &
\multicolumn{1}{c|}{rs1} &
\multicolumn{1}{c|}{001} &
\multicolumn{1}{c|}{rd} &
\multicolumn{1}{c|}{1010011} & FLT.H \\
\cline{2-11}


&
\multicolumn{4}{|c|}{1010010} &
\multicolumn{2}{c|}{rs2} &
\multicolumn{1}{c|}{rs1} &
\multicolumn{1}{c|}{000} &
\multicolumn{1}{c|}{rd} &
\multicolumn{1}{c|}{1010011} & FLE.H \\
\cline{2-11}


&
\multicolumn{4}{|c|}{1110010} &
\multicolumn{2}{c|}{00000} &
\multicolumn{1}{c|}{rs1} &
\multicolumn{1}{c|}{001} &
\multicolumn{1}{c|}{rd} &
\multicolumn{1}{c|}{1010011} & FCLASS.H \\
\cline{2-11}


&
\multicolumn{4}{|c|}{1100010} &
\multicolumn{2}{c|}{00000} &
\multicolumn{1}{c|}{rs1} &
\multicolumn{1}{c|}{rm} &
\multicolumn{1}{c|}{rd} &
\multicolumn{1}{c|}{1010011} & FCVT.W.H \\
\cline{2-11}


&
\multicolumn{4}{|c|}{1100010} &
\multicolumn{2}{c|}{00001} &
\multicolumn{1}{c|}{rs1} &
\multicolumn{1}{c|}{rm} &
\multicolumn{1}{c|}{rd} &
\multicolumn{1}{c|}{1010011} & FCVT.WU.H \\
\cline{2-11}


&
\multicolumn{4}{|c|}{1110010} &
\multicolumn{2}{c|}{00000} &
\multicolumn{1}{c|}{rs1} &
\multicolumn{1}{c|}{000} &
\multicolumn{1}{c|}{rd} &
\multicolumn{1}{c|}{1010011} & FMV.X.H \\
\cline{2-11}


&
\multicolumn{4}{|c|}{1101010} &
\multicolumn{2}{c|}{00000} &
\multicolumn{1}{c|}{rs1} &
\multicolumn{1}{c|}{rm} &
\multicolumn{1}{c|}{rd} &
\multicolumn{1}{c|}{1010011} & FCVT.H.W \\
\cline{2-11}


&
\multicolumn{4}{|c|}{1101010} &
\multicolumn{2}{c|}{00001} &
\multicolumn{1}{c|}{rs1} &
\multicolumn{1}{c|}{rm} &
\multicolumn{1}{c|}{rd} &
\multicolumn{1}{c|}{1010011} & FCVT.H.WU \\
\cline{2-11}


&
\multicolumn{4}{|c|}{1111010} &
\multicolumn{2}{c|}{00000} &
\multicolumn{1}{c|}{rs1} &
\multicolumn{1}{c|}{000} &
\multicolumn{1}{c|}{rd} &
\multicolumn{1}{c|}{1010011} & FMV.H.X \\
\cline{2-11}


&
\multicolumn{10}{c}{} & \\
&
\multicolumn{10}{c}{\bf RV64Zfh标准扩展(RV32Zfh之外的部分)} & \\
\cline{2-11}


&
\multicolumn{4}{|c|}{1100010} &
\multicolumn{2}{c|}{00010} &
\multicolumn{1}{c|}{rs1} &
\multicolumn{1}{c|}{rm} &
\multicolumn{1}{c|}{rd} &
\multicolumn{1}{c|}{1010011} & FCVT.L.H \\
\cline{2-11}


&
\multicolumn{4}{|c|}{1100010} &
\multicolumn{2}{c|}{00011} &
\multicolumn{1}{c|}{rs1} &
\multicolumn{1}{c|}{rm} &
\multicolumn{1}{c|}{rd} &
\multicolumn{1}{c|}{1010011} & FCVT.LU.H \\
\cline{2-11}


&
\multicolumn{4}{|c|}{1101010} &
\multicolumn{2}{c|}{00010} &
\multicolumn{1}{c|}{rs1} &
\multicolumn{1}{c|}{rm} &
\multicolumn{1}{c|}{rd} &
\multicolumn{1}{c|}{1010011} & FCVT.H.L \\
\cline{2-11}


&
\multicolumn{4}{|c|}{1101010} &
\multicolumn{2}{c|}{00011} &
\multicolumn{1}{c|}{rs1} &
\multicolumn{1}{c|}{rm} &
\multicolumn{1}{c|}{rd} &
\multicolumn{1}{c|}{1010011} & FCVT.H.LU \\
\cline{2-11}


\end{tabular}
\end{center}
\end{small}
\caption{RISC-V指令列表}
\end{table}



\FloatBarrier
表~\ref{rvgcsrnames}列出了当前被分配了CSR地址的CSR。计时器、计数器,和浮点CSR是这个规范中定义的仅有的CSR。
% Table~\ref{rvgcsrnames} lists the CSRs that have
% currently been allocated CSR addresses.  The timers, counters, and
% floating-point CSRs are the only CSRs defined in this specification.

\begin{table}[htb!]
\begin{center}
\begin{tabular}{|l|l|l|l|}
\hline
编号   & 权限 & 名称 & 描述 \\
\hline
\multicolumn{4}{|c|}{浮点控制和状态寄存器} \\
\hline
\tt 0x001 & 读/写  &\tt fflags     & 浮点增长异常。 \\
\tt 0x002 & 读/写  &\tt frm        & 浮点动态舍入模式。 \\
\tt 0x003 & 读/写  &\tt fcsr       & 浮点控制和状态寄存器({\tt frm} + {\tt fflags})。 \\
\hline
\multicolumn{4}{|c|}{计数器和计时器} \\
\hline
\tt 0xC00 & 只读  &\tt cycle      & 用于RDCYCLE指令的周期计数器。\\
\tt 0xC01 & 只读  &\tt time       & 用于RDTIME指令的计时器。 \\
\tt 0xC02 & 只读  &\tt instret    & 用于RDINSTRET指令的指令退场计数器。\\
\tt 0xC80 & 只读  &\tt cycleh     & {\tt cycle}的高32位,RV32I专用。 \\
\tt 0xC81 & 只读  &\tt timeh      & {\tt time}的高32位,RV32I专用。 \\
\tt 0xC82 & 只读  &\tt instreth   & {\tt instret}的高32位,RV32I专用。 \\
\hline
\end{tabular}
\end{center}
\caption{RISC-V控制和状态寄存器(CSR)地址映射。}
\label{rvgcsrnames}
\end{table}

% \begin{table}[htb!]
%   \begin{center}
%   \begin{tabular}{|l|l|l|l|}
%   \hline
%   Number    & Privilege & Name & Description \\
%   \hline
%   \multicolumn{4}{|c|}{Floating-Point Control and Status Registers} \\
%   \hline
%   \tt 0x001 & Read/write  &\tt fflags     & Floating-Point Accrued Exceptions. \\
%   \tt 0x002 & Read/write  &\tt frm        & Floating-Point Dynamic Rounding Mode. \\
%   \tt 0x003 & Read/write  &\tt fcsr       & Floating-Point Control and Status
%   Register ({\tt frm} + {\tt fflags}). \\
%   \hline
%   \multicolumn{4}{|c|}{Counters and Timers} \\
%   \hline
%   \tt 0xC00 & Read-only  &\tt cycle      & Cycle counter for RDCYCLE instruction. \\
%   \tt 0xC01 & Read-only  &\tt time       & Timer for RDTIME instruction. \\
%   \tt 0xC02 & Read-only  &\tt instret    & Instructions-retired counter for RDINSTRET instruction. \\
%   \tt 0xC80 & Read-only  &\tt cycleh     & Upper 32 bits of {\tt cycle}, RV32I only. \\
%   \tt 0xC81 & Read-only  &\tt timeh      & Upper 32 bits of {\tt time}, RV32I only. \\
%   \tt 0xC82 & Read-only  &\tt instreth   & Upper 32 bits of {\tt instret}, RV32I only. \\
%   \hline
%   \end{tabular}
%   \end{center}
%   \caption{RISC-V control and status register (CSR) address map.}
%   \label{rvgcsrnames}
%   \end{table}
\chapter{扩充的RISC-V}
% \chapter{Extending RISC-V}
\label{extensions}

除了支持标准通用目的软件开发,RISC-V的另一个目标是为更加专门的指令集扩展或更加定制的加速器提供一个基础。
指令编码空间和可选的可变长度指令编码的设计使得在构建更加定制的处理器时,更容易利用那些用于标准ISA工具链的软件开发工作。
例如,意图为只使用标准I基础的实现持续提供完全的软件支持,也许还有许多非标准的指令集扩展。
% In addition to supporting standard general-purpose software
% development, another goal of RISC-V is to provide a basis for more
% specialized instruction-set extensions or more customized
% accelerators.  The instruction encoding spaces and optional
% variable-length instruction encoding are designed to make it easier to
% leverage software development effort for the standard ISA toolchain
% when building more customized processors.  For example, the intent is
% to continue to provide full software support for implementations that
% only use the standard I base, perhaps together with many non-standard
% instruction-set extensions.

这章描述了各种可以扩展基础RISC-V ISA的方法,以及管理由独立工作组开发的指令集扩展的策略。
本卷只考虑非特权ISA,尽管相同的方法和术语也被用于第二卷中描述的监管器级别的扩展。
% This chapter describes various ways in which the base RISC-V ISA can
% be extended, together with the scheme for managing instruction-set
% extensions developed by independent groups.  This volume only deals
% with the unprivileged ISA, although the same approach and terminology is
% used for supervisor-level extensions described in the second volume.

\section{扩展术语}
% \section{Extension Terminology}

本节定义了一些用于描述RISC-V扩展的标准术语。
% This section defines some standard terminology for describing RISC-V
% extensions.
\vspace{-0.2in}
\subsection*{标准扩展vs非标准扩展}
% \subsection*{Standard versus Non-Standard Extension}

任何RISC-V处理器实现必须支持一个基础整数ISA(RV32I、RV32E、RV64I或RV128I)。
此外,一个实现可以支持一个或更多的扩展。我们把扩展划分为两个宽泛的种类:{\em 标准扩展}vs{\em 非标准扩展}。
% Any RISC-V processor implementation must support a base integer ISA
% (RV32I, RV32E, RV64I, or RV128I).  In addition, an implementation may
% support one or more extensions.  We divide extensions into two broad
% categories: {\em standard} versus {\em non-standard}.
\begin{itemize}
\item 一个标准扩展是,一种通常有用的扩展,它被设计为不与任何其它标志扩展相冲突。
当前,在这本手册的其它章节中描述的“MAFDQLCBTPV”,或者是完整的、或者是计划中的标准扩展。
% A standard extension is one that is generally useful and that is
%   designed to not conflict with any other standard extension.
%   Currently, ``MAFDQLCBTPV'', described in other chapters of this
%   manual, are either complete or planned standard extensions.
\item 一个非标准扩展,可以是高度专用的、并且可以与其它标准或非标准扩展冲突的扩展。
  我们预计随着时间的推移,将会开发出多种多样的非标准扩展,而其中的某些最终会被提升为标准扩展。
% A non-standard extension may be highly specialized and may
%   conflict with other standard or non-standard extensions.  We
%   anticipate a wide variety of non-standard extensions will be
%   developed over time, with some eventually being promoted to standard
%   extensions.
\end{itemize}

\vspace{-0.2in}
\subsection*{指令编码空间和前缀}
% \subsection*{Instruction Encoding Spaces and Prefixes}

指令编码空间是一定数目的指令位,在这些指令位中编码了基础ISA或ISA扩展。
RISC-V支持多种指令长度,但是即使在单一指令长度中,也有各种尺寸的可用编码空间。
例如,基础ISA被定义在一个30位编码空间(32位指令的位31-2)之中,同时原子扩展“A”容纳在一个25位编码空间(位31-7)之中。
% An instruction encoding space is some number of instruction bits
% within which a base ISA or ISA extension is encoded.  RISC-V supports
% varying instruction lengths, but even within a single instruction
% length, there are various sizes of encoding space available.  For
% example, the base ISAs are defined within a 30-bit encoding space (bits
% 31--2 of the 32-bit instruction), while the atomic extension ``A''
% fits within a 25-bit encoding space (bits 31--7).

我们使用术语“{\em 前缀}”来指代一个指令编码空间的{\em 右边}的位(因为RISC-V中的指令获取是小字节序的,
右边的位被存储在更早的内存地址处,因此形成了一个按照指令获取次序的前缀)。
标准基础ISA编码的前缀是两位“11”域,在32位字的位1-0中,而标准原子扩展“A”的前缀是七位的“0101111”域,
保持在代表AMO主操作码的32位字的位6-0中。编码格式的一个怪癖是,在32位指令格式中,
用于编码次要操作码的3位的funct3域虽然不与主操作码位相接,但是却被认为是22位指令空间的前缀的一部分。
% We use the term {\em prefix} to refer to the bits to the {\em right}
% of an instruction encoding space (since instruction fetch in RISC-V is
% little-endian, the
% bits to the right are stored at earlier memory addresses, hence form a
% prefix in instruction-fetch order).  The prefix for the standard base
% ISA encoding is the two-bit ``11'' field held in bits 1--0 of the
% 32-bit word, while the prefix for the standard atomic extension ``A''
% is the seven-bit ``0101111'' field held in bits 6--0 of the 32-bit
% word representing the AMO major opcode.  A quirk of the encoding
% format is that the 3-bit funct3 field used to encode a minor opcode is
% not contiguous with the major opcode bits in the 32-bit instruction
% format, but is considered part of the prefix for 22-bit instruction
% spaces.

尽管一个指令编码空间可以是任何尺寸的,采用一组较小的常见尺寸将简化把独立开发的扩展打包进单一的全局编码中的过程。
表~\ref{encodingspaces}为RISC-V给出了建议的尺寸。
% Although an instruction encoding space could be of any size, adopting
% a smaller set of common sizes simplifies packing independently
% developed extensions into a single global encoding.
% Table~\ref{encodingspaces} gives the suggested sizes for RISC-V.

\begin{table}[H]
\begin{center}
\begin{tabular}{|c|l|r|r|r|r|}
\hline
\multicolumn{1}{|c|}{Size} & \multicolumn{1}{|c|}{Usage} &
\multicolumn{4}{|c|}{\# 可用的标准指令长度} \\ \cline{3-6}
 & &
\multicolumn{1}{|c|}{16-bit} &
\multicolumn{1}{|c|}{32-bit} &
\multicolumn{1}{|c|}{48-bit} &
\multicolumn{1}{|c|}{64-bit} \\ \hline \hline
14位 & 压缩16位编码的象限 & 3       &         &         &         \\ \hline \hline
22位 & 基础32位编码中的次要操作码   &         & $2^{8}$ & $2^{20}$ & $2^{35}$ \\ \hline
25位 & 基础32位编码中的主要操作码  &         &      32 & $2^{17}$ & $2^{32}$ \\ \hline
30位 & 基础32位编码的象限       &         &       1 & $2^{12}$ & $2^{27}$ \\ \hline \hline
32位 & 48位编码中的次要操作码        &         &         & $2^{10}$ & $2^{25}$ \\ \hline
37位 & 48位编码中的主要操作码        &         &         &       32 & $2^{20}$ \\ \hline
40位 & 48位编码的象限            &         &         &        4 & $2^{17}$ \\ \hline \hline
45位 & 64位编码中的子级次要操作码    &         &         &          & $2^{12}$ \\ \hline
48位 & 64位编码中的次要操作码        &         &         &          & $2^{9}$  \\ \hline
52位 & 64位编码中的主要操作码        &         &         &          &      32\\ \hline
\end{tabular}
\end{center}
\caption{建议的标准RISC-V指令编码空间尺寸。}
\label{encodingspaces}
\end{table}

\vspace{-0.2in}
\subsection*{绿色地带扩展vs棕色地带扩展}
% \subsection*{Greenfield versus Brownfield Extensions}

我们使用术语{\em 绿色地带扩展}来描述一个这样的扩展,它在一个新的指令编码空间开始发展,并因此只能在前缀级别引起编码冲突。
我们使用术语{\em 棕色地带扩展}来描述一个扩展,它符合在先前定义的指令空间中的现有编码。
一个棕色地带扩展必须联系到一个特定的绿色地带父级编码,而对于相同的绿色地带父级编码,可能有多个棕色地带扩展。
例如,基础ISA是一个30位指令空间的绿色地带编码,同时FDQ浮点扩展都是添加到父级基础ISA的30位编码空间的棕色地带扩展。
% We use the term {\em greenfield extension} to describe an extension
% that begins populating a new instruction encoding space, and hence can
% only cause encoding conflicts at the prefix level.  We use the term
% {\em brownfield extension} to describe an extension that fits around
% existing encodings in a previously defined instruction space.  A
% brownfield extension is necessarily tied to a particular greenfield
% parent encoding, and there may be multiple brownfield extensions to
% the same greenfield parent encoding.  For example, the base ISAs are
% greenfield encodings of a 30-bit instruction space, while the FDQ
% floating-point extensions are all brownfield extensions adding to the
% parent base ISA 30-bit encoding space.

注意,我们认为标准A扩展具有绿色地带编码,因为它在全32位基础指令编码的最左侧的位中定义了一个全新的、先前为空的25位编码空间,
即使它的标准前缀把它定位在了父基础ISA的30位编码空间之中。
仅仅改变它的一个7位前缀可以把A扩展移动到一个不同的30位编码空间,同时只需要担心前缀级别的冲突,而在编码空间自身之中不会有冲突。
% Note that we consider the standard A extension to have a greenfield
% encoding as it defines a new previously empty 25-bit encoding space in
% the leftmost bits of the full 32-bit base instruction encoding, even
% though its standard prefix locates it within the 30-bit encoding space
% of its parent base ISA.
% Changing only its single 7-bit prefix could move the
% A extension to a different 30-bit encoding space while only worrying
% about conflicts at the prefix level, not within the encoding space
% itself.

\begin{table}[H]
{
\begin{center}
\begin{tabular}{|r|c|c|}
\hline
 & 添加状态 & 没有新状态 \\ \hline
绿色地带编码 & RV32I(30), RV64I(30) & A(25) \\\hline
棕色地带编码 & F(I), D(F), Q(D) & M(I) \\
\hline
\end{tabular}
\end{center}
}
\caption{标准指令集扩展的二维特征。
  % Two-dimensional characterization of standard instruction-set extensions.
  }
\label{exttax}
\end{table}

表~\ref{exttax}显示了置于一种简单的二维分类中的基础和标准扩展。
一个轴是该扩展属于绿色地带的还是棕色地带的,而另一个轴是该扩展是否添加了架构的状态。
对于绿色地带扩展,括号中给出的是指令编码空间的尺寸。对于棕色地带扩展,括号中给出的是其构建所基于的扩展的名字
(绿色地带或者棕色地带)。额外的用户级架构状态通常暗示了对监管器级别系统的改变,或者对标准调用约定的可能的改变。
% Table~\ref{exttax} shows the bases and standard extensions placed in a
% simple two-dimensional taxonomy.  One axis is whether the extension is
% greenfield or brownfield, while the other axis is whether the
% extension adds architectural state.  For greenfield extensions, the
% size of the instruction encoding space is given in parentheses.  For
% brownfield extensions, the name of the extension (greenfield or
% brownfield) it builds upon is given in parentheses.  Additional
% user-level architectural state usually implies changes to the
% supervisor-level system or possibly to the standard calling
% convention.

注意RV64I并不被认为是RV32I的一个扩展,而是一种不同的完整的基础编码。
% Note that RV64I is not considered an extension of RV32I, but a
% different complete base encoding.

\vspace{-0.2in}
\subsection*{标准-可兼容全局编码}
% \subsection*{Standard-Compatible Global Encodings}

对于一个确实的RISC-V实现,一个ISA的完整的或{\em 全局的}编码必须为每个所包含的指令编码空间分配一个唯一的不冲突的前缀。
基础扩展和每个标准扩展各拥有一个已分配的标准前缀,以确保它们都可以在全局编码中共存。
% A complete or {\em global} encoding of an ISA for an actual RISC-V
% implementation must allocate a unique non-conflicting prefix for every
% included instruction encoding space.  The bases and every standard
% extension have each had a standard prefix allocated to ensure they can
% all coexist in a global encoding.

{\em 标准-可兼容}全局编码是,一种基础扩展和每个所包含的标准扩展都拥有它们的标准前缀的编码。
标准-可兼容全局编码可以含有与所包含的标准扩展不相冲突的非标准扩展。标准-可兼容全局编码也可以为非标准扩展使用标准前缀,
如果相关联的标准扩展并没有被包含在全局编码之中。换句话说,一个标准扩展如果被包含在一个标准-可兼容全局编码之中,
那么必须使用它的标准前缀,但是否则的话,它的前缀可以自由地重新分配。
这些限制允许常见的工具链把任何RISC-V标准-可兼容全局编码的标准子集作为目标。
% A {\em standard-compatible} global encoding is one where the base and
% every included standard extension have their standard prefixes.  A
% standard-compatible global encoding can include non-standard
% extensions that do not conflict with the included standard extensions.
% A standard-compatible global encoding can also use standard prefixes
% for non-standard extensions if the associated standard extensions are
% not included in the global encoding.  In other words, a standard
% extension must use its standard prefix if included in a
% standard-compatible global encoding, but otherwise its prefix is free
% to be reallocated.  These constraints allow a common toolchain to
% target the standard subset of any RISC-V standard-compatible global
% encoding.

\vspace{-0.2in}
\subsection*{保证的非标准编码空间}
% \subsection*{Guaranteed Non-Standard Encoding Space}

为了支持专有的自定义扩展的开发,部分编码空间被保证永远不会被标准扩展使用。
% To support development of proprietary custom extensions, portions of
% the encoding space are guaranteed to never be used by standard
% extensions.

\section{RISC-V扩展设计理念}
% \section{RISC-V Extension Design Philosophy}

我们试图通过鼓励扩展开发者们在指令编码空间中操作,以及通过提供工具来为这些分配独有的前缀来将其打包进标准-可兼容全局编码,
来支持大量的独立开发的扩展。某些扩展被自然地实现为现有扩展的棕色地带扩充,而将分享分配给它们的父级绿色地带扩展的任何前缀。
标准扩展前缀避免了核心功能编码中的虚假的不兼容,同时允许定制更加深奥的扩展。
% We intend to support a large number of independently developed
% extensions by encouraging extension developers to operate within
% instruction encoding spaces, and by providing tools to pack these into
% a standard-compatible global encoding by allocating unique prefixes.
% Some extensions are more naturally implemented as brownfield
% augmentations of existing extensions, and will share whatever prefix
% is allocated to their parent greenfield extension.  The standard
% extension prefixes avoid spurious incompatibilities in the encoding of
% core functionality, while allowing custom packing of more esoteric
% extensions.

这种把RISC-V扩展重新打包进不同的标准-可兼容全局编码的能力可以有多种使用的方式。
% This capability of repacking RISC-V extensions into different
% standard-compatible global encodings can be used in a number of ways.

一种使用情况是,开发高度特化的定制加速器,是为了运行来自重要应用领域的内核而设计。
这些加速器可能希望丢弃除了基础整数ISA之外的所有ISA,而只加入手头任务所需要的扩展。
基础ISA被设计为,呈现了关于一个硬件实现的最低需求,而被编码为只使用了32位指令编码空间一小部分。
% One use-case is developing highly specialized custom accelerators,
% designed to run kernels from important application domains.  These
% might want to drop all but the base integer ISA and add in only the
% extensions that are required for the task in hand.  The base ISAs have
% been designed to place minimal requirements on a hardware
% implementation, and has been encoded to use only a small fraction of a
% 32-bit instruction encoding space.

另一种使用情况是,为一种新的类型的指令集扩展构建一个研究原型。研究人员可能不想把工作扩展到实现一个可变长度的指令获取单元,
并因此愿意使用一个简单的32位定宽指令编码来构建他们的扩展的原型。然而,这个新的扩展可能太大,而不能与32位空间中的标准扩展共存。
如果研究实验不需要所有的标准扩展,标准-可兼容全局编码可以丢弃不使用的标准扩展,而重用它们的前缀,
以把所提出的扩展放置在非标准位置中,来简化研究原型的工程。标准工具将仍然能够把基础扩展和任何存在的标准扩展作为目标,
以减少开发时间。一旦指令集扩展被评估和优化过,然后它就可以被打包进一个更大的、可变长度的编码空间而变得可用,
以避免与所有标准扩展冲突。
% Another use-case is to build a research prototype for a new type of
% instruction-set extension.  The researchers might not want to expend
% the effort to implement a variable-length instruction-fetch unit, and
% so would like to prototype their extension using a simple 32-bit
% fixed-width instruction encoding.  However, this new extension might
% be too large to coexist with standard extensions in the 32-bit space.
% If the research experiments do not need all of the standard
% extensions, a standard-compatible global encoding might drop the
% unused standard extensions and reuse their prefixes to place the
% proposed extension in a non-standard location to simplify engineering
% of the research prototype.  Standard tools will still be able to
% target the base and any standard extensions that are present to reduce
% development time.  Once the instruction-set extension has been
% evaluated and refined, it could then be made available for packing
% into a larger variable-length encoding space to avoid conflicts with
% all standard extensions.

下面的章节描述了使用新指令集扩展开发实现的越来越复杂的策略。
这些策略主要是为了用于高度定制的、教育的、或者实验的架构,而不是用于RISC-V ISA开发的主线。
% The following sections describe increasingly sophisticated strategies
% for developing implementations with new instruction-set extensions.
% These are mostly intended for use in highly customized, educational,
% or experimental architectures rather than for the main line of RISC-V
% ISA development.

\section{定宽32位指令格式下的扩展}
% \section{Extensions within fixed-width 32-bit instruction format}
\label{fix32b}

在这节中,我们对向只支持基础定宽32位指令格式的实现添加扩展的内容进行了讨论。
% In this section, we discuss adding extensions to implementations that
% only support the base fixed-width 32-bit instruction format.

\begin{commentary}
  我们预计,最简单的定宽32位编码将在许多受限的加速器和研究原型中变得流行。
% We anticipate the simplest fixed-width 32-bit encoding will be popular for
% many restricted accelerators and research prototypes.
\end{commentary}

\subsection*{可用的30位指令编码空间}
% \subsection*{Available 30-bit instruction encoding spaces}

在标准编码中,可用的30位指令编码空间中的三个(2位前缀是00、01和10的那些)被用于启用可选的压缩指令扩展。
然而,如果压缩指令集扩展是不需要的,那么这三个额外的30位编码空间就变得可用了。这使32位格式中的可用编码空间变成了四倍。
% In the standard encoding, three of the available 30-bit instruction
% encoding spaces (those with 2-bit prefixes 00, 01, and 10) are used to
% enable the optional compressed instruction extension.  However, if the
% compressed instruction-set extension is not required, then these three
% further 30-bit encoding spaces become available.  This quadruples the
% available encoding space within the 32-bit format.

\subsection*{可用的25位指令编码空间}
% \subsection*{Available 25-bit instruction encoding spaces}

一个25位指令编码空间对应于基础和标准扩展编码中的一个主要操作码。
% A 25-bit instruction encoding space corresponds to a major opcode in
% the base and standard extension encodings.

有四个主要的操作码被明确指定用于自定义扩展(表~\ref{opcodemap}),它们中的每个都代表一个25位编码空间。
% 其中的两个(将是OP-IMM-64和OP-64)为RV128基础编码的最终使用而保留,但是可以被用于RV32和RV64的非标准扩展。
% There are four major opcodes expressly designated for custom extensions
% (Table~\ref{opcodemap}), each of which represents a 25-bit encoding
% space.  Two of these are reserved for eventual use in the RV128 base
% encoding (will be OP-IMM-64 and OP-64), but can be used for
% non-standard extensions for RV32 and RV64.

两个保留用于RV64的主要操作码(OP-IMM-32和OP-32)也可以被只用于RV32的非标准扩展。
% The two major opcodes reserved for RV64 (OP-IMM-32 and OP-32) can also be
% used for non-standard extensions to RV32 only.

如果实现不需要浮点,那么七个保留用于标准浮点扩展的主要的操作码(LOAD-FP、STORE-FP、MADD、MSUB、NMSUB、NMADD、OP-FP)可以被重用于非标准扩展。
类似地,AMO主操作码可以被重用,如果不需要标准原子扩展的话。
% If an implementation does not require floating-point, then the seven
% major opcodes reserved for standard floating-point extensions
% (LOAD-FP, STORE-FP, MADD, MSUB, NMSUB, NMADD, OP-FP) can be reused for
% non-standard extensions.  Similarly, the AMO major opcode can be
% reused if the standard atomic extensions are not required.

如果实现不需要超过32位长的指令,那么额外的四个主要的操作码是可用的(在表~\ref{opcodemap}中被标记为灰色的那些)。
% If an implementation does not require instructions longer than
% 32-bits, then an additional four major opcodes are available (those
% marked in gray in Table~\ref{opcodemap}).

基础RV32I编码只使用11个主要的操作码和3个保留的操作码,留给了扩展18个可用的操作码。
基础RV64I编码只使用13个主要的操作码和3个保留的操作码,留给了扩展16个可用的操作码。
% The base RV32I encoding uses only 11 major opcodes plus 3 reserved
% opcodes, leaving up to 18 available for extensions.  The base RV64I
% encoding uses only 13 major opcodes plus 3 reserved opcodes, leaving
% up to 16 available for extensions.

\subsection*{可用的22位指令编码空间}
% \subsection*{Available 22-bit instruction encoding spaces}

一个22位编码空间对应于基础和标准扩展编码中的一个funct3次要操作码空间。
一些主要的操作码有一个没有被完全占用的funct3域的次要操作码,留下了一些可用的22位编码空间。
% A 22-bit encoding space corresponds to a funct3 minor opcode space in
% the base and standard extension encodings.  Several major opcodes have
% a funct3 field minor opcode that is not completely occupied, leaving
% available several 22-bit encoding spaces.

通常一个主要的操作码在指令余下的位中选择用于编码操作数的格式,并且理想情况下,扩展应当遵循主要的操作码的操作数格式,以简化硬件解码。
% Usually a major opcode selects the format used to encode operands in
% the remaining bits of the instruction, and ideally, an extension
% should follow the operand format of the major opcode to simplify
% hardware decoding.

\subsection*{其它空间}
% \subsection*{Other spaces}

在特定的主要的操作码下可以使用更小的空间,并且不是所有的次要操作码都被完全填满。
% Smaller spaces are available under certain major opcodes, and not all
% minor opcodes are entirely filled.

\section{添加对齐的64位指令扩展}
% \section{Adding aligned 64-bit instruction extensions}

对于基础32位定宽指令格式来说,为太大的扩展提供空间的最简单的方法是添加自然对齐的64位指令。
该实现仍然必须支持32位基础指令格式,但是可以要求64位指令在64位边界对齐,以简化指令获取,必要时使用32位NOP指令作为对齐的填充。
% The simplest approach to provide space for extensions that are too
% large for the base 32-bit fixed-width instruction format is to add
% naturally aligned 64-bit instructions.  The implementation must still
% support the 32-bit base instruction format, but can require that
% 64-bit instructions are aligned on 64-bit boundaries to simplify
% instruction fetch, with a 32-bit NOP instruction used as alignment
% padding where necessary.

为了简化标准工具的使用,64位指令的编码应当像表~\ref{instlengthcode}中描述的那样。然而,实现可能为64位指令选择了一个非标准的指令长度编码,
同时为32位指令保留了标准编码。例如,如果压缩指令是不需要的,那么64位指令可以在指令的前两位中使用一个或更多的零位来编码。
% To simplify use of standard tools, the 64-bit instructions should be
% encoded as described in Figure~\ref{instlengthcode}.  However, an
% implementation might choose a non-standard instruction-length encoding
% for 64-bit instructions, while retaining the standard encoding for
% 32-bit instructions.  For example, if compressed instructions are not
% required, then a 64-bit instruction could be encoded using one or more
% zero bits in the first two bits of an instruction.

\begin{commentary}
  我们期待生产指令获取单元的处理器产生器能够自动地处理任何所支持的可变长度指令编码的组合
% We anticipate processor generators that produce instruction-fetch
% units capable of automatically handling any combination of supported
% variable-length instruction encodings.
\end{commentary}

\section{支持VLIW编码}
% \section{Supporting VLIW encodings}

尽管RISC-V并不是作为一个纯VLIW机器的基础而设计,也可以使用一些替代的方法,将VLIW编码作为扩展添加。
但是在所有情况中,都必须支持基础32位编码,以允许任何标准软件工具的使用。
% Although RISC-V was not designed as a base for a pure VLIW machine,
% VLIW encodings can be added as extensions using several alternative
% approaches. In all cases, the base 32-bit encoding has to be supported
% to allow use of any standard software tools.

\subsection*{固定尺寸指令组}
% \subsection*{Fixed-size instruction group}

最简单的方法是,在编码的VLIW操作中定义一个单一的、大型的、自然对齐的指令格式(例如,128位)。
在一个传统的VLIW中,这个方法往往会浪费指令内存来容纳NOP,但是一个兼容RISC-V的实现也将不得不支持基础32位指令,
从而限制了将VLIW代码尺寸扩张到VLIW加速的函数。
% The simplest approach is to define a single large naturally aligned
% instruction format (e.g., 128 bits) within which VLIW operations are
% encoded.  In a conventional VLIW, this approach would tend to waste
% instruction memory to hold NOPs, but a RISC-V-compatible
% implementation would have to also support the base 32-bit
% instructions, confining the VLIW code size expansion to
% VLIW-accelerated functions.

\subsection*{编码长度组}
% \subsection*{Encoded-Length Groups}

另一个方法是,使用表~\ref{instlengthcode}中的标准长度编码来编码并行的指令组,这允许NOP被压缩在VLIW指令之外。
例如,一个64位指令可以容纳两个28位操作,同时一个96位指令可以容纳三个28位操作,等等。
或者,一个48位指令可以容纳一个42位操作,同时一个96位指令可以容纳两个42位操作,等等。
% Another approach is to use the standard length encoding from
% Figure~\ref{instlengthcode} to encode parallel instruction groups,
% allowing NOPs to be compressed out of the VLIW instruction.  For
% example, a 64-bit instruction could hold two 28-bit operations, while
% a 96-bit instruction could hold three 28-bit operations, and so on.
% Alternatively, a 48-bit instruction could hold one 42-bit operation,
% while a 96-bit instruction could hold two 42-bit operations, and so
% on.

这个方法具有为容纳单一操作的指令保留了基础ISA编码的优势,但是劣势在于,
需要为VLIW指令中的操作、以及更大指令组的未对齐的指令获取,使用新的28位或42位编码。
一个简化方法是,不允许VLIW指令跨越特定的微架构的重要边界(例如,缓存行或者虚拟内存页)。
% This approach has the advantage of retaining the base ISA encoding for
% instructions holding a single operation, but has the disadvantage of
% requiring a new 28-bit or 42-bit encoding for operations within the
% VLIW instructions, and misaligned instruction fetch for larger groups.
% One simplification is to not allow VLIW instructions to straddle
% certain microarchitecturally significant boundaries (e.g., cache lines
% or virtual memory pages).

\subsection*{固定尺寸指令扎}
% \subsection*{Fixed-Size Instruction Bundles}

另一个方法,与Itanium类似,是使用一个较大的自然对齐的固定指令扎尺寸(例如,128位),并行操作组在该指令扎上进行编码。
这个方法简化了指令获取,但是把复杂度转移给了组执行引擎。为了保持RISC-V的兼容性,将必须仍然支持基础32位指令。
% Another approach, similar to Itanium, is to use a larger naturally
% aligned fixed instruction bundle size (e.g., 128 bits) across which
% parallel operation groups are encoded.  This simplifies instruction
% fetch, but shifts the complexity to the group execution engine.  To
% remain RISC-V compatible, the base 32-bit instruction would still have
% to be supported.

\subsection*{前缀中的End-Of-Group位}
% \subsection*{End-of-Group bits in Prefix}

上述方法没有一个为VLIW指令中的单独操作保留了RISC-V编码。然而另一个方法是在定宽32位编码中重新赋予两个前缀位新的用途。
一个前缀位可以被用于在被设置时指示“组的结束”,同时第二个位可以在被清除时表示正在谓词下执行。
由VLIW扩展感知不到的工具生成的标准RISC-V 32位指令将把两个前缀位都设置为(11),
并因此具有正确的语义:每条指令都位于组的结尾并且都不是谓词。
% None of the above approaches retains the RISC-V encoding for the
% individual operations within a VLIW instruction.  Yet another approach
% is to repurpose the two prefix bits in the fixed-width 32-bit
% encoding.  One prefix bit can be used to signal ``end-of-group'' if
% set, while the second bit could indicate execution under a predicate
% if clear.  Standard RISC-V 32-bit instructions generated by tools
% unaware of the VLIW extension would have both prefix bits set (11) and
% thus have the correct semantics, with each instruction at the end of a
% group and not predicated.

这个方法的主要劣势是,基础ISA缺少复杂的谓词支持,而那在一个激进的VLIW系统中通常是需要的;并且在标准30位编码空间中,
很难增加空间以指定更多的谓词寄存器。
% The main disadvantage of this approach is that the base ISAs lack the
% complex predication support usually required in an aggressive VLIW
% system, and it is difficult to add space to specify more predicate
% registers in the standard 30-bit encoding space.

\chapter{ISA扩展命名约定}
% \chapter{ISA Extension Naming Conventions}
\label{naming}

这章描述了RISC-V ISA扩展的命名策略,它被用于简洁地描述一个硬件实现中现有指令的集合,
或者被一个应用二进制接口(ABI)所使用的指令的集合。
% This chapter describes the RISC-V ISA extension naming scheme that is
% used to concisely describe the set of instructions present in a
% hardware implementation, or the set of instructions used by an
% application binary interface (ABI).

\begin{commentary}
  RISC-V ISA为了支持多种多样的实现而设计,带有各种实验的指令集扩展。
  我们已经发现,一个有组织的命名策略对软件工具和文档具有简化作用。
% The RISC-V ISA is designed to support a wide variety of
% implementations with various experimental instruction-set extensions.
% We have found that an organized naming scheme simplifies software
% tools and documentation.
\end{commentary}

\section{大小写敏感性}
% \section{Case Sensitivity}

ISA命名字符串是大小写敏感的。
% The ISA naming strings are case insensitive.

\section{基础整数ISA}
% \section{Base Integer ISA}
RISC-V ISA字符串以RV32I、RV32E、RV64I或RV128I开始,表示了对于基础整数ISA所支持的地址空间尺寸的位数。
% RISC-V ISA strings begin with either RV32I, RV32E, RV64I, or RV128I
% indicating the supported address space size in bits for the base
% integer ISA.

\section{指令集扩展的命名}
% \section{Instruction-Set Extension Names}

标准ISA被赋予了由单个字母组成的命名。
例如,最初的四个对于整数基础的标准扩展是:用于整数乘法和除法的“M”,用于原子内存指令的“A”,用于单精度浮点指令的“F”,和用于双精度浮点指令的“D”。
任何RISC-V指令集的变体可以被简洁地描述为,将基础整数前缀与所包含的扩展的命名的结合,例如,“RV64IMAFD”。
% Standard ISA extensions are given a name consisting of a single
% letter.  For example, the first four standard
% extensions to the integer bases are:
% ``M'' for integer multiplication and division,
% ``A'' for atomic memory instructions,
% ``F'' for single-precision floating-point instructions, and
% ``D'' for double-precision floating-point instructions.
% Any RISC-V instruction-set variant can be succinctly described by
% concatenating the base integer prefix with the names of the included
% extensions, e.g., ``RV64IMAFD''.

我们也定义了一个缩写“G”来代表“IMAFDZicsr\_Zifencei”基础和扩展,因为这是为了表示我们的标准通用目的ISA。
% We have also defined an abbreviation ``G'' to represent the ``IMAFDZicsr\_Zifencei''
% base and extensions, as this is intended to represent our standard
% general-purpose ISA.

对RISC-V ISA的标准扩展被赋予了其它保留的字母,例如用于四精度浮点的“Q”,或者用于16位压缩指令格式的“C”。
% Standard extensions to the RISC-V ISA are given other reserved
% letters, e.g., ``Q'' for quad-precision floating-point, or
% ``C'' for the 16-bit compressed instruction format.

有些ISA扩展依赖于其它扩展的存在,例如,“D”扩展依赖于“F”扩展,而“F”扩展依赖于“Zicsr”扩展。
这些依赖可能隐含在ISA命名之中:例如,RV32IF等价于RV32IFZicsr,而RV32ID等价于RV32IFD和RV32IFDZicsr。
% Some ISA extensions depend on the presence of other extensions, e.g., ``D''
% depends on ``F'' and ``F'' depends on ``Zicsr''.  These dependences may be
% implicit in the ISA name: for example, RV32IF is equivalent to RV32IFZicsr,
% and RV32ID is equivalent to RV32IFD and RV32IFDZicsr.

\section{版本号}
% \section{Version Numbers}
认识到指令集可能随着时间而扩展或改变,我们在扩展的名字后面编码了扩展的版本号。
版本号被划分为主版本号和次版本号,使用“p”分割。如果次版本是“0”,那么“p0”可以从版本字符串中被忽略。
主版本号的改变表示了一种向后兼容性的损失,而只改变次版本号必须是向后兼容的。
例如,在这本手册的1.0发布版本中定义的原始的64位标准ISA可以被写全为“RV64I1p0M1p0A1p0F1p0D1p0”,
而“RV64I1M1A1F1D1”是更简洁的写法。
% Recognizing that instruction sets may expand or alter over time, we
% encode extension version numbers following the extension name.  Version
% numbers are divided into major and minor version numbers, separated by
% a ``p''.  If the minor version is ``0'', then ``p0'' can be omitted
% from the version string.  Changes in major version numbers imply a
% loss of backwards compatibility, whereas changes in only the minor
% version number must be backwards-compatible.  For example, the
% original 64-bit standard ISA defined in release 1.0 of this manual can
% be written in full as ``RV64I1p0M1p0A1p0F1p0D1p0'', more concisely as
% ``RV64I1M1A1F1D1''.

我们在第二个发布版本中介绍了版本号策略。因此我们把一个标准扩展的默认版本定义为在那个时间的当前版本,例如,“RV32I”等价于“RV32I2”。
% We introduced the version numbering scheme with the second release.  Hence, we
% define the default version of a standard extension to be the version present at that
% time, e.g., ``RV32I'' is equivalent to ``RV32I2''.

\section{着重说明}
% \section{Underscores}

下划线“\_”可以被用于分割ISA扩展,以增强可读性,和提供歧义消除,例如“RV32I2\_M2\_A2”。
% Underscores ``\_'' may be used to separate ISA extensions to
% improve readability and to provide disambiguation, e.g., ``RV32I2\_M2\_A2''.

因为用于打包SIMD的“P”扩展可能会与版本号中的小数点相混淆,所以如果它跟在一个数字后面,那么在它前面必须加下划线。
例如,“rv32i2p2”意味着RV32I的2.2版本,而“rv32i2\_p2”意味着带有P扩展的2.0版本的RV32I的2.0版本。
% Because the ``P'' extension for Packed SIMD can be confused for the decimal
% point in a version number, it must be preceded by an underscore if it follows
% a number.  For example, ``rv32i2p2'' means version 2.2 of RV32I, whereas
% ``rv32i2\_p2'' means version 2.0 of RV32I with version 2.0 of the P extension.

\section{附加的标准扩展的命名}
% \section{Additional Standard Extension Names}

标准扩展也可以使用一个“Z”、后面跟着一个按字母顺序排列的名字和一个可选的版本号来命名。
例如,“Zifencei”命名了第~\ref{chap:zifencei}章中描述的指令获取屏障扩展;“Zifencei2”和“Zifencei2p0”描述了相同扩展的2.0版本。
% Standard extensions can also be named using a single ``Z'' followed by an
% alphabetical name and an optional version number.  For example,
% ``Zifencei'' names the instruction-fetch fence extension described in
% Chapter~\ref{chap:zifencei}; ``Zifencei2'' and ``Zifencei2p0'' name version
% 2.0 of same.

跟在“Z”后面的第一个字母通常暗示了相关字母顺序最近的扩展种类,IMAFDQCVH。
例如,对于用于未对齐原子的“Zam”扩展,字母“a”表示该扩展与“A”标准扩展相关。
如果命名了多个“Z”扩展,它们应当首先按种类排序,然后在每个种类中按字母顺序排序——例如,“Zicsr\_Zifencei\_Zam”。
% The first letter following the ``Z'' conventionally indicates the most closely
% related alphabetical extension category, IMAFDQCVH.  For the ``Zam''
% extension for misaligned atomics, for example, the letter ``a'' indicates the
% extension is related to the ``A'' standard extension.  If multiple ``Z''
% extensions are named, they should be ordered first by category, then
% alphabetically within a category---for example, ``Zicsr\_Zifencei\_Zam''.

使用“Z”前缀的扩展必须使用一个下划线与其它多字母的扩展分割,例如,“RV32IMACZicsr\_Zifencei”。
% Extensions with the ``Z'' prefix must be separated
% from other multi-letter extensions by an underscore, e.g.,
% ``RV32IMACZicsr\_Zifencei''.

\section{监管器级指令集扩展}
% \section{Supervisor-level Instruction-Set Extensions}

标准监管器级指令集扩展在第二卷中定义,但是在命名时使用“S”作为前缀,后面跟着按字母顺序排列的名称和一个可选的版本号。
监管器级别的扩展必须通过一个下划线与其他多字母的扩展进行分割。
% Standard supervisor-level instruction-set extensions are defined in Volume II,
% but are named using ``S'' as a prefix, followed by an alphabetical name and an
% optional version number.  Supervisor-level extensions must be separated from
% other multi-letter extensions by an underscore.

标准监管器级扩展应当被列在标准非特权扩展之后。如果列出了多个监管器级别的扩展,它们应当按字母顺序排列。
% Standard supervisor-level extensions should be listed after standard
% unprivileged extensions.  If multiple supervisor-level extensions are listed,
% they should be ordered alphabetically.

\section{机器级指令集扩展}
% \section{Machine-level Instruction-Set Extensions}

标准机器级指令集扩展使用三个字母“Zxm”作为前缀。
% Standard machine-level instruction-set extensions are prefixed with the three
% letters ``Zxm''.

标准机器级扩展应当被列在标准更低特权级扩展之后。如果列出了多个机器级扩展,它们应当按字母顺序排列。
% Standard machine-level extensions should be listed after standard
% lesser-privileged extensions.  If multiple machine-level extensions are listed,
% they should be ordered alphabetically.

\section{非标准扩展的命名}
% \section{Non-Standard Extension Names}

非标准扩展使用一个单独的“X”、后面跟着一个按字母顺序排列的名字和一个可选的版本号来命名。
例如,“Xhwacha”命名了Hwacha向量获取ISA扩展;“Xhwacha2”和“Xhwacha2p0”命名了相同扩展的2.0版本。
% Non-standard extensions are named using a single ``X'' followed by an
% alphabetical name and an optional version number.
% For example, ``Xhwacha'' names the Hwacha vector-fetch ISA extension;
% ``Xhwacha2'' and ``Xhwacha2p0'' name version 2.0 of same.

非标准扩展必须被列在所有的标准扩展之后。它们必须通过一条下划线来与其它多字母的扩展分割。
例如,一个带有非标准扩展Argle和Bargle的ISA可以被命名为“RV64IZifencei\_Xargle\_Xbargle”。
% Non-standard extensions must be listed after all standard extensions.
% They must be separated from other multi-letter extensions
% by an underscore.  For example, an ISA with non-standard extensions
% Argle and Bargle may be named ``RV64IZifencei\_Xargle\_Xbargle''.

如果列出了多个非标准扩展,它们应当按照字母顺序排序。
% If multiple non-standard extensions are listed, they should be ordered
% alphabetically.

\section{子集命名约定}
% \section{Subset Naming Convention}
表~\ref{isanametable}总结了标准化的扩展命名。
% Table~\ref{isanametable} summarizes the standardized extension names.
~\\
\begin{table}[h]
\center
\begin{tabular}{|l|c|c|}
\hline
子集 & 命名 & 隐含 \\
\hline
\hline
\multicolumn{3}{|c|}{基础ISA}\\
\hline
整数 & I & \\
约简的整数 & E & \\
\hline
\hline
\multicolumn{3}{|c|}{标准非特权扩展}\\
\hline
整数乘法和除法 & M & \\
原子性 & A & \\
单精度浮点 & F & Zicsr \\
双精度浮点 & D & F \\
\hline
通用 & G & IMAFDZicsr\_Zifencei \\
\hline
四精度浮点 & Q & D\\
16位压缩指令 & C & \\
打包的SIMD扩展 & P & \\
向量扩展 & V & D \\
监视级扩展 & H & \\
控制和状态寄存器访问 & Zicsr & \\
指令获取屏障 & Zifencei & \\
未对齐原子性 & Zam & A \\
全存储排序 & Ztso & \\
\hline
\hline
\multicolumn{3}{|c|}{标准监管器级扩展}\\
\hline
监管器级扩展“def” & Sdef & \\
\hline
\hline
\multicolumn{3}{|c|}{标准机器级扩展}\\
\hline
机器级扩展“jkl” & Zxmjkl & \\
\hline
\hline
\multicolumn{3}{|c|}{非标准扩展}\\
\hline
非标准扩展“mno” & Xmno & \\
\hline
\end{tabular}
\caption{
  % Standard ISA extension names.  The table also defines the
  % canonical order in which extension names must appear in the name
  % string, with top-to-bottom in table indicating first-to-last in the
  % name string, e.g., RV32IMACV is legal, whereas RV32IMAVC is not.
  标准ISA扩展命名。该表也定义了扩展的名字在命名字符串中必须出现在的传统次序,
  表中从上到下的顺序表示命名字符串中从开始到结束的顺序,例如,RV32IMACV是合法的,而RV32IMAVC是不合法的。
  }
\label{isanametable}
\end{table}



\chapter{历史和鸣谢}
% \chapter{History and Acknowledgments}
\label{history}

\section{“为什么要开发一个新的ISA?”伯克利小组的理由}
% \section{``Why Develop a new ISA?'' Rationale from Berkeley Group}

我们开发RISC-V来支持我们自己在研究和教育中的需求,这里,
我们组尤其对研究想法的实际硬件实现(自从这个规范的初次编辑以来,我们已经完成了11个不同的RISC-V的硅制品)、
以及为学生们在课堂中的探索提供真实实现(RISC-V处理器RTL设计已经用在了伯克利的多个本科和研究生课程之中)感兴趣。
在我们目前的研究中,受到传统晶体管规模的终结所引发的能量受限的驱使,我们特别感兴趣于转向专用化和异构化的加速器。
我们希望有一种高度灵活的和可扩展的基础ISA,围绕它来构建我们的研究工作。
% We developed RISC-V to support our own needs in research and
% education, where our group is particularly interested in actual
% hardware implementations of research ideas (we have completed eleven
% different silicon fabrications of RISC-V since the first edition of
% this specification), and in providing real implementations for
% students to explore in classes (RISC-V processor RTL designs have been
% used in multiple undergraduate and graduate classes at Berkeley).  In
% our current research, we are especially interested in the move towards
% specialized and heterogeneous accelerators, driven by the power
% constraints imposed by the end of conventional transistor scaling.  We
% wanted a highly flexible and extensible base ISA around which to build
% our research effort.

我们被重复问到的一个问题是“为什么要开发一个新的ISA?”使用一个现有的商业ISA的最显而易见的好处是有大而广泛的软件生态系统的支持,
无论是开发工具还是移植的应用,都可以被用于研究和教学。其它好处包括存在大量的文档和教程样例。
然而,我们使用商业指令集用于研究和教学的经验是,这些好处在实际中是比较小的,并且比不过其劣势:
% A question we have been repeatedly asked is ``Why develop a new ISA?''
% The biggest obvious benefit of using an existing commercial ISA is the
% large and widely supported software ecosystem, both development tools
% and ported applications, which can be leveraged in research and
% teaching.  Other benefits include the existence of large amounts of
% documentation and tutorial examples.  However, our experience of using
% commercial instruction sets for research and teaching is that these
% benefits are smaller in practice, and do not outweigh the
% disadvantages:

\begin{itemize}
\item {\bf 商业ISA是有专利的。} 除了SPARC V8,它是一个开放的IEEE标准~\cite{sparcieee1994},大多数商业ISA的所有者都小心地保卫着他们的知识产权,
而不欢迎自由地提供有竞争力的实现。这对于只使用软件模拟器的学术研究和教学来说还不是太大的问题,
但是对于那些希望分享确实的RTL实现的团队来说却是主要关注的事情。
对于不愿意相信几乎没有源代码的商业ISA实现的群体来说,这也是个主要的问题——但是他们被禁止创造他们自己的干净实现。
我们不能保证所有的RISC-V实现都将免受第三方专利的侵权,但是我们可以保证我们不会尝试起诉RISC-V的实现者。
% {\bf Commercial ISAs are proprietary.}  Except for SPARC V8,
%   which is an open IEEE standard~\cite{sparcieee1994}, most owners of
%   commercial ISAs carefully guard their intellectual property and do
%   not welcome freely available competitive implementations.  This is
%   much less of an issue for academic research and teaching using only
%   software simulators, but has been a major concern for groups wishing
%   to share actual RTL implementations.  It is also a major concern for
%   entities who do not want to trust the few sources of commercial ISA
%   implementations, but who are prohibited from creating their own
%   clean room implementations.  We cannot guarantee that all RISC-V
%   implementations will be free of third-party patent infringements,
%   but we can guarantee we will not attempt to sue a RISC-V
%   implementor.

\item {\bf 商业ISA只流行于特定的市场领域。} 在编写时,最显而易见的例子是,ARM架构对服务器空间的支持并不好,
而英特尔x86架构(或者同样地,几乎其余的所有架构)都不能很好地支持移动空间——尽管英特尔和ARM都在尝试进入相互的市场段。
另一个例子是ARC和Tensilica,它们提供可扩展的核心,但是聚焦于嵌入式空间。
这种市场分割冲淡了支持特定商业ISA的好处,因为实际上,软件生态系统只针对特定领域而存在,而不得不为了其它领域而构建。
  % {\bf Commercial ISAs are only popular in certain market
  % domains.}  The most obvious examples at time of writing are that
  % the ARM architecture is not well supported in the server space, and
  % the Intel x86 architecture (or for that matter, almost every other
  % architecture) is not well supported in the mobile space, though both
  % Intel and ARM are attempting to enter each other's market segments.
  % Another example is ARC and Tensilica, which provide extensible cores
  % but are focused on the embedded space.  This market segmentation
  % dilutes the benefit of supporting a particular commercial ISA as in
  % practice the software ecosystem only exists for certain domains, and
  % has to be built for others.

\item {\bf 商业ISA来来往往。} 先前的研究基础设施围绕着商业ISA构建,那些商业ISA已经不再流行(SPARC、MIPS)或者甚至不再生产了(Alpha)。
  这些都失去了一个活跃的软件生态系统的好处,而围绕ISA和支持工具的持续不断的知识产权问题一直在阻碍着感兴趣的第三方继续支持ISA的能力。
  一个开放的ISA可能也会失去流行性,但是任何感兴趣的团体都可以继续使用和开发这个生态系统。
  % {\bf Commercial ISAs come and go.}  Previous research
  % infrastructures have been built around commercial ISAs that are no
  % longer popular (SPARC, MIPS) or even no longer in production
  % (Alpha).  These lose the benefit of an active software ecosystem,
  % and the lingering intellectual property issues around the ISA and
  % supporting tools interfere with the ability of interested third
  % parties to continue supporting the ISA.  An open ISA might also lose
  % popularity, but any interested party can continue using and
  % developing the ecosystem.

\item  {\bf 流行的商业ISA是复杂的。} 对于在硬件中的支持常见软件栈和操作系统的级别,居主导的商业ISA(x86和ARM)的实现都是非常复杂的。
更糟的是,几乎所有的复杂度都是由于坏的、或者至少是过时的ISA设计决策,而不是由于真正的提高效率的特性。
% {\bf Popular commercial ISAs are complex.}  The dominant
%   commercial ISAs (x86 and ARM) are both very complex to implement in
%   hardware to the level of supporting common software stacks and
%   operating systems.  Worse, nearly all the complexity is due to bad,
%   or at least outdated, ISA design decisions rather than features that
%   truly improve efficiency.

\item {\bf 只有商业ISA自己是不足以带起应用的。} 即使我们扩展了实现一个商业ISA的工作,这也仍然不足以使ISA运行现有的应用。
大多数应用需要一个完整的ABI(应用程序二进制接口)来运行,而不仅仅是用户级ISA。
大多数ABI依赖于库,而这又反过来依赖于操作系统的支持。
为了运行一个现有的操作系统,需要实现监管器级别的ISA和操作系统期望的设备接口。
与用户级ISA相比,这些通常更加不明确,并且实现起来也要更复杂得多。
% {\bf Commercial ISAs alone are not enough to bring up
%   applications.}  Even if we expend the effort to implement a
%   commercial ISA, this is not enough to run existing applications for
%   that ISA.  Most applications need a complete ABI (application binary
%   interface) to run, not just the user-level ISA.  Most ABIs rely on
%   libraries, which in turn rely on operating system support.  To run an
%   existing operating system requires implementing the supervisor-level
%   ISA and device interfaces expected by the OS.  These are usually
%   much less well-specified and considerably more complex to
%   implement than the user-level ISA.

\item {\bf 流行的商业ISA并非为了扩展而设计。} 居主导的商业ISA不会特地为扩展设计,并且因此,随着它们的指令集的增长,增加了相当多的指令编码复杂度。
  诸如Tensilica(被Cadence收购)和ARC(被Synopsys收购)的公司虽然围绕着可扩展性构建了ISA和工具链,但是更加聚焦于嵌入式应用,
  而不是通用目的的计算系统。
  % {\bf Popular commercial ISAs were not designed for extensibility.}  The
  % dominant commercial ISAs were not particularly designed for
  % extensibility, and as a consequence have added considerable
  % instruction encoding complexity as their instruction sets have
  % grown.  Companies such as Tensilica (acquired by Cadence) and ARC
  % (acquired by Synopsys) have built ISAs and toolchains around
  % extensibility, but have focused on embedded applications rather than
  % general-purpose computing systems.

\item {\bf 修改后的商业ISA是一种新的ISA。} 我们的目标之一是支持架构研究,包括主要ISA扩展在内。
甚至小型扩展也会减少使用标准ISA的收益,因为,为了使用扩展,不得不修改编译器和从源代码重新构建应用。
更大的扩展引入了新的架构状态,也需要对操作系统的修改。最终,修改后的商业ISA变成了一个新的ISA,
但是它仍然保留了基础ISA的所有的遗留的包袱。
% {\bf A modified commercial ISA is a new ISA.} One of our main
%   goals is to support architecture research, including major ISA
%   extensions.  Even small extensions diminish the benefit of using a
%   standard ISA, as compilers have to be modified and applications
%   rebuilt from source code to use the extension.  Larger extensions
%   that introduce new architectural state also require modifications to
%   the operating system.  Ultimately, the modified commercial ISA
%   becomes a new ISA, but carries along all the legacy baggage of the
%   base ISA.
\end{itemize}

我们的立场是,ISA或许是计算系统中最重要的接口,并且这样一个重要的接口应当没有理由成为专利。
居主导的商业ISA是基于30年前就已经广为人知的指令集概念。软件开发者应当能够瞄准一种开放的标准硬件目标,
而商业处理器设计者应当在实现质量上竞争。
% Our position is that the ISA is perhaps the most important interface
% in a computing system, and there is no reason that such an important
% interface should be proprietary.  The dominant commercial ISAs are
% based on instruction-set concepts that were already well known over 30
% years ago.  Software developers should be able to target an open
% standard hardware target, and commercial processor designers should
% compete on implementation quality.

我们远不是第一个考虑适合硬件实现的开发ISA设计的。我们也考虑过其它的开放ISA设计,
其中最接近我们目标的是OpenRISC架构~\cite{openriscarch}。但我们决定不采用OpenRISC ISA,是由于一些技术原因:
% We are far from the first to contemplate an open ISA design suitable
% for hardware implementation.  We also considered other existing open
% ISA designs, of which the closest to our goals was the OpenRISC
% architecture~\cite{openriscarch}.  We decided against adopting the
% OpenRISC ISA for several technical reasons:

\begin{itemize}
\item OpenRISC具有条件代码和分支延迟槽,这使更高性能的实现复杂化。
% OpenRISC has condition codes and branch delay slots, which
%   complicate higher performance implementations.
\item OpenRISC使用固定32位编码和16位立即数,这妨碍了更密集的指令编码,并限制了ISA的后续扩展的空间。
% OpenRISC uses a fixed 32-bit encoding and 16-bit immediates,
%   which precludes a denser instruction encoding and limits space for
%   later expansion of the ISA.
\item OpenRISC不支持2008年修订的IEEE 754浮点标准。
% OpenRISC does not support the 2008 revision to the IEEE 754
%   floating-point standard.
\item 在我们开始时,OpenRISC的64位设计还没有完成。
% The OpenRISC 64-bit design had not been completed when we began.
\end{itemize}

通过从一片空白开始,我们可以设计一个满足我们所有目标的ISA,尽管理所当然地,
这比我们一开始所计划的要采取的工作量要大得多。我们现在已经调查了在构建RISC-V ISA基础设施方面相当多的工作,
包括文档、编译器工具链、操作系统端口、参考ISA模拟器、FPGA实现、有效的ASIC实现、架构测试套件和教学材料。
从这本手册的上一次编辑开始,RISC-V ISA在学术和工业中已经有了相当大的应用,
而我们已经创造了非营利性的RISC-V基金会来保护和推广该标准。RISC-V基金会网站在\url{https://riscv.org},
它包括了关于基金会成员和各种使用RISC-V的开源项目的最新信息。
% By starting from a clean slate, we could design an ISA that met all of
% our goals, though of course, this took far more effort than we had
% planned at the outset.  We have now invested considerable effort in
% building up the RISC-V ISA infrastructure, including documentation,
% compiler tool chains, operating system ports, reference ISA
% simulators, FPGA implementations, efficient ASIC implementations,
% architecture test suites, and teaching materials. Since the last
% edition of this manual, there has been considerable uptake of the
% RISC-V ISA in both academia and industry, and we have created the
% non-profit RISC-V Foundation to protect and promote the standard.  The
% RISC-V Foundation website at \url{https://riscv.org} contains the latest
% information on the Foundation membership and various open-source
% projects using RISC-V.


\section{从ISA手册1.0版的修订历史}
% \section{History from Revision 1.0 of ISA manual}

RISC-V ISA和指令集手册构建在一些较早的项目之上。监管器级机器的某些方面和手册的整体格式可以追溯到开始于1992年UC伯克利和ICSI的T0(Torrent-0)向量微处理器项目。
T0是一个基于MIPS-II ISA的向量处理器,由克尔斯泰·阿桑诺维奇作为主要的架构和RTL设计者,以及布莱恩·金斯伯里和伯特兰·伊里索作为主要的VLSI实现者。
ICSI的大卫·约翰逊是对T0 ISA设计、尤其是监管器模式,以及手册文本的主要贡献者。约翰·豪瑟也提供了关于T0 ISA设计的相当多的反馈。
% The RISC-V ISA and instruction-set manual builds upon several earlier
% projects.  Several aspects of the supervisor-level machine and the
% overall format of the manual date back to the T0 (Torrent-0) vector
% microprocessor project at UC Berkeley and ICSI, begun in 1992.  T0 was
% a vector processor based on the MIPS-II ISA, with Krste Asanovi\'{c}
% as main architect and RTL designer, and Brian Kingsbury and Bertrand
% Irrisou as principal VLSI implementors.  David Johnson at ICSI was a
% major contributor to the T0 ISA design, particularly supervisor mode,
% and to the manual text.  John Hauser also provided considerable
% feedback on the T0 ISA design.

麻省理工学院在2000年开始的Scale(用于低能耗的软件控制架构)项目,在T0项目基础设施上构建,
改良了监管器级的接口;并通过丢弃分支延迟槽,移除了MIPS标量ISA。
罗尼·克拉辛斯基和克里斯托弗·巴顿是麻省理工学院的Scale向量线程处理器的主要架构师,
而马克·汉普顿为Scale移植了基于GCC的编译器基础设施和工具。
% The Scale (Software-Controlled Architecture for Low Energy) project at
% MIT, begun in 2000, built upon the T0 project infrastructure, refined
% the supervisor-level interface, and moved away from the MIPS scalar
% ISA by dropping the branch delay slot.  Ronny Krashinsky and
% Christopher Batten were the principal architects of the Scale
% Vector-Thread processor at MIT, while Mark Hampton ported the
% GCC-based compiler infrastructure and tools for Scale.

在2002年秋季学期,T0 MIPS标量处理器规范的一个轻微编辑的版本(MIPS-6371)被用于教授MIT 6.371的VLSI系统入门课程,
由克里斯·特曼和克尔斯泰·阿桑诺维奇作为讲师。克里斯·特曼为课程(当时没有TAI)贡献了大部分的实验材料。
2005年春季,在麻省理工学院,6.371课程演化为实验性的6.884复杂数字设计课程,由阿文和克尔斯泰·阿桑诺维奇教授,
该课程成为一个常规的春季课程6.375。在6.884/6.375中使用了基于Scale MIPS标量ISA的一个约简版本,
命名为SMIPS。克里斯托弗·巴顿是早期提供了这些的课程的助教,围绕SMIPS ISA开发了相当大量的文档和实验材料。
这个相同的SMIPS实验材料被助教李云燮采纳和强化,用于UC伯克利2009年秋季的CS250 VLSI系统设计课程,
该课程由约翰·沃兹内克、克尔斯泰·阿桑诺维奇和约翰·拉扎罗教授。
% A lightly edited version of the T0 MIPS scalar processor specification
% (MIPS-6371) was used in teaching a new version of the MIT 6.371
% Introduction to VLSI Systems class in the Fall 2002 semester, with
% Chris Terman and Krste Asanovi\'{c} as lecturers.  Chris Terman
% contributed most of the lab material for the class (there was no
% TA!). The 6.371 class evolved into the trial 6.884 Complex Digital
% Design class at MIT, taught by Arvind and Krste Asanovi\'{c} in Spring
% 2005, which became a regular Spring class 6.375.  A reduced version of
% the Scale MIPS-based scalar ISA, named SMIPS, was used in 6.884/6.375.
% Christopher Batten was the TA for the early offerings of these classes
% and developed a considerable amount of documentation and lab material
% based around the SMIPS ISA.  This same SMIPS lab material was adapted
% and enhanced by TA Yunsup Lee for the UC Berkeley Fall 2009 CS250 VLSI
% Systems Design class taught by John Wawrzynek, Krste Asanovi\'{c}, and
% John Lazzaro.

Maven(向量线程引擎的可塑阵列)项目是一种第二代向量线程架构。它由克里斯托弗·巴顿主导设计,
当时他是UC伯克利的一名开始于2007年夏季的交换学者。青木秀田,一名来自日立的客座工业研究员,
给出了关于早期Maven ISA和微架构设计的大量反馈。Maven基础设施是基于Scale基础设施,
但是Maven ISA进一步地移除了Scale中定义的MIPS ISA变体,而使用一个统一的浮点和整数寄存器文件。
设计Maven是为了支持带有备用数据并行加速器的实验。李云燮是各种Maven向量单元的主要实现者,
同时里马斯·阿维齐尼斯是各种Maven标量单元的主要实现者。李云燮和克里斯托弗·巴顿将GCC移植到新的Maven ISA中共同工作。
克里斯托弗·塞利奥提供了Maven的一种传统的向量指令集(“Flood”)变体的原始定义。
% The Maven (Malleable Array of Vector-thread ENgines) project was a
% second-generation vector-thread architecture.  Its design was led by
% Christopher Batten when he was an Exchange Scholar at UC Berkeley starting
% in summer 2007.  Hidetaka Aoki, a visiting industrial fellow from
% Hitachi, gave considerable feedback on the early Maven ISA and
% microarchitecture design.  The Maven infrastructure was based on the
% Scale infrastructure but the Maven ISA moved further away from the
% MIPS ISA variant defined in Scale, with a unified floating-point and
% integer register file.  Maven was designed to support experimentation
% with alternative data-parallel accelerators.  Yunsup Lee was the main
% implementor of the various Maven vector units, while Rimas Avi\v{z}ienis
% was the main implementor of the various Maven scalar units.
% Yunsup Lee and Christopher Batten ported GCC to work with the new
% Maven ISA.  Christopher Celio provided the initial definition of a
% traditional vector instruction set (``Flood'') variant of Maven.

基于所有这些先前项目的经验,在2010年夏天,开始了RISC-V ISA的定义,由安德鲁·沃特曼、李云燮、克尔斯泰·阿桑诺维奇和大卫·帕特森作为主要设计者。
RISC-V 32位指令子集的一个最初版本被用于UC伯克利2010年秋季CS250 VLSI系统设计课程之中,由李云燮作为助教。
RISC-V与较早的MIPS灵感的设计有明显的不同。约翰·豪瑟贡献了浮点ISA的定义,包括符号注入指令和一个寄存器编码策略,
它允许浮点值的内部重新编码。
% Based on experience with all these previous projects, the RISC-V ISA
% definition was begun in Summer 2010, with Andrew Waterman, Yunsup Lee,
% Krste Asanovi\'{c}, and David Patterson as principal designers.
% An initial version of the RISC-V
% 32-bit instruction subset was used in the UC Berkeley Fall 2010 CS250
% VLSI Systems Design class, with Yunsup Lee as TA.  RISC-V is a clean
% break from the earlier MIPS-inspired designs.  John Hauser contributed
% to the floating-point ISA definition, including the sign-injection
% instructions and a register encoding scheme that permits
% internal recoding of floating-point values.

\section{从ISA手册2.0版的修订历史}
% \section{History from Revision 2.0 of ISA manual}

已经完成了RISC-V处理器的多个实现,包括一些硅制品,就像表~\ref{silicon}中显示的那样。
% Multiple implementations of RISC-V processors have been completed,
% including several silicon fabrications, as shown in
% Figure~\ref{silicon}.

\begin{table*}[!h]
\begin{center}
\begin{tabular}{|l|r|l|l|}
\hline
\multicolumn{1}{|c|}{名称} & \multicolumn{1}{|c|}{流片日期} & \multicolumn{1}{|c|}{处理} & \multicolumn{1}{|c|}{ISA} \\ \hline
\hline
Raven-1 & 2011年5月29日 & ST 28nm FDSOI & RV64G1\_Xhwacha1 \\ \hline
EOS14 & 2012年4月1日 & IBM 45nm SOI & RV64G1p1\_Xhwacha2 \\ \hline
EOS16 & 2012年8月17日 & IBM 45nm SOI & RV64G1p1\_Xhwacha2 \\ \hline
Raven-2 & 2012年8月22日 & ST 28nm FDSOI & RV64G1p1\_Xhwacha2 \\ \hline
EOS18 & 2013年2月6日 & IBM 45nm SOI & RV64G1p1\_Xhwacha2 \\ \hline
EOS20 & 2013年7月3日 & IBM 45nm SOI & RV64G1p99\_Xhwacha2 \\ \hline
Raven-3 & 2013年9月26日 & ST 28nm SOI & RV64G1p99\_Xhwacha2 \\ \hline
EOS22 & 2014年3月7日 & IBM 45nm SOI & RV64G1p9999\_Xhwacha3 \\ \hline
\end{tabular}
\end{center}
\vspace{-0.15in}
\caption{已制造的RISC-V测试芯片。}
\label{silicon}
\end{table*}

第一个被制造的RISC-V处理器是用Verilog编写的,并且在2011年作为Raven-1测试芯片,以一种从ST预先生产的\wunits{28}{纳米}FDSOI技术制造。
在克尔斯泰·阿桑诺维奇的建议下,李云燮和安德鲁·沃特曼共同开发和制造了两个核心:
1)一个带有错误检测反转的RV64标量核心,和
2)一个带有附着64位浮点向量单元的RV64核心。第一个微架构被非正式地称之为“TrainWreck”,因为可用来完成设计的时间较短,而且使用了不成熟的设计库。
% The first RISC-V processors to be fabricated were written in Verilog and
% manufactured in a pre-production \wunits{28}{nm} FDSOI technology from
% ST as the Raven-1 testchip in 2011.  Two cores were developed by Yunsup
% Lee and Andrew Waterman, advised by Krste Asanovi\'{c}, and fabricated
% together: 1) an RV64 scalar core with error-detecting flip-flops, and 2)
% an RV64 core with an attached 64-bit floating-point vector unit.  The
% first microarchitecture was informally known as ``TrainWreck'', due to
% the short time available to complete the design with immature design
% libraries.

随后,在克尔斯泰·阿桑诺维奇的建议下,安德鲁·沃特曼、里马斯·阿维齐尼斯和李云燮开发了一个干净的微架构,
用于有序解耦的RV64核,并且,延续了铁路的主题,以乔治·史蒂芬森的成功的蒸汽机车设计“Rocket”为代号。
Rocket是用Chisel编写的,后者是UC伯克利开发的一种新的硬件设计语言。
Rocket使用的IEEE浮点单元由约翰·豪瑟、安德鲁·沃特曼和布莱恩·理查兹开发。在这之后,Rocket被进一步精炼和开发,
并以\wunits{28}{纳米}FDSOI又制造了两次(Raven-2,、Raven-3),并为了一个光电项目以IBM \wunits{45}{纳米}SOI技术制造了五次(EOS14、EOS16、EOS18、EOS20、EOS22)。
让Rocket的设计可以用作一种参数化的RISC-V处理器生成器的工作正在进行中。
% Subsequently, a clean microarchitecture for an in-order decoupled RV64
% core was developed by Andrew Waterman, Rimas Avi\v{z}ienis, and Yunsup
% Lee, advised by Krste Asanovi\'{c}, and, continuing the railway theme,
% was codenamed ``Rocket'' after George Stephenson's successful steam
% locomotive design.  Rocket was written in Chisel, a new hardware
% design language developed at UC Berkeley.  The IEEE floating-point
% units used in Rocket were developed by John Hauser, Andrew
% Waterman, and Brian Richards.
% Rocket has since been refined and developed further, and has been
% fabricated two more times in \wunits{28}{nm} FDSOI (Raven-2, Raven-3),
% and five times in IBM \wunits{45}{nm} SOI technology (EOS14, EOS16,
% EOS18, EOS20, EOS22) for a photonics project.  Work is ongoing to make
% the Rocket design available as a parameterized RISC-V processor
% generator.

EOS14-EOS22芯片包括了Hwacha的早期版本,它是一个64位的IEEE浮点向量单元,在克尔斯泰·阿桑诺维奇的建议下,
由李云燮、安德鲁·沃特曼、海·沃、欧伯特、全阮和史蒂芬·提格开发。EOS16-EOS22芯片包括了带有缓存一致协议的两个核,
该协议是在克尔斯泰·阿桑诺维奇的建议下,由亨利·库克和安德鲁·沃特曼开发的。EOS14硅已经成功地以\wunits{1.25}{GHz}运行了。
EOS16硅遇到了一个来自IBM焊接点库的故障。EOS18和EOS20也已经成功地以\wunits{1.35}{GHz}运行。
% EOS14--EOS22 chips include early versions of Hwacha, a 64-bit IEEE
% floating-point vector unit, developed by Yunsup Lee, Andrew Waterman,
% Huy Vo, Albert Ou, Quan Nguyen, and Stephen Twigg, advised by Krste
% Asanovi\'{c}.  EOS16--EOS22 chips include dual cores with a
% cache-coherence protocol developed by Henry Cook and Andrew Waterman,
% advised by Krste Asanovi\'{c}.  EOS14 silicon has successfully run at
% \wunits{1.25}{GHz}. EOS16 silicon suffered from a bug in the IBM pad
% libraries.  EOS18 and EOS20 have successfully run at \wunits{1.35}{GHz}.

Raven测试芯片的贡献者包括李云燮、安德鲁·沃特曼、里马斯·阿维齐尼斯、布莱恩·齐默、扎瓦·夸克、
鲁齐卡·杰夫蒂、米洛万·布拉戈耶维奇、阿尔贝托·普盖利、斯蒂芬·贝利、本·凯勒、皮凤秋、布莱恩·理查兹、
鲍里沃耶·尼科利和克尔斯泰·阿桑诺维奇。
% Contributors to the Raven testchips include Yunsup Lee, Andrew Waterman,
% Rimas Avi\v{z}ienis, Brian Zimmer, Jaehwa Kwak, Ruzica Jevti\'{c},
% Milovan Blagojevi\'{c}, Alberto Puggelli, Steven Bailey, Ben Keller,
% Pi-Feng Chiu, Brian Richards, Borivoje Nikoli\'{c}, and Krste
% Asanovi\'{c}.

EOS测试芯片的贡献者包括李云燮、里马斯·阿维齐尼斯、安德鲁·沃特曼、亨利·库克、海·沃、李代伟、孙晨、欧伯特、全阮、
史蒂芬·提格、弗拉基米尔·斯托亚诺维奇和克尔斯泰·阿桑诺维奇。
% Contributors to the EOS testchips include Yunsup Lee, Rimas
% Avi\v{z}ienis, Andrew Waterman, Henry Cook, Huy Vo, Daiwei Li, Chen Sun,
% Albert Ou, Quan Nguyen, Stephen Twigg, Vladimir Stojanovi\'{c}, and
% Krste Asanovi\'{c}.

安德鲁·沃特曼和李云燮开发了C++ ISA模拟器“Spike”,用作开发中的一个黄金模型,
并且以用于庆祝美国横贯大陆铁路的完成的黄金钉来命名。Spike已经可以作为一个BSD开源项目而获得了。
% Andrew Waterman and Yunsup Lee developed the C++ ISA simulator
% ``Spike'', used as a golden model in development and named after the
% golden spike used to celebrate completion of the US transcontinental
% railway.  Spike has been made available as a BSD open-source project.

安德鲁·沃特曼完成了一篇RISC-V压缩指令集的初步设计的硕士论文~\cite{waterman-ms}。
% Andrew Waterman completed a Master's thesis with a preliminary design
% of the RISC-V compressed instruction set~\cite{waterman-ms}. 

已经完成了RISC-V的各种FPGA的实现,主要作为Par实验室项目研究撤离的集成演示的一部分。
最大的FPGA设计是运行一个研究操作系统的三个缓存一致RV64IMA处理器。FPGA实现的贡献者包括安德鲁·沃特曼、李云燮、里马斯·阿维齐尼斯和克尔斯泰·阿桑诺维奇。
% Various FPGA implementations of the RISC-V have been completed,
% primarily as part of integrated demos for the Par Lab project research
% retreats.  The largest FPGA design has 3 cache-coherent RV64IMA
% processors running a research operating system.  Contributors to the
% FPGA implementations include Andrew Waterman, Yunsup Lee, Rimas
% Avi\v{z}ienis, and Krste Asanovi\'{c}.

RISC-V处理器已经用在了UC伯克利的一些课程之中。Rocket被用在2011年秋季推出的CS250中,作为班级项目的基础,布莱恩·齐默担任助教。
对于2012年春季的本科CS152课程,克里斯托弗·塞利奥使用Chisel来写了一个教育用RV32处理器的套件,以“坦克引擎Thomas”和朋友们居住的岛屿命名为“Sodor”。
该套件包括了一个微代码核,一个无管道核,和2级、3级与5级流水线核,并在一个BSD许可证下公开可用。
该套件后续再次被更新和使用是在2013年春季的CS152中,李云燮担任助教,以及2014年春季,由埃里克·洛夫担任助教。
克里斯托弗·塞利奥也开发了一个乱序的RV64设计,称之为BOOM(伯克利乱序机器),伴有流水线可视化,它被用在CS152课程之中。
CS152课程也使用了由安德鲁·沃特曼和亨利·库克开发的Rocket核的缓存一致版本。
% RISC-V processors have been used in several classes at UC Berkeley.
% Rocket was used in the Fall 2011 offering of CS250 as a basis for class
% projects, with Brian Zimmer as TA.  For the undergraduate CS152 class in
% Spring 2012, Christopher Celio used Chisel to write a suite of educational
% RV32 processors, named ``Sodor'' after the island on which ``Thomas the
% Tank Engine'' and friends live.  The suite includes a microcoded core,
% an unpipelined core, and 2, 3, and 5-stage pipelined cores, and is
% publicly available under a BSD license.  The suite was subsequently
% updated and used again in CS152 in Spring 2013, with Yunsup Lee as TA,
% and in Spring 2014, with Eric Love as TA.
% Christopher Celio also developed an out-of-order RV64 design known as BOOM
% (Berkeley Out-of-Order Machine), with accompanying pipeline
% visualizations, that was used in the CS152 classes.  The CS152 classes
% also used cache-coherent versions of the Rocket core developed by Andrew
% Waterman and Henry Cook.

整个2013年的夏天,RoCC(Rocket定制协处理器)接口被定义为简化了定制加速器向Rocket核的添加。
Rocket和RoCC接口在乔纳森·巴赫拉赫教授的2013年秋季CS250 VLSI课程中得到了广泛的使用,
为RoCC接口构建了一些学生加速器项目。Hwacha向量单元已经被重写为一个RoCC协处理器。
% Over the summer of 2013, the RoCC (Rocket Custom Coprocessor)
% interface was defined to simplify adding custom accelerators to the
% Rocket core.  Rocket and the RoCC interface were used extensively in
% the Fall 2013 CS250 VLSI class taught by Jonathan Bachrach, with
% several student accelerator projects built to the RoCC interface.  The
% Hwacha vector unit has been rewritten as a RoCC coprocessor.

在2013年春天,两个伯克利的本科生,全阮和欧伯特,已经成功地将Linux移植到RISC-V上运行。
% Two Berkeley undergraduates, Quan Nguyen and Albert Ou, have
% successfully ported Linux to run on RISC-V in Spring 2013.

在2014年1月,科林·施密特成功地为RISC-V 2.0完成了一个LLVM后端。
% Colin Schmidt successfully completed an LLVM backend for RISC-V 2.0 in
% January 2014.

在2014年3月,大流士·拉德在Bluespec为GCC的移植贡献了软浮点ABI支持。
% Darius Rad at Bluespec contributed soft-float ABI support to the GCC port in
% March 2014.

约翰·豪瑟贡献了浮点分类指令的定义
% John Hauser contributed the definition of the floating-point classification
% instructions.

我们还了解了一些其它的RISC-V核的实现,包括一个由汤米·索恩用Verilog的实现,和一个由里希尔·尼希尔用Bluespec的实现。
% We are aware of several other RISC-V core implementations, including
% one in Verilog by Tommy Thorn, and one in Bluespec by Rishiyur Nikhil.

\section*{鸣谢}
% \section*{Acknowledgments}

感谢克里斯托弗·F·巴顿、普雷斯顿·布里格斯、克里斯托弗·塞利奥、大卫·奇斯纳尔、斯特凡·弗洛伊德伯格、约翰·豪瑟、本·凯勒、
里希尔·尼希尔、迈克尔·泰勒、汤米·索恩和罗伯特·沃森关于规范2.0版本草案ISA的评论。
% Thanks to Christopher F. Batten, Preston Briggs, Christopher Celio, David
% Chisnall, Stefan Freudenberger, John Hauser, Ben Keller, Rishiyur
% Nikhil, Michael Taylor, Tommy Thorn, and Robert Watson for comments on
% the draft ISA version 2.0 specification.

\section{从2.1版的修订历史}
% \section{History from Revision 2.1}

从引入2014年5月冻结的2.0版本依赖,RISC-V ISA的应用已经非常迅速,在这样一个简短的历史小节中有太多的活动要记录。
或许最重要的单一事件就是,在2015年8月,非盈利RISC-V基金会的成立。基金会现在将接管官方RISC-V ISA标准的管理工作,
而官方网站{\tt riscv.org}是获得关于RISC-V标准的新闻和更新的最佳场所。
% Uptake of the RISC-V ISA has been very rapid since the introduction of
% the frozen version 2.0 in May 2014, with too much activity to record
% in a short history section such as this.  Perhaps the most important
% single event was the formation of the non-profit RISC-V Foundation in
% August 2015. The Foundation will now take over stewardship of the
% official RISC-V ISA standard, and the official website {\tt riscv.org}
% is the best place to obtain news and updates on the RISC-V standard.

\section*{鸣谢}
% \section*{Acknowledgments}

感谢斯科特·比默、艾伦·J·鲍姆、克里斯托弗·塞利奥、大卫·奇斯纳尔、保罗·克莱顿、帕默·达贝尔特、
简·格雷、迈克尔·汉伯格和约翰·豪瑟对2.0版本规范的评论。
% Thanks to Scott Beamer, Allen J. Baum, Christopher Celio, David Chisnall,
% Paul Clayton, Palmer Dabbelt, Jan Gray, Michael Hamburg, and John
% Hauser for comments on the version 2.0 specification.

\section{从2.2版的修订历史}
% \section{History from Revision 2.2}

\section*{鸣谢}
% \section*{Acknowledgments}

感谢雅各布·巴赫迈耶、亚历克斯·布拉德伯里、戴维·霍纳、斯特凡·奥雷尔和约瑟夫·迈尔斯对2.1版本规范的评论。
% Thanks to Jacob Bachmeyer, Alex Bradbury, David Horner, Stefan O'Rear,
% and Joseph Myers for comments on the version 2.1 specification.

\section{2.3版的修订历史}
% \section{History for Revision 2.3}

RISC-V的应用持续以惊人的速度发展着。
% Uptake of RISC-V continues at breakneck pace.

基于保罗·邦齐尼的一个提议,约翰·豪瑟和安德鲁·沃特曼贡献了一个超管级ISA扩展。
% John Hauser and Andrew Waterman contributed a hypervisor ISA extension
% based upon a proposal from Paolo Bonzini.

丹尼尔·卢斯蒂格、阿文、克尔斯泰·阿桑诺维奇、谢克德·弗勒、保罗·洛文斯坦、雅廷·曼尔卡、卢克·马兰杰、玛格丽特·马托诺西、
维贾亚南德·纳加拉扬、里希尔·尼希尔、乔纳斯·奥伯豪斯、克里斯托弗·普尔特、何塞·雷诺、彼得·苏厄尔、萨米特·萨卡尔、
卡罗琳·特里普、穆拉里达兰·维贾亚拉加万、安德鲁·沃特曼、德里克·威廉姆斯、安德鲁·赖特和张思卓贡献了内存一致模型。
% Daniel Lustig, Arvind, Krste Asanovi\'{c}, Shaked Flur, Paul Loewenstein, Yatin
% Manerkar, Luc Maranget, Margaret Martonosi, Vijayanand Nagarajan, Rishiyur
% Nikhil, Jonas Oberhauser, Christopher Pulte, Jose Renau, Peter Sewell, Susmit
% Sarkar, Caroline Trippel, Muralidaran Vijayaraghavan, Andrew Waterman, Derek
% Williams, Andrew Wright, and Sizhuo Zhang contributed the memory consistency
% model.

\section{赞助}
% \section{Funding}

部分RISC-V架构和实现的开发由如下赞助者所赞助:
% Development of the RISC-V architecture and implementations has been
% partially funded by the following sponsors.
\begin{itemize}

\item {\bf Par实验室:}研究由微软(Award \#024263)和英特尔(Award \#024894)赞助,
并由U.C.Discovery(Award \#DIG07-10227)提供匹配资助。额外的支持来自于Par 实验室附属的诺基亚、英伟达、甲骨文和三星。
% {\bf Par Lab:} Research supported by Microsoft (Award \#024263) and Intel (Award
%     \#024894) funding and by matching funding by U.C. Discovery
%     (Award \#DIG07-10227). Additional support came from Par Lab
%     affiliates Nokia, NVIDIA, Oracle, and Samsung.

\item {\bf 项目Isis}:DoE Award DE-SC0003624。 
% \{\bf Project Isis:} DoE Award DE-SC0003624.

\item {\bf ASPIRE实验室}:DARPA PERFECT工程,Award HR0011-12-2-0016。DARPA POEM工程Award HR0011-11-C-0100。
未来架构研究中心(C-FAR),一个由半导体研究公司资助的STARnet中心。
额外的支持来自于ASPIRE工业赞助者,英特尔,和ASPIRE附属,谷歌,惠普企业,华为,诺基亚,英伟达,甲骨文,和三星。
% {\bf ASPIRE Lab}: DARPA PERFECT program, Award
%     HR0011-12-2-0016.  DARPA POEM program Award HR0011-11-C-0100.  The
%     Center for Future Architectures Research (C-FAR), a STARnet center
%     funded by the Semiconductor Research Corporation.  Additional
%     support from ASPIRE industrial sponsor, Intel, and ASPIRE
%     affiliates, Google, Hewlett Packard Enterprise, Huawei, Nokia,
%     NVIDIA, Oracle, and Samsung.

\end{itemize}

本文的内容并不能必然地反映出美国政府的立场和政策,并且不应被推断出官方的认可。
% The content of this paper does not necessarily reflect the position or the
% policy of the US government and no official endorsement should be
% inferred. 


\appendix
%%%%Nice fonts in diagrams
%%Images
\makeatletter
% for the fig2dev version on P desktop
\gdef\SetFigFont#1#2#3#4#5{%
  \reset@font\fontsize{12}{#2pt}%
%  \fontfamily{#3}\fontseries{#4}\fontshape{#5}%
  \fontfamily{\sfdefault}\fontseries{#4}\fontshape{#5}%
  \selectfont}
\makeatother

\chapter{RVWMO 说明材料(0.1版本)}
\label{sec:memorymodelexplanation}
这节使用了更加非正式的语言和具体的例子,提供了更多关于RVWMO(第~\ref{ch:memorymodel}章)的解释。
这些解释都是为了澄清该公理和保留的程序次序规则的含义和目的。
这个附录应当被视为评注;而所有的规范性材料都在第~\ref{ch:memorymodel}章和ISA规范的主体的其余部分中提供。
当前的所有已知的差异性都被列在了第~\ref{sec:memory:discrepancies}节。任何其它的差异性都是无意的。
% This section provides more explanation for RVWMO (Chapter~\ref{ch:memorymodel}), using more informal language and concrete examples.
% These are intended to clarify the meaning and intent of the axioms and preserved program order rules.
% This appendix should be treated as commentary; all normative material is provided in Chapter~\ref{ch:memorymodel} and in the rest of the main body of the ISA specification.
% All currently known discrepancies are listed in Section~\ref{sec:memory:discrepancies}.
% Any other discrepancies are unintentional.

\section{为什么用 RVWMO?}
\label{sec:whyrvwmo}

内存一致性模型遵循着从弱到强的松散谱系。
弱内存模型允许更多的硬件实现的灵活性,并提供理论上比强模型更好的性能、每瓦特的性能、能量、可扩展性,和硬件验证开销,但代价是更复杂的编程模型。
强模型提供了更简单的编程模型,但是对于可以在流水线和内存系统中执行的各种(非推测性的)硬件优化,要强加更多的约束开销,并且反过来在能量、区域开销和验证负担方面强加一些成本。
% Memory consistency models fall along a loose spectrum from weak to strong.
% Weak memory models allow more hardware implementation flexibility and deliver arguably better performance, performance per watt, power, scalability, and hardware verification overheads than strong models, at the expense of a more complex programming model.
% Strong models provide simpler programming models, but at the cost of imposing more restrictions on the kinds of (non-speculative) hardware optimizations that can be performed in the pipeline and in the memory system, and in turn imposing some cost in terms of power, area overhead, and verification burden.

RISC-V选择了RVWMO内存模型,它是释放一致性的一个变体。这将它置于了内存模型谱系的两个极端之间。
RVWMO内存模型使架构师能够构建简单的实现、激进的实现,将实现深深地嵌入到一个更大的系统之中,并服务于复杂的内存系统交互,或者任何其它的可能性,所有这些同时又能够以足够强大的高性能支持编程语言内存模型。
% RISC-V has chosen the RVWMO memory model, a variant of release consistency.
% This places it in between the two extremes of the memory model spectrum.
% The RVWMO memory model enables architects to build simple implementations, aggressive implementations, implementations embedded deeply inside a much larger system and 
% subject to complex memory system interactions, or any number of other possibilities, all while simultaneously being strong enough to support programming language memory models at high performance.

为了促进来自其它架构的代码的移植,一些硬件实现可以选择实现Ztso扩展,它默认提供了更严格的RVTSO次序的语义。
为RVWMO编写的代码是与RVTSO自动且固有地兼容的,但是假定RVTSO写的代码不保证在RVWMO实现上能够正确地运行。
事实上,大多数RVWMO实现都将(并且应当)简单地拒绝运行RVTSO专用的二进制文件。
每个实现必须因此进行选择,或者优先兼容RVTSO代码(例如,为了便于来自x86的移植),或者反之优先兼容其他实现了RVWMO的RISC-V核。
% To facilitate the porting of code from other architectures, some hardware implementations may choose to implement the Ztso extension, which provides stricter RVTSO ordering semantics by default.
% Code written for RVWMO is automatically and inherently compatible with RVTSO, but code written assuming RVTSO is not guaranteed to run correctly on RVWMO implementations.
% In fact, most RVWMO implementations will (and should) simply refuse to run RVTSO-only binaries.
% Each implementation must therefore choose whether to prioritize compatibility with RVTSO code (e.g., to facilitate porting from x86) or whether to instead prioritize compatibility with other RISC-V cores implementing RVWMO.

在RVTSO下,代码中为RVWMO所写的一些屏障和/或内存次序注释可能变得冗余;
在Ztso实现上默认采用RVWMO的代价是获取那些在实现上已经变成no-op的屏障(例如:FENCE~R,RW 和 FENCE~RW,W)的增量开销。
然而,如果希望兼容非Ztso的实现,这些屏障在代码中仍然必须存在。
% Some fences and/or memory ordering annotations in code written for RVWMO may become redundant under RVTSO; 
% the cost that the default of RVWMO imposes on Ztso implementations is the incremental overhead of fetching those fences (e.g., FENCE~R,RW and FENCE~RW,W) which become no-ops on that implementation.
% However, these fences must remain present in the code if compatibility with non-Ztso implementations is desired.

\section{Litmus 测试}\label{sec:litmustests}
这章的解释使用了{\em litmus测试},或者说,为测试或突出显示内存模型的一个特定部分而设计的小型程序。
图~\ref{fig:litmus:sample}显示了带有两个硬件线程的litmus测试的一个例子。
作为对这个图和对本章中之后所有图的约定,我们假定{\tt s0}-{\tt s2}在所有硬件线程中都被预先设置为相同的值,并且{\tt s0}持有由{\tt x}标签的地址,{\tt s1}持有{\tt y}的,而{\tt s2}持有{\tt z}的,这里{\tt x}、{\tt y}和{\tt z}是对齐到8字节边界的不相交的内存位置。
每张图在左侧显示了litmus测试的代码,在右侧则是一个特定的有效或无效执行的可视化。
% The explanations in this chapter make use of {\em litmus tests}, or small programs designed to test or highlight one particular aspect of a memory model.
% Figure~\ref{fig:litmus:sample} shows an example of a litmus test with two harts.
% As a convention for this figure and for all figures that follow in this chapter, we assume that {\tt s0}--{\tt s2} are pre-set to the same value in all harts and that {\tt s0} holds the address labeled {\tt x}, {\tt s1} holds {\tt y}, and {\tt s2} holds {\tt z}, where {\tt x}, {\tt y}, and {\tt z} are disjoint memory locations aligned to 8 byte boundaries.
% Each figure shows the litmus test code on the left, and a visualization of one particular valid or invalid execution on the right.

\begin{figure}[h!]
  \centering
    \begin{tabular}{m{.4\linewidth}m{.05\linewidth}m{.4\linewidth}}
    \tt\small
    \begin{tabular}{cl||cl}
    \multicolumn{2}{c}{Hart 0} & \multicolumn{2}{c}{Hart 1} \\
    \hline
          & $\vdots$    &     & $\vdots$    \\
          & li t1,1     &     & li t4,4     \\
      (a) & sw t1,0(s0) & (e) & sw t4,0(s0) \\
          & $\vdots$    &     & $\vdots$    \\
          & li t2,2     &     &             \\
      (b) & sw t2,0(s0) &     &             \\
          & $\vdots$    &     & $\vdots$    \\
      (c) & lw a0,0(s0) &     &             \\
          & $\vdots$    &     & $\vdots$    \\
          & li t3,3     &     & li t5,5     \\
      (d) & sw t3,0(s0) & (f) & sw t5,0(s0) \\
          & $\vdots$    &     & $\vdots$    \\
    \end{tabular}
    & &
    \input{figs/litmus_sample.pdf_t}
\end{tabular}
    \caption{一个litmus测试的示例和一个被禁止的执行({\tt a0=1})。}
  \label{fig:litmus:sample}
\end{figure}

Litmus测试被用于理解特定具体情境中的内存模型的含义。
例如,在图~\ref{fig:litmus:sample}的litmus测试中,根据运行时来自各个硬件线程的指令流的动态交错情况,第一个硬件线程中的{\tt a0}的最终的值可以是2、4或5。
然而,在这个例子中,硬件线程0中的{\tt a0}的最终的值将永远都不会是1或3;按直觉,值1在加载执行时将不再可见,而值3在加载执行时尚未成为可见的。
我们下面来分析这个测试和一些其它的测试。
% Litmus tests are used to understand the implications of the memory model in specific concrete situations.
% For example, in the litmus test of Figure~\ref{fig:litmus:sample}, the final value of {\tt a0} in the first hart can be either 2, 4, or 5, depending on the dynamic interleaving of the instruction stream from each hart at runtime.
% However, in this example, the final value of {\tt a0} in Hart 0 will never be 1 or 3; intuitively, the value 1 will no longer be visible at the time the load executes, and the value 3 will not yet be visible by the time the load executes.
% We analyze this test and many others below.

\begin{table}[h]
  \centering\small
  \begin{tabular}{|c|l|}
    \hline
    边       & 全名 (和解释) \\
    \hline
    \sf rf   & 读从(Reads From) (从各存储到返回该存储写入值的加载) \\
    \hline
    \sf co   & 一致性(Coherence) (关于存储到各地址的一个总的次序) \\
    \hline
    \sf fr   & 从读(From-Reads) (从各加载到读取加载所返回值的存储的共同后继) \\
    \hline
    \sf ppo  & 保留程序次序(Preserved Program Order) \\
    \hline
    \sf fence & 通过一个FENCE指令强行采取的排序 \\
    \hline
    \sf addr & 地址依赖(Address Dependency) \\
    \hline
    \sf ctrl & 控制依赖(Control Dependency) \\
    \hline
    \sf data & 数据依赖(Data Dependency) \\
    \hline
  \end{tabular}
  \caption{在这个附录中绘制的litmus测试图表的要点}
  \label{tab:litmus:key}
\end{table}

每个litmus测试的右侧显示的图展示了正在被考虑的特定执行候选的一个可视化的表示。
这些图标使用在内存模型文献中常见的符号,来限制可能的全局内存次序(可能在执行中产生问题)的集合。
它也是附录~\ref{sec:herd}中展示的\textsf{herd}模型的基础。
表~\ref{tab:litmus:key}中解释了该符号。
在列出的关系中,在硬件线程之间、{\sf co}边、{\sf fr}边和{\sf ppo}边之间的{\sf rf}边直接限制了全局内存次序(正如通过{\sf ppo}也可以限制{\sf fence}、{\sf addr}、{\sf data},和一些{\sf ctrl})。
其它边(例如infa-hart {\sf rf}边)是信息性的,但是不会限制全局内存次序。
% The diagram shown to the right of each litmus test shows a visual representation of the particular execution candidate being considered.
% These diagrams use a notation that is common in the memory model literature for constraining the set of possible global memory orders that could produce the execution in question.
% It is also the basis for the \textsf{herd} models presented in Appendix~\ref{sec:herd}.
% This notation is explained in Table~\ref{tab:litmus:key}.
% Of the listed relations, {\sf rf} edges between harts, {\sf co} edges, {\sf fr} edges, and {\sf ppo} edges directly constrain the global memory order (as do {\sf fence}, {\sf addr}, {\sf data}, and some {\sf ctrl} edges, via {\sf ppo}).
% Other edges (such as intra-hart {\sf rf} edges) are informative but do not constrain the global memory order.

例如,在表~\ref{fig:litmus:sample}中,{\tt a0=1}只可能发生在(c)读取由(a)写入的值、且下列情形之一为真的时候:
% For example, in Figure~\ref{fig:litmus:sample}, {\tt a0=1} could occur only if (c) reads the value written by (a) and one of the following were true:
\begin{itemize}
  \item 在全局内存次序中(以及在一致性次序{\sf co}中),(b)出现在(a)之前。然而这违反了RVWMO PPO规则~\ref{ppo:->st}。从(b)到(a)的{\sf co}边突出了这一矛盾。
  % (b) appears before (a) in global memory order (and in the coherence order {\sf co}).  However, this violates RVWMO PPO rule~\ref{ppo:->st}.  The {\sf co} edge from (b) to (a) highlights this contradiction.
  \item 在全局内存次序中(以及在一致性次序{\sf co}中),(a)出现在(b)之前。然而,在这种情况中,加载值公理将被违反,因为在程序次序中,(a)不是先于(c)的最近匹配的存储。从(c)到(b)的{\sf fr}边突出了这一矛盾。
  % (a) appears before (b) in global memory order (and in the coherence order {\sf co}).  However, in this case, the Load Value Axiom would be violated, because (a) is not the latest matching store prior to (c) in program order.  The {\sf fr} edge from (c) to (b) highlights this contradiction.
\end{itemize}
由于这些场景都不满足RVWMO公理,结果{\tt a0=1}就被禁止了。
% Since neither of these scenarios satisfies the RVWMO axioms, the outcome {\tt a0=1} is forbidden.

除了在这个附录中描述的内容,在\url{https://github.com/litmus-tests/litmus-tests-riscv}中还提供了一套超过七千个的石蕊测试。
% Beyond what is described in this appendix, a suite of more than seven thousand litmus tests is available at \url{https://github.com/litmus-tests/litmus-tests-riscv}.

\begin{commentary}
  litmus测试项目也提供了关于如何在RISC-V硬件上运行litmus测试,和如何将结果与操作和公理模型进行比较的指令。
  % The litmus tests repository also provides instructions on how to run
  % the litmus tests on RISC-V hardware and how to compare the results
  % with the operational and axiomatic models.
\end{commentary}

\begin{commentary}
  在未来,我们期望把这些关于内存模型的litmus测试也改编作为RISC-V一致性测试套件的一部分而使用。
  % In the future, we expect to adapt these memory model litmus tests for use as part of the RISC-V compliance test suite as well.
\end{commentary}

\section{RVWMO规则的解释}
在这节中,我们提供了对所有RVWMO规则和公理的解释和例子。
% In this section, we provide explanation and examples for all of the RVWMO rules and axioms.

\subsection{保留程序次序和全局内存次序}
保留程序次序代表了必须在全局内存次序中被遵循的程序次序的子集。
概念上,从其它硬件线程和/或观察者的角度,来自相同硬件线程的、按照保留程序次序被排序的事件,必须以该次序出现。
% Preserved program order represents the subset of program order that must be respected within the global memory order.
% Conceptually, events from the same hart that are ordered by preserved program order must appear in that order from the perspective of other harts and/or observers.
% Events from the same hart that are not ordered by preserved program order, on the other hand, may appear reordered from the perspective of other harts and/or observers.

% 换句话说,来自相同硬件线程的、没有按保留程序次序排序的事件,从其它硬件线程和/或观察者的角度,可以以新的次序出现。

非正式地讲,全局内存次序代表了加载和存储所执行的次序。
正式的内存模型文献已经从围绕执行概念构建的规范中移出,但是该思想对于建立非正式的直觉仍然是有用的。
对于加载,当它的返回值被确定时,它被称为已执行的。对于存储,不是当它在流水线内部被执行时、而是只有当它的值已经被传播到全局可见的存储时,它才被称为已执行的。
在这个意义上,全局内存次序也代表了一致性协议和/或余下的内存系统的贡献:把每个硬件线程发出的(可能被重新排序的)内存访问交错到所有硬件线程都赞成的单一的总次序之中。
% Informally, the global memory order represents the order in which loads and stores perform.
% The formal memory model literature has moved away from specifications built around the concept of performing, but the idea is still useful for building up informal intuition.
% A load is said to have performed when its return value is determined.
% A store is said to have performed not when it has executed inside the pipeline, but rather only when its value has been propagated to globally visible memory.
% In this sense, the global memory order also represents the contribution of the coherence protocol and/or the rest of the memory system to interleave the (possibly reordered) memory accesses being issued by each hart into a single total order agreed upon by all harts.

加载执行的次序并不总是直接对应于那两个加载所返回的值的相对生存时间。
特别地,对相同的地址,一个加载$b$可以在另一个加载$a$之前执行(例如,$b$可以在$a$之前执行,并且在全局内存次序中,$b$可以出现在$a$之前),但是尽管如此,$a$可以返回一个比$b$更早旧的值。
这种差异性(在其它事情之中)捕获了核心与内存之间安置的缓冲的重新排序效果。
例如,$b$可能已经返回了$a$存储在存储缓冲区中的一个值,同时$a$可能已经忽略了较新的存储,反而从内存中读取了一个较旧的值。
为了解释这个情况,在每次加载执行的时候,它返回的值由加载值公理决定,而不只是通过确定在全局内存次序中最近对相同地址的存储来严格地决定,正如下面描述的那样。
% The order in which loads perform does not always directly correspond to the relative age of the values those two loads return.
% In particular, a load $b$ may perform before another load $a$ to the same address (i.e., $b$ may execute before $a$, and $b$ may appear before $a$ in the global memory order), but $a$ may nevertheless return an older value than $b$.
% This discrepancy captures (among other things) the reordering effects of buffering placed between the core and memory.
% For example, $b$ may have returned a value from a store in the store buffer, while $a$ may have ignored that younger store and read an older value from memory instead.
% To account for this, at the time each load performs, the value it returns is determined by the load value axiom, not just strictly by determining the most recent store to the same address in the global memory order, as described below.

\subsection{\nameref*{rvwmo:ax:load}}
\label{sec:memory:loadvalueaxiom}
\begin{tabular}{p{1cm}|p{12cm}} &
\nameref{rvwmo:ax:load}: \loadvalueaxiom
\end{tabular}

保留程序次序不需要遵循“在重叠的地址上,一个存储跟随着一个加载”的次序。
这种复杂度的提升是因为,在几乎所有实现中存储缓冲区都是随处可见的。
非正式地说,当存储仍然在存储缓冲区中的时候,加载可以通过从存储转发来执行(返回一个值),并因此出现在了存储自身的执行(写回到全局可见内存)之前。
因此,任何其它的硬件线程将观察到,加载在存储之前执行。
% Preserved program order is {\em not} required to respect the ordering of a store followed by a load to an overlapping address.
% This complexity arises due to the ubiquity of store buffers in nearly all implementations.
% Informally, the load may perform (return a value) by forwarding from the store while the store is still in the store buffer, and hence before the store itself performs (writes back to globally visible memory).
% Any other hart will therefore observe the load as performing before the store.

\begin{figure}[h!]
  \centering
  \begin{tabular}{m{.4\linewidth}@{\qquad}m{.45\linewidth}}
  {
    \tt\small
    \begin{tabular}{cl||cl}
    \multicolumn{2}{c}{Hart 0} & \multicolumn{2}{c}{Hart 1} \\
    \hline
          & li t1, 1    &     & li t1, 1    \\
      (a) & sw t1,0(s0) & (e) & sw t1,0(s1) \\
      (b) & lw a0,0(s0) & (f) & lw a2,0(s1) \\
      (c) & fence r,r   & (g) & fence r,r   \\
      (d) & lw a1,0(s1) & (h) & lw a3,0(s0) \\
      \hline
      \multicolumn{4}{c}{输出结果 {\tt a0=1}, {\tt a1=0}, {\tt a2=1}, {\tt a3=0}}
    \end{tabular}
  }
  &
  \input{figs/litmus_sb_fwd.pdf_t}
  \end{tabular}
  \caption{一个存储缓冲区转发石蕊测试(允许的输出结果)
    % A store buffer forwarding litmus test (outcome permitted)
    }
  \label{fig:litmus:storebuffer}
\end{figure}

考虑表~\ref{fig:litmus:storebuffer}的litmus测试。
当在一个带有存储缓冲区(store buffers)的实现上运行这个程序时,它可能得到{\tt a0=1}, {\tt a1=0}, {\tt a2=1}, {\tt a3=0}的最终输出结果,如下:
% Consider the litmus test of Figure~\ref{fig:litmus:storebuffer}.
% When running this program on an implementation with store buffers, it is possible to arrive at the final outcome
% {\tt a0=1}, {\tt a1=0}, {\tt a2=1}, {\tt a3=0}
% as follows:
\begin{itemize}
  \item (a) 执行并进入第一个硬件线程的私有存储缓冲区(store buffer)  % executes and enters the first hart's private store buffer
  \item (b) 执行并从存储缓冲区中的(a)转发它的返回值1  % executes and forwards its return value 1 from (a) in the store buffer
  \item (c) 当所有之前的加载操作(例如,(b))都已经完成时执行  % executes since all previous loads (i.e., (b)) have completed
  \item (d) 执行并从内存读取值0  % executes and reads the value 0 from memory
  \item (e) 执行并进入第二个硬件线程的私有存储缓冲区  % executes and enters the second hart's private store buffer
  \item (f) 执行并从存储缓冲区中的(e)转发它的值1  % executes and forwards its return value 1 from (e) in the store buffer
  \item (g) 从所有之前的加载操作(例如,(f))都已经完成时执行  % executes since all previous loads (i.e., (f)) have completed
  \item (h) 执行并从内存读取值0  % executes and reads the value 0 from memory
  \item (a) 从第一个硬件线程的存储缓冲区排放到内存  % drains from the first hart's store buffer to memory
  \item (e) 从第二个硬件线程的存储缓冲区排放到内存  % drains from the second hart's store buffer to memory
\end{itemize}
因此,内存模型必须能够解释这种行为。
% Therefore, the memory model must be able to account for this behavior.

换句话说,假设保留程序次序确实包括了下列假定的规则:
在保留的程序次序中,内存访问$a$先于内存访问$b$(并因此也在全局内存次序中先于$b$),如果在程序次序中$a$先于$b$,并且$a$和$b$访问相同的内存位置,$a$是一个写,而$b$是一个读。把这个称作“规则X”。然后我们得到如下结果:
% To put it another way, suppose the definition of preserved program order did include the following hypothetical rule:
% memory access $a$ precedes memory access $b$ in preserved program order (and hence also in the global memory order) if $a$ precedes $b$ in program order and $a$ and $b$ are accesses to the same memory location, $a$ is a write, and $b$ is a read.  Call this ``Rule X''.  Then we get the following:

\begin{itemize}
  \item (a) 先于 (b): 根据规则 X
  \item (b) 先于 (d): 根据规则 \ref{ppo:fence}
  \item (d) 先于 (e): 根据加载值公理。  否则,如果(e)先于(d),那么将需要(d)返回值1。(这是一个完全合法的执行;它只是并非问题所在)% Otherwise, if (e) preceded (d), then (d) would be required to return the value 1.  (This is a perfectly legal execution; it's just not the one in question)
  \item (e) 先于 (f): 根据规则 X
  \item (f) 先于 (h): 根据规则 \ref{ppo:fence}
  \item (h) 先于 (a): 根据加载值公理,同上。
\end{itemize}
全局内存次序必须是一个总次序,而不能有循环,因为循环将暗示该循环内的每个事件都发生在它自己之前,这是不可能的。
因此,上面提出的执行将被禁止,并因此,规则X的添加将禁止带有存储缓冲区转发的实现,这显然是不可取的。
% The global memory order must be a total order and cannot be cyclic, because a cycle would imply that every event in the cycle happens before itself, which is impossible.
% Therefore, the execution proposed above would be forbidden, and hence the addition of rule X would forbid implementations with store buffer forwarding, which would clearly be undesirable.

尽管如此,即使在全局内存次序中,(b)先于(a)且/或(f)先于(e),这个例子中唯一合理的可能性也是,对于(b),返回由(a)所写的值,而(f)和(e)类似。
这种情况的组合导致了加载值公理的定义中的第二个选项。即使在全局内存次序中,(b)先于(a),由于在(b)执行的时候(a)还位于存储缓冲区中,(a)将仍然对(b)可见。
因此,即使在全局内存次序中(b)先于(a),(b)也应当返回由(a)所写的值,因为在程序次序中(a)先于(b)。对于(e)和(f)也类似。
% Nevertheless, even if (b) precedes (a) and/or (f) precedes (e) in the global memory order, the only sensible possibility in this example is for (b) to return the value written by (a), and likewise for (f) and (e).  
% This combination of circumstances is what leads to the second option in the definition of the load value axiom.
% Even though (b) precedes (a) in the global memory order, (a) will still be visible to (b) by virtue of sitting in the store buffer at the time (b) executes.
% Therefore, even if (b) precedes (a) in the global memory order, (b) should return the value written by (a) because (a) precedes (b) in program order.
% Likewise for (e) and (f).

\begin{figure}[h!]
  \centering
  \begin{tabular}{m{.4\linewidth}@{\qquad}m{.4\linewidth}}
  {
    \tt\small
    \begin{tabular}{cl||cl}
    \multicolumn{2}{c}{Hart 0} & \multicolumn{2}{c}{Hart 1} \\
    \hline
          & li t1, 1    &     & li t1, 1      \\
      (a) & sw t1,0(s0) &     & LOOP:         \\
      (b) & fence w,w   & (d) & lw a0,0(s1)   \\
      (c) & sw t1,0(s1) &     & beqz a0, LOOP \\
          &             & (e) & sw t1,0(s2)   \\
          &             & (f) & lw a1,0(s2)   \\
          &             &     & xor a2,a1,a1  \\
          &             &     & add s0,s0,a2  \\
          &             & (g) & lw a2,0(s0)   \\
      \hline
      \multicolumn{4}{c}{输出结果: {\tt a0=1}, {\tt a1=1}, {\tt a2=0}}
    \end{tabular}
  }
  &
  \input{figs/litmus_ppoca.pdf_t}
  \end{tabular}
  \caption{“PPOCA”存储缓冲区转发litmus测试(允许的输出结果)}
  \label{fig:litmus:ppoca}
\end{figure}

在图~\ref{fig:litmus:ppoca}中显示了另一个突出存储缓冲区行为的测试。
在这个例子中,由于控制依赖,(d)的次序排在(e)之前,而由于地址依赖,(f)的次序排在(g)之前。
然而,即使(f)返回了由(e)所写的值,(e)的次序也并不需要排在(f)之前。
这个可能对应到下列事件序列:
% Another test that highlights the behavior of store buffers is shown in Figure~\ref{fig:litmus:ppoca}.
% In this example, (d) is ordered before (e) because of the control dependency, and (f) is ordered before (g) because of the address dependency.
% However, (e) is {\em not} necessarily ordered before (f), even though (f) returns the value written by (e).
% This could correspond to the following sequence of events:
\begin{itemize}
  \item (e) 推测地执行,并进入第二个硬件线程的私有存储缓冲区(但是没有排放到内存) % executes speculatively and enters the second hart's private store buffer (but does not drain to memory)
  \item (f) 推测地执行,并从存储缓冲区中的(e)转发它的值1 % executes speculatively and forwards its return value 1 from (e) in the store buffer
  \item (g) 推测地执行,并从内存读取值0  % executes speculatively and reads the value 0 from memory
  \item (a) 执行,进入第一个硬件线程的私有存储缓冲区,并排放到内存  % executes, enters the first hart's private store buffer, and drains to memory
  \item (b) 执行,并退场  % executes and retires
  \item (c) 执行,进入第一个硬件线程的私有存储缓冲区,并排放到内存  % executes, enters the first hart's private store buffer, and drains to memory
  \item (d) 执行,并从内存读取值1  % executes and reads the value 1 from memory
  \item (e), (f), 和 (g) 提交,因为推测是正确的  % commit, since the speculation turned out to be correct
  \item (e) 从存储缓冲区排放到内存  % drains from the store buffer to memory
\end{itemize}

\subsection{\nameref*{rvwmo:ax:atom}}
\label{sec:memory:atomicityaxiom}
\begin{tabular}{p{1cm}|p{12cm}} &
\nameref{rvwmo:ax:atom} (对于对齐的原子): \atomicityaxiom
\end{tabular}

RISC-V架构把原子性的概念从排序的概念中解耦出来。
不像诸如TSO的架构,RISC-V在RVWMO下的原子性不会默认采用任何排序需求。
排序的语义仅仅由PPO规则保证,否则就是适用的。
% The RISC-V architecture decouples the notion of atomicity from the notion of ordering.  
% Unlike architectures such as TSO, RISC-V atomics under RVWMO do not impose any ordering requirements by default.  
% Ordering semantics are only guaranteed by the PPO rules that otherwise apply.

RISC-V包含两种类型的原子性:AMO和LR/SC对。通过下列方式,它们在概念上有不同的表现。
LR/SC的行为就像是,旧值被带到核心,修改,然后写回到内存,所有这些保留都维持在该内存位置。
另一方面,AMO在概念上表现得像是,它们直接在内存中执行。AMO因此有固有的原子性,而LR/SC对的原子性在某种意义上略有不同,
即在内存位置方面,在起初的硬件线程持有该保留的期间,不会被另一个硬件线程所修改。
% RISC-V contains two types of atomics: AMOs and LR/SC pairs.
% These conceptually behave differently, in the following way.
% LR/SC behave as if the old value is brought up to the core, modified, and written back to memory, all while a reservation is held on that memory location.
% AMOs on the other hand conceptually behave as if they are performed directly in memory.
% AMOs are therefore inherently atomic, while LR/SC pairs are atomic in the slightly different sense that the memory location in question will not be modified by another hart during the time the original hart holds the reservation.

\begin{figure}[h!]
  \centering\small
  \begin{verbbox}
  (a) lr.d a0, 0(s0)
  (b) sd   t1, 0(s0)
  (c) sc.d t2, 0(s0)
  \end{verbbox}
  \theverbbox
  ~~~~~~
  \begin{verbbox}
  (a) lr.d a0, 0(s0)
  (b) sw   t1, 4(s0)
  (c) sc.d t2, 0(s0)
  \end{verbbox}
  \theverbbox
  ~~~~~~
  \begin{verbbox}
  (a) lr.w a0, 0(s0)
  (b) sw   t1, 4(s0)
  (c) sc.w t2, 0(s0)
  \end{verbbox}
  \theverbbox
  ~~~~~~
  \begin{verbbox}
  (a) lr.w a0, 0(s0)
  (b) sw   t1, 4(s0)
  (c) sc.w t2, 8(s0)
  \end{verbbox}
  \theverbbox
  \caption{在所有的四个(独立的)代码片段中,存储条件(c)是被允许的,但是不保证成功
    % In all four (independent) code snippets, the store-conditional (c) is permitted but not guaranteed to succeed
    }
  \label{fig:litmus:lrsdsc}
\end{figure}

原子性公理禁止在全局内存次序中,来自其它硬件线程的存储,在一个LR核、与该LR配对的SC之间交错。
原子性公理没有禁止在程序次序或全局内存次序中,加载在成对的操作之间交错,也没有禁止在程序次序或全局内存次序中,
来自相同硬件线程的存储或者对非重叠位置的存储出现在成对的操作之间。例如,图~\ref{fig:litmus:lrsdsc}中的SC指令可以(但是不保证)成功。
那些成功没有一个将违背原子性公理,因为其间的非条件存储是与成对的加载-存储指令和存储-条件指令来自相同的硬件线程。
这样,一个在缓存行粒度追踪内存访问(并因此将看到图~\ref{fig:litmus:lrsdsc}中的四个片段是完全相同的)的内存系统将不会强制让碰巧(假)共享了相同缓存行另一部分作为保留所正在持有的内存位置的存储-条件指令失败。
% The atomicity axiom forbids stores from other harts from being interleaved in global memory order between an LR and the SC paired with that LR.
% The atomicity axiom does not forbid loads from being interleaved between the paired operations in program order or in the global memory order, nor does it forbid stores from the same hart or stores to non-overlapping locations from appearing between the paired operations in either program order or in the global memory order.
% For example, the SC instructions in Figure~\ref{fig:litmus:lrsdsc} may (but are not guaranteed to) succeed.
% None of those successes would violate the atomicity axiom, because the intervening non-conditional stores are from the same hart as the paired load-reserved and store-conditional instructions.
% This way, a memory system that tracks memory accesses at cache line granularity (and which therefore will see the four snippets of Figure~\ref{fig:litmus:lrsdsc} as identical) will not be forced to fail a store-conditional instruction that happens to (falsely) share another portion of the same cache line as the memory location being held by the reservation.

这个原子性公理也技术性地支持了LR和SC接触不同地址和/或使用不同访问尺寸的情形;然而,在实际中,预计这种行为的使用情形会很稀少。同样地,那种在一个LR/SC对之间,来自相同硬件线程的存储与该LR或SC引用的内存位置实际重叠的情景,与其间的存储可能简单地落在相同的缓存行上的情景相比,也是稀少的。
% The atomicity axiom also technically supports cases in which the LR and SC touch different addresses and/or use different access sizes; however, use cases for such behaviors are expected to be rare in practice.
% Likewise, scenarios in which stores from the same hart between an LR/SC pair actually overlap the memory location(s) referenced by the LR or SC are expected to be rare compared to scenarios where the intervening store may simply fall onto the same cache line.

\subsection{\nameref*{rvwmo:ax:prog}}
\label{sec:memory:progress}
\begin{tabular}{p{1cm}|p{12cm}} &
\nameref{rvwmo:ax:prog}: \progressaxiom
\end{tabular}

进程公理确保了一个最小的向前进程保证。它确保了来自一个硬件线程的存储将在有限数量的时间之内,最终变得对于系统中的其它硬件线程可见,
并且来自其它硬件线程的加载将最终能够读取那些值(或由该值而来的后继值)。没有这个规则的话,举个例子,
一个自旋锁无限地在一个值上旋转,将变得合法,甚至是在有来自另一个硬件线程的存储正在等待该自旋锁解锁的时候。
% The progress axiom ensures a minimal forward progress guarantee.
% It ensures that stores from one hart will eventually be made visible to other harts in the system in a finite amount of time, and that loads from other harts will eventually be able to read those values (or successors thereof).
% Without this rule, it would be legal, for example, for a spinlock to spin infinitely on a value, even with a store from another hart waiting to unlock the spinlock.

进程公理并不试图在一个RISC-V实现中的硬件线程上采用任何其它的公平、延迟或者服务质量的概念。任何更强的公平性概念都取决于剩余的ISA和/或平台和/或设备的定义和实现。
% The progress axiom is intended not to impose any other notion of fairness, latency, or quality of service onto the harts in a RISC-V implementation.
% Any stronger notions of fairness are up to the rest of the ISA and/or up to the platform and/or device to define and implement.

在几乎所有的情况中,向前进程公理都将被任何标准的缓存一致协议所满足。带有非一致性缓存的实现可能不得不提供一些其它的机制来确保所有的存储(或者由此而来的后继者)对于所有硬件线程的最终可见性。
% The forward progress axiom will in almost all cases be naturally satisfied by any standard cache coherence protocol.
% Implementations with non-coherent caches may have to provide some other mechanism to ensure the eventual visibility of all stores (or successors thereof) to all harts.

\subsection{重叠地址排序(规则~\ref{ppo:->st}-\ref{ppo:amoforward})
% Overlapping-Address Orderings (Rules~\ref{ppo:->st}--\ref{ppo:amoforward})
}
\label{sec:memory:overlap}
\begin{tabular}{p{1cm}|p{12cm}}
  & Rule \ref{ppo:->st}: \ppost \\
  & Rule \ref{ppo:rdw}: \ppordw \\
  & Rule \ref{ppo:amoforward}: \ppoamoforward \\
\end{tabular}

相同地址排序,如果后者是一个存储,那么是简单的:一个加载或存储永远不会被重新排序到一个与后面的存储重叠的内存位置。
从微架构的视角,总的来说,如果推测被证明是无效的,很难或者说不可能来撤销一个推测性重排的存储,因此这种行为被模型简单地禁止了。
换句话说,不需要从一个存储到后一个加载的相同地址排序。正如在~\ref{sec:memory:loadvalueaxiom}节中讨论的那样,这反映了将值从缓冲的存储转发到之后的加载的实现的可观察的行为。
% Same-address orderings where the latter is a store are straightforward: a load or store can never be reordered with a later store to an overlapping memory location.  From a microarchitecture perspective, generally speaking, it is difficult or impossible to undo a speculatively reordered store if the speculation turns out to be invalid, so such behavior is simply disallowed by the model.
% Same-address orderings from a store to a later load, on the other hand, do not need to be enforced.
% As discussed in Section~\ref{sec:memory:loadvalueaxiom}, this reflects the observable behavior of implementations that forward values from buffered stores to later loads.

相同地址的加载-加载排序的要求要微妙得多。基础要求是,较新的加载一定不能返回比同一个硬件线程中对相同地址进行的较旧的加载所返回的值更旧的值。
这通常被称为“CoRR”(读-读对的一致性),或者称为更宽泛的“一致性”或者“各位置的顺序连贯性”需求的一部分。
过去,一些架构已经放松了相同地址的加载-加载排序,但是事后看来,这通常会让编程模型变得过于复杂,并且因此RVWMO需要强制执行CoRR排序。
然而,因为全局内存次序对应于加载执行的次序,而不是值被返回的次序,所以,从全局内存次序的角度,捕获CoRR的需求需要一点间接性。
% Same-address load-load ordering requirements are far more subtle.
% The basic requirement is that a younger load must not return a value that is older than a value returned by an older load in the same hart to the same address.  This is often known as ``CoRR'' (Coherence for Read-Read pairs), or as part of a broader ``coherence'' or ``sequential consistency per location'' requirement.
% Some architectures in the past have relaxed same-address load-load ordering, but in hindsight this is generally considered to complicate the programming model too much, and so RVWMO requires CoRR ordering to be enforced.
% However, because the global memory order corresponds to the order in which loads perform rather than the ordering of the values being returned, capturing CoRR requirements in terms of the global memory order requires a bit of indirection.

\begin{figure}[h!]
  \center
  \begin{tabular}{m{.4\linewidth}@{\qquad}m{.4\linewidth}}
    {\tt\small
    \begin{tabular}{cl||cl}
    \multicolumn{2}{c}{Hart 0} & \multicolumn{2}{c}{Hart 1} \\
    \hline
          & li t1, 1    &     & li~ t2, 2    \\
      (a) & sw t1,0(s0) & (d) & lw~ a0,0(s1) \\
      (b) & fence w, w  & (e) & sw~ t2,0(s1) \\
      (c) & sw t1,0(s1) & (f) & lw~ a1,0(s1) \\
          &             & (g) & xor t3,a1,a1 \\
          &             & (h) & add s0,s0,t3 \\
          &             & (i) & lw~ a2,0(s0) \\
      \hline
      \multicolumn{4}{c}{输出结果: {\tt a0=1}, {\tt a1=2}, {\tt a2=0}}
    \end{tabular}
  }
  &
  \input{figs/litmus_mp_fenceww_fri_rfi_addr.pdf_t}
  \end{tabular}
  \caption{石蕊测试MP+fence.w.w+fir-rfi-addr(允许的输出结果)。 
  % Litmus test MP+fence.w.w+fri-rfi-addr (outcome permitted)
  }
  \label{fig:litmus:frirfi}
\end{figure}

考虑图~\ref{fig:litmus:frirfi}的石蕊测试,它是更一般的“fri-rfi”式样的一个特别的实例。术语“fri-rfi”表示序列(d)、(e)、(f):(d)“从读取”来自相同硬件线程的(e)(例如,从一个比(e)更早的写读取),而(f)从来自相同硬件线程的(e)读取。
% Consider the litmus test of Figure~\ref{fig:litmus:frirfi}, which is one particular instance of the more general ``fri-rfi'' pattern.
% The term ``fri-rfi'' refers to the sequence (d), (e), (f): (d) ``from-reads'' (i.e., reads from an earlier write than) (e) which is the same hart, and (f) reads from (e) which is in the same hart.

从微架构的视角,输出结果{\tt a0=1},{\tt a1=2},{\tt a2=0}是合法的(比起各种其它更加不怎么微妙的输出结果)。直观地讲,下列将产生上述提及的输出结果:
% From a microarchitectural perspective, outcome {\tt a0=1}, {\tt a1=2}, {\tt a2=0} is legal (as are various other less subtle outcomes).  Intuitively, the following would produce the outcome in question:
\begin{itemize}
  \item (d) 暂停(不论出于什么原因;或许它在等待一些其它先前的指令时就暂停了)
  \item (e) 执行,并进入存储缓冲区(但是还没有排放到内存)
  \item (f) 执行,并从存储缓冲区中的(e)转发
  \item (g)、(h)和(i)执行
  \item (a) 执行并排放到内存,(b)执行,且(c)执行并排放到内存
  \item (d) 解除暂停并执行
  \item (e) 从存储缓冲区排放到内存
\end{itemize}
这个对应于全局内存次序(f)、(i)、(a)、(c)、(d)、(e)。注意,即使(f)在(d)之前执行,由(f)返回的值也比由(d)返回的值更新。
因此,这个执行是合法的,并且不会违背CoRR需求。
% This corresponds to a global memory order of (f), (i), (a), (c), (d), (e).
% Note that even though (f) performs before (d), the value returned by (f) is newer than the value returned by (d).
% Therefore, this execution is legal and does not violate the CoRR requirements.

类似地,如果两个背靠背的加载返回了相同存储所写入的值,那么在全局内存次序中,它们也可以乱序出现而不违背CoRR。注意这与说两个加载返回相同的值是不相同的,因为两个不同的存储也可以写入相同的值。
% Likewise, if two back-to-back loads return the values written by the same store, then they may also appear out-of-order in the global memory order without violating CoRR.  Note that this is not the same as saying that the two loads return the same value, since two different stores may write the same value.

\begin{figure}[h!]
  \centering
  \begin{tabular}{m{.4\linewidth}@{\qquad\quad}m{.6\linewidth}}
  {
    \tt\small
    \begin{tabular}{cl||cl}
    \multicolumn{2}{c}{Hart 0} & \multicolumn{2}{c}{Hart 1} \\
    \hline
          & li t1, 1    & (d) & lw~ a0,0(s1) \\
      (a) & sw t1,0(s0) & (e) & xor t2,a0,a0 \\
      (b) & fence w, w  & (f) & add s4,s2,t2 \\
      (c) & sw t1,0(s1) & (g) & lw~ a1,0(s4) \\
          &             & (h) & lw~ a2,0(s2) \\
          &             & (i) & xor t3,a2,a2 \\
          &             & (j) & add s0,s0,t3 \\
          &             & (k) & lw~ a3,0(s0) \\
      \hline
      \multicolumn{4}{c}{输出结果: {\tt a0=1}, {\tt a1=$v$}, {\tt a2=$v$}, {\tt a3=0}}
    \end{tabular}
  }
  &
  \input{figs/litmus_rsw.pdf_t}
   \end{tabular}
  \caption{石蕊测试RSW(允许的输出结果) 
  % Litmus test RSW (outcome permitted)
  }
  \label{fig:litmus:rsw}
\end{figure}

考虑图~\ref{fig:litmus:rsw}的石蕊测试。
输出结果{\tt a0=1}、{\tt a1=$v$}、{\tt a2=$v$}、{\tt a3=0}(这里$v$是由另一个硬件线程所写入的某个值)
可以通过允许(g)和(h)重排而被观察到。这个做法可能是推测性的,并且该推测可以被微架构证明(例如,通过监视缓存失效情况而没有发现),
因为无论如何,在(g)之后重新执行(h)都将返回相同存储所写入的值。因此假设,无论如何a1和a2都将最终由相同的存储写入相同的值,(g)和(h)可以被合法地重新排序。
对应这个执行的全局内存次序将是(h)、(k)、(a)、(c)、(d)、(g)。
% Consider the litmus test of Figure~\ref{fig:litmus:rsw}.
% The outcome {\tt a0=1}, {\tt a1=$v$},  {\tt a2=$v$}, {\tt a3=0} (where $v$ is some value written by another hart) can be observed by allowing (g) and (h) to be reordered.  This might be done speculatively, and the speculation can be justified by the microarchitecture (e.g., by snooping for cache invalidations and finding none) because replaying (h) after (g) would return the value written by the same store anyway.
% Hence assuming {\tt a1} and {\tt a2} would end up with the same value written by the same store anyway, (g) and (h) can be legally reordered.
% The global memory order corresponding to this execution would be (h),(k),(a),(c),(d),(g).

图~\ref{fig:litmus:rsw}中{\tt a1}不等于{\tt a2}的测试的执行实际上需要(g)在全局内存次序中出现在(h)之前。允许(h)在全局内存次序中出现在(g)之前,那种情况中将导致违反CoRR,因为接下来(h)将返回一个比(g)所返回的更旧的值。
因此,PPO规则~\ref{ppo:rdw}禁止这种CoRR违背的发生。严格来说,PPO规则~\ref{ppo:rdw}在所有情况中执行CoRR都达成了一种小心的平衡,同时又足够弱,以允许在真实的微架构中经常出现的“RSW”和“fri-rfi”式样。
% Executions of the test in Figure~\ref{fig:litmus:rsw} in which {\tt a1} does not equal {\tt a2} do in fact require that (g) appears before (h) in the global memory order.
% Allowing (h) to appear before (g) in the global memory order would in that case result in a violation of CoRR, because then (h) would return an older value than that returned by (g).
% Therefore, PPO rule~\ref{ppo:rdw} forbids this CoRR violation from occurring.
% As such, PPO rule~\ref{ppo:rdw} strikes a careful balance between enforcing CoRR in all cases while simultaneously being weak enough to permit ``RSW'' and ``fri-rfi'' patterns that commonly appear in real microarchitectures.

还有一个重叠地址规则:PPO规则~\ref{ppo:amoforward}简单地陈述了,一个值不能从AMO或SC返回到后续的加载,直到AMO或SC已经全局执行(在SC的情况中,
还要是成功的全局执行)。从概念的观点,AMO和SC指令理应在内存中原子性地执行,这有点是顺理成章的。然而,特别地,PPO规则~\ref{ppo:amoforward}陈述了,
硬件可能甚至不会无意识地转发正在被AMOSWAP存储到后续加载的值,即使对于AMOSWAP,
该存储的值实际在语义上并不依赖于内存中的先前的值,就像对于其它AMO的情况一样。
即使当从SC转发存储的值在语义上不依赖于所配对的LR所返回的值时也同样如此。
% There is one more overlapping-address rule: PPO rule~\ref{ppo:amoforward} simply states that a value cannot be returned from an AMO or SC to a subsequent load until the AMO or SC has (in the case of the SC, successfully) performed globally.
% This follows somewhat naturally from the conceptual view that both AMOs and SC instructions are meant to be performed atomically in memory.
% However, notably, PPO rule~\ref{ppo:amoforward} states that hardware may not even non-speculatively forward the value being stored by an AMOSWAP to a subsequent load, even though for AMOSWAP that store value is not actually semantically dependent on the previous value in memory, as is the case for the other AMOs.
% The same holds true even when forwarding from SC store values that are not semantically dependent on the value returned by the paired LR.

上面这三个PPO规则也应用在当上述提到的内存访问只有部分重叠的情况中。
这是可能发生的,例如,当使用了不同尺寸的访问来访问相同的对象。也要注意,为了两个内存访问重叠,
两个重叠的内存操作的基地址不需要必定是相同的。当使用了未对齐的内存访问的时候,重叠地址PPO规则独立地应用到每个组件内存访问。
% The three PPO rules above also apply when the memory accesses in question only overlap partially.
% This can occur, for example, when accesses of different sizes are used to access the same object.
% Note also that the base addresses of two overlapping memory operations need not necessarily be the same for two memory accesses to overlap.
% When misaligned memory accesses are being used, the overlapping-address PPO rules apply to each of the component memory accesses independently.

% ----------  word文档初稿中没有这一段的翻译,可能是后期版本加上去的,需再三核查  ----------------
% \begin{comment}
% The formal model captures this as follows:
% \begin{itemize}
%   \item (a) precedes (b) in preserved program order because both are stores to the same address, and (b) is a store (Rule~\ref{ppo:->st}).  Therefore, (c) cannot return the value written by (a), because (b) is a later store to the same address in both program order and the global memory order, and so returning the value written by (a) would violate the load value axiom.
%   \item (c) precedes (d) in preserved program order because both are accesses to the same address, and (d) is a store.  (c) also precedes (d) in program order.  Therefore, (c) is not able to return the value written by (d), because neither option in the load value axiom applies.
% \end{itemize}
% \end{comment}

\subsection{屏障(规则~\ref{ppo:fence})}
\label{sec:mm:fence}
\begin{tabular}{p{1cm}|p{12cm}} &
Rule \ref{ppo:fence}: \ppofence
\end{tabular}

默认情况下,FENCE指令确保程序次序中所有的来自屏障之前的指令的内存访问(即,“前趋集”)在全局内存次序中,早于程序次序中来自屏障之后的指令的内存访问(即,“后继集”)出现。
然而,屏障可以选择性地把前趋集和/或后继集进一步限制到一个更小的内存访问集合,以提供某些加速。
特别地,屏障拥有限制前趋集和/或后继集的PR、PW、SR和SW位。当且仅当设置了PR位(对应于PW)时,前趋集包括加载(对应于存储)。
类似地,当且仅当设置了SR(对应于SW)时,后继集包括加载(对应于存储)。
% By default, the FENCE instruction ensures that all memory accesses from instructions preceding the fence in program order (the ``predecessor set'') appear earlier in the global memory order than memory accesses from instructions appearing after the fence in program order (the ``successor set'').
% However, fences can optionally further restrict the predecessor set and/or the successor set to  a smaller set of memory accesses in order to provide some speedup.
% Specifically, fences have PR, PW, SR, and SW bits which restrict the predecessor and/or successor sets.
% The predecessor set includes loads (resp.\@ stores) if and only if PR (resp.\@ PW) is set.
% Similarly, the successor set includes loads (resp.\@ stores) if and only if SR (resp.\@ SW) is set.

当前的FENCE编码有关于四个位PR、PW、SR和SW的九个非平凡的组合,加上一个额外的编码FENCE.TSO,它有助于“获取+释放”或RVTSO语义的映射。
剩余的七个组合没有前趋集和/或后继集,并因此都是no-op。对于十个非平凡的选项,只有六个是在实际中经常使用的:
% The FENCE encoding currently has nine non-trivial combinations of the four bits PR, PW, SR, and SW, plus one extra encoding FENCE.TSO which facilitates mapping of ``acquire+release'' or RVTSO semantics.
% The remaining seven combinations have empty predecessor and/or successor sets and hence are no-ops.
% Of the ten non-trivial options, only six are commonly used in practice:
{
\begin{itemize}
  \item FENCE RW,RW
  \item FENCE.TSO
  \item FENCE RW,W
  \item FENCE R,RW
  \item FENCE R,R
  \item FENCE W,W
\end{itemize}
}
使用PR、PW、SR和SW的任何其它组合的FENCE指令是被保留的。我们强烈建议编程人员坚持使用这六个。其它的组合可能与内存模型有未知的或者不期望的交互。
% FENCE instructions using any other combination of PR, PW, SR, and SW are reserved.  We strongly recommend that programmers stick to these six.
% Other combinations may have unknown or unexpected interactions with the memory model.

最后,我们注意到,由于RISC-V使用一种多重拷贝原子的内存模型,编程人员因此可以以一种线程本地的方式来推断屏障位。在非多重拷贝原子的内存模型中,没有“屏障累积性”的复杂性概念。
% Finally, we note that since RISC-V uses a multi-copy atomic memory model, programmers can reason about fences bits in a thread-local manner.  There is no complex notion of ``fence cumulativity'' as found in memory models that are not multi-copy atomic.

\subsection{显式同步(规则~\ref{ppo:acquire}-\ref{ppo:pair})}
% \subsection{Explicit Synchronization (Rules~\ref{ppo:acquire}--\ref{ppo:pair})}
\label{sec:memory:acqrel}
\begin{tabular}{p{1cm}|p{12cm}}
  & 规则 \ref{ppo:acquire}: \ppoacquire \\
  & 规则 \ref{ppo:release}: \pporelease \\
  & 规则 \ref{ppo:rcsc}: \pporcsc \\
  & 规则 \ref{ppo:pair}: \ppopair \\
\end{tabular}

一个{\em 获取}操作,正如应当被用在临界区开始处那样,需要在程序次序中的所有之后的内存操作也都在全局内存次序中在获取操作之后。这确保了,例如,临界区之内的所有的加载和存储,相对于正在保护它们的同步变量,都是最新的。
获取次序可以通过两种方式之一而采用:通过一个acquire注释,它采用只相对于同步变量自身的次序,或者通过一个FENCE R, RW,它采用相对于所有先前的加载的次序。
% An {\em acquire} operation, as would be used at the start of a critical section, requires all memory operations following the acquire in program order to also follow the acquire in the global memory order.
% This ensures, for example, that all loads and stores inside the critical section are up to date with respect to the synchronization variable being used to protect it.
% Acquire ordering can be enforced in one of two ways: with an acquire annotation, which enforces ordering with respect to just the synchronization variable itself, or with a FENCE~R,RW, which enforces ordering with respect to all previous loads.

\begin{figure}[h!]
  \centering\small
  \begin{verbatim}
          sd           x1, (a1)     # 任意不相关的存储 
          ld           x2, (a2)     # 任意不相关的加载  
          li           t0, 1        # 初始化交换值 
      again:
          amoswap.w.aq t0, t0, (a0) # 尝试获取锁 
          bnez         t0, again    # 如果被占用则重试 
          # ...
          # Critical section.
          # ...
          amoswap.w.rl x0, x0, (a0) # 通过存入0来释放锁 
          sd           x3, (a3)     # 任意不相关的存储 
          ld           x4, (a4)     # 任意不相关的加载 
  \end{verbatim}
  \caption{带原子性的自旋锁}
  \label{fig:litmus:spinlock_atomics}
\end{figure}

考虑图~\ref{fig:litmus:spinlock_atomics}。
因为这个例子使用{\em aq},临界区中的加载和存储被保证在全局内存次序中出现在用于获取锁的AMOSWAP之后。然而,假设{\tt a0}、{\tt a1}和{\tt a2}指向不同的内存位置,临界区中的加载和存储可能会、或可能不会在全局内存次序中,出现在例子开始的“任意不相干的加载”之后。
% Consider Figure~\ref{fig:litmus:spinlock_atomics}.
% Because this example uses {\em aq}, the loads and stores in the critical section are guaranteed to appear in the global memory order after the AMOSWAP used to acquire the lock.  However, assuming {\tt a0}, {\tt a1}, and {\tt a2} point to different memory locations, the loads and stores in the critical section may or may not appear after the ``Arbitrary unrelated load'' at the beginning of the example in the global memory order.

\begin{figure}[h!]
  \centering\small
  \begin{verbatim}
          sd           x1, (a1)     # 任意不相关的存储 
          ld           x2, (a2)     # 任意不相关的加载 
          li           t0, 1        # 初始化交换值 
      again:
          amoswap.w    t0, t0, (a0) # 尝试获取锁
          fence        r, rw        # 强制采用“acquire”内存次序
          bnez         t0, again    # 如果被占用则重试
          # ...
          # Critical section.
          # ...
          fence        rw, w        # 强制采用“release”内存次序
          amoswap.w    x0, x0, (a0) # 通过存入0来释放锁 
          sd           x3, (a3)     # 任意不相关的存储 
          ld           x4, (a4)     # 任意不相关的加载
  \end{verbatim}
  \caption{带屏障的自旋锁}
  \label{fig:litmus:spinlock_fences}
\end{figure}

现在,考虑图~\ref{fig:litmus:spinlock_fences}中的替代方案。
在这种情况中,即使AMOSWAP不会采用带有{\em aq}位的次序,尽管如此,屏障也会使获取AMOSWAP在全局内存次序中早于临界区中的所有加载和存储出现。
然而,注意,在这种情况中,屏障也会强制采用额外的次序:它也需要程序开始处的“任意不相干的加载”在全局内存次序中比临界区的加载和存储出现得更早。(然而,这个特别的屏障并不强制采用任何相对于片段开始处的“任意不相干的存储”的次序。)
通过这种方式,屏障强加的次序比通过{\em .aq}采用的次序会稍微地更粗糙些。
% Now, consider the alternative in Figure~\ref{fig:litmus:spinlock_fences}.
% In this case, even though the AMOSWAP does not enforce ordering with an {\em aq} bit, the fence nevertheless enforces that the acquire AMOSWAP appears earlier in the global memory order than all loads and stores in the critical section.
% Note, however, that in this case, the fence also enforces additional orderings: it also requires that the ``Arbitrary unrelated load'' at the start of the program appears earlier in the global memory order than the loads and stores of the critical section.  (This particular fence does not, however, enforce any ordering with respect to the ``Arbitrary unrelated store'' at the start of the snippet.)
% In this way, fence-enforced orderings are slightly coarser than orderings enforced by {\em .aq}.

释放次序和获取次序的效果完全相同,只是方向相反。释放的语义需要在程序次序中所有的先于释放操作的加载和存储也要在全局内存次序中先于释放操作。
这确保了,例如,在临界区中的内存访问在全局内存次序中出现在锁释放存储之前。
正如和获取的语义一样,释放的语义可以使用relase注释或者用FENCE RW, W操作来强制采用。使用相同的例子,临界区中的加载和存储和代码片段末尾处的“任意不相干存储”之间的次序只由图~\ref{fig:litmus:spinlock_fences}的FENCE RW,W采用,而不是图~\ref{fig:litmus:spinlock_atomics}中的{\em rl}。
% Release orderings work exactly the same as acquire orderings, just in the opposite direction.  Release semantics require all loads and stores preceding the release operation in program order to also precede the release operation in the global memory order.
% This ensures, for example, that memory accesses in a critical section appear before the lock-releasing store in the global memory order.  Just as for acquire semantics, release semantics can be enforced using release annotations or with a FENCE~RW,W operation.  Using the same examples, the ordering between the loads and stores in the critical section and the ``Arbitrary unrelated store'' at the end of the code snippet is enforced only by the FENCE~RW,W in Figure~\ref{fig:litmus:spinlock_fences}, not by the {\em rl} in Figure~\ref{fig:litmus:spinlock_atomics}.

单独使用RCpc注释,存储-释放到加载-获取的次序是不会被强制采用的。这有助于在TSO和/或RCpc内存模型下所写的代码的移植。
为了强制采用存储-释放到加载-获取的次序,代码必须使用store-release-RCsc和load-acquire-RCsc操作,以便应用PPO规则\ref{ppo:rcsc}。
对于许多C/C++中的使用情形,只有RCpc来举一些例子是足够的,但是对于许多C/C++、Java和Linux中的其它的使用情形是不够的;详情请见~\ref{sec:memory:porting}节。
% With RCpc annotations alone, store-release-to-load-acquire ordering is not enforced.  This facilitates the porting of code written under the TSO and/or RCpc memory models.  
% To enforce store-release-to-load-acquire ordering, the code must use store-release-RCsc and load-acquire-RCsc operations so that PPO rule \ref{ppo:rcsc} applies.
% RCpc alone is sufficient for many use cases in C/C++ but is insufficient for many other use cases in C/C++, Java, and Linux, to name just a few examples; see Section~\ref{sec:memory:porting} for details.

PPO规则~\ref{ppo:pair}说明了,一个SC必须在全局内存次序中出现在它所配对的LR之后。由于固有的数据依赖,这将自然地从LR/SC的常见使用开始去执行一个原子的读-修改-写操作。
然而,PPO规则~\ref{ppo:pair}也会应用,即使当正在存储的值在句法上并不依赖于所配对的LR所返回的值。
% PPO rule~\ref{ppo:pair} indicates that an SC must appear after its paired LR in the global memory order.
% This will follow naturally from the common use of LR/SC to perform an atomic read-modify-write operation due to the inherent data dependency.
% However, PPO rule~\ref{ppo:pair} also applies even when the value being stored does not syntactically depend on the value returned by the paired LR.

最后,我们注意到,只使用屏障,编程人员在分析排序注释的时候,不需要担心“累积性”。
% Lastly, we note that just as with fences, programmers need not worry about ``cumulativity'' when analyzing ordering annotations.

\subsection{句法依赖(规则~\ref{ppo:addr}-\ref{ppo:ctrl})}
\label{sec:memory:dependencies}
\begin{tabular}{p{1cm}|p{12cm}}
  & 规则 \ref{ppo:addr}: \ppoaddr \\
  & 规则 \ref{ppo:data}: \ppodata \\
  & 规则 \ref{ppo:ctrl}: \ppoctrl \\
\end{tabular}

从一个加载到相同硬件线程中的后续内存操作的依赖是RVWMO内存模型所考虑的。Alpha内存模型由于选择{\em 不}强制采用这些依赖而著名,但是大多数现代硬件和软件内存模型都考虑允许依赖指令被重新排序,是过于混乱和违反直觉的。此外,现代代码有时会故意使用这种依赖,作为一种特别轻量级的排序实施机制。
% Dependencies from a load to a later memory operation in the same hart are respected by the RVWMO memory model.
% The Alpha memory model was notable for choosing {\em not} to enforce the ordering of such dependencies, but most modern hardware and software memory models consider allowing dependent instructions to be reordered too confusing and counterintuitive.
% Furthermore, modern code sometimes intentionally uses such dependencies as a particularly lightweight ordering enforcement mechanism.

第~\ref{sec:memorymodel:dependencies}节中的术语工作如下。
无论何时,当写入每个目的寄存器的值是源寄存器的函数的时候,指令被称为携带了从它们的源寄存器到它们的目的寄存器的依赖。
对于大多数指令,这意味着,目的寄存器携带了一个来自所有源寄存器的依赖。然而,也有一些著名的例外。在内存指令的情形中,写入目的寄存器的值最终来自于内存系统,而不是直接来自源寄存器,并因此这样打破了所携带的来自源寄存器的依赖链。
在无条件跳转的情形中,写入目的寄存器的值来自于当前的{\tt pc}(它永远不会被内存模型认为是一个源寄存器),并因此类似地,JALR(仅有的带有源寄存器的跳转)不会携带一个从{\em rs1}到{\em rd}的依赖。
% The terms in Section~\ref{sec:memorymodel:dependencies} work as follows.
% Instructions are said to carry dependencies from their source register(s) to their destination register(s) whenever the value written into each destination register is a function of the source register(s).
% For most instructions, this means that the destination register(s) carry a dependency from all source register(s).
% However, there are a few notable exceptions.
% In the case of memory instructions, the value written into the destination register ultimately comes from the memory system rather than from the source register(s) directly, and so this breaks the chain of dependencies carried from the source register(s).
% In the case of unconditional jumps, the value written into the destination register comes from the current {\tt pc} (which is never considered a source register by the memory model), and so likewise, JALR (the only jump with a source register) does not carry a dependency from {\em rs1} to {\em rd}.

\begin{verbbox}
(a) fadd  f3,f1,f2
(b) fadd  f6,f4,f5
(c) csrrs a0,fflags,x0
\end{verbbox}
\begin{figure}[h!]
  \centering\small
  \theverbbox
  \caption{通过{\tt fflags},一个(a)和(b)都会隐含地累积进去的目的寄存器,(c)有一个关于(a)和(b)的句法依赖。
    % (c) has a syntactic dependency on both (a) and (b) via {\tt fflags}, a destination register that both (a) and (b) implicitly accumulate into
    }
  \label{fig:litmus:fflags}
\end{figure}

累积到一个目的寄存器,而不是写入它,这个概念反应了类似{\tt fflags}的CSR的行为。特别地,累积进一个寄存器不会冲击到任何先前的写入或对相同寄存器的累积。
例如,在图~\ref{fig:litmus:fflags}中,(c)有一个同时关于(a)和(b)的句法依赖。
% The notion of accumulating into a destination register rather than writing into it reflects the behavior of CSRs such as {\tt fflags}.
% In particular, an accumulation into a register does not clobber any previous writes or accumulations into the same register.
% For example, in Figure~\ref{fig:litmus:fflags}, (c) has a syntactic dependency on both (a) and (b).

类似其它现代内存模型,RVWMO内存模型使用句法依赖,而不是语义依赖。换句话说,这个定义依赖于正在被不同指令访问的寄存器的标识,
而不是那些寄存器的实际的内容。这意味着地址依赖、控制依赖、或者内存依赖必须是强制采用的,即使计算看起来是可以“优化掉”的。
这个选择确保了RVWMO保留了与使用这些假句法依赖作为轻量级排序机制的代码的兼容性。
% Like other modern memory models, the RVWMO memory model uses syntactic rather than semantic dependencies.
% In other words, this definition depends on the identities of the
% registers being accessed by different instructions, not the actual
% contents of those registers.  This means that an address, control, or
% data dependency must be enforced even if the calculation could seemingly
% be ``optimized away''.
% This choice ensures that RVWMO remains compatible with code that uses these false syntactic dependencies as a lightweight ordering mechanism.

\begin{verbbox}
ld  a1,0(s0)
xor a2,a1,a1
add s1,s1,a2
ld  a5,0(s1)
\end{verbbox}
\begin{figure}[h!]
  \centering\small
  \theverbbox
  \caption{一个句法地址依赖  
  % A syntactic address dependency
  }
  \label{fig:litmus:address}
\end{figure}

例如,在图~\ref{fig:litmus:address}中,存在一个从第一个指令生成的内存操作到最后一个指令生成的内存操作的句法地址依赖,即使{\tt a1} XOR {\tt a1}是零,
并因此不会影响到第二个加载所访问的地址。
% For example, there is a syntactic address
% dependency from the memory operation generated by the first instruction to the memory operation generated by the last instruction in
% Figure~\ref{fig:litmus:address}, even though {\tt a1} XOR {\tt a1} is zero and
% hence has no effect on the address accessed by the second load.

使用依赖作为轻量级同步机制的好处是,排序的强制性需求仅仅被限制在上述提及的特定的两个指令。
其它非依赖的指令可以被激进的实现自由地重排。一个替代方案是使用一个加载-获取,但是这将强制第一个加载相对于所有后继指令的次序。
另一个替代方案是使用FENCE R, R,但是这将包含所有先前的加载和{\em 所有}后继的加载,使这个选择更加昂贵。
% The benefit of using dependencies as a lightweight synchronization mechanism is that the ordering enforcement requirement is limited only to the specific two instructions in question.
% Other non-dependent instructions may be freely reordered by aggressive implementations.
% One alternative would be to use a load-acquire, but this would enforce ordering for the first load with respect to {\em all} subsequent instructions.
% Another would be to use a FENCE~R,R, but this would include all previous and all subsequent loads, making this option more expensive.

\begin{verbbox}
      lw  x1,0(x2)
      bne x1,x0,next
      sw  x3,0(x4)
next: sw  x5,0(x6)
\end{verbbox}
\begin{figure}[h!]
  \centering\small
  \theverbbox
  \caption{一个句法控制依赖 
  % A syntactic control dependency
  }
  \label{fig:litmus:control1}
\end{figure}

一个控制依赖总是扩展到程序次序中跟在原始目标之后的所有的指令,在这个意义上,控制依赖的表现不同于地址依赖和数据依赖。
考虑表~\ref{fig:litmus:control1}:尽管{\tt 下}一个指令将总是执行,但是由上一个指令生成的内存操作仍然有一个来自第一个指令生成的内存操作的控制依赖。
% Control dependencies behave differently from address and data dependencies in the sense that a control dependency always extends to all instructions following the original target in program order.
% Consider Figure~\ref{fig:litmus:control1}: the instruction at {\tt next} will always execute, but the memory operation generated by that last instruction nevertheless still has a control dependency from the memory operation generated by the first instruction.

\begin{verbbox}
        lw  x1,0(x2)
        bne x1,x0,next
  next: sw  x3,0(x4)
\end{verbbox}
\begin{figure}[h!]
  \centering\small
  \theverbbox
  \caption{另一个句法控制依赖  
  % Another syntactic control dependency
  }
  \label{fig:litmus:control2}
\end{figure}

类似地,考虑图~\ref{fig:litmus:control2}。
即使两个分支的最终结果都有相同的目标,仍然存在从这个片段的第一个指令生成的内存操作,到最后一个指令生成的内存操作的一个控制依赖。控制依赖的这个定义比其它环境中(例如,C++)可能看到的要强一些,但是它符合文献中控制依赖的标准定义。
% Likewise, consider Figure~\ref{fig:litmus:control2}.
% Even though both branch outcomes have the same target, there is still a control dependency from the memory operation generated by the first instruction in this snippet to the memory operation generated by the last instruction.
% This definition of control dependency is subtly stronger than what might be seen in other contexts (e.g., C++), but it conforms with standard definitions of control dependencies in the literature.

显然,PPO规则\ref{ppo:addr} - \ref{ppo:ctrl}也是有意设计的,以尊重来源于成功的存储条件指令的输出的依赖。
通常,一个SC指令将跟随一个检测输出结果是否成功的条件分支;这暗示了将会有一个从SC指令生成的存储操作到分支随后的任何内存操作的控制依赖。
PPO规则~\ref{ppo:ctrl}反过来暗示了,任何后继的存储操作将在全局内存次序中比SC生成的存储操作出现得更晚。
然而,由于控制依赖、地址依赖和数据依赖是定义在内存操作上的,并且由于一次不成功的SC不会生成内存操作,所以在不成功的SC和它的依赖指令之间不会强制排序。并且,由于只有当SC成功的时候,SC才被定义为携带从它的源寄存器到{\em rd}的依赖,一次不成功的SC不会影响全局内存次序。
% Notably, PPO rules \ref{ppo:addr}--\ref{ppo:ctrl} are also intentionally designed to respect dependencies that originate from the output of a successful store-conditional instruction.
% Typically, an SC instruction will be followed by a conditional branch checking whether the outcome was successful; this implies that there will be a control dependency from the store operation generated by the SC instruction to any memory operations following the branch.
% PPO rule~\ref{ppo:ctrl} in turn implies that any subsequent store operations will appear later in the global memory order than the store operation generated by the SC.
% However, since control, address, and data dependencies are defined over memory operations, and since an unsuccessful SC does not generate a memory operation, no order is enforced between unsuccessful SC and its dependent instructions.
% Moreover, since SC is defined to carry dependencies from its source registers to {\em rd} only when the SC is successful, an unsuccessful SC has no effect on the global memory order.


\begin{figure}[h!]
  \centering
  \begin{tabular}{m{.4\linewidth}m{0.05\linewidth}m{.4\linewidth}}
  {
    \tt\small
    \begin{tabular}{cl||cl}
    \multicolumn{4}{c}{初始值: 0(s0)=1; 0(s2)=1} \\
    \\
    \multicolumn{2}{c}{Hart 0} & \multicolumn{2}{c}{Hart 1} \\
    \hline
      (a) & ld a0,0(s0)    & (e) & ld a3,0(s2) \\
      (b) & lr a1,0(s1)    & (f) & sd a3,0(s0) \\
      (c) & sc a2,a0,0(s1) &                    \\
      (d) & sd a2,0(s2)    &                    \\
      \hline
      \multicolumn{4}{c}{输出结果: {\tt a0=0}, {\tt a3=0}}
    \end{tabular}
  }
  & &
  \input{figs/litmus_lb_lrsc.pdf_t}
  \end{tabular}
  \caption{LB石蕊测试的一种变体(禁止的输出结果)  
  % A variant of the LB litmus test (outcome forbidden)
  }
  \label{fig:litmus:successdeps}
\end{figure}

此外,选择尊重源自于存储-条件指令的依赖确保了,特定的类似无中生有的行为会被阻止。
考虑图~\ref{fig:litmus:successdeps}。假设一个假想的实现可以偶然地做到提前保证存储-条件操作将会成功。
在这种情形中,(c)将提前返回0到{\tt a2}(在实际执行之前),从而允许序列(d)、(e)、(f)、(a)、然后是(b)的执行,接着(c)可能只在那一点(成功地)执行。这将表示(c)把它自己的成功的值写到了{\tt 0(s1)}!幸运的是,由于RVWMO尊重源自于由成功的SC指令生成的存储的依赖的事实,这个情形和其它类似的情形被阻止了。
% In addition, the choice to respect dependencies originating at store-conditional instructions ensures that certain out-of-thin-air-like behaviors will be prevented.
% Consider Figure~\ref{fig:litmus:successdeps}.
% Suppose a hypothetical implementation could occasionally make some early guarantee that a store-conditional operation will succeed.
% In this case, (c) could return 0 to {\tt a2} early (before actually executing), allowing the sequence (d), (e), (f), (a), and then (b) to execute, and then (c) might execute (successfully) only at that point.
% This would imply that (c) writes its own success value to {\tt 0(s1)}!
% Fortunately, this situation and others like it are prevented by the fact that RVWMO respects dependencies originating at the stores generated by successful SC instructions.

我们也注意到,指令之间的句法依赖,只有当它们采取句法地址依赖、句法控制依赖,和/或句法数据依赖的形式时才会有效力。
例如,~\ref{sec:source-dest-regs}节中,在两个“F”指令之间通过“累积CSR”形成的句法依赖并{\em 不}表示这两个“F”指令必须按次序执行。这种依赖将只会用于之后最终建立从两个“F”指令到之后的上述提及的访问CSR标志的CSR指令的依赖。
% We also note that syntactic dependencies between instructions only have any force when they take the form of a syntactic address, control, and/or data dependency.
% For example: a syntactic dependency between two ``F'' instructions via one of the ``accumulating CSRs'' in Section~\ref{sec:source-dest-regs} does {\em not} imply that the two ``F'' instructions must be executed in order.
% Such a dependency would only serve to ultimately set up later a dependency from both ``F'' instructions to a later CSR instruction accessing the CSR flag in question.

\subsection{流水线依赖(规则~\ref{ppo:addrdatarfi}-\ref{ppo:addrpo})}
% \subsection{Pipeline Dependencies (Rules~\ref{ppo:addrdatarfi}--\ref{ppo:addrpo})}
\label{sec:memory:ppopipeline}
\begin{tabular}{p{1cm}|p{12cm}}
  & 规则 \ref{ppo:addrdatarfi}: \ppoaddrdatarfi \\
  & 规则 \ref{ppo:addrpo}: \ppoaddrpo \\
%  & Rule \ref{ppo:ctrlcfence}: \ppoctrlcfence \\
%  & Rule \ref{ppo:addrpocfence}: \ppoaddrpocfence \\
\end{tabular}

\begin{figure}[h!]
  \centering
  \begin{tabular}{m{.4\linewidth}m{.05\linewidth}m{.4\linewidth}}
  {
    \tt\small
    \begin{tabular}{cl||cl}
    \multicolumn{2}{c}{Hart 0} & \multicolumn{2}{c}{Hart 1} \\
    \hline
          & li t1, 1    & (d) & lw a0, 0(s1)   \\
      (a) & sw t1,0(s0) & (e) & sw a0, 0(s2)   \\
      (b) & fence w, w  & (f) & lw a1, 0(s2)   \\
      (c) & sw t1,0(s1) &     & xor a2,a1,a1   \\
          &             &     & add s0,s0,a2   \\
          &             & (g) & lw a3,0(s0)    \\   
      \hline
      \multicolumn{4}{c}{输出结果: {\tt a0=1}, {\tt a3=0}}
    \end{tabular}
  } & &
  \input{figs/litmus_datarfi.pdf_t}
  \end{tabular}

  \caption{根据PPO规则~\ref{ppo:addrdatarfi}和从(d)到(e)的数据依赖,在全局内存次序中,(d)必须也先于(f)(禁止的输出结果)
    % Because of PPO rule~\ref{ppo:addrdatarfi} and the data dependency from (d) to (e), (d) must also precede (f) in the global memory order (outcome forbidden)
    }
  \label{fig:litmus:addrdatarfi}
\end{figure}

PPO规则~\ref{ppo:addrdatarfi}和\ref{ppo:addrpo}反应了几乎所有的真实的处理器流水线实现的行为。规则~\ref{ppo:addrdatarfi}陈述了一个加载不能从一个存储转发,
直到那个存储的地址和数据是已知的时候。考虑图~\ref{fig:litmus:addrdatarfi}:在(e)的数据被决定之前,(f)不能执行,因为(f)必须返回由(e)写入的值(或者在全局内存次序中由某些更加靠后的所写入的值),
而在(d)有机会执行之前,旧的值必须不能被(e)的写回所冲击。因此,(f)将永远不会在(d)的执行之前执行。
% PPO rules~\ref{ppo:addrdatarfi} and \ref{ppo:addrpo} reflect behaviors of almost all real processor pipeline implementations.
% Rule~\ref{ppo:addrdatarfi} states that a load cannot forward from a store until the address and data for that store are known.
% Consider Figure~\ref{fig:litmus:addrdatarfi}:
% (f) cannot be executed until the data for (e) has been resolved, because (f) must return the value written by (e) (or by something even later in the global memory order), and the old value must not be clobbered by the writeback of (e) before (d) has had a chance to perform.
% Therefore, (f) will never perform before (d) has performed.

\begin{figure}[h!]
  \centering
  \begin{tabular}{m{.4\linewidth}m{.05\linewidth}m{.4\linewidth}}
  {
    \tt\small
    \begin{tabular}{cl||cl}
    \multicolumn{2}{c}{Hart 0} & \multicolumn{2}{c}{Hart 1} \\
    \hline
          & li t1, 1    &     & li t1, 1       \\
      (a) & sw t1,0(s0) & (d) & lw a0, 0(s1)   \\
      (b) & fence w, w  & (e) & sw a0, 0(s2)   \\
      (c) & sw t1,0(s1) & (f) & sw t1, 0(s2)   \\
          &             & (g) & lw a1, 0(s2)   \\
          &             &     & xor a2,a1,a1   \\
          &             &     & add s0,s0,a2   \\
          &             & (h) & lw a3,0(s0)    \\   

      \hline
      \multicolumn{4}{c}{输出结果: {\tt a0=1}, {\tt a3=0}}
    \end{tabular}
  } & &
  \input{figs/litmus_datacoirfi.pdf_t}
  \end{tabular}

  \caption{根据PPO规则12和从(d)到(e)的数据依赖,在全局内存次序中,(d)必须也先于(f)(禁止的输出结果)
    % Because of the extra store between (e) and (g), (d) no longer necessarily precedes (g) (outcome permitted)
    }
  \label{fig:litmus:addrdatarfi_no}
\end{figure}

如果在(e)和(f)之间,有另一个针对相同地址的存储,就像图~\ref{fig:litmus:addrdatarfi_no}中的那样,那么(f)将不再依赖于(e)正在被决定的数据,并因此(f)关于为(e)生产数据的(d)的依赖将被打破。
% If there were another store to the same address in between (e) and (f), as in Figure~\ref{fig:litmus:addrdatarfi_no}, then (f) would no longer be dependent on the data of (e) being resolved, and hence the dependency of (f) on (d), which produces the data for (e), would be broken.

规则~\ref{ppo:addrpo}制定了一个和之前规则相似的观点:一个存储不能在内存执行,直到所有的先前可能访问相同地址的加载自身已经被执行。
这样的加载看起来必须在存储之前执行,但是如果存储在加载有机会读取旧值之前就覆写了内存中的值,它就不能这么做了。类似地,一个存储通常不能执行,直到它已知了先前的指令不会由于地址解析失败而引发异常,而从这个意义上讲,规则~\ref{ppo:addrpo}可以被视作规则~\ref{ppo:ctrl}的某种特殊情况。
% Rule~\ref{ppo:addrpo} makes a similar observation to the previous rule: a store cannot be performed at memory until all previous loads that might access the same address have themselves been performed.
% Such a load must appear to execute before the store, but it cannot do so if the store were to overwrite the value in memory before the load had a chance to read the old value.
% Likewise, a store generally cannot be performed until it is known that preceding instructions will not cause an exception due to failed address resolution, and in this sense, rule~\ref{ppo:addrpo} can be seen as somewhat of a special case of rule~\ref{ppo:ctrl}.

\begin{figure}[h!]
  \centering
  \begin{tabular}{m{.4\linewidth}m{.05\linewidth}m{.4\linewidth}}
    \tt\small
    \begin{tabular}{cl||cl}
    \multicolumn{2}{c}{Hart 0} & \multicolumn{2}{c}{Hart 1} \\
    \hline
        &             &     & li t1, 1       \\
    (a) & lw a0,0(s0) & (d) & lw a1, 0(s1)   \\
    (b) & fence rw,rw & (e) & lw a2, 0(a1)   \\
    (c) & sw s2,0(s1) & (f) & sw t1, 0(s0)   \\
    \hline
    \multicolumn{4}{c}{输出结果: {\tt a0=1}, {\tt a1=t}}
    \end{tabular}  
    & &
    \input{figs/litmus_addrpo.pdf_t}
  \end{tabular}
  \caption{因为在(e)和(g)之间的额外的存储,(d)不再必须先于(g)了(允许的输出结果)
    % Because of the address dependency from (d) to (e), (d) also precedes (f) (outcome forbidden)
    }
  \label{fig:litmus:addrpo}
\end{figure}

考虑图~\ref{fig:litmus:addrpo}:在(e)的地址被决定之前,(f)不能执行,因为它可能会导致地址匹配;也就是说,{\tt a1=s0}。
因此,在(d)已经执行并证实地址是否确实重叠之前,(f)不能被送到内存。
% Consider Figure~\ref{fig:litmus:addrpo}:
% (f) cannot be executed until the address for (e) is resolved, because it may turn out that the addresses match; i.e., that {\tt a1=s0}.  Therefore, (f) cannot be sent to memory before (d) has executed and confirmed whether the addresses do indeed overlap.

\section{超出主存范围}
% \section{Beyond Main Memory}

RVWMO当前不会尝试正式地描述FENCE.I、SFENCE.VMA、I/O屏障和PMA的行为如何。所有这些的行为都将在未来的形式化中描述。
与之同时,FENCE.I的行为描述在第~\ref{chap:zifencei}章中,SFENCE.VMA的行为描述在RISC-V指令集特权架构手册之中,而I/O屏障的行为和PMA的效果将在下面描述。
% RVWMO does not currently attempt to formally describe how FENCE.I, SFENCE.VMA, I/O fences, and PMAs behave.
% All of these behaviors will be described by future formalizations.
% In the meantime, the behavior of FENCE.I is described in Chapter~\ref{chap:zifencei}, the behavior of SFENCE.VMA is described in the RISC-V Instruction Set Privileged Architecture Manual, and the behavior of I/O fences and the effects of PMAs are described below.

\subsection{一致性和可缓存性}
% \subsection{Coherence and Cacheability}

RISC-V特权ISA定义了物理内存属性(PMA),除此之外,它指定了地址空间的各部分是否是一致的和/或可缓存的。完整的细节见RISC-V特权ISA规范。这里,我们简单地讨论每个PMA中的各种细节是如何关系到内存模型的:
% The RISC-V Privileged ISA defines Physical Memory Attributes (PMAs) which specify, among other things, whether portions of the address space are coherent and/or cacheable.
% See the RISC-V Privileged ISA Specification for the complete details.
% Here, we simply discuss how the various details in each PMA relate to the memory model:

\begin{itemize}
  \item 主内存vs I/O,以及I/O内存次序PMA:定义的内存模型适用于主内存区域。I/O次序在下面讨论。
  % Main memory vs.\@ I/O, and I/O memory ordering PMAs: the memory model as defined applies to main memory regions.  I/O ordering is discussed below.
  \item 支持的访问类型和原子性PMA:内存模型简单地被应用在每个区域所支持的任何原语之上。
  % Supported access types and atomicity PMAs: the memory model is simply applied on top of whatever primitives each region supports.
  \item 可缓存性PMA:可缓存性PMA总体上不会影响内存模型。非可缓存的区域可能比可缓存的区域有更多的限制性行为,但是所允许的行为集合无论如何都不会再变化。然而,一些平台相关的和/或设备相关的可缓存性设置可能有区别。
  %  PMAs: the cacheability PMAs in general do not affect the memory model.  Non-cacheable regions may have more restrictive behavior than cacheable regions, but the set of allowed behaviors does not change regardless.  However, some platform-specific and/or device-specific cacheability settings may differ.
  \item 一致性PMA:在PMA中标记为非一致性的内存区域,其内存一致性模型当前是平台相关的和/或设备相关的:加载值公理、原子性公理和进程公理都可能被非一致性内存所违背。然而请注意,一致性内存不需要硬件缓存一致性协议。RISC-V特权ISA规范建议,主内存的硬件不一致区域是不鼓励的,但是内存模型与硬件一致性、软件一致性、由只读性内存导致的隐含一致性、由只有一个拥有权限的代理导致的隐含一致性,或者其它的一致性,相兼容。
  % Coherence PMAs: The memory consistency model for memory regions marked as non-coherent in PMAs is currently platform-specific and/or device-specific: the load-value axiom, the atomicity axiom, and the progress axiom all may be violated with non-coherent memory.  Note however that coherent memory does not require a hardware cache coherence protocol.  The RISC-V Privileged ISA Specification suggests that hardware-incoherent regions of main memory are discouraged, but the memory model is compatible with hardware coherence, software coherence, implicit coherence due to read-only memory, implicit coherence due to only one agent having access, or otherwise.
  \item 幂等性PMA:幂等性PMA被用于指定那些加载和/或存储可能有副作用的内存区域,而这反过来被微架构用来决定,例如,预取是否合法。这个区别不影响内存模型。
  % Idempotency PMAs: Idempotency PMAs are used to specify memory regions for which loads and/or stores may have side effects, and this in turn is used by the microarchitecture to determine, e.g., whether prefetches are legal.  This distinction does not affect the memory model.
\end{itemize}

\subsection{I/O排序}
% \subsection{I/O Ordering}

对于I/O,通常不会应用加载值公理和原子性公理,因为读和写都可能有设备相关的副作用,并可能把由最近的存储所“写”的值以外的值返回到相同的地址。
无论如何,下面保留的程序次序规则通常仍然适用于对I/O内存的访问:在全局内存次序中,内存访问$a$先于内存访问$b$,如果在程序次序中$a$先于$b$,并且满足如下的一个或多个条件:
% For I/O, the load value axiom and atomicity axiom in general do not apply, as both reads and writes might have device-specific side effects and may return values other than the value ``written'' by the most recent store to the same address.
% Nevertheless, the following preserved program order rules still generally apply for accesses to I/O memory:
% memory access $a$ precedes memory access $b$ in global memory order if $a$ precedes $b$ in program order and one or more of the following holds:
\begin{enumerate}
  \item 在如第~\ref{ch:memorymodel}章中定义的保留的程序次序中,$a$先于$b$,除了只适用于从一个内存操作到另一个内存操作、和从一个I/O操作到另一个I/O操作的获取和释放次序注释的例外,但从一个内存操作到一个I/O操作,或者反过来,则不是。
  % $a$ precedes $b$ in preserved program order as defined in Chapter~\ref{ch:memorymodel}, with the exception that acquire and release ordering annotations apply only from one memory operation to another memory operation and from one I/O operation to another I/O operation, but not from a memory operation to an I/O nor vice versa
  \item $a$和$b$是对于一个I/O区域中重叠地址的访问 
  % $a$ and $b$ are accesses to overlapping addresses in an I/O region
  \item $a$和$b$是对于相同的强排序的I/O区域的访问
  % $a$ and $b$ are accesses to the same strongly ordered I/O region
  \item $a$和$b$是对于I/O区域的访问,且关联到被$a$或$b$所访问的I/O区域的通道是通道1
  % $a$ and $b$ are accesses to I/O regions, and the channel associated with the I/O region accessed by either $a$ or $b$ is channel 1
  \item $a$和$b$是对于关联到相同通道(除了通道0)的I/O区域的访问
  % $a$ and $b$ are accesses to I/O regions associated with the same channel (except for channel 0)
\end{enumerate}

注意FENCE指令在其前驱集和后继集之中区分了主内存操作和I/O操作。
为了强制在I/O操作和主内存操作之间排序,代码必须使用一个带有PI、PO、SI和/或SO,加上PR、PW、SR和/或SW的FENCE。
例如,为了强制在一个对主内存的写和一个对设备寄存器的I/O写之间排序,需要一个FENCE W, O或者更强的指令。
% Note that the FENCE instruction distinguishes between main memory operations and I/O operations in its predecessor and successor sets.
% To enforce ordering between I/O operations and main memory operations, code must use a FENCE with PI, PO, SI, and/or SO, plus PR, PW, SR, and/or SW.
% For example, to enforce ordering between a write to main memory and an I/O write to a device register, a FENCE~W,O or stronger is needed.

\begin{verbbox}
  sd t0, 0(a0)
  fence w,o
  sd a0, 0(a1)
\end{verbbox}
\begin{figure}[h!]
  \centering\small
  \theverbbox
  \caption{有序的内存和I/O访问}
  \label{fig:litmus:wo}
\end{figure}

当一个屏障被实际使用时,实现必须假定设备可能尝试在接收到MMIO信号之后立即访问内存,以及来自该设备到内存的后继的内存访问必须观察到所有次序优先于该MMIO操作的访问的影响。
换句话说,在图~\ref{fig:litmus:wo}中,假设{\tt 0(a0)}是在主内存中的,而{\tt 0(a1)}是在I/O内存中的一个设备寄存器的地址。如果设备在接收到MMIO写的时候访问{\tt 0(a0)},那么根据RVWMO内存模型的规则,在概念上,该加载必须出现在第一个对0(a0)的存储之后。
在一些实现中,确保这一点的仅有的方法将是要求第一个存储确实在MMIO写被发出之前完成。其它实现可能找到了更加激进的方式,同时其它实现仍然可能不需要对I/O和主内存访问做任何完全不同的事。
无论如何,RVWMO内存模型不在这些选项中做区分;它只是简单地提供了一种与实现无关的机制来指定必须强制采用的次序。
% When a fence is in fact used, implementations must assume that the device may attempt to access memory immediately after receiving the MMIO signal, and subsequent memory accesses from that device to memory must observe the effects of all accesses ordered prior to that MMIO operation.
% In other words, in Figure~\ref{fig:litmus:wo}, suppose {\tt 0(a0)} is in main memory and {\tt 0(a1)} is the address of a device register in I/O memory.
% If the device accesses {\tt 0(a0)} upon receiving the MMIO write, then that load must conceptually appear after the first store to {\tt 0(a0)} according to the rules of the RVWMO memory model.
% In some implementations, the only way to ensure this will be to require that the first store does in fact complete before the MMIO write is issued.
% Other implementations may find ways to be more aggressive, while others still may not need to do anything different at all for I/O and main memory accesses.
% Nevertheless, the RVWMO memory model does not distinguish between these options; it simply provides an implementation-agnostic mechanism to specify the orderings that must be enforced.

许多架构包括了“次序”和“完成”屏障的独立概念,尤其是当它与I/O相关时(与常规的主内存相反)。
次序屏障简单地确保了内存操作保持有序,而完成屏障确保了,在任何后继的访问变得可见之前,前趋的访问都已经被完成。
RISC-V不会明确地区分次序屏障和完成屏障。反之,这种区分是从FENCE位的不同用法简单地推断出来的。
% Many architectures include separate notions of ``ordering'' and ``completion'' fences, especially as it relates to I/O (as opposed to regular main memory).
% Ordering fences simply ensure that memory operations stay in order, while completion fences ensure that predecessor accesses have all completed before any successors are made visible.
% RISC-V does not explicitly distinguish between ordering and completion fences.
% Instead, this distinction is simply inferred from different uses of the FENCE bits.

对于遵守RISC-V Unix平台规范的实现,I/O设备和DMA操作被要求一致性地访问内存,并且通过强排序的I/O通道完成。
因此,访问常规的主内存区域,如果该区域同时被外部设备访问,那么也可以使用标准同步机制。
不遵守Unix平台规范的实现和/或在不会一致性地访问内存的设备中,将需要使用机制(这目前是平台相关的或者设备相关的)来强制一致性。
% For implementations that conform to the RISC-V Unix Platform Specification, I/O devices and DMA operations are required to access memory coherently and via strongly ordered I/O channels.
% Therefore, accesses to regular main memory regions that are concurrently accessed by external devices can also use the standard synchronization mechanisms.
% Implementations that do not conform to the Unix Platform Specification and/or in which devices do not access memory coherently will need to use mechanisms (which are currently platform-specific or device-specific) to enforce coherency.

地址空间中的I/O区域应当被考虑为在那些区域的PMA中的非可缓存的区域。这种区域可以被PMA认为是一致性的,如果它们还没有被任何代理缓存的话。
% I/O regions in the address space should be considered non-cacheable regions in the PMAs for those regions.  Such regions can be considered coherent by the PMA if they are not cached by any agent.

这一节中的次序保证可能不适用于在RISC-V核心和设备之间的平台相关的边界。特别地,经过外部总线(例如,PCIe)发送的I/O访问可能在它们到达它们的最终目的地之前被重新排序。在那种情景中,必须根据那些外部设备和总线的平台相关的规则来强制实行排序。
% The ordering guarantees in this section may not apply beyond a platform-specific boundary between the RISC-V cores and the device.  In particular, I/O accesses sent across an external bus (e.g., PCIe) may be reordered before they reach their ultimate destination.  Ordering must be enforced in such situations according to the platform-specific rules of those external devices and buses.

\section{代码移植和映射指南}
\label{sec:memory:porting}

\begin{table}[h!]
  \centering
  \begin{tabular}{|l|l|}
    \hline
    x86/TSO 操作 & RVWMO 映射 \\
    \hline
    \hline
    加载             & \tt l\{b|h|w|d\}; fence r,rw               \\
    \hline
    存储             & \tt fence rw,w; s\{b|h|w|d\}               \\
    \hline
    \multirow{2}{*}{原子 RMW}
    & \tt amo<op>.\{w|d\}.aqrl \textrm{OR} \\
    & \tt loop:\@ lr.\{w|d\}.aq; <op>; sc.\{w|d\}.aqrl; bnez loop \\
    \hline
    屏障             & \tt fence rw,rw \\
    \hline
  \end{tabular}
  \caption{从TSO操作到RISC-V操作的映射  
  % Mappings from TSO operations to RISC-V operations
  }
  \label{tab:tsomappings}
\end{table}

表~\ref{tab:tsomappings}提供了一份从TSO内存操作到RISC-V内存指令的映射。
通常的x86加载和存储都是固有的acquire-RCpc和release-RCpc操作:TSO默认强制所有的加载-加载、加载-存储,和存储-存储排序。
因此,在RVWMO下,所有的TSO加载必须被映射到随后跟有FENCE R, RW的加载,而所有的TSO存储必须被映射到跟在存储之后的FENCE RW, W。
TSO原子读-修改-写和使用LOCK前缀的x86指令是完全排序的,并且可以或者通过一个同时设置了{\em aq}和{\em rl}的AMO实现,或者通过一个设置了{\em aq}的LR、上述提及的算数操作、一个同时设置了{\em aq}和{\em rl}的SC,还有一个检查成功条件的条件分支来实现。
在最后一种情况中,在LR上的{\em rl}注释(由于不明的原因)是多余的,并且可以被省略。
% Table~\ref{tab:tsomappings} provides a mapping from TSO memory operations onto RISC-V memory instructions.
% Normal x86 loads and stores are all inherently acquire-RCpc and release-RCpc operations: TSO enforces all load-load, load-store, and store-store ordering by default.
% Therefore, under RVWMO, all TSO loads must be mapped onto a load followed by FENCE~R,RW, and all TSO stores must be mapped onto FENCE~RW,W followed by a store.
% TSO atomic read-modify-writes and x86 instructions using the LOCK prefix are fully ordered and can be implemented either via an AMO with both {\em aq} and {\em rl} set, or via an LR with {\em aq} set, the arithmetic operation in question, an SC with both {\em aq} and {\em rl} set, and a conditional branch checking the success condition.
% In the latter case, the {\em rl} annotation on the LR turns out (for non-obvious reasons) to be redundant and can be omitted.

表~\ref{tab:tsomappings}的替代方案也是可行的。一个TSO存储可以被映射到设置了{\em rl}的AMOSWAP上。
然而,由于RVWMO PPO规则~\ref{ppo:amoforward}禁止值从AMO到后继加载的转发,对存储使用AMOSWAP可能对性能产生负面的影响。
一个TSO加载可以使用设置了{\em aq}的LR来映射:所有的这种LR指令将是无配对的,但是事实上本身并不排除使用LR进行加载。
然而,再次强调,这种映射也可能对性能有负面影响,如果它把比最初意图更多的压力放在了保留机制上的话。
% Alternatives to Table~\ref{tab:tsomappings} are also possible.
% A TSO store can be mapped onto AMOSWAP with {\em rl} set.
% However, since RVWMO PPO Rule~\ref{ppo:amoforward} forbids forwarding of values from AMOs to subsequent loads, the use of AMOSWAP for stores may negatively affect performance.
% A TSO load can be mapped using LR with {\em aq} set: all such LR instructions will be unpaired, but that fact in and of itself does not preclude the use of LR for loads.
% However, again, this mapping may also negatively affect performance if it puts more pressure on the reservation mechanism than was originally intended.

\begin{table}[h!]
  \centering
  \begin{tabular}{|l|l|}
    \hline
    Power 操作 & RVWMO 映射 \\
    \hline
    \hline
    加载              & \tt l\{b|h|w|d\}  \\
    \hline
    加载-保留      & \tt lr.\{w|d\}  \\
    \hline
    存储             & \tt s\{b|h|w|d\}  \\
    \hline
    存储-条件 & \tt sc.\{w|d\}  \\
    \hline
    \tt lwsync        & \tt fence.tso \\
    \hline
    \tt sync          & \tt fence rw,rw \\
    \hline
    \tt isync         & \tt fence.i; fence r,r \\
    \hline
  \end{tabular}
  \caption{从Power操作到RISC-V操作的映射}
  \label{tab:powermappings}
\end{table}

表~\ref{tab:powermappings}提供了一份从Power内存操作到RISC-V内存指令的映射。
Power ISYNC在RISC-V上映射到一个后跟有FENCE R, R的FENCE.I上;后一个屏障是必须的,因为ISYNC被用于定义一种“控制+控制屏障”的依赖,而它在RVWMO中是不存在的。
% Table~\ref{tab:powermappings} provides a mapping from Power memory operations onto RISC-V memory instructions.
% Power ISYNC maps on RISC-V to a FENCE.I followed by a FENCE~R,R; the latter fence is needed because ISYNC is used to define a ``control+control fence'' dependency that is not present in RVWMO.

\begin{table}[h!]
  \centering
  \begin{tabular}{|l|l|}
    \hline
    ARM 操作             & RVWMO 映射 \\
    \hline
    \hline
    加载                      & \tt l\{b|h|w|d\}  \\
    \hline
    加载-获取              & \tt fence rw, rw; l\{b|h|w|d\}; fence r,rw  \\
    \hline
    加载-独占            & \tt lr.\{w|d\}  \\
    \hline
    加载-获取-独占    & \tt lr.\{w|d\}.aqrl \\
    \hline
    存储                     & \tt s\{b|h|w|d\}  \\
    \hline
    存储-释放             & \tt fence rw,w; s\{b|h|w|d\}  \\
    \hline
    存储-独占           & \tt sc.\{w|d\}  \\
    \hline
    存储-释放-独占   & \tt sc.\{w|d\}.rl  \\
    \hline
    \tt dmb                   & \tt fence rw,rw \\
    \hline
    \tt dmb.ld                & \tt fence r,rw \\
    \hline
    \tt dmb.st                & \tt fence w,w \\
    \hline
    \tt isb                   & \tt fence.i; fence r,r \\
    \hline
  \end{tabular}
  \caption{从ARM操作到RISC-V操作的映射}
  \label{tab:armmappings}
\end{table}

表~\ref{tab:armmappings}提供了一份从ARM内存操作到RISC-V内存指令的映射。由于RISC-V目前没有带{\em aq}或{\em rl}注释的不修饰的加载和存储的操作码,ARM加载-获取和存储-释放操作应当代之以使用屏障来映射。
而且,为了强制采用存储-释放到加载-获取的次序,在存储-释放和加载-获取之间必须有一个FENCE RW, RW;
表~\ref{tab:armmappings}通过把屏障放置在每个获取操作之前,强制采用了这个次序。
ARM的load-exclusive和store-exclusive指令可以类似地映射到与它们对等的RISC-V LR和SC上,但是并非把FENCE RW, RW放在设置了{\em aq}的LR之前,而是我们也简单地用设置{\em rl}代替。
ARM ISB在RISC-V上映射到后跟有FENCE R, R的FENCE.I上,类似于ISYNC映射Power的方式。
% Table~\ref{tab:armmappings} provides a mapping from ARM memory operations onto RISC-V memory instructions.
% Since RISC-V does not currently have plain load and store opcodes with {\em aq} or {\em rl} annotations, ARM load-acquire and store-release operations should be mapped using fences instead.
% Furthermore, in order to enforce store-release-to-load-acquire ordering, there must be a FENCE~RW,RW between the store-release and load-acquire; Table~\ref{tab:armmappings} enforces this by always placing the fence in front of each acquire operation.
% ARM load-exclusive and store-exclusive instructions can likewise map onto their RISC-V LR and SC equivalents, but instead of placing a FENCE~RW,RW in front of an LR with {\em aq} set, we simply also set {\em rl} instead.
% ARM ISB maps on RISC-V to FENCE.I followed by FENCE~R,R similarly to how ISYNC maps for Power.

\begin{table}[h!]
  \centering
  \begin{tabular}{|l|l|}
    \hline
    Linux 操作           & RVWMO 映射 \\
    \hline
    \hline
    \tt smp\_mb()             & \tt fence rw,rw \\
    \hline
    \tt smp\_rmb()            & \tt fence r,r \\
    \hline
    \tt smp\_wmb()            & \tt fence w,w \\
    \hline
    \tt dma\_rmb()            & \tt fence r,r \\
    \hline
    \tt dma\_wmb()            & \tt fence w,w \\
    \hline
    \tt mb()                  & \tt fence iorw,iorw \\
    \hline
    \tt rmb()                 & \tt fence ri,ri \\
    \hline
    \tt wmb()                 & \tt fence wo,wo \\
    \hline
    \tt smp\_load\_acquire()   & \tt l\{b|h|w|d\}; fence r,rw \\
    \hline
    \tt smp\_store\_release()  & \tt fence.tso; s\{b|h|w|d\}  \\
    \hline
    \hline
    Linux 构造            & RVWMO AMO 映射        \\
    \hline
    \tt atomic\_<op>\_relaxed  & \tt amo<op>.\{w|d\}      \\
    \hline
    \tt atomic\_<op>\_acquire  & \tt amo<op>.\{w|d\}.aq   \\
    \hline
    \tt atomic\_<op>\_release  & \tt amo<op>.\{w|d\}.rl   \\
    \hline
    \tt atomic\_<op>           & \tt amo<op>.\{w|d\}.aqrl \\
    \hline
    \hline
    Linux 构造            & RVWMO LR/SC 映射\\
    \hline
    \tt atomic\_<op>\_relaxed  & \tt loop:\@ lr.\{w|d\}; <op>; sc.\{w|d\}; bnez loop \\
    \hline
    \tt atomic\_<op>\_acquire  & \tt loop:\@ lr.\{w|d\}.aq; <op>; sc.\{w|d\}; bnez loop \\
    \hline
    \multirow{2}{*}{\tt atomic\_<op>\_release}
      & \tt loop:\@ lr.\{w|d\}; <op>; sc.\{w|d\}.aqrl$^*$; bnez loop \textrm{OR} \\
      & \tt fence.tso; loop:\@ lr.\{w|d\}; <op>; sc.\{w|d\}$^*$; bnez loop \\
    \hline
    \tt atomic\_<op>           & \tt loop:\@ lr.\{w|d\}.aq; <op>; sc.\{w|d\}.aqrl; bnez loop \\
    \hline
  \end{tabular}
  \caption{从Linux内存原语到RISC-V原语的映射。其它的构造(例如自旋锁)应当相应地服从。非一致性DMA的平台或设备可能需要额外的同步(例如缓存冲刷或无效性机制);当前任何这样的额外同步都将是设备相关的。
    % Mappings from Linux memory primitives to RISC-V primitives.  Other constructs (such as spinlocks) should follow accordingly.  Platforms or devices with non-coherent DMA may need additional synchronization (such as cache flush or invalidate mechanisms); currently any such extra synchronization will be device-specific.
    }
  \label{tab:linuxmappings}
\end{table}

表~\ref{tab:linuxmappings}提供了一份Linux内存排序宏到RISC-V内存指令的映射。
Linux屏障{\tt dma\_rmb()}和{\tt dma\_wmb()}分别映射到FENCE R, R和FENCE W, W,因为RISC-V Unix平台需要一致性DMA,但是在非一致性DMA平台上将分别被映射到FENCE RI, RI和FENCE WO, WO。
非一致性DMA的平台也可以要求一种“缓存行可以被冲刷和/或无效化”的机制。这种机制将是设备相关的、和/或在未来对ISA的扩展中标准化的。
% Table~\ref{tab:linuxmappings} provides a mapping of Linux memory ordering macros onto RISC-V memory instructions.
% The Linux fences {\tt dma\_rmb()} and {\tt dma\_wmb()} map onto FENCE~R,R and FENCE~W,W, respectively, since the RISC-V Unix Platform requires coherent DMA, but would be mapped onto FENCE~RI,RI and FENCE~WO,WO, respectively, on a platform with non-coherent DMA.
% Platforms with non-coherent DMA may also require a mechanism by which cache lines can be flushed and/or invalidated.
% Such mechanisms will be device-specific and/or standardized in a future extension to the ISA.

Linux对于释放操作的映射可能看起来比必要的更强,但是需要这些映射去覆盖某些Linux需要比更直观的映射将提供的更强的次序的情况。
特别地,在本文正在被编写的时候,Linux正在积极地讨论,在一个临界区中的访问和相同硬件线程中的一个后继的临界区中的访问之间,
是否需要加载-加载、加载-存储,和存储-存储的次序,并由相同的同步对象进行保护。
不是所有的FENCE RW, W/FENCE R, RW映射和{\em aq}/{\em rl}映射的组合都能提供这种次序。围绕这个问题有这样一些方法,包括:
% The Linux mappings for release operations may seem stronger than necessary, but these mappings are needed to cover some cases in which Linux requires stronger orderings than the more intuitive mappings would provide.
% In particular, as of the time this text is being written, Linux is actively debating whether to require load-load, load-store, and store-store orderings between accesses in one critical section and accesses in a subsequent critical section in the same hart and protected by the same synchronization object.
% Not all combinations of FENCE~RW,W/FENCE~R,RW mappings with {\em aq}/{\em rl} mappings combine to provide such orderings.
% There are a few ways around this problem, including:
\begin{enumerate}
  \item 永远使用FENCE RW, W/FENCE R, RW,并且永远不使用{\em aq}/{\em rl}。这是足够的,但是不可取,因为它违背了{\em aq}/{\em rl}修饰符的目的。
  % Always use FENCE~RW,W/FENCE~R,RW, and never use {\em aq}/{\em rl}.  This suffices but is undesirable, as it defeats the purpose of the {\em aq}/{\em rl} modifiers.
  \item 永远使用{\em aq}/{\em rl},并永远不使用FENCE RW, W/FENCE R, RW。这目前不会起作用,因为缺少带有{\em aq}和{\em rl}修饰符的加载和存储操作码。
  % Always use {\em aq}/{\em rl}, and never use FENCE~RW,W/FENCE~R,RW.  This does not currently work due to the lack of load and store opcodes with {\em aq} and {\em rl} modifiers.
  \item 加强释放操作的映射,使得它们将在现有的任何种类的获取映射中强制采用充分的次序。这是当前推荐的方案,该方案展示在表~\ref{tab:linuxmappings}中。
  % Strengthen the mappings of release operations such that they would enforce sufficient orderings in the presence of either type of acquire mapping.  This is the currently recommended solution, and the one shown in Table~\ref{tab:linuxmappings}.
\end{enumerate}

\begin{figure}[h!]
  \centering\small
  \begin{verbbox}
Linux 代码:
(a)  int r0 = *x;
(bc) spin_unlock(y, 0);
     ...
     ...
(d)  spin_lock(y);
(e)  int r1 = *z;
  \end{verbbox}
  \theverbbox
  ~~~~~~~~~~
  \begin{verbbox}
RVWMO 映射:
(a) lw           a0, 0(s0)
(b) fence.tso  // vs. fence rw,w
(c) sd           x0,0(s1)
    ...
    loop:
(d) amoswap.d.aq a1,t1,0(s1)
    bnez         a1,loop
(e) lw           a2,0(s2)
  \end{verbbox}
  \theverbbox
  \caption{Linux中临界区之间的次序}
  \label{fig:litmus:lkmm_ll}
\end{figure}

例如,Linux社区当前正在讨论临界区次序规则,该规则将要求图~\ref{fig:litmus:lkmm_ll}中的(a)被排序在(e)之前。如果确实那样要求,那么把(b)映射为FENCE RW, W将是不充分的。也就是说,随着Linux内核内存模型的演化,这些映射也将随之变化。
% For example, the critical section ordering rule currently being debated by the Linux community would require (a) to be ordered before (e) in Figure~\ref{fig:litmus:lkmm_ll}.
% If that will indeed be required, then it would be insufficient for (b) to map as FENCE~RW,W.
% That said, these mappings are subject to change as the Linux Kernel Memory Model evolves.

\begin{table}[h!]
  \centering
  \begin{tabular}{|l|l|}
    \hline
    C/C++ 构造                            & RVWMO 映射 \\
    \hline
    \hline
    Non-atomic load                            & \tt l\{b|h|w|d\}               \\
    \hline
    \tt atomic\_load(memory\_order\_relaxed)   & \tt l\{b|h|w|d\}               \\
    \hline
    %\tt atomic\_load(memory\_order\_consume)   & \multicolumn{2}{l|}{\tt l\{b|h|w|d\}; fence r,rw}   \\
    %\hline
    \tt atomic\_load(memory\_order\_acquire)   & \tt l\{b|h|w|d\}; fence r,rw    \\
    \hline
    \tt atomic\_load(memory\_order\_seq\_cst)  & \tt fence rw,rw; l\{b|h|w|d\}; fence r,rw       \\
    \hline
    \hline
    非原子性存储                           & \tt s\{b|h|w|d\}               \\
    \hline
    \tt atomic\_store(memory\_order\_relaxed)  & \tt s\{b|h|w|d\}               \\
    \hline
    \tt atomic\_store(memory\_order\_release)  & \tt fence rw,w; s\{b|h|w|d\}  \\
    \hline
    \tt atomic\_store(memory\_order\_seq\_cst) & \tt fence rw,w; s\{b|h|w|d\}  \\
    \hline
    \hline
    \tt atomic\_thread\_fence(memory\_order\_acquire)  & \tt fence r,rw \\
    \hline
    \tt atomic\_thread\_fence(memory\_order\_release)  & \tt fence rw,w \\
    \hline
    \tt atomic\_thread\_fence(memory\_order\_acq\_rel) & {\tt fence.tso} \\
    \hline
    \tt atomic\_thread\_fence(memory\_order\_seq\_cst) & \tt fence rw,rw \\
    \hline
    \hline
    C/C++ 构造                          & RVWMO AMO 映射        \\
    \hline
    \tt atomic\_<op>(memory\_order\_relaxed)  & \tt amo<op>.\{w|d\}      \\
    \hline
    \tt atomic\_<op>(memory\_order\_acquire)  & \tt amo<op>.\{w|d\}.aq   \\
    \hline
    \tt atomic\_<op>(memory\_order\_release)  & \tt amo<op>.\{w|d\}.rl   \\
    \hline
    \tt atomic\_<op>(memory\_order\_acq\_rel) & \tt amo<op>.\{w|d\}.aqrl \\
    \hline
    \tt atomic\_<op>(memory\_order\_seq\_cst) & \tt amo<op>.\{w|d\}.aqrl \\
    \hline
    \hline
    C/C++ 构造                           & RVWMO LR/SC 映射\\
    \hline
    \multirow{2}{*}{\tt atomic\_<op>(memory\_order\_relaxed)}
      & \tt loop:\@ lr.\{w|d\}; <op>; sc.\{w|d\}; \\
      & \tt bnez loop \\
    \hline
    \multirow{2}{*}{\tt atomic\_<op>(memory\_order\_acquire)}
      & \tt loop:\@ lr.\{w|d\}.aq; <op>; sc.\{w|d\}; \\
      & \tt bnez loop \\
    \hline
    \multirow{2}{*}{\tt atomic\_<op>(memory\_order\_release)}
      & \tt loop:\@ lr.\{w|d\}; <op>; sc.\{w|d\}.rl; \\
      & \tt bnez loop \\
    \hline
    \multirow{2}{*}{\tt atomic\_<op>(memory\_order\_acq\_rel)}
      & \tt loop:\@ lr.\{w|d\}.aq; <op>; sc.\{w|d\}.rl; \\
      & \tt bnez loop \\
    \hline
    \multirow{2}{*}{\tt atomic\_<op>(memory\_order\_seq\_cst)}
      & \tt loop:\@ lr.\{w|d\}.aqrl; <op>; \\
      & \tt sc.\{w|d\}.rl; bnez loop \\
    \hline
  \end{tabular}
  \caption{从C/C++原语到RISC-V原语的映射。
    % Mappings from C/C++ primitives to RISC-V primitives.
    }
  \label{tab:c11mappings}
\end{table}

\begin{table}[h!]
  \centering
  \begin{tabular}{|l|l|}
    \hline
    C/C++ 构造                            & RVWMO 映射 \\
    \hline
    \hline
    非原子性加载                            & \tt l\{b|h|w|d\}               \\
    \hline
    \tt atomic\_load(memory\_order\_relaxed)   & \tt l\{b|h|w|d\}               \\
    \hline
    \tt atomic\_load(memory\_order\_acquire)   & \tt l\{b|h|w|d\}.aq  \\
    \hline
    \tt atomic\_load(memory\_order\_seq\_cst)  & \tt l\{b|h|w|d\}.aq  \\
    \hline
    \hline
    非原子性存储                          & \tt s\{b|h|w|d\}               \\
    \hline
    \tt atomic\_store(memory\_order\_relaxed)  & \tt s\{b|h|w|d\}               \\
    \hline
    \tt atomic\_store(memory\_order\_release)  & \tt s\{b|h|w|d\}.rl  \\
    \hline
    \tt atomic\_store(memory\_order\_seq\_cst) & \tt s\{b|h|w|d\}.rl \\
    \hline
    \hline
    \tt atomic\_thread\_fence(memory\_order\_acquire)  & \tt fence r,rw \\
    \hline
    \tt atomic\_thread\_fence(memory\_order\_release)  & \tt fence rw,w \\
    \hline
    \tt atomic\_thread\_fence(memory\_order\_acq\_rel) & {\tt fence.tso} \\
    \hline
    \tt atomic\_thread\_fence(memory\_order\_seq\_cst) & \tt fence rw,rw \\
    \hline
    \hline
    C/C++ 构造                           & RVWMO AMO 映射    \\
    \hline
    \tt atomic\_<op>(memory\_order\_relaxed)  & \tt amo<op>.\{w|d\}      \\
    \hline
    \tt atomic\_<op>(memory\_order\_acquire)  & \tt amo<op>.\{w|d\}.aq   \\
    \hline
    \tt atomic\_<op>(memory\_order\_release)  & \tt amo<op>.\{w|d\}.rl   \\
    \hline
    \tt atomic\_<op>(memory\_order\_acq\_rel) & \tt amo<op>.\{w|d\}.aqrl \\
    \hline
    \tt atomic\_<op>(memory\_order\_seq\_cst) & \tt amo<op>.\{w|d\}.aqrl \\
    \hline
    \hline
    C/C++ 构造                           & RVWMO LR/SC 映射\\
    \hline
    \tt atomic\_<op>(memory\_order\_relaxed)  & \tt lr.\{w|d\}; <op>; sc.\{w|d\} \\
    \hline
    \tt atomic\_<op>(memory\_order\_acquire)  & \tt lr.\{w|d\}.aq; <op>; sc.\{w|d\} \\
    \hline
    \tt atomic\_<op>(memory\_order\_release)  & \tt lr.\{w|d\}; <op>; sc.\{w|d\}.rl \\
    \hline
    \tt atomic\_<op>(memory\_order\_acq\_rel) & \tt lr.\{w|d\}.aq; <op>; sc.\{w|d\}.rl \\
    \hline
    \tt atomic\_<op>(memory\_order\_seq\_cst) & \tt lr.\{w|d\}.aq$^*$; <op>; sc.\{w|d\}.rl \\
    \hline
    \multicolumn{2}{l}{为了能与表~\ref{tab:c11mappings}中的各个代码映射相互操作,$^*$必须是{\tt lr.\{w|d\}.aqrl}  
    % $^*$must be {\tt lr.\{w|d\}.aqrl} in order to interoperate with code mapped per Table~\ref{tab:c11mappings}
    }
  \end{tabular}
  \caption{假设的从C/C++原语到RISC-V原语的映射,如果引入了原生的加载-获取和存储释放操作码的话。
    % Hypothetical mappings from C/C++ primitives to RISC-V primitives, if native load-acquire and store-release opcodes are introduced.
    }
  \label{tab:c11mappings_hypothetical}
\end{table}

表~\ref{tab:c11mappings}提供了一份C11/C++11原子操作到RISC-V内存指令的映射。
如果引入了带有{\em aq}和{\em rl}修饰符的加载和存储操作码,那么表~\ref{tab:c11mappings_hypothetical}中的映射就将足够了。
然而要注意,只有当原子的{\tt atomic\_<op>(内存\_次序\_seq\_cst)}被使用一个同时设置了{\em aq}和{\em rl}的LR映射时,这两个映射才会正确地互通。
% Table~\ref{tab:c11mappings} provides a mapping of C11/C++11 atomic operations onto RISC-V memory instructions.
% If load and store opcodes with {\em aq} and {\em rl} modifiers are introduced, then the mappings in Table~\ref{tab:c11mappings_hypothetical} will suffice.
% Note however that the two mappings only interoperate correctly if {\tt atomic\_<op>(memory\_order\_seq\_cst)} is mapped using an LR that has both {\em aq} and {\em rl} set.

任何AMO可以通过一个LR/SC对来模拟,但是必须注意确保任何源自于LR的PPO次序也是源自于SC的,并且任何在SC终止的PPO次序也会使得在LR处终止。
例如,LR必须也要服从任何AMO拥有的数据依赖,使得加载操作否则将没有任何数据依赖的概念。类似地,相同硬件线程中的其它地方的一个FENCE R, R的影响也必须适用于SC,否则将不会遵从该屏障。
模拟器可以系通过简单地把AMO映射到{\tt lr.aq;~<op>;~sc.aqrl}上来达成这个效果,与其它地方的用于全排序原子性的映射相匹配。
% Any AMO can be emulated by an LR/SC pair, but care must be taken to ensure that any PPO orderings that originate from the LR are also made to originate from the SC, and that any PPO orderings that terminate at the SC are also made to terminate at the LR.
% For example, the LR must also be made to respect any data dependencies that the AMO has, given that load operations do not otherwise have any notion of a data dependency.
% Likewise, the effect a FENCE~R,R elsewhere in the same hart must also be made to apply to the SC, which would not otherwise respect that fence.
% The emulator may achieve this effect by simply mapping AMOs onto {\tt lr.aq;~<op>;~sc.aqrl}, matching the mapping used elsewhere for fully ordered atomics.

这些C11/C++11映射需要平台为所有内存提供下列物理内存属性(正如RISC-V特权ISAS中定义的那样):
% These C11/C++11 mappings require the platform to provide the following Physical Memory Attributes (as defined in the RISC-V Privileged ISA) for all memory:
\begin{itemize}
  \item 主内存
  \item 连贯性
  \item AMOArithmetic
  \item RsrvEventual
\end{itemize}
具有不同属性的平台可能需要不同的映射,或者需要平台相关的SW(例如,内存映射I/O)。
% Platforms with different attributes may require different mappings, or require platform-specific SW (e.g., memory-mapped I/O).

\section{实现指南}
% \section{Implementation Guidelines}

RVWMO和RVTSO内存模型绝不排除微架构采用复杂的推测技术或其它形式的优化来提供更高的性能。
模型也不采用任何需要使用任何一个特定的缓存层次的需求,甚至一点也不使用缓存一致性协议。
相反,这些模型只指定了可以暴露给软件的行为。微架构可以自由地使用任何流水线设计,任何一致性或非一致性缓存层次,任何片上互连,等等,
只要设计只认可满足内存模型规则的执行。也就是说,为了帮助人们理解内存模型的实际实现,本节中我们提供了一些关于架构师和编程人员应当如何解释模型的规则的指导。
% The RVWMO and RVTSO memory models by no means preclude microarchitectures from employing sophisticated speculation techniques or other forms of optimization in order to deliver higher performance.
% The models also do not impose any requirement to use any one particular cache hierarchy, nor even to use a cache coherence protocol at all.
% Instead, these models only specify the behaviors that can be exposed to software.
% Microarchitectures are free to use any pipeline design, any coherent or non-coherent cache hierarchy, any on-chip interconnect, etc., as long as the design only admits executions that satisfy the memory model rules.
% That said, to help people understand the actual implementations of the memory model, in this section we provide some guidelines on how architects and programmers should interpret the models' rules.

RVWMO和RVTSO都是多重拷贝原子性(或者“其它多重拷贝原子性”)的:任何对一个硬件线程(除了最初发出它的那个硬件线程)可见的存储的值,在概念上必须也对系统中的所有的其它硬件线程可见。
换句话说,硬件线程可以从它们自己的先前的存储进行转发,时机在那些存储变得对所有的硬件线程全局可见之前,但是提前在硬件线程之间进行转发是不被允许的。
多重拷贝原子性可以通过多种方式被采用。它可能由于缓存和存储缓冲区的物理设计而固有地存在,它可以通过一种单一写者/多重读者的缓存一致性协议来采用,或者它可以由于某些其它机制而存在。
% Both RVWMO and RVTSO are multi-copy atomic (or ``other-multi-copy-atomic''): any store value that is visible to a hart other than the one that originally issued it must also be conceptually visible to all other harts in the system.
% In other words, harts may forward from their own previous stores before those stores have become globally visible to all harts, but no early inter-hart forwarding is permitted.
% Multi-copy atomicity may be enforced in a number of ways.
% It might hold inherently due to the physical design of the caches and store buffers, it may be enforced via a single-writer/multiple-reader cache coherence protocol, or it might hold due to some other mechanism.

尽管多重拷贝原子性的确在微架构上采用了一些限制,但是它是保持内存模型免于变得极度复杂的关键属性之一。
例如,一个硬件线程不可以从一个邻居硬件线程的私有存储缓冲区合法地转发一个值(当然,除非它这么做,不会有新的非法行为变得架构可见)。
一个缓存一致性协议也不能在其已经无效化了所有来自其它缓存的更旧的拷贝之前,从一个硬件线程向另一个硬件线程转发一个值。
当然,微架构可以(并且高性能实现可能会)通过推测或其它的优化来暗中违背这些规则,只要任何不合规的行为都不会暴露给编程人员。
% Although multi-copy atomicity does impose some restrictions on the microarchitecture, it is one of the key properties keeping the memory model from becoming extremely complicated.
% For example, a hart may not legally forward a value from a neighbor hart's private store buffer (unless of course it is done in such a way that no new illegal behaviors become architecturally visible).
% Nor may a cache coherence protocol forward a value from one hart to another until the coherence protocol has invalidated all older copies from other caches.
% Of course, microarchitectures may (and high-performance implementations likely will) violate these rules under the covers through speculation or other optimizations, as long as any non-compliant behaviors are not exposed to the programmer.

作为一份解释RVWMO中的PPO规则的粗略指南,从软件的角度,我们期待看到以下情形:
% As a rough guideline for interpreting the PPO rules in RVWMO, we expect the following from the software perspective:
\begin{itemize}
  \item 编程人员将有规律地和积极地使用PPO规则\ref{ppo:->st}和\ref{ppo:fence}-\ref{ppo:pair}。
  % programmers will use PPO rules \ref{ppo:->st} and \ref{ppo:fence}--\ref{ppo:pair} regularly and actively.
  \item 专业的编程人员将使用PPO规则\ref{ppo:addr}-\ref{ppo:ctrl}来加速重要数据结构的关键路径。
  % expert programmers will use PPO rules \ref{ppo:addr}--\ref{ppo:ctrl} to speed up critical paths of important data structures.
  \item 即使是专业的编程人员也将很少直接使用PPO规则\ref{ppo:rdw}-\ref{ppo:amoforward}和\ref{ppo:addrdatarfi}-\ref{ppo:addrpo},如果它们有的话。
  % even expert programmers will rarely if ever use PPO rules \ref{ppo:rdw}--\ref{ppo:amoforward} and \ref{ppo:addrdatarfi}--\ref{ppo:addrpo} directly.  These are included to facilitate common microarchitectural optimizations (rule~\ref{ppo:rdw}) and the operational formal modeling approach (rules \ref{ppo:amoforward} and \ref{ppo:addrdatarfi}--\ref{ppo:addrpo}) described in Section~\ref{sec:operational}.  They also facilitate the process of porting code from other architectures that have similar rules.
\end{itemize}

% -------------  word文档里有这一段,但是tex源文件中没有,可能是这版tex去掉了这段话,需要再三审核。--------------
% 包含这些规则是为了方便在B.3节中描述的常见微架构优化(规则2)和操作形式化模型方法(规则3和12-13)。它们也有助于从拥有相似规则的其它架构移植代码的进程。


从硬件的角度,我们也希望看到下列情况:
% We also expect the following from the hardware perspective:
\begin{itemize}
  \item PPO规则\ref{ppo:->st}和\ref{ppo:amoforward}-\ref{ppo:release}反映了好理解的规则,应当不会给架构师带来什么惊喜。
  % PPO rules \ref{ppo:->st} and \ref{ppo:amoforward}--\ref{ppo:release} reflect well-understood rules that should pose few surprises to architects.
  \item PPO规则\ref{ppo:rdw}反映了一个自然的和常见的硬件优化,但那是一个非常微妙的优化,因此值得仔细地复查。
  % PPO rule \ref{ppo:rdw} reflects a natural and common hardware optimization, but one that is very subtle and hence is worth double checking carefully.
  \item PPO规则\ref{ppo:rcsc}可能对于架构师来说不会立刻感到显然,但是它是标准内存模型的需求。
  % PPO rule \ref{ppo:rcsc} may not be immediately obvious to architects, but it is a standard memory model requirement
  \item 加载值公理、原子性公理,和PPO规则\ref{ppo:pair}-\ref{ppo:addrpo}反映了大多数硬件实现将自然地采用的规则,除非它们包含了极度的优化。当然,尽管如此,实现应当确保复查这些规则。硬件也必须确保句法依赖不会“被优化掉”。
  % The load value axiom, the atomicity axiom, and PPO rules \ref{ppo:pair}--\ref{ppo:addrpo} reflect rules that most hardware implementations will enforce naturally, unless they contain extreme optimizations.  Of course, implementations should make sure to double check these rules nevertheless.  Hardware must also ensure that syntactic dependencies are not ``optimized away''.
\end{itemize}

架构可以自由地实现任何的内存模型规则,正如它们选择的那样保守。例如,一个硬件实现可以选择做到下列中的任何或者所有:
% Architectures are free to implement any of the memory model rules as conservatively as they choose.  For example, a hardware implementation may choose to do any or all of the following:
  \begin{itemize}
    \item 无论各位实际如何设置,把所有的屏障都按它们是FENCE RW, RW(或者FENCE IORW, IORW,如果涉及了I/O)来解释
    % interpret all fences as if they were FENCE~RW,RW (or FENCE~IORW,IORW, if I/O is involved), regardless of the bits actually set
    \item 把所有的带PW和SR的屏障都按它们是FENCE RW, RW(或者FENCE IORW, IORW,如果涉及了I/O)来实现,因为无论如何,带SR的PW都是四个可能的主内存次序组件中最昂贵的
    % implement all fences with PW and SR as if they were FENCE~RW,RW (or FENCE~IORW,IORW, if I/O is involved), as PW with SR is the most expensive of the four possible main memory ordering components anyway
    \item 像~\ref{sec:memory:porting}节中描述的那样模拟{\em aq}和{\em rl}
    % emulate {\em aq} and {\em rl} as described in Section~\ref{sec:memory:porting}
    \item 强制所有的相同地址的加载-加载次序,即使存在在诸如“fri-rfi”和“RSW”等式样
    %  all same-address load-load ordering, even in the presence of patterns such as ``fri-rfi'' and ``RSW''
    \item 禁止任何从存储缓冲区中的存储到相同地址的后继的AMO或LR的值的转发
    % forbid any forwarding of a value from a store in the store buffer to a subsequent AMO or LR to the same address
    \item 禁止任何从存储缓冲区中的AMO或SC到相同地址的后继的加载的值的转发
    % forbid any forwarding of a value from an AMO or SC in the store buffer to a subsequent load to the same address
    \item 在所有内存访问上实现TSO,并忽略任何不包括PW和SR次序的主内存屏障(例如,就像Ztso实现会做的那样)
    % implement TSO on all memory accesses, and ignore any main memory fences that do not include PW and SR ordering (e.g., as Ztso implementations will do)
    \item 把所有的原子性实现为RCsc或者甚至是全排序的,无论注释如何
    % implement all atomics to be RCsc or even fully ordered, regardless of annotation
  \end{itemize}

实现了RVTSO的架构可以安全地做到下列事情:
% Architectures that implement RVTSO can safely do the following:
\begin{itemize}
  \item 忽略所有不同时具有PW和SR的屏障(除非该屏障也排序了I/O)
  % Ignore all fences that do not have both PW and SR (unless the fence also orders I/O)
  \item 忽略所有除了规则\ref{ppo:fence}至\ref{ppo:rcsc}之外的PPO规则,因为在RVTSO的假设下,剩下的规则与其它的PPO规则是多余的
  % Ignore all PPO rules except for rules \ref{ppo:fence} through \ref{ppo:rcsc}, since the rest are redundant with other PPO rules under RVTSO assumptions
\end{itemize}

其它的一般注意事项:
% Other general notes:

\begin{itemize}
  \item 从内存模型的观点来看,静默存储(例如,写入与内存位置已经存在的值相同的值的存储)的行为与任何其它的存储相像。
  类似地,不实际改变内存中的值的AMO(例如,一个{\em rs2}中的值比内存中当前的值更小的AMOMAX)仍然在语义上被认为是存储操作。
  尝试实现静默存储的微架构必须小心地确保仍然服从内存模型,特别是在诸如RSW(章节~\ref{sec:memory:overlap})的情形中,它们往往与静默存储不兼容。
  % Silent stores (i.e., stores that write the same value that already exists at a memory location) behave like any other store from a memory model point of view.  Likewise, AMOs which do not actually change the value in memory (e.g., an AMOMAX for which the value in {\em rs2} is smaller than the value currently in memory) are still semantically considered store operations.  Microarchitectures that attempt to implement silent stores must take care to ensure that the memory model is still obeyed, particularly in cases such as RSW (Section~\ref{sec:memory:overlap}) which tend to be incompatible with silent stores.
  \item 写可以被融合(即,对相同地址的两个连续的写可以被融合)或归并(即,对于相同地址的两个背靠背的写,可以省略较早的那个),只要结果行为不会违反内存模型语义。
  % Writes may be merged (i.e., two consecutive writes to the same address may be merged) or subsumed (i.e., the earlier of two back-to-back writes to the same address may be elided) as long as the resulting behavior does not otherwise violate the memory model semantics.
\end{itemize}

写归并的问题可以从下面的例子来理解:
% The question of write subsumption can be understood from the following example:
\begin{figure}[h!]
  \centering
  \begin{tabular}{m{.4\linewidth}m{.1\linewidth}m{.4\linewidth}}
    \tt\small
    \begin{tabular}{cl||cl}
    \multicolumn{2}{c}{Hart 0} & \multicolumn{2}{c}{Hart 1} \\
    \hline
        & li t1, 3    &     & li  t3, 2    \\
        & li t2, 1    &     &              \\
    (a) & sw t1,0(s0) & (d) & lw  a0,0(s1) \\
    (b) & fence w, w  & (e) & sw  a0,0(s0) \\
    (c) & sw t2,0(s1) & (f) & sw  t3,0(s0) \\
    \end{tabular}
  & &
    \input{figs/litmus_subsumption.pdf_t}
  \end{tabular}
  \caption{写归并石蕊测试,允许的执行。
  %  Write subsumption litmus test, allowed execution.
  }
  \label{fig:litmus:subsumption}
\end{figure}

如前所写,如果加载(d)读取值$1$,然后(a)在全局内存次序中必须先于(f):
% As written, if the load ~(d) reads value~$1$, then (a) must precede (f) in the global memory order:
\begin{itemize}
  \item 由于规则2,在全局内存次序中,(a)先于(c)
  % (a) precedes (c) in the global memory order because of rule 2
  \item 由于加载值公理,在全局内存次序中,(c)先于(d)
  % (c) precedes (d) in the global memory order because of the Load Value axiom
  \item 由于规则7,在全局内存次序中,(d)先于(e)
  % (d) precedes (e) in the global memory order because of rule 7
  \item 由于规则1,在全局内存次序中,(e)先于(f)
  % (e) precedes (f) in the global memory order because of rule 1
\end{itemize}
换句话说,地址在{\tt s0}中的内存位置的最终取值必定是$2$(由存储(f)写入的值),而不能是$3$(由存储(a)写入的值)。
% In other words the final value of the memory location whose address is in {\tt s0} must be~$2$ (the value written by the store~(f)) and cannot be~$3$ (the value written by the store~(a)).

一个非常激进的微架构可能会错误地决定丢弃(e),因为(f)取代了它,而这可能反过来导致微架构打破(d)和(f)之间现在已消除的依赖
(并因此也会打破(a)和(f)之间的)。这将违反内存模型规则,因而它是被禁止的。
写归并可能在其他情形中是合法的,如果,例如在(d)和(e)之间没有数据依赖的话。
% A very aggressive microarchitecture might erroneously decide to discard (e), as (f) supersedes it, and this may in turn lead the microarchitecture to break the now-eliminated dependency between (d) and (f) (and hence also between (a) and (f)).
% This would violate the memory model rules, and hence it is forbidden.
% Write subsumption may in other cases be legal, if for example there were no data dependency between (d) and (e).

\subsection{未来可能的扩展}
% \subsection{Possible Future Extensions}

我们希望任何或所有的下列可能的未来扩展都将与RVWMO内存模型兼容:
% We expect that any or all of the following possible future extensions would be compatible with the RVWMO memory model:

\begin{itemize}
  \item ‘V’向量ISA扩展  
  % `V' vector ISA extensions
  \item ‘J’JIT扩展 
  % `J' JIT extension
  \item 用于设置了{\em aq}和{\em rl}的加载和存储操作码的原生编码
  % Native encodings for load and store opcodes with {\em aq} and {\em rl} set
  \item 限制到特定地址的屏障
  % Fences limited to certain addresses
  \item 缓存的写回/冲刷/无效化/等等……指令
  % Cache writeback/flush/invalidate/etc.\@ instructions
\end{itemize}

\section{已知问题}
% \section{Known Issues}
\label{sec:memory:discrepancies}

\subsection{混合尺寸的RSW}
% \subsection{Mixed-size RSW}
\label{sec:memory:discrepancies:mixedrsw}

\begin{figure}[h!]
  \centering\small
  {\tt
    \begin{tabular}{cl||cl}
    \multicolumn{2}{c}{Hart 0} & \multicolumn{2}{c}{Hart 1} \\
    \hline
          & li t1, 1    &     & li t1, 1    \\
      (a) & lw a0,0(s0) & (d) & lw a1,0(s1) \\
      (b) & fence rw,rw & (e) & amoswap.w.rl a2,t1,0(s2) \\
      (c) & sw t1,0(s1) & (f) & ld a3,0(s2) \\
          &             & (g) & lw a4,4(s2) \\
          &             &     & xor a5,a4,a4  \\
          &             &     & add s0,s0,a5  \\
          &             & (h) & sw a2,0(s0)   \\
      \hline
      \multicolumn{4}{c}{输出结果: {\tt a0=1}, {\tt a1=1}, {\tt a2=0}, {\tt a3=1}, {\tt a4=0}}
    \end{tabular}
  }
  \caption{混合尺寸的差异(被公理模型所允许,被操作模型所禁止)
    % Mixed-size discrepancy (permitted by axiomatic models, forbidden by operational model)
    }
  \label{fig:litmus:discrepancy:rsw1}
\end{figure}

\begin{figure}[h!]
  \centering\small
  {\tt
    \begin{tabular}{cl||cl}
    \multicolumn{2}{c}{Hart 0} & \multicolumn{2}{c}{Hart 1} \\
    \hline
          & li t1, 1    &     & li t1, 1      \\
      (a) & lw a0,0(s0) & (d) & ld a1,0(s1)   \\
      (b) & fence rw,rw & (e) & lw a2,4(s1)   \\
      (c) & sw t1,0(s1) &     & xor a3,a2,a2  \\
          &             &     & add s0,s0,a3  \\
          &             & (f) & sw a2,0(s0)   \\
      \hline
      \multicolumn{4}{c}{输出结果: {\tt a0=0}, {\tt a1=1}, {\tt a2=0}}
    \end{tabular}
  }
  \caption{混合尺寸的差异(被公理模型所允许,被操作模型所禁止)
    % Mixed-size discrepancy (permitted by axiomatic models, forbidden by operational model)
    }
  \label{fig:litmus:discrepancy:rsw2}
\end{figure}

\begin{figure}[h!]
  \centering\small
  {\tt
    \begin{tabular}{cl||cl}
    \multicolumn{2}{c}{Hart 0} & \multicolumn{2}{c}{Hart 1} \\
    \hline
          & li t1, 1    &     & li t1, 1      \\
      (a) & lw a0,0(s0) & (d) & sw t1,4(s1)   \\
      (b) & fence rw,rw & (e) & ld a1,0(s1)   \\
      (c) & sw t1,0(s1) & (f) & lw a2,4(s1)   \\
          &             &     & xor a3,a2,a2  \\
          &             &     & add s0,s0,a3  \\
          &             & (g) & sw a2,0(s0)   \\
      \hline
      \multicolumn{4}{c}{Outcome: {\tt a0=1}, {\tt a1=0x100000001}, {\tt a1=1}}
    \end{tabular}
  }
  \caption{混合尺寸的差异(被公理模型所允许,被操作模型所禁止)}
  \label{fig:litmus:discrepancy:rsw3}
\end{figure}

在混合尺寸的RSW变体家族中,操作规范和公理规范之间有一种已知的差异性,显示在表~\ref{fig:litmus:discrepancy:rsw1}—\ref{fig:litmus:discrepancy:rsw3}中。
为了解决这个差异性,我们可以选择添加一些像是下列新的PPO规则的东西:在保留的程序次序中(并因此也在全局内存次序中),
内存操作$a$先于内存操作$b$,如果在程序次序中a先于b,a和b都访问常规主内存(而不是I/O区域),$a$是一个加载,$b$是一个存储,
那么在$a$和$b$之间存在一个加载$m$,$a$和$m$都读取一个位$x$,在$a$和$m$之间没有写$x$的存储,且在PPO中$m$先于$b$。换句话说,在{\sf herd}语法中,
我们可以选择向PPO添加“{\tt (po-loc \& rsw);ppo;[W]}”。许多实现已经自然地采用了这种次序。严格来说,即使这个规则不是官方的,
尽管如此,我们也推荐实现者采用它,以便确保未来可能把这个规则添加到RVWMO的向前兼容性。
% There is a known discrepancy between the operational and axiomatic specifications within the family of mixed-size RSW variants shown in Figures~\ref{fig:litmus:discrepancy:rsw1}--\ref{fig:litmus:discrepancy:rsw3}.
% To address this, we may choose to add something like the following new PPO rule:
% Memory operation $a$ precedes memory operation $b$ in preserved program order (and hence also in the global memory order) if $a$ precedes $b$ in program order, $a$ and $b$ both access regular main memory (rather than I/O regions), $a$ is a load, $b$ is a store, there is a load $m$ between $a$ and $b$, there is a byte $x$ that both $a$ and $m$ read, there is no store between $a$ and $m$ that writes to $x$, and $m$ precedes $b$ in PPO.
% In other words, in {\sf herd} syntax, we may choose to add ``{\tt (po-loc \& rsw);ppo;[W]}'' to PPO.
% Many implementations will already enforce this ordering naturally.
% As such, even though this rule is not official, we recommend that implementers enforce it nevertheless in order to ensure forwards compatibility with the possible future addition of this rule to RVWMO.


\chapter{形式化的内存模型规范(0.1版本)}
% \chapter{Formal Memory Model Specifications, Version 0.1}
为了便于对RVWMO的形式化分析,这章使用不同的工具和建模方法呈现了一组形式化。
任何差异性都是无意的;希望这些模型确实描述了相同的合法行为集合。
% To facilitate formal analysis of RVWMO, this chapter presents a set of formalizations using different tools and modeling approaches.  Any discrepancies are unintended; the expectation is that the models describe exactly the same sets of legal behaviors.

这个附录应当被视为注释;所有的规范材料都提供在第~\ref{ch:memorymodel}章和ISA规范主体的剩余部分中。第~\ref{sec:memory:discrepancies}节中列出了所有当前已知的差异性。任何其它的差异性都是无意的。
% This appendix should be treated as commentary; all normative material is provided in Chapter~\ref{ch:memorymodel} and in the rest of the main body of the ISA specification.
% All currently known discrepancies are listed in Section~\ref{sec:memory:discrepancies}.
% Any other discrepancies are unintentional.

\clearpage
\section{Alloy中的形式公理规范}
% \section{Formal Axiomatic Specification in Alloy}
\label{sec:alloy}

\lstdefinelanguage{alloy}{
  morekeywords={abstract, sig, extends, pred, fun, fact, no, set, one, lone, let, not, all, iden, some, run, for},
  morecomment=[l]{//},
  morecomment=[s]{/*}{*/},
  commentstyle=\color{green!40!black},
  keywordstyle=\color{blue!40!black},
  moredelim=**[is][\color{red}]{@}{@},
  escapeinside={!}{!},
}
\lstset{language=alloy}
\lstset{aboveskip=0pt}
\lstset{belowskip=0pt}

我们在Alloy(\url{http://alloy.mit.edu})中呈现了RVWMO内存模型的一种形式规范。
这个模型可以从\url{https://github.com/daniellustig/riscv-memory-model}在线获得。
% We present a formal specification of the RVWMO memory model in Alloy (\url{http://alloy.mit.edu}).
% This model is available online at \url{https://github.com/daniellustig/riscv-memory-model}.

该线上材料也包含了一些石蕊测试和一些关于Alloy可以怎样被用于对第~\ref{sec:memory:porting}节中的一些映射进行模型检查的例子。
% The online material also contains some litmus tests and some examples of how Alloy can be used to model check some of the mappings in Section~\ref{sec:memory:porting}.

\begin{figure}[h!]
  {
  \tt\bfseries\centering\footnotesize
  \begin{lstlisting}
////////////////////////////////////////////////////////////////////////////////
// =RVWMO PPO=

// 保留的程序次序
fun ppo : Event->Event {
  // same-address ordering
  po_loc :> Store
  + rdw
  + (AMO + StoreConditional) <: rfi

  // 显式同步
  + ppo_fence
  + Acquire <: ^po :> MemoryEvent
  + MemoryEvent <: ^po :> Release
  + RCsc <: ^po :> RCsc
  + pair

  // 句法依赖
  + addrdep
  + datadep
  + ctrldep :> Store

  // 管道依赖
  + (addrdep+datadep).rfi
  + addrdep.^po :> Store
}

// 全局内存次序尊重保留的程序次序
fact { ppo in ^gmo }
\end{lstlisting}}
  \caption{Alloy中的形式化RVWMO内存模型(1/5:PPO)  
  % The RVWMO memory model formalized in Alloy (1/5: PPO)
  }
  \label{fig:alloy1}
\end{figure}
\begin{figure}[h!]
  {
  \tt\bfseries\centering\footnotesize
  \begin{lstlisting}
////////////////////////////////////////////////////////////////////////////////
// =RVWMO 公理=

// 加载值公理
fun candidates[r: MemoryEvent] : set MemoryEvent {
  (r.~^gmo & Store & same_addr[r]) // 在gmo中写前趋r
  + (r.^~po & Store & same_addr[r]) // 在po中写前趋r
}

fun latest_among[s: set Event] : Event { s - s.~^gmo }

pred LoadValue {
  all w: Store | all r: Load |
    w->r in rf <=> w = latest_among[candidates[r]]
}

// 原子性公理
pred Atomicity {
  all r: Store.~pair |            // 从lr开始,
    no x: Store & same_addr[r] |  // 对于相同的地址,没有存储x
      x not in same_hart[r]       // 使得x来自不同的硬件线程,
      and x in r.~rf.^gmo         // 在gmo中x跟着(存储r的“读从”),
      and r.pair in x.^gmo        // 以及在gmo中,r跟着x
}

// 进程公理被隐去:Alloy只考虑有限的执行

pred RISCV_mm { LoadValue and Atomicity /* and Progress */ }

\end{lstlisting}}
  \caption{Alloy中的形式化RVWMO内存模型(2/5:公理)
    % The RVWMO memory model formalized in Alloy (2/5: Axioms)
    }
  \label{fig:alloy2}
\end{figure}
\begin{figure}[h!]
  {
  \tt\bfseries\centering\footnotesize
  \begin{lstlisting}
////////////////////////////////////////////////////////////////////////////////
// 内存的基础模型

sig Hart {  // 硬件线程
  start : one Event
}
sig Address {}
abstract sig Event {
  po: lone Event // 程序次序
}

abstract sig MemoryEvent extends Event {
  address: one Address,
  acquireRCpc: lone MemoryEvent,
  acquireRCsc: lone MemoryEvent,
  releaseRCpc: lone MemoryEvent,
  releaseRCsc: lone MemoryEvent,
  addrdep: set MemoryEvent,
  ctrldep: set Event,
  datadep: set MemoryEvent,
  gmo: set MemoryEvent,  // 全局内存次序
  rf: set MemoryEvent
}
sig LoadNormal extends MemoryEvent {} // l{b|h|w|d}
sig LoadReserve extends MemoryEvent { // lr
  pair: lone StoreConditional
}
sig StoreNormal extends MemoryEvent {}       // s{b|h|w|d}
// 模型中所有的StoreConditionals 都假定是成功的
sig StoreConditional extends MemoryEvent {}  // sc
sig AMO extends MemoryEvent {}               // amo
sig NOP extends Event {}

fun Load : Event { LoadNormal + LoadReserve + AMO }
fun Store : Event { StoreNormal + StoreConditional + AMO }

sig Fence extends Event {
  pr: lone Fence, // 操作位
  pw: lone Fence, // 操作位
  sr: lone Fence, // 操作位
  sw: lone Fence  // 操作位
}
sig FenceTSO extends Fence {}

/* Alloy编码细节:操作码位要么被设置(被编码,例如 as f.pr in iden),
 * 要么不被设置(f.pr not in iden)。这些位不能被用作其它任何用途 */
fact { pr + pw + sr + sw in iden }
// 对排序注释也是类似的
fact { acquireRCpc + acquireRCsc + releaseRCpc + releaseRCsc in iden }
// 不要试图通过pr/pw/sr/sw来编码FenceTSO;只把它当做is那样使用
fact { no FenceTSO.(pr + pw + sr + sw) }
\end{lstlisting}}
  \caption{Alloy中的形式化RVWMO内存模型(3/5:内存的模型)}
  \label{fig:alloy3}
\end{figure}

\begin{figure}[h!]
  {
  \tt\bfseries\centering\footnotesize
  \begin{lstlisting}
////////////////////////////////////////////////////////////////////////////////
// =基本模型规则 =

// 次序注释组
fun Acquire : MemoryEvent { MemoryEvent.acquireRCpc + MemoryEvent.acquireRCsc }
fun Release : MemoryEvent { MemoryEvent.releaseRCpc + MemoryEvent.releaseRCsc }
fun RCpc : MemoryEvent { MemoryEvent.acquireRCpc + MemoryEvent.releaseRCpc }
fun RCsc : MemoryEvent { MemoryEvent.acquireRCsc + MemoryEvent.releaseRCsc }

// 没有像store - acquire或者load - release那样的东西,除非两者结合
fact { Load & Release in Acquire }
fact { Store & Acquire in Release }

// FENCE PPO
fun FencePRSR : Fence { Fence.(pr & sr) }
fun FencePRSW : Fence { Fence.(pr & sw) }
fun FencePWSR : Fence { Fence.(pw & sr) }
fun FencePWSW : Fence { Fence.(pw & sw) }

fun ppo_fence : MemoryEvent->MemoryEvent {
    (Load  <: ^po :> FencePRSR).(^po :> Load)
  + (Load  <: ^po :> FencePRSW).(^po :> Store)
  + (Store <: ^po :> FencePWSR).(^po :> Load)
  + (Store <: ^po :> FencePWSW).(^po :> Store)
  + (Load  <: ^po :> FenceTSO) .(^po :> MemoryEvent)
  + (Store <: ^po :> FenceTSO) .(^po :> Store)
}

// 辅助定义
fun po_loc : Event->Event { ^po & address.~address }
fun same_hart[e: Event] : set Event { e + e.^~po + e.^po }
fun same_addr[e: Event] : set Event { e.address.~address }

// 初始化存储
fun NonInit : set Event { Hart.start.*po }
fun Init : set Event { Event - NonInit }
fact { Init in StoreNormal }
fact { Init->(MemoryEvent & NonInit) in ^gmo }
fact { all e: NonInit | one e.*~po.~start }  // 各事件都确实地在one硬件线程中
fact { all a: Address | one Init & a.~address } // one初始化各地址的存储
fact { no Init <: po and no po :> Init }
\end{lstlisting}}
  \caption{Alloy中的形式化RVWMO内存模型(4/5:基础模型规则)}
  \label{fig:alloy4}
\end{figure}

\begin{figure}[h!]
  {
  \tt\bfseries\centering\footnotesize
  \begin{lstlisting}
// po
fact { acyclic[po] }

// gmo
fact { total[^gmo, MemoryEvent] } // gmo是在所有MemoryEvents上的总次序

//rf
fact { rf.~rf in iden } // 每个读返回唯一写的值
fact { rf in Store <: address.~address :> Load }
fun rfi : MemoryEvent->MemoryEvent { rf & (*po + *~po) }

//dep
fact { no StoreNormal <: (addrdep + ctrldep + datadep) }
fact { addrdep + ctrldep + datadep + pair in ^po }
fact { datadep in datadep :> Store }
fact { ctrldep.*po in ctrldep }
fact { no pair & (^po :> (LoadReserve + StoreConditional)).^po }
fact { StoreConditional in LoadReserve.pair } // 假设所有的SC都成功了

// rdw
fun rdw : Event->Event {
  (Load <: po_loc :> Load)  // start with all same_address load-load pairs,
  - (~rf.rf)                // subtract pairs that read from the same store,
  - (po_loc.rfi)            // and subtract out "fri-rfi" patterns
}

// 过滤出冗余的实例和/或可视化效果
fact { no gmo & gmo.gmo } // 保持可视化效果整洁
fact { all a: Address | some a.~address }

////////////////////////////////////////////////////////////////////////////////
// =可选:操作码编码约束=

// 神圣的屏障列表
fact { Fence in
  Fence.pr.sr
  + Fence.pw.sw
  + Fence.pr.pw.sw
  + Fence.pr.sr.sw
  + FenceTSO
  + Fence.pr.pw.sr.sw
}

pred restrict_to_current_encodings {
  no (LoadNormal + StoreNormal) & (Acquire + Release)
}

////////////////////////////////////////////////////////////////////////////////
// =Alloy捷径 =
pred acyclic[rel: Event->Event] { no iden & ^rel }
pred total[rel: Event->Event, bag: Event] {
  all disj e, e': bag | e->e' in rel + ~rel
  acyclic[rel]
}
\end{lstlisting}}
  \caption{Alloy中的形式化RVWMO内存模型(5/5:辅助内容)
    % The RVWMO memory model formalized in Alloy (5/5: Auxiliaries)
    ]}
  \label{fig:alloy5}
\end{figure}



\clearpage
\section{Herd中的形式公理规范}
% \section{Formal Axiomatic Specification in Herd}
\label{sec:herd}

工具\textsf{herd}把一个内存模型和一个石蕊测试作为输入,并在内存模型顶端模拟测试的执行。内存模型使用领域专用语言\textsc{Cat}写成。
这节提供了两个RVWMO的\textsc{Cat}内存模型。第一个模型,图~\ref{fig:herd2},对于一个Cat模型而言尽可能多地遵循了\emph{全局内存次序}、第~\ref{ch:memorymodel}章、RVWMO的定义。第二个模型,图~\ref{fig:herd3},是一个等价的、更加有效的的、基于部分次序的RVWMO模型。
% The tool \textsf{herd} takes a memory model and a litmus test as input and simulates the execution of the test on top of the memory model. Memory models are written in the domain specific language \textsc{Cat}. This section provides two \textsc{Cat} memory model of RVWMO. The first model, Figure~\ref{fig:herd2}, follows the \emph{global memory order}, Chapter~\ref{ch:memorymodel}, definition of~RVWMO, as much as is possible for a \textsc{Cat} model. The second model, Figure~\ref{fig:herd3}, is an equivalent, more efficient, partial order based RVWMO model.

模拟器~\textsf{herd}是\textsf{DIY}工具套件的一部分——软件和文件见 \url{http://diy.inria.fr} 。模型和更多内容可以从~url{http://diy.inria.fr/cats7/riscv/} 在线获得。
% The simulator~\textsf{herd} is part of the \textsf{diy} tool suite --- see \url{http://diy.inria.fr} for software and documentation. The models and more are available online at~\url{http://diy.inria.fr/cats7/riscv/}.

\begin{figure}[h!]
  {
  \tt\bfseries\centering\footnotesize
  \begin{lstlisting}
(*************)
(* Utilities *)
(*************)

(* All fence relations *)
let fence.r.r = [R];fencerel(Fence.r.r);[R]
let fence.r.w = [R];fencerel(Fence.r.w);[W]
let fence.r.rw = [R];fencerel(Fence.r.rw);[M]
let fence.w.r = [W];fencerel(Fence.w.r);[R]
let fence.w.w = [W];fencerel(Fence.w.w);[W]
let fence.w.rw = [W];fencerel(Fence.w.rw);[M]
let fence.rw.r = [M];fencerel(Fence.rw.r);[R]
let fence.rw.w = [M];fencerel(Fence.rw.w);[W]
let fence.rw.rw = [M];fencerel(Fence.rw.rw);[M]
let fence.tso =
  let f = fencerel(Fence.tso) in
  ([W];f;[W]) | ([R];f;[M])

let fence = 
  fence.r.r | fence.r.w | fence.r.rw |
  fence.w.r | fence.w.w | fence.w.rw |
  fence.rw.r | fence.rw.w | fence.rw.rw |
  fence.tso

(* Same address, no W to the same address in-between *)
let po-loc-no-w = po-loc \ (po-loc?;[W];po-loc)
(* Read same write *)
let rsw = rf^-1;rf
(* Acquire, or stronger  *)
let AQ = Acq|AcqRel
(* Release or stronger *)
and RL = RelAcqRel
(* All RCsc *)
let RCsc = Acq|Rel|AcqRel
(* Amo events are both R and W, relation rmw relates paired lr/sc *)
let AMO = R & W
let StCond = range(rmw)

(*************)
(* ppo rules *)
(*************)

(* Overlapping-Address Orderings *)
let r1 = [M];po-loc;[W]
and r2 = ([R];po-loc-no-w;[R]) \ rsw
and r3 = [AMO|StCond];rfi;[R]
(* Explicit Synchronization *)
and r4 = fence
and r5 = [AQ];po;[M]
and r6 = [M];po;[RL]
and r7 = [RCsc];po;[RCsc]
and r8 = rmw
(* Syntactic Dependencies *)
and r9 = [M];addr;[M]
and r10 = [M];data;[W]
and r11 = [M];ctrl;[W]
(* Pipeline Dependencies *)
and r12 = [R];(addr|data);[W];rfi;[R]
and r13 = [R];addr;[M];po;[W]

let ppo = r1 | r2 | r3 | r4 | r5 | r6 | r7 | r8 | r9 | r10 | r11 | r12 | r13
\end{lstlisting}
  }
  \caption{riscv-defs.cat,一个关于保留程序次序的Herd定义(1/3)
    % {\tt riscv-defs.cat}, a herd definition of preserved program order (1/3)
    }
  \label{fig:herd1}
\end{figure}

\begin{figure}[ht!]
  {
  \tt\bfseries\centering\footnotesize
  \begin{lstlisting}
Total

(* Notice that herd has defined its own rf relation *)

(* Define ppo *)
include "riscv-defs.cat"

(********************************)
(* Generate global memory order *)
(********************************)

let gmo0 = (* precursor: ie build gmo as an total order that include gmo0 *)
  loc & (W\FW) * FW | # Final write after any write to the same location
  ppo |               # ppo compatible
  rfe                 # includes herd external rf (optimization)

(* Walk over all linear extensions of gmo0 *)
with  gmo from linearizations(M\IW,gmo0)

(* Add initial writes upfront -- convenient for computing rfGMO *)
let gmo = gmo | loc & IW * (M\IW)

(**********)
(* Axioms *)
(**********)

(* Compute rf according to the load value axiom, aka rfGMO *)
let WR = loc & ([W];(gmo|po);[R])
let rfGMO = WR \ (loc&([W];gmo);WR)

(* Check equality of herd rf and of rfGMO *)
empty (rf\rfGMO)|(rfGMO\rf) as RfCons

(* Atomicity axiom *)
let infloc = (gmo & loc)^-1
let inflocext = infloc & ext
let winside  = (infloc;rmw;inflocext) & (infloc;rf;rmw;inflocext) & [W]
empty winside as Atomic
\end{lstlisting}
  }
  \caption{riscv.cat,RVWMO内存模型的一个Herd版本(2/3)
    % {\tt riscv.cat}, a herd version of the RVWMO memory model (2/3)
    }
  \label{fig:herd2}
\end{figure}

\begin{figure}[h!]
  {
  \tt\bfseries\centering\footnotesize
  \begin{lstlisting}
Partial

(***************)
(* Definitions *)
(***************)

(* Define ppo *)
include "riscv-defs.cat"

(* Compute coherence relation *)
include "cos-opt.cat"

(**********)
(* Axioms *)
(**********)

(* Sc per location *)
acyclic co|rf|fr|po-loc as Coherence

(* Main model axiom *)
acyclic co|rfe|fr|ppo as Model

(* Atomicity axiom *)
empty rmw & (fre;coe) as Atomic
\end{lstlisting}
  }
  \caption{{\tt riscv.cat},RVWMO内存模型的一种备选的Herd表示(3/3)
    % {\tt riscv.cat}, an alternative herd presentation of the RVWMO memory model (3/3)
    }
  \label{fig:herd3}
\end{figure}



\clearpage
\section{一个内存操作模型}
% \section{An Operational Memory Model}
\label{sec:operational}
这是操作风格中对RVWMO内存模型的一个备选的表示。
它旨在确实地承认与公理化表示相同的扩展行为:对于任何给定的程序,承认一个执行,当且仅当公理化表示也允许它。
% This is an alternative presentation of the RVWMO memory model in
% operational style.
%
% It aims to admit exactly the same extensional behavior as the
% axiomatic presentation: for any given program, admitting an execution
% if and only if the axiomatic presentation allows it.

公理化表示被定义为关于完整候选执行的谓词。相比之下,这种操作表示具有一种抽象的微架构风味:它被表示为一个状态机,
其中状态是对于硬件机器状态的一种抽象表示,并且带有显式的乱序执行和推测性执行(但是从更实现相关的微架构细节中抽象出来,
例如寄存器重命名、存储缓冲区、缓存层次、缓存协议,等等)。尽管如此,它可以提供有用的直觉。它也可以增量地构造执行,
使交互地和随机地探索更大样例的行为成为可能,同时公理化模型需要完整的候选执行,在此之上,公理可以得到检查。
% The axiomatic presentation is defined as a predicate on complete
% candidate executions.  In contrast, this operational presentation has
% an abstract microarchitectural flavor: it is expressed as a state
% machine, with states that are an abstract representation of hardware
% machine states, and with explicit out-of-order and speculative
% execution
% (but abstracting from more implementation-specific microarchitectural
% details such as register renaming, store buffers, cache hierarchies, cache protocols, etc.).
% As such, it can provide useful intuition.
% It can also
% construct executions incrementally, making it possible to
% interactively and randomly explore the behavior of larger examples,
% while the axiomatic model requires complete candidate executions
% over which the axioms can be checked.

操作表示覆盖了混合尺寸的执行,可能带有2的不同乘幂字节尺寸的重叠的内存访问。未对齐的访问被打断为单字节访问。
% The operational presentation covers mixed-size execution, with
% potentially overlapping memory accesses of different power-of-two byte
% sizes.  Misaligned accesses are broken up into single-byte accesses.

操作模型,与RISC-V ISA语义片段(RV64I和A)一起,被集成进{\tt rmem}探究工具中(url{https://github.com/rems-project/rmem})。
{\tt rmem}可以彻底地、伪随机地和交互地探究石蕊测试(见\ref{sec:litmustests})和小型ELF二进制文件。
在{\tt rmem}中,ISA语义使用Sail显示表达(关于Sail语言,见url{https://github.com/rems-project/sail},以及RISC-V ISA模型,
见url{https://github.com/rems-project/sail-riscv}),
而并发语义使用Lem表达(关于Lem语言,见url{https://github.com/rems-project/lem})。
% The operational model, together with a fragment of the RISC-V ISA
% semantics (RV64I and A), are integrated into the {\tt rmem} exploration
% tool (\url{https://github.com/rems-project/rmem}).  {\tt rmem} can
% explore litmus tests (see \ref{sec:litmustests}) and small ELF
% binaries exhaustively,
% pseudo-randomly and interactively.  In {\tt rmem}, the ISA semantics
% is expressed explicitly in Sail (see
% \url{https://github.com/rems-project/sail} for the Sail language, and
% \url{https://github.com/rems-project/sail-riscv} for the RISC-V ISA
% model), and the concurrency semantics is expressed in Lem (see
% \url{https://github.com/rems-project/lem} for the Lem language).

{\tt rmem}有一个命令行接口和一个网络接口。网络接口完全运行在客户端,
并且与一个石蕊测试库一起在线提供:\url{http//www.cl.cam.ac.uk/~pes20/rmem}。
命令行接口比网络接口更快,尤其是在穷举模式中。
% {\tt rmem} has a command-line interface and a web-interface.
% The web-interface runs entirely on the client side, and is provided
% online together with a library of litmus tests:
% \url{http://www.cl.cam.ac.uk/~pes20/rmem}.  The command-line interface
% is faster than the web-interface, specially in exhaustive mode.

% A library of RISC-V litmus tests can be downloaded from
% \url{https://github.com/litmus-tests/litmus-tests-riscv}.
% This repository also provides instructions on how to run the litmus
% tests on RISC-V hardware and how to compare the results with the
% operational and axiomatic models.  The library is also available
% through the web-interface.


% TODO: compare with the herd and alloy versions

下面是关于模型状态和过渡的一个非正式的介绍。正式的模型描述在下一个小节开始。
% Below is an informal introduction of the model states and transitions.
% The description of the formal model starts in the next subsection.

术语:与公理化表示相对,这里每个内存操作都或者是一个加载,或者是一个存储。
因此,AMO产生了两种截然不同的内存操作,加载和存储。当与“指令”连起来使用时,术语“加载”和“存储”指代产生此种内存操作的指令。
因此,二者都包括AMO指令。术语“获取”指代一种带有acquire-RCpc或acquire-RCsc注释的指令(或者它的内存操作)。
术语“释放”指代一种带有release-RCpc或release-RCsc注释的指令(或者它的内存操作)。
% Terminology: In contrast to the axiomatic presentation, here every memory operation is either a load or a store.
% Hence, AMOs give rise to two distinct memory operations, a load and a store.
% When used in conjunction with ``instruction'', the terms ``load'' and ``store'' refer to instructions that give rise to such memory operations.
% As such, both include AMO instructions.
% The term ``acquire'' refers to an instruction (or its memory operation) with the acquire-RCpc or acquire-RCsc annotation.
% The term ``release'' refers to an instruction (or its memory operation) with the release-RCpc or release-RCsc annotation.

\paragraph{模型状态}
一个模型状态由一份共享内存和一组硬件线程状态组成。
% A model state consists of a shared memory and a tuple of hart states.
\begin{center}
\sffamily
\begin{tabular}{ccc}
\cline{1-1}\cline{3-3}
\multicolumn{1}{|c|}{Hart 0} & \bf \dots & \multicolumn{1}{|c|}{Hart $n$} \\
\cline{1-1}\cline{3-3}
$\big\uparrow$ $\big\downarrow$ & & $\big\uparrow$ $\big\downarrow$ \\
\hline
\multicolumn{3}{|c|}{Shared Memory} \\
\hline
\end{tabular}
\end{center}
共享内存状态记录所有的至今已经传播的内存存储操作,以它们传播的次序记录(这可以变得更加高效,但是为了简化表示,我们采用了这种方式)。
% The shared memory state records all the memory store operations that have propagated so far, in the order they propagated (this can be made more efficient, but for simplicity of the presentation we keep it this way).

每个硬件线程状态主要由一个指令实例树组成,其中一些已经\emph{完成}了,而另一些还没有完成。未完成的指令实例可能被\emph{重新启动},例如,如果它们依赖于一个乱序的或者推测的加载,结果该加载出现了错误。
% Each hart state consists principally of a tree of instruction instances, some of which have been \emph{finished}, and some of which have not.
% Non-finished instruction instances can be subject to \emph{restart}, e.g.~if they depend on an out-of-order or speculative load that turns out to be unsound.

在指令树中,条件分支和间接跳转指令可能多个后继。当这种指令完成时,任何没有被采用的备选路径都被丢弃。
% Conditional branch and indirect jump instructions may have multiple successors in the instruction tree.
% When such instruction is finished, any un-taken alternative paths are discarded.

指令树中的每个指令实例都有一个状态,它包括了内部指令语义(用于该指令的ISA伪代码)的执行状态。
模型在Sail中对内部指令语义使用了一种形式化描述。可以将一条指令的执行状态想成伪代码控制状态、伪代码调用栈和局部变量值的一种表示。
一个指令实例状态也包括了关于该实例的内存和寄存器足迹、它的寄存器读写、它的内存操作、它是否完成,等等的信息。
% Each instruction instance in the instruction tree has a state that
% includes an execution state of the intra-instruction semantics (the
% ISA pseudocode for this instruction).
% The model uses a formalization of the intra-instruction semantics in Sail.
% One can think of the execution state of an instruction as a representation of the pseudocode control state, pseudocode call stack, and local variable values.
% An instruction instance state also includes information about the instance's memory and register footprints, its register reads and writes, its memory operations, whether it is finished, etc.

\paragraph{模型过渡}
% \paragraph{Model transitions}
对于任何模型状态,模型都定义了允许的过渡集合,每个过渡都是一个到达新的抽象机器状态的单一原子步骤。
一条指令的执行往往将涉及许多过渡,而它们在操作模型执行中与来自其它指令引发的过渡进行交错。
每个过渡由一个单独的指令实例引发;它将改变那个实例的状态,并且它可能依赖于、或者改变它的余下的硬件线程状态和共享内存状态,
但是它不会依赖于其它硬件线程的状态,并且它也不会去改变它们:下面引入的过渡定义在~\ref{sec:omm:transitions}节中,
每个过渡都带有一个先决条件和一个过渡后模型状态的构造。
% The model defines, for any model state, the set of allowed transitions, each of which is a single atomic step to a new abstract machine state.
% Execution of a single instruction will typically involve many
% transitions, and they may be interleaved in operational-model
% execution with transitions arising from other instructions. 
% Each transition arises from a single instruction instance; it will
% change the state of that instance, and it may depend on or change the
% rest of its hart state and the shared memory state, but it does not depend on other hart states, and it will not change them.
% The transitions are introduced below and defined in Section~\ref{sec:omm:transitions}, with a precondition and a construction of the post-transition model state for each.

\noindent 用于所有指令的过渡:
\begin{itemize}
\item \nameref{omm:fetch}: 这个过渡代表了一个新指令实例的获取和解码,作为一个先前的获取指令实例(或者初始获取地址)的一个程序次序后继。
% This transition represents a fetch and
%   decode of a new instruction instance, as a program order successor
%   of a previously fetched instruction instance (or the initial fetch
%   address).

模型假定指令内存是固定的;它不描述自我修改的代码的行为。
特别地,\nameref{omm:fetch}过渡不会生成内存加载操作,而在过渡中也不会涉及共享内存。
反而,模型依赖于在给定一个内存位置时提供操作码的外部指示。
% The model assumes the instruction memory is fixed; it does not
% describe the behavior of self-modifying code. 
% In particular, the \nameref{omm:fetch} transition does not generate memory load operations, and the shared memory is not involved in the transition.
% Instead, the model depends on an external oracle that provides an opcode when given a memory location.
%


\item[$\circ$] \nameref{omm:reg_write}: 这是对一个寄存器值的写。
% This is a write of a register value.
\item[$\circ$] \nameref{omm:reg_read}: 这是对一个寄存器值的写,该值来自写该寄存器的最近的程序次序前趋指令实例。
% This is a read of a register
  % value from the most recent program-order-predecessor instruction instance that writes to that register.
\item[$\circ$] \nameref{omm:sail_interp}: 这覆盖了伪代码内部运算:算数、函数调用,等等。
% This covers pseudocode internal computation: arithmetic, function calls, etc.
\item[$\circ$] \nameref{omm:finish}: 在指令伪代码完成的这一时刻,指令不能被重启,内存访问不能被丢弃,而所有的内存效果已经发生。
对于条件分支和间接跳转指令,任何从未被写到{\em pc}寄存器的地址获取的程序次序后继,与它们之后的指令实例的子树一起,都被丢弃,
% At this point the instruction pseudocode is done, the instruction cannot be restarted, memory accesses cannot be discarded, and all memory effects have taken place.
% For conditional branch and indirect jump instructions, any program order successors that were fetched from an address that is not the one that was written to the {\em pc} register are discarded, together with the sub-tree of instruction instances below them.
\end{itemize}

\noindent 专用于加载指令的过渡:
\begin{itemize}
\item[$\circ$] \nameref{omm:initiate_load}: 在这一时刻,加载指令的内存足迹是暂时已知的(它可能发生改变,如果较早的指令被重启的话),
  并且它的单个内存加载指令被满足而可以开始。
  % At this point the memory
  % footprint of the load instruction is provisionally known (it could change if
  % earlier instructions are restarted) and its individual memory load operations can start being satisfied.
\item \nameref{omm:sat_by_forwarding}: 通过转发,这会从程序次序中之前的内存存储操作,部分地或完全地满足单个内存加载操作。
% This partially or entirely
%   satisfies a single memory load operation by forwarding, from
%   program-order-previous memory store operations.
\item \nameref{omm:sat_from_mem}: 这会从内存完全地满足单个内存加载操作的未完成部分。
% This entirely satisfies the outstanding slices of a single memory load operation, from memory.
\item[$\circ$] \nameref{omm:complete_loads}: 在这一时刻,指令的所有的内存加载操作都已经被完全地满足,且指令的伪代码可以继续执行。
  一个加载指令可以被重启,直到\nameref{omm:finish}过渡。但是,在某些条件下,模型可能把一个加载指令视为不可重启的,即使是在它结束之前(例如,见\nameref{omm:prop_store})。
% At this point all the memory load operations of the instruction have been entirely satisfied and the instruction pseudocode can continue executing.
% A load instruction can be subject to being restarted until the \nameref{omm:finish} transition.
% But, under some conditions, the model might treat a load instruction as non-restartable even before it is finished (e.g.~see \nameref{omm:prop_store}).
\end{itemize}

\noindent 专用于存储指令的过渡:
\begin{itemize}
\item[$\circ$] \nameref{omm:initiate_store_footprint}: 在这一时刻,存储的内存足迹是暂时已知的。
% At this point the memory footprint of the store is provisionally known.
\item[$\circ$] \nameref{omm:instantiate_store_value}: 在这一时刻,内存存储操作具有它们的值,并且程序次序后继的内存加载操作可以通过从这些值转发而被满足。
% At this point the memory store operations have their values and program-order-successor memory load operations can be satisfied by forwarding from them.
\item[$\circ$] \nameref{omm:commit_stores}: 在这一时刻,存储指令被保证发生(指令不再会被重启或丢弃),而它们可以通过被传播到内存而开始。
% At this point the store operations are guaranteed to happen (the instruction can no longer be restarted or discarded), and they can start being propagated to memory.
\item \nameref{omm:prop_store}:这会把单个的内存存储操作传播到内存。
%  This propagates a single memory store operation to memory.
\item[$\circ$] \nameref{omm:complete_stores}: 在这一时刻,指令的所有的内存存储操作都已经被传播到内存,而指令的伪代码可以继续执行。
% At this point all the memory store operations of the instruction have been propagated to memory, and the instruction pseudocode can continue executing.
\end{itemize}

\noindent 专用于{\tt sc}指令的过渡:
\begin{itemize}
\item \nameref{omm:early_sc_fail}: 这会引起{\tt sc}失败,或者是一个自发的失败,或者是因为没有与程序次序之前的{\tt lr}配对。
% This causes the {\tt sc} to fail, either a spontaneous fail or because it is not paired with a program-order-previous {\tt lr}.
\item \nameref{omm:paired_sc}: 这个过渡表示{\tt sc}与一个{\tt lr}配对了,并且可能会成功。
% This transition indicates the {\tt sc} is paired with an {\tt lr} and might succeed.
\item \nameref{omm:commit_sc}: 这是对提交\nameref{omm:commit_stores}和\nameref{omm:prop_store}操作过渡的一个原子性的执行,它只有当存储来自于{\tt lr}读取的内容没有被覆写时才会被启用。
% This is an atomic execution of the transitions \nameref{omm:commit_stores} and \nameref{omm:prop_store}, it is enabled only if the stores from which the {\tt lr} read from have not been overwritten.
\item \nameref{omm:late_sc_fail}: 这会引起{\tt sc}失败,或者是一个自发的失败,或者是因为存储来自于{\tt lr}读取的内容已经被覆写。
% This causes the {\tt sc} to fail, either a spontaneous fail or because the stores from which the {\tt lr} read from have been overwritten.
\end{itemize}

\noindent 专用于AMO指令的过渡: 
\begin{itemize}
\item \nameref{omm:do_amo}: 这是一个所有需要去满足加载操作、执行必要的算术、和传播存储操作的过渡的原子性操作。
% This is an atomic execution of all the transitions needed to satisfy the load operation, do the required arithmetic, and propagate the store operation.
\end{itemize}

\noindent 专用于屏障指令的过渡: % Transitions specific to fence instructions:
\begin{itemize}
\item[$\circ$] \nameref{omm:commit_fence}
\end{itemize}

标记有~$\circ$标签的过渡,只要它们的先决条件被满足,那么总是可以立即执行,而不需要排除其它的行为;而$\bullet$不行。
尽管\nameref{omm:fetch}指令被标记为$\bullet$,但是只要它不是无限地多次执行,它就可以被立即执行。
% The transitions labeled~$\circ$ can always be taken eagerly, as soon as their precondition is satisfied, without excluding other behavior; the $\bullet$ cannot.
% Although \nameref{omm:fetch} is marked with a $\bullet$, it can be taken eagerly as long as it is not taken infinitely many times.

一个非AMO加载指令的实例,在被获取之后,往往将按这个次序经历如下的过渡:
% An instance of a non-AMO load instruction, after being fetched, will typically experience the following transitions in this order:
\begin{enumerate}
\item \nameref{omm:reg_read}
\item \nameref{omm:initiate_load}
\item \nameref{omm:sat_by_forwarding} 和/或 \nameref{omm:sat_from_mem} (与满足实例的所有加载操作的需求数目相同)
\item \nameref{omm:complete_loads}
\item \nameref{omm:reg_write}
\item \nameref{omm:finish}
\end{enumerate}
在上述过渡之前、之间和之后,可能出现任意数目的伪代码内部步骤过渡。此外,在下一个程序位置中的用于获取指令的Fetch指令过渡将一直可用,直到其被执行为止。
% Before, between and after the transitions above, any number of \nameref{omm:sail_interp} transitions may appear.
% In addition, a \nameref{omm:fetch} transition for fetching the instruction in the next program location will be available until it is taken.

这样结束了关于操作模型的非正式描述。接下来的章节描述了正式的操作模型。
% This concludes the informal description of the operational model.
% The following sections describe the formal operational model.

\subsection{指令内的伪码执行}\label{sec:omm:pseudocode_exec}
% \subsection{Intra-instruction Pseudocode Execution}\label{sec:omm:pseudocode_exec}
每个指令实例的指令内语义被表达为一个状态机,它本质上运行指令伪代码。
给定一个伪代码执行状态,它会计算下一个状态。大多数状态标识了一个由伪代码所请求的、挂起的内存或寄存器状态,这是内存模型必须做的。
这些状态是(这是一个带标签的联合;标签用小写字母表示):
% The intra-instruction semantics for each instruction instance is expressed as a state machine, essentially running the instruction pseudocode.
% Given a pseudocode execution state, it computes the next state.  Most
% states identify a pending memory or register operation, requested by
% the pseudocode, which the memory model has to do.  The
% states are (this is a tagged union; tags in small-caps):

\begin{center}
\begin{tabular}{l@{ \quad-\quad }l}
{\sc Load\_mem}({\it kind}, {\it address}, {\it size}, {\it load\_continuation})
    & 内存加载操作  \\
{\sc Early\_sc\_fail}({\it res\_continuation})
    & 允许 {\tt sc} 提前失败 \\
{\sc Store\_ea}({\it kind}, {\it address}, {\it size}, {\it next\_state})
    & 内存存储有效地址  \\
{\sc Store\_memv}({\it mem\_value}, {\it store\_continuation})
    & 内存存储值 \\
{\sc Fence}({\it kind}, {\it next\_state})
    & 屏障  \\
{\sc Read\_reg}({\it reg\_name}, {\it read\_continuation})
    & 寄存器读  \\
{\sc Write\_reg}({\it reg\_name}, {\it reg\_value}, {\it next\_state})
    & 寄存器写  \\
{\sc Internal}({\it next\_state})
    & 伪代码内部步骤  \\
{\sc Done}
    & 伪代码的结束  \\
\end{tabular}
\end{center}
Here:
\begin{tightlist}
\item {\it mem\_value}和{\it reg\_value}是字节的列表;  %   {\it mem\_value} and {\it reg\_value} are lists of bytes;
\item {\it address}是一个XLEN位的整数;   %  {\it address} is an integer of XLEN bits;
\item 对于加载/存储,{\it kind}标识了它是否是{\tt lr/sc}、acquire-RCpc/release-RCpc、acquire-RCsc/release-RCsc、acquire-release-RCsc;  
%  for load/store, {\it kind} identifies whether it is {\tt lr/sc}, acquire-RCpc/release-RCpc, acquire-RCsc/release-RCsc, acquire-release-RCsc;
\item 对于屏障,{\it kind}标识了它是否是一个普通的屏障还是TSO屏障,以及(对于普通屏障)前趋和后继次序的位;
% for fence, {\it kind} identifies whether it is a normal or TSO, and (for normal fences) the predecessor and successor ordering bits;
\item {\it reg\_name}标识了一个寄存器和它的一部分(开始和结束位的索引);以及
% {\it reg\_name} identifies a register and a slice thereof (start and
%   end bit indices); and
\item continuation描述了指令实例将如何继续处理每个可能由周围内存模型所提供的值({\it load\_continuation}和{\it read\_continuation}取得从内存加载的值和读取从先前寄存器写入的值,{\it store\_continuation}对于一个失败的sc取假,而在所有其它情形中取真,而如果{\tt sc}{\it 失败},{\it res\_continuation}取假,否则取{\it 真})。
% the continuations describe how the instruction instance will continue for each value that might be provided by the surrounding memory model (the {\it load\_continuation} and {\it read\_continuation} take the value loaded from memory and read from the previous register write, the {\it store\_continuation} takes {\it false} for an {\tt sc} that failed and {\it true} in all other cases, and {\it res\_continuation} takes {\it false} if the {\tt sc} fails and {\it true} otherwise).
\end{tightlist}

\begin{commentary}
  例如,给定加载指令\verb!lw x1,0(x2)!,一次执行往往将如下运行。初始执行状态将从给定的操作码的伪代码计算。
  这可能预计是{\sc Read\_reg}({\tt x2}, {\it read\_continuation})。
  把寄存器{\tt x2}最近写入的值(如果有必要,指令语义将被阻塞,直到寄存器的值可用),也就是{\tt 0x4000},输入到{\it read\_continuation},
  返回{\sc Load\_mem}({\tt plain\_load}, {\tt 0x4000}, {\tt 4}, {\it load\_continuation})。把从内存地址{\tt 0x4000}加载的{\tt 4}字节的值,也就是{\tt 0x42},
  输入到{\it load\_continuation},返回{\sc Write\_reg}({\tt x1}, {\tt 0x42}, {\sc Done})。在上述状态之间可能出现许多{\sc Internal}({\it next\_state})状态。
% For example, given the load instruction \verb!lw x1,0(x2)!,
% an execution will typically go as follows.
% The initial execution state will be computed from the pseudocode for the given opcode.
% This can be expected to be {\sc Read\_reg}({\tt x2}, {\it read\_continuation}).
% Feeding the most recently written value of register {\tt x2} (the instruction semantics will be blocked if necessary until the register value is available), say {\tt 0x4000}, to {\it read\_continuation} returns {\sc Load\_mem}({\tt plain\_load}, {\tt 0x4000}, {\tt 4}, {\it load\_continuation}).
% Feeding the 4-byte value loaded from memory location {\tt 0x4000}, say {\tt 0x42}, to {\it load\_continuation} returns
% {\sc Write\_reg}({\tt x1}, {\tt 0x42}, {\sc Done}).
% Many {\sc Internal}({\it next\_state}) states may appear before and between the states above.
\end{commentary}

注意,写内存被分为两个步骤,{\sc Store\_ea}和{\sc Store\_memv}:第一个步骤制造存储暂时已知的内存足迹,而第二个步骤添加要被存储的值。我们确保这些在伪代码中是成对的({\sc Store\_ea}后面跟着{\sc Store\_memv}),但是在它们之间可能会有其它的步骤。
% Notice that writing to memory is split into two steps, {\sc Store\_ea} and {\sc Store\_memv}: the first one makes the memory footprint of the store provisionally known, and the second one adds the value to be stored.
% We ensure these are paired in the pseudocode ({\sc Store\_ea} followed by {\sc Store\_memv}), but there may be other steps between them.
\begin{commentary}
  可以观察到,{\sc Store\_ea}可以发生在要存储的值被决定之前。例如,对于操作模型所允许的石蕊测试LB+fence.r.rw+data-po(正如RVWMO所允许的那样),硬件线程1中的第一个存储必须在它的值被决定之前就采取{\sc Store\_ea}步骤,以便第二个存储可以看到它是一个非重叠的内存足迹,以允许第二个存储被乱序提交而不违反一致性。
% It is observable that the {\sc Store\_ea} can occur before the value to be stored is determined.
% For example, for the litmus test LB+fence.r.rw+data-po to be allowed by the operational model (as it is by RVWMO), the first store in Hart 1 has to take the {\sc Store\_ea} step before its value is determined, so that the second store can see it is to a non-overlapping memory footprint, allowing the second store to be committed out of order without violating coherence.
\end{commentary}

除了明确执行一次加载和一次存储的AMO,每个指令的伪代码最多执行一次存储或一次加载。那些内存访问然后通过硬件线程语义被分割为架构上的原子性单元(见下面的\nameref{omm:initiate_load}和\nameref{omm:initiate_store_footprint})。
% The pseudocode of each instruction performs at most one store or one load, except for AMOs that perform exactly one load and one store.
% Those memory accesses are then split apart into the architecturally atomic units by the hart semantics (see \nameref{omm:initiate_load} and \nameref{omm:initiate_store_footprint} below).

非正式地,一个寄存器读的每一位都应当由来自最近的(以程序次序)可以写该位的指令实例的一个寄存器写(或者来自硬件线程的初始寄存器状态,如果没有这样的写的话)来满足。因此,有必要知道每个指令实例的寄存器写的足迹,我们在指令实例被创建时(见下面的\nameref{omm:fetch}的动作)计算这个足迹。我们确保在伪代码中,每个指令对每个寄存器位最多执行一次寄存器写,并且也保证不会尝试读取它刚刚写入的寄存器的值。
% Informally, each bit of a register read should be satisfied from a register write by the most recent (in program order) instruction instance that can write that bit (or from the hart's initial register state if there is no such write).
% Hence, it is essential to know the register write footprint of each instruction instance, which we calculate when the instruction instance is created (see the action of \nameref{omm:fetch} below).
% We ensure in the pseudocode that each instruction does at most one register write to each register bit, and also that it does not try to read a register value it just wrote.

每个寄存器读必须等待合适的寄存器写被执行(正如上面描述的那样),从这个事实浮现了模型中的数据流依赖(地址和数据)。
% Data-flow dependencies (address and data) in the model emerge from the
% fact that each register read has to wait for the appropriate register write to be executed (as described above).

\subsection{指令实例状态}\label{sec:omm:inst_state}
% \subsection{Instruction Instance State}\label{sec:omm:inst_state}
每个指令实例$i$都拥有一个状态,包括:
% Each instruction instance $i$ has a state comprising:
\begin{itemize}
\item {\it program\_loc},指令被从此获取的内存地址; % , the memory address from which the instruction was fetched;
\item {\it instruction\_kind},识别这是否是一个加载、存储、AMO、屏障、分支/跳转,或者一个‘简单的’指令(这也包括了一种类似于描述伪指令执行状态的类型); % identifying whether this is a load, store, AMO, fence, branch/jump or a `simple' instruction (this also includes a {\it kind} similar to the one described for the pseudocode execution states);
\item {\it src\_regs},源{\it reg\_name}的集合(包括系统寄存器),由指令的伪代码静态决定; %  , the set of source {\it reg\_name}s (including system registers), as statically determined from the pseudocode of the instruction;
\item {\it dst\_regs},目的{\it reg\_name}(包括系统寄存器),由指令的伪代码静态决定; % , the destination {\it reg\_name}s (including system registers), as statically determined from the pseudocode of the instruction;
\item {\it pseudocode\_state}(或者有时只简写为‘state’),如下之一(这是一个带标签的联合;标签用小写字母表示):  %  (or sometimes just `state' for short), one of (this is a tagged union; tags in small-caps):
  \begin{center}
  \begin{tabular}{l@{ \quad-\quad }l}
  {\sc Plain}({\it isa\_state})
    & 准备制造一个伪代码过渡 \\
  {\sc Pending\_mem\_loads}({\it load\_continuation})
    & 请求内存加载操作 \\
  {\sc Pending\_mem\_stores}({\it store\_continuation})
    & 请求内存存储操作 \\
%   {\sc Pending\_exception}({\it exception}) & performing an exception;
  \end{tabular}
  \end{center}

\item {\it reg\_reads},寄存器读实例已经被执行,对于每个实例,包括寄存器写它从中读取的片段; % , the register reads the instance has performed, including, for each one, the register write slices it read from;
\item {\it reg\_writes},寄存器写实例已经被执行; % , the register writes the instance has performed;
\item {\it mem\_loads},一组内存加载操作的集合,并且对于每个加载操作,是尚未满足的片段(还没有被满足的字节的索引),而对于已满足的片段,则是满足它的存储片段(每个片段由一个内存存储操作和它的字节索引的子集组成)。  % , a set of memory load operations, and for each one
  % the as-yet-unsatisfied slices (the byte indices that have not been
  % satisfied yet), and, for the satisfied slices, the store slices
  % (each consisting of a memory store operation and subset of its byte indices) that satisfied it.
% ------   下面两条word里没有,可能是tex新版本刚加入的内容 ----------
% \item {\it mem\_stores}, a set of memory store operations, and for each one a flag that indicates whether it has been propagated (passed to the shared memory) or not.
% \item information recording whether the instance is committed, finished, etc.
\end{itemize}

每个内存加载操作包括了一个内存足迹(地址和尺寸)。每个内存存储操作包括了一个内存足迹和(当值可用时)一个值。
% Each memory load operation includes a memory footprint (address and size).
% Each memory store operations includes a memory footprint, and, when available, a value.

一个带有非空{\it mem\_loads}的加载指令实例,如果所有的加载操作都被满足(换句话说,没有未满足的加载片段),那么被称之为被完全满足的。
% A load instruction instance with a non-empty {\it mem\_loads}, for which all the load operations are satisfied (i.e.~there are no unsatisfied load slices) is said to be {\it entirely satisfied}.

非正式地,一个指令实例被称为具有{\it 完全决定的}数据,如果为它的源寄存器提供输入的加载(和{\tt sc})指令被完成了。
类似地,它被称为具有完全决定的内存足迹,如果为它的内存操作地址寄存器提供输入的加载(和{\tt sc})指令被完成了。
正式地讲,我们首先定义了{\it 完全决定的寄存器写}的概念:一个来自指令实例$i$的{\it reg\_writes}的$w$被称之为{\it 完全决定的},如果满足了下列条件之一:
% Informally, an instruction instance is said to have {\it fully determined data} if the load (and {\tt sc}) instructions feeding its source registers are finished.
% Similarly, it is said to have a {\it fully determined memory footprint} if the load (and {\tt sc}) instructions feeding its memory operation address register are finished.
%
% Formally, we first define the notion of {\it fully determined register write}: a register write $w$ from {\it reg\_writes} of instruction instance $i$ is said to be {\it fully determined} if one of the following conditions hold:
\begin{enumerate}
\item i被完成了;或者 %  $i$ is finished; or
\item w所写的值不会被i制造的内存操作所影响(即,一个从内存加载的值或者sc的结果),并且,对于i已经制造的每个影响w的寄存器读,源于i读取的寄存器写都是完全决定的(或者i从初始寄存器状态读取)。 
% the value written by $w$ is not affected by a memory operation that $i$ has made (i.e. a value loaded from memory or the result of {\tt sc}), and, for every register read that $i$ has made, that affects $w$, the register write from which $i$ read is fully determined (or $i$ read from the initial register state).
\end{enumerate}
现在,一个指令实例$i$被称为具有{\it 完全决定的}数据,如果对于每个来自{\it reg\_reads}的寄存器读$r$,$r$从中读取的寄存器写都是完全决定的。
一个指令实例$i$被称为{\it 具有完全决定的内存足迹},如果对于来自{\it reg\_reads}的每个输入到$i$的内存操作地址的寄存器读$r$,$r$从中读取的寄存器写都是完全决定的。
% Now, an instruction instance $i$ is said to have  {\it fully determined data} if for every register read $r$ from {\it reg\_reads}, the register writes that $r$ reads from are fully determined.
% An instruction instance $i$ is said to have a {\it fully determined memory footprint} if for every register read $r$ from {\it reg\_reads} that feeds into $i$'s memory operation address, the register writes that $r$ reads from are fully determined.
\begin{commentary}
  对于每次寄存器写, {\tt rmem}工具都会记录,在执行写的这一时刻,该指令已经读取的来自其它指令的寄存器写的集合。通过小心地安排由工具覆盖的指令的伪代码,我们能够做到这点,使得这确实是这次写所依赖的寄存器写的集合。
% The {\tt rmem} tool records, for every register write, the set of register writes from other instructions that have been read by this instruction at the point of performing the write.
% By carefully arranging the pseudocode of the instructions covered by the tool we were able to make it so that this is exactly the set of register writes on which the write depends on.
\end{commentary}


\subsection{硬件线程状态}
% \subsection{Hart State}
单个硬件线程的模型状态包括:
% The model state of a single hart comprises:
\begin{itemize}
\item {\it hart\_id},一个关于硬件线程的唯一的标识符; % , a unique identifier of the hart;
%\item {\it register\_data}, the name, bit width, and start bit index for each register;
\item {\it initial\_register\_state},各寄存器的初始寄存器状态;  % , the initial register value for each register;
\item {\it initial\_fetch\_address},初始指令获取地址; % , the initial instruction fetch address;
\item {\it instruction\_tree},以程序次序,一个已经被获取到的(并且没有被丢弃的)指令实例的树。 % , a tree of the instruction instances that have been fetched (and not discarded), in program order.
\end{itemize}


\subsection{共享内存状态}
% \subsection{Shared Memory State}
共享内存的模型状态包括一个内存存储操作的列表,按它们传播到共享内存的次序排序。
% The model state of the shared memory comprises a list of memory store operations, in the order they propagated to the shared memory.

当一个存储操作被传播到共享内存时,它被简单地添加到列表的末端。当一个加载操作从内存被满足时,对于加载操作的每一个字节,将返回最近对应的存储片段。
% When a store operation is propagated to the shared memory it is simply added to the end of the list.
% When a load operation is satisfied from memory, for each byte of the load operation, the most recent corresponding store slice is returned.

\begin{commentary}
  对于大多数目的,将共享内存想象为一个数组会更简单,即,一个从内存位置到内存存储操作片段的映射,这里每个内存位置被映射到一个最近的针对该位置的内存存储操作的1字节的片段。
  然而,对于适当地处理{\tt sc}指令,这种抽象不够详细。RVWMO\nameref{rvwmo:ax:atom}允许来自与{\tt sc}相同的硬件线程的存储操作,以在{\tt sc}的存储操作和与之配对的{\tt lr}将读取自的存储操作之间进行干预。
  为了允许这种存储操作进行干预,而禁止其它的,数组抽象必须被扩展以记录更多的信息。这里,我们使用一个列表,因为它很简单,但是更有效和可扩展的实现应当更可能使用某些更好的东西。
% For most purposes, it is simpler to think of the shared memory as an array, i.e., a map from memory locations to memory store operation slices, where each memory location is mapped to a one-byte slice of the most recent memory store operation to that location.
% However, this abstraction is not detailed enough to properly handle the {\tt sc} instruction.
% The RVWMO \nameref{rvwmo:ax:atom} allows store operations from the same hart as the {\tt sc} to intervene between the store operation of the {\tt sc} and the store operations the paired {\tt lr} read from.
% To allow such store operations to intervene, and forbid others, the array abstraction must be extended to record more information.
% Here, we use a list as it is very simple, but a more efficient and scalable implementations should probably use something better.
\end{commentary}


\subsection{过渡}\label{sec:omm:transitions}

下面的每个段落都描述了一种系统过渡。描述开始于一个基于当前系统状态之上的条件。
只有当条件被满足时,过渡可以发生在当前状态。条件后面跟着一个行动,它在过渡发生时应用到该状态,以生成新的系统状态。
% Each of the paragraphs below describes a single kind of system transition.
% The description starts with a condition over the current system state.
% The transition can be taken in the current state only if the condition is satisfied.
% The condition is followed by an action that is applied to that state when the transition is taken, in order to generate the new system state.

\paragraph{获取指令}\label{omm:fetch}
可以从地址{\it loc}获取指令实例$i$的一个可能的程序次序后继,如果:
% A possible program-order-successor of instruction instance $i$ can be fetched from address {\it loc} if:
\begin{enumerate}
\item 它还没有被获取到,即,在硬件线程的{\it 指令树}中,$i$没有直接的来自{\it loc}的后继;并且 % it has not already been fetched, i.e., none of the immediate successors of $i$ in the hart's {\it instruction\_tree} are from {\it loc}; and
\item 如果$i$的伪代码已经向{\em pc}写入了一个地址,那么{\it loc}必须是那个地址,否则{\it loc}就是:  % if $i$'s pseudocode has already written an address to {\em pc}, then {\it loc} must be that address, otherwise {\it loc} is:
  \begin{itemize}
  \item 对于一个条件分支,是后继地址或分支目标地址;  % for a conditional branch, the successor address or the branch target address;
  \item 对于一个(直接的)跳转和链接指令({\tt jal}),是目标地址;  % for a (direct) jump and link instruction ({\tt jal}), the target address;
  \item 对于一个间接跳转指令({\tt jalr}),是任何地址;以及  % for an indirect jump instruction ({\tt jalr}), any address; and
  \item 对于任何其它的指令,是$i.\textit{program\_loc}+4$。 %  for any other instruction, $i.\textit{program\_loc}+4$.
  \end{itemize}
\end{enumerate}
% \fixme{Does an instruction at the end of memory need special-case treatment?}

行动:在{\it loc}处的程序内存中为指令构建一个新的初始化的指令实例$i’$,它带有从指令伪代码计算的状态{\sc Plain}({\it isa\_state}),包括从伪代码可以获得的静态信息,例如它的{\it instruction\_kind},{\it src\_regs},和{\it dst\_regs},并把$i’$添加到硬件线程的指令树作为$i$的一个后继。
% Action: construct a freshly initialized instruction instance $i'$ for the instruction in the program memory at {\it loc}, with state {\sc Plain}({\it isa\_state}), computed from the instruction pseudocode, including the static information available from the pseudocode such as its {\it instruction\_kind}, {\it src\_regs}, and {\it dst\_regs}, and add $i'$ to the hart's {\it instruction\_tree} as a successor of $i$.

\begin{commentary}

  可能的下一个获取地址({\it loc})在获取$i$之后是立即可用的,并且模型不需要等待伪代码写到{\em pc};这允许乱序执行,并且推测过去的条件分支和跳转。对于大多数指令,这些地址很容易从指令的伪代码获得。
  唯一的例外是间接跳转指令({\tt jalr}),那里地址依赖于寄存器中持有的值。
  原则上,数学模型应当允许这里推测到任意的地址。在{\tt rmem}工具中的穷举搜索通过多次运行穷举搜索来处理这个问题,对于每个间接跳转,都带有不断增长的、可能的下一个获取地址的集合。
  最初的搜索使用空集合,因此在间接跳转指令之后没有获取,直到指令的伪代码写到{\em pc},然后我们使用该值来获取下一条指令。在开始穷举搜索的下一次迭代之前,我们为每个间接跳转(按代码位置分组)收集了它在先前搜索迭代中的所有执行中写到{\em pc}的值的集合,并使用它作为指令的可能的下一个获取的地址。
  当没有检测到新的获取地址时,这个过程终止。
% The possible next fetch addresses ({\it loc}) are available immediately after fetching $i$ and the model does not need to wait for the pseudocode to write to {\em pc}; this allows out-of-order execution, and speculation past conditional branches and jumps.
% For most instructions these addresses are easily obtained from the instruction pseudocode.
% The only exception to that is the indirect jump instruction ({\tt jalr}), where the address depends on the value held in a register.
% %
% In principle the mathematical model should allow speculation to
% arbitrary addresses here. 
% %
% The exhaustive search in the {\tt rmem} tool handles this by running the exhaustive search multiple times with a growing set of possible next fetch addresses for each indirect jump.
% The initial search uses empty sets, hence there is no fetch after indirect jump instruction until the pseudocode of the instruction writes to {\em pc}, and then we use that value for fetching the next instruction.
% Before starting the next iteration of exhaustive search, we collect for each indirect jump (grouped by code location) the set of values it wrote to {\em pc} in all the executions in the previous search iteration, and use that as possible next fetch addresses of the instruction.
% This process terminates when no new fetch addresses are detected.
\end{commentary}

\paragraph{初始化内存加载操作}\label{omm:initiate_load}
一个状态{\sc Plain}({\sc Load\_mem}({\it kind}, {\it address}, {\it size}, {\it load\_continuation}))中的指令实例$i$总是可以初始化对应的内存加载操作。
% An instruction instance $i$ in state {\sc Plain}({\sc Load\_mem}({\it kind}, {\it address}, {\it size}, {\it load\_continuation})) can always initiate the corresponding memory load operations.
行动:
% Action:
\begin{enumerate}
\item 构造合适的内存加载操作$mlos$:  % Construct the appropriate memory load operations $mlos$:
  \begin{itemize}
  \item 如果{\it address}被对齐到{\it size},那么$mlos$是一个从{\it address}开始的{\it size}字节的单个内存加载操作; %  if {\it address} is aligned to {\it size} then $mlos$ is a single memory load operation of {\it size} bytes from {\it address};
  \item 否则,$mlos$是一组数量为{\it size}的内存加载操作的集合,每个加载操作负责地址$\textit{address}\ldots\textit{address}+\textit{size}-1$中的一个字节。  % otherwise, $mlos$ is a set of {\it size} memory load operations, each of one byte, from the addresses $\textit{address}\ldots\textit{address}+\textit{size}-1$.
  \end{itemize}
\item 把$i$的{\it mem\_loads}设置到$mlos$;并且  % set {\it mem\_loads} of $i$ to $mlos$; and
\item 更新$i$的状态为{\sc Pending\_mem\_loads}({\it load\_continuation})。  % update the state of $i$ to {\sc Pending\_mem\_loads}({\it load\_continuation}).
\end{enumerate}

\begin{commentary}
  在~\ref{sec:rvwmo:primitives}节中说到,未对齐的内存访问可能在任意粒度被分解。这里我们把它们分解为一个字节的访问,因为这个粒度包含了所有其它的粒度。
% In Section~\ref{sec:rvwmo:primitives} it is said that misaligned memory accesses may be decomposed at any granularity.
% Here we decompose them to one-byte accesses as this granularity subsumes all others.
\end{commentary}

\paragraph{通过从未传播的状态转发来满足内存加载操作}\label{omm:sat_by_forwarding}
对于一个状态{\sc Pending\_mem\_loads}({\it load\_continuation})中的非AMO加载指令实例$i$,和$i.\textit{mem\_loads}$中的一个具有未满足片段的内存加载操作$mlo$,内存加载操作可以通过从未被程序次序之前的存储指令实例所传播的内存存储操作转发,而被部分地或完全地满足,如果:
% For a non-AMO load instruction instance $i$ in state {\sc Pending\_mem\_loads}({\it load\_continuation}), and a memory load operation $mlo$ in $i.\textit{mem\_loads}$ that has unsatisfied slices, the memory load operation can be partially or entirely satisfied by forwarding from unpropagated memory store operations by store instruction instances that are program-order-before $i$ if:
\begin{enumerate}
\item 所有的程序次序在先的设置了{\tt .sr}和{\tt .pw}的{\tt fence}指令都完成了;  % all program-order-previous {\tt fence} instructions with {\tt .sr} and {\tt .pw} set are finished;
\item 对于每个程序次序在先的设置了{\tt .sr}和{\tt .pr},但是没有设置{\tt .pw}的{\tt fence}指令,$f$,如果$f$没有完成,那么所有的程序次序在$f$之前的加载指令都被完全地满足了;  % for every program-order-previous {\tt fence} instruction, $f$, with {\tt .sr} and {\tt .pr} set, and {\tt .pw} not set, if $f$ is not finished then all load instructions that are program-order-before $f$ are entirely satisfied;
\item 对于每个没有完成的程序次序在先的{\tt fence.tso}指令,$f$,所有的程序次序先于$f$的加载指令都被完全地满足了;  % for every program-order-previous {\tt fence.tso} instruction, $f$, that is not finished, all load instructions that are program-order-before $f$ are entirely satisfied;
% \item all program-order-previous {\tt fence.i} instructions are finished;
\item 如果$i$是一个load-acquire-RCsc,所有的程序次序在先的store-relaase-RCsc都完成了;  % if $i$ is a load-acquire-RCsc, all program-order-previous store-releases-RCsc are finished;
\item 如果$i$是一个load-acquire-release,所有的程序次序在先的指令都完成了;  % if $i$ is a load-acquire-release, all program-order-previous instructions are finished;
\item 所有的未完成的程序次序在先的load-acquire指令都完全地被满足了;并且  % all non-finished program-order-previous load-acquire instructions are entirely satisfied; and
\item 所有的程序次序在先的store-acquire-release指令都被满足了;  % all program-order-previous store-acquire-release instructions are finished;
\end{enumerate}

令$msoss$是所有来自程序次序先于$i$的、并且已经计算出要存储的值的非{\tt sc}存储指令实例的未被转发的内存存储操作片段的集合,它们与$mlo$的未被满足的片段重叠,并且不会被干扰的存储操作、或从一个干扰的加载读取的存储操作所替代。最后一个条件要求,对于$msoss$中的每个来自指令$i’$的内存存储操作片段$msos$:
% Let $msoss$ be the set of all unpropagated memory store operation slices from non-{\tt sc} store instruction instances that are program-order-before $i$ and have already calculated the value to be stored, that overlap with the unsatisfied slices of $mlo$, and which are not superseded by intervening store operations or store operations that are read from by an intervening load.
% The last condition requires, for each memory store operation slice $msos$ in $msoss$ from instruction $i'$:
\begin{tightlist}
\item 没有程序次序在$i$和$i’$之间的存储指令带有与$msos$重叠的内存存储操作;并且 % that there is no store instruction program-order-between $i$ and $i'$ with a memory store operation overlapping $msos$; and
\item 没有程序次序在$i$和$i’$之间的加载指令被来自不同的硬件线程的重叠的内存存储操作片段所满足。  % that there is no load instruction program-order-between $i$ and $i'$ that was satisfied from an overlapping memory store operation slice from a different hart.
\end{tightlist}

行动:
% Action:
\begin{enumerate}
\item 更新$i.\textit{mem\_loads}$,以表示$mlo$被$msoss$满足了;并且
% update $i.\textit{mem\_loads}$ to indicate that $mlo$ was satisfied by $msoss$; and
\item 重启任何违背一致性的推测性指令,作为这个的结果,也就是说,对于每个未完成的作为i的程序次序后继的指令$i’$,和$i’$的每个由$msoss’$满足的内存加载操作$mlo’$,如果在$msoss’$中存在一个内存存储操作片段$msos’$,和一个来自$msoss$中一个不同的内存存储操作的重叠的内存存储操作片段,并且$msos’$并非来自$i$的一个程序次序后继指令,那么重新启动$i’$和它的{\em 重启依赖}。
% restart any speculative instructions which have violated coherence as a result of this, i.e., for every non-finished instruction $i'$ that is a program-order-successor of $i$, and every memory load operation $mlo'$ of $i'$ that was satisfied from $msoss'$, if there exists a memory store operation slice $msos'$ in $msoss'$, and an overlapping memory store operation slice from a different memory store operation in $msoss$, and $msos'$ is not from an instruction that is a program-order-successor of $i$, restart $i'$ and its {\em restart-dependents}.
\end{enumerate}
此处,指令$j$的{\em 重启依赖}是指:
% Where, the {\em restart-dependents} of instruction $j$ are:
\begin{tightlist}
\item $j$的程序次序后继,如果它有关于$j$的寄存器写的数据流依赖;  % program-order-successors of $j$ that have data-flow dependency on a register write of $j$;
\item $j$的程序次序后继,如果它有一个内存加载操作,其读取自j的一个内存存储操作(通过转发);  % program-order-successors of $j$ that have a memory load operation that reads from a memory store operation of $j$ (by forwarding);
\item 如果$j$是一个load-acquire,那么是$j$的所有的程序次序后继;  % if $j$ is a load-acquire, all the program-order-successors of $j$;
\item 如果$j$是一个加载,对于每个设置了{\tt .sr}和{\tt .pr},但是没有设置{\tt .pw}的{\tt fence},$f$,如果它是$j$的一个程序次序后继,那么是所有的是$f$的程序次序后继的加载指令;  %  if $j$ is a load, for every {\tt fence}, $f$, with {\tt .sr} and {\tt .pr} set, and {\tt .pw} not set, that is a program-order-successor of $j$, all the load instructions that are program-order-successors of $f$;
\item 如果$j$是一个加载,对于每个{\tt fence.tso},$f$,如果是$j$的一个程序次序后继,那么是所有的是$f$的程序次序后继的加载指令;以及  % if $j$ is a load, for every {\tt fence.tso}, $f$, that is a program-order-successor of $j$, all the load instructions that are program-order-successors of $f$;
% and 
\item(递归地)所有的上述指令实例的所有重启依赖。  %  (recursively) all the restart-dependents of all the instruction instances above.
\end{tightlist}

\begin{commentary}
  向一个内存加载转发内存存储操作可能只满足该加载的某些片段,而使其它片段仍是未满足的。
% Forwarding memory store operations to a memory load might satisfy only some slices of the load, leaving other slices unsatisfied.

一个程序次序在先的不可用的存储操作,当它变得可用时,在采用上面的过渡的时候,可能使得$msoss$暂时不可靠(违反一致性)。那样的存储将阻止加载的完成(见\nameref{omm:finish}),并将在存储操作被传播时,导致它重启(见\nameref{omm:prop_store})。
% A program-order-previous store operation that was not available when taking the transition above might make $msoss$ provisionally unsound (violating coherence) when it becomes available.
% That store will prevent the load from being finished (see \nameref{omm:finish}), and will cause it to restart when that store operation is propagated (see \nameref{omm:prop_store}).

上述过渡条件的结果就是,store-release-RCsc内存存储操作不能被转发到load-acquire-RCsc指令:$msoss$不包括来自已完成存储的内存存储操作(因为那些必定是已传播的内存存储操作),而且当加载是acquire-RCsc的时候,上述条件需要所有的程序次序在先的store-release-RCsc都被完成。
% A consequence of the transition condition above is that store-release-RCsc memory store operations cannot be forwarded to load-acquire-RCsc instructions:
% $msoss$ does not include memory store operations from finished stores (as those must be propagated memory store operations), and the condition above requires all program-order-previous store-releases-RCsc to be finished when the load is acquire-RCsc.
\end{commentary}


\paragraph{从内存满足内存加载操作}\label{omm:sat_from_mem}
对于一个非AMO加载指令或者一个“AMO的满足、提交和传播操作”过渡上下文中的AMO指令的指令实例$i$,任何$i.\textit{mem\_loads}$中的有未满足片段的内存加载操作mlo,可以从内存被满足,如果通过未传播的存储转发来满足内存加载操作的所有条件都被满足了的话。
行动:令$msoss$是来自覆盖了$mlo$的未满足片段的内存的内存存储操作片段,并应用通过未传播的存储转发满足内存加载操作的行为。
% For an instruction instance $i$ of a non-AMO load instruction or an AMO instruction in the context of the ``\nameref{omm:do_amo}'' transition, any memory load operation $mlo$ in $i.\textit{mem\_loads}$ that has unsatisfied slices, can be satisfied from memory if all the conditions of \nameref{omm:sat_by_forwarding} are satisfied.
% Action: let $msoss$ be the memory store operation slices from memory covering the unsatisfied slices of $mlo$, and apply the action of \nameref{omm:sat_by_forwarding}.

\begin{commentary}
  注意通过未传播的存储转发来满足内存加载操作可能会使内存加载操作的某些片段无法满足,那些片段将不得不通过再次采取过渡来满足,或者采取从内存满足内存加载操作来满足。另一方面,从内存满足内存加载操作将总是满足内存加载操作的所有的未满足片段。
% Note that \nameref{omm:sat_by_forwarding} might leave some slices of the memory load operation unsatisfied, those will have to be satisfied by taking the transition again, or taking \nameref{omm:sat_from_mem}.
% \nameref{omm:sat_from_mem}, on the other hand, will always satisfy all the unsatisfied slices of the memory load operation.
\end{commentary}


\paragraph{完整加载操作}\label{omm:complete_loads}
状态{\sc Pending\_mem\_loads}({\it load\_continuation})中的一个加载指令实例$i$可以是完整的(不要与完成相混淆),如果所有的内存加载操作$i.\textit{mem\_loads}$被完全地满足了(即,没有未满足的片段)。
行动:将i的状态更新为{\sc Plain}({\it load\_continuation(mem\_value)}),这里{\it mem\_value}是从所有满足$i.\textit{mem\_loads}$的内存存储操作片段集合而成的。
% A load instruction instance $i$ in state {\sc Pending\_mem\_loads}({\it load\_continuation}) can be completed (not to be confused with finished) if all the memory load operations $i.\textit{mem\_loads}$ are entirely satisfied (i.e.~there are no unsatisfied slices).
% Action: update the state of $i$ to {\sc Plain}({\it load\_continuation(mem\_value)}), where {\it mem\_value} is assembled from all the memory store operation slices that satisfied $i.\textit{mem\_loads}$.


\paragraph{早期{\tt sc}失败}\label{omm:early_sc_fail}
状态{\sc Plain}({\sc Early\_sc\_fail}({\it res\_continuation}))中的一个{\tt sc}指令实例$i$可以总是被造成失败。行动:把$i$的状态更新到{\sc Plain}({\it res\_continuation(false)})。
% An {\tt sc} instruction instance $i$ in state {\sc Plain}({\sc Early\_sc\_fail}({\it res\_continuation})) can always be made to fail.
% Action: update the state of $i$ to {\sc Plain}({\it res\_continuation(false)}).


\paragraph{配对的{\tt sc} }\label{omm:paired_sc}
状态{\sc Plain}({\sc Early\_sc\_fail}({\it res\_continuation}))中的一个{\tt sc}指令实例$i$可以继续它的(可能成功的)执行,如果$i$与一个{\tt lr}配对的话。行动:把$i$的状态更新到{\sc Plain}({\it res\_continuation(true)})。
% An {\tt sc} instruction instance $i$ in state {\sc Plain}({\sc Early\_sc\_fail}({\it res\_continuation})) can continue its (potentially successful) execution if $i$ is paired with an {\tt lr}.
% Action: update the state of $i$ to {\sc Plain}({\it res\_continuation(true)}).


\paragraph{初始化内存存储操作足迹}\label{omm:initiate_store_footprint}
状态{\sc Plain}({\sc Store\_ea}({\it kind}, {\it address}, {\it size}, {\it next\_state}))中的一个指令实例$i$可以总是宣布它的挂起的内存操作足迹。
行动:
% An instruction instance $i$ in state {\sc Plain}({\sc Store\_ea}({\it kind}, {\it address}, {\it size}, {\it next\_state})) can always announce its pending memory store operation footprint.
% Action:
\begin{enumerate}
\item 构造合适的内存存储操作$msos$(不带有存储的值): % construct the appropriate memory store operations $msos$ (without the store value):
  \begin{itemize}
  \item 如果{\it address}对齐到{\it size},那么$msos$是一个单独的针对{\it address}的{\it size}字节的内存存储操作; % if {\it address} is aligned to {\it size} then $msos$ is a single memory store operation of {\it size} bytes to {\it address};
  \item 否则,$msos$是一组数量为{\it size}的内存存储操作集合,每个长度1字节,针对地址$\textit{address}\ldots\textit{address}+\textit{size}-1$。  %  otherwise, $msos$ is a set of {\it size} memory store operations, each of one-byte size, to the addresses $\textit{address}\ldots\textit{address}+\textit{size}-1$.
  \end{itemize}
\item 把$i.\textit{mem\_stores}$设置为$msos$;并且  % set $i.\textit{mem\_stores}$ to $msos$; and
\item 把$i$的状态更新到{\sc Plain}({\it next\_state})。  % update the state of $i$ to {\sc Plain}({\it next\_state}).
\end{enumerate}

\begin{commentary}
  注意,在采取上述过渡之后,内存存储操作还没有拥有它们的值。把这个过渡从下面的过渡分离出来的重要性在于,它允许其它的程序次序后继的存储指令观察到这个指令的内存足迹,并且如果它们不重叠的话,尽可能早地乱序传播(即,在数据寄存器的值变得可用之前)。
% Note that after taking the transition above the memory store operations do not yet have their values.
% The importance of splitting this transition from the transition below is that it allows other program-order-successor store instructions to observe the memory footprint of this instruction, and if they don't overlap, propagate out of order as early as possible (i.e.~before the data register value becomes available).
\end{commentary}


\paragraph{初始化内存存储操作的值}\label{omm:instantiate_store_value}
状态{\sc Plain}({\sc Store\_memv}({\it mem\_value}中的一个指令实例i可以总是初始化内存存储操作$i.\textit{mem\_stores}$的值。
行动:
% An instruction instance $i$ in state {\sc Plain}({\sc Store\_memv}({\it mem\_value}, {\it store\_continuation})) can always instantiate the values of the memory store operations $i.\textit{mem\_stores}$.
% Action:
\begin{enumerate}
\item 在内存存储操作{\it mem\_value}之间分割出$i.\textit{mem\_stores}$;以及 % split {\it mem\_value} between the memory store operations $i.\textit{mem\_stores}$; and
\item 把$i$的状态更新到{\sc Pending\_mem\_stores}({\it store\_continuation})。 % update the state of $i$ to {\sc Pending\_mem\_stores}({\it store\_continuation}).
\end{enumerate}


\paragraph{提交存储指令}\label{omm:commit_stores}
一个非{\tt sc}存储指令,或者一个在“提交和传播一个{\tt sc}的存储操作”的上下文中的{\tt sc}指令,如果在状态{\sc Pending\_mem\_stores}({\it store\_continuation})之中,那么可以被提交(不要与传播相混淆),如果:
% An uncommitted instruction instance $i$ of a non-{\tt sc} store instruction or an {\tt sc} instruction in the context of the ``\nameref{omm:commit_sc}'' transition, in state {\sc Pending\_mem\_stores}({\it store\_continuation}), can be committed (not to be confused with propagated) if:
\begin{enumerate}
\item $i$具有完全决定的数据; % $i$ has fully determined data;
\item 所有的程序次序在先的条件分支和间接跳转指令都完成了; % all program-order-previous conditional branch and indirect jump instructions are finished;
\item 所有的程序次序在先的设置了{\tt .sw}的{\tt fence}指令都完成了; % all program-order-previous {\tt fence} instructions with {\tt .sw} set are finished;
\item 所有的程序次序在先的{\tt fence.tso}指令都完成了; % all program-order-previous {\tt fence.tso} instructions are finished;
% \item all program-order-previous {\tt fence.i} instructions are finished;
\item 所有的程序次序在先的load-acquire指令都完成了; % all program-order-previous load-acquire instructions are finished;
\item 所有的程序次序在先的store-acquire-release指令都完成了; %  all program-order-previous store-acquire-release instructions are finished;
\item 如果$i$是一个store-release,所有的程序次序在先的指令都完成了; %  if $i$ is a store-release, all program-order-previous instructions are finished;
\item\label{omm:commit_store:prev_addrs} 所有的程序次序在先的内存访问指令具有完全决定的内存足迹; % all program-order-previous memory access instructions have a fully determined memory footprint;
\item\label{omm:commit_store:prev_stores} 所有的程序次序在先的存储指令,除了失败的{\tt sc},都已经初始化并因此具有非空的{\it mem\_stores};并且  % all program-order-previous store instructions, except for {\tt sc} that failed, have initiated and so have non-empty {\it mem\_stores}; and
\item\label{omm:commit_store:prev_loads} 所有的程序次序在先的加载指令都已经初始化并因此具有非空的{\it mem\_loads}。 % all program-order-previous load instructions have initiated and so have non-empty {\it mem\_loads}.
\end{enumerate}
行动:记录i被提交了。
% Action: record that $i$ is committed.

\begin{commentary}
  注意,如果条件\ref{omm:commit_store:prev_addrs}被满足了,条件\ref{omm:commit_store:prev_stores}和\ref{omm:commit_store:prev_loads}也被满足,或者将在采取某些立即的过渡之后被满足。
  因此,对它们的要求并不会增强模型。通过要求它们,我们保证了先前的内存访问指令已经采取了足够的过渡,使得它们的内存操作对\nameref{omm:prop_store}的条件检查可见,这是指令将采取的下一个过渡,使得那个条件更加简单。
% Notice that if condition \ref{omm:commit_store:prev_addrs} is satisfied the conditions \ref{omm:commit_store:prev_stores} and \ref{omm:commit_store:prev_loads} are also satisfied, or will be satisfied after taking some eager transitions.
% Hence, requiring them does not strengthen the model.
% By requiring them, we guarantee that previous memory access instructions have taken enough transitions to make their memory operations visible for the condition check of \nameref{omm:prop_store}, which is the next transition the instruction will take, making that condition simpler.
\end{commentary}


\paragraph{传播存储操作}\label{omm:prop_store}
对于状态{\sc Pending\_mem\_stores}({\it store\_continuation})中的一个提交的指令实例$i$,和一个在$i.\textit{mem\_stores}$之中的未传播的内存存储操作$mso$,$mso$可以被传播,如果:
% For a committed instruction instance $i$ in state {\sc Pending\_mem\_stores}({\it store\_continuation}), and an unpropagated memory store operation $mso$ in $i.\textit{mem\_stores}$, $mso$ can be propagated if:
\begin{enumerate}
\item 所有的与$mso$重叠的程序次序在先的存储指令的内存存储操作都已经传播;  % all memory store operations of program-order-previous store instructions that overlap with $mso$ have already propagated;
\item 所有的与$mso$重叠的程序次序在先的加载指令的内存加载操作都已经被满足,并且(加载指令)是{\em 不可重启的}(见下面的定义);并且  % all memory load operations of program-order-previous load instructions that overlap with $mso$ have already been satisfied, and (the load instructions) are {\em non-restartable} (see definition below); and
\item 所有的通过转发$mso$被满足的内存加载操作都被完全地满足。  % all memory load operations that were satisfied by forwarding $mso$ are entirely satisfied.
\end{enumerate}
此处,一个{\em 未完成的}指令实例$i$是不可重启的,如果:
% Where a non-finished instruction instance $j$ is {\em non-restartable} if:
\begin{enumerate}
\item 不存在一个存储指令s和s的一个未传播的内存存储操作mso,使得对mso应用“传播存储操作”过渡的行为将导致j的重启;并且 % there does not exist a store instruction $s$ and an unpropagated memory store operation $mso$ of $s$ such that applying the action of the ``\nameref{omm:prop_store}'' transition to $mso$ will result in the restart of $j$; and
\item 不存在一个未完成的加载指令l和l的一个内存加载操作mlo,使得对mlo应用“通过从未传播的存储转发来满足内存加载操作”/“从内存满足内存加载操作”过渡(甚至mlo已经被满足)将导致j的重启。  % there does not exist a non-finished load instruction $l$ and a memory load operation $mlo$ of $l$ such that applying the action of the ``\nameref{omm:sat_by_forwarding}''/``\nameref{omm:sat_from_mem}'' transition (even if $mlo$ is already satisfied) to $mlo$ will result in the restart of $j$.
\end{enumerate}
行动:
% Action:
\begin{enumerate}
\item 使用$mso$更新共享内存的状态;  % update the shared memory state with $mso$;
\item 更新$i.\textit{mem\_stores}$,以表示$mso$被传播了;以及  % update $i.\textit{mem\_stores}$ to indicate that $mso$ was propagated; and
\item 重启任何因为这个的结果而已经违背了一致性的推测性指令,也就是说,对于每个程序次序晚于$i$的未完成的指令$i’$和$msoss’$中的与$mso$重叠而不来自$mso$的每个$i’$的内存加载操作$mlo’$,并且$msos’$不是来自$i$的一个程序次序后继,重启$i’$和它的{\em 重启依赖}(见:\nameref{omm:sat_by_forwarding})。  % restart any speculative instructions which have violated coherence as a result of this, i.e., for every non-finished instruction $i'$ program-order-after $i$ and every memory load operation $mlo'$ of $i'$ that was satisfied from $msoss'$, if there exists a memory store operation slice $msos'$ in $msoss'$ that overlaps with $mso$ and is not from $mso$, and $msos'$ is not from a program-order-successor of $i$, restart $i'$ and its {\em restart-dependents} (see \nameref{omm:sat_by_forwarding}).
\end{enumerate}


\paragraph{提交和传播一个{\tt sc}的存储操作}\label{omm:commit_sc}
一个来自硬件线程$h$的、状态{\sc Pending\_mem\_stores}({\it store\_continuation})中的、未提交的{\tt sc}指令实例$i$,带有一个配对的{\tt lr} $i’$,其已经被某些存储片段$msoss$所满足,可以被同时提交和传播,如果:
% An uncommitted {\tt sc} instruction instance $i$, from hart $h$, in state {\sc Pending\_mem\_stores}({\it store\_continuation}), with a paired {\tt lr} $i'$ that has been satisfied by some store slices $msoss$, can be committed and propagated at the same time if:
\begin{enumerate}
\item $i’$被完成; % $i'$ is finished;
\item 每个已经转发到$i’$的内存存储操作都是已传播的; % every memory store operation that has been forwarded to $i'$ is propagated;
\item \nameref{omm:commit_stores}的条件被满足; % the conditions of \nameref{omm:commit_stores} is satisfied;
\item \nameref{omm:prop_store}的条件被满足(注意一个{\tt sc}指令只能有一个内存存储操作);以及  % the conditions of \nameref{omm:prop_store} is satisfied (notice that an {\tt sc} instruction can only have one memory store operation); and
\item 在共享内存中,对于每个来自$msoss$的存储片段$msos$,从$msos$被传播到内存后的任何时刻,$msos$都还没有被来自不是$h$的硬件线程的存储覆写。  % for every store slice $msos$ from $msoss$, $msos$ has not been overwritten, in the shared memory, by a store that is from a hart that is not $h$, at any point since $msos$ was propagated to memory.
\end{enumerate}
行动:
% Action:
\begin{enumerate}
\item 应用\nameref{omm:commit_stores}的行动;以及  % apply the actions of \nameref{omm:commit_stores}; and
\item 应用\nameref{omm:prop_store}的行动。  %  apply the action of \nameref{omm:prop_store}.
\end{enumerate}


\paragraph{Late {\tt sc} fail}\label{omm:late_sc_fail}
状态{\sc Pending\_mem\_stores}({\it store\_continuation})中的一个{\tt sc}指令实例$i$,如果还没有传播它的内存存储操作,那么总是可以被造成失败。
行动:
% An {\tt sc} instruction instance $i$ in state {\sc Pending\_mem\_stores}({\it store\_continuation}), that has not propagated its memory store operation, can always be made to fail.
% Action:
\begin{enumerate}
\item 清除$i.\textit{mem\_stores}$;以及 % clear $i.\textit{mem\_stores}$; and
\item 把$i$的状态更新到{\sc Plain}({\it store\_continuation(false)})。  % update the state of $i$ to {\sc Plain}({\it store\_continuation(false)}).
\end{enumerate}

\begin{commentary}
  为了效率,{\tt rmem}工具只在不可能采用一个提交和传播一个{\tt sc}存储操作过渡的时候,才允许这个过渡。这不影响所允许的最终状态的集合,但是当交互式地探究时,如果sc应当失败,那么应当使用早期sc失败过渡,而不是期待这个过渡。
% For efficiency, the {\tt rmem} tool allows this transition only when it is not possible to take the \nameref{omm:commit_sc} transition.
% This does not affect the set of allowed final states, but when explored interactively, if the {\tt sc} should fail one should use the \nameref{omm:early_sc_fail} transition instead of waiting for this transition.
\end{commentary}

\paragraph{完整存储操作}\label{omm:complete_stores}
状态{\sc Pending\_mem\_stores}({\it store\_continuation})中的一个存储指令实例$i$,如果在$i.\textit{mem\_stores}$中的所有内存存储操作都已经被传播,那么可以总是完整的(不要与完成相混淆)。行动:把$i$的状态更新到{\sc Plain}({\it store\_continuation(true)})。
% A store instruction instance $i$ in state {\sc Pending\_mem\_stores}({\it store\_continuation}), for which all the memory store operations in $i.\textit{mem\_stores}$ have been propagated, can always be completed (not to be confused with finished).
% Action: update the state of $i$ to {\sc Plain}({\it store\_continuation(true)}).


\paragraph{满足、提交和传播一个AMO的操作}\label{omm:do_amo}
状态{\sc Pending\_mem\_loads}({\it load\_continuation})中的一个AMO指令实例$i$可以执行它的内存访问,如果可以执行下列顺序的过渡而没有干扰的过渡:
% An AMO instruction instance $i$ in state {\sc Pending\_mem\_loads}({\it load\_continuation}) can perform its memory access if it is possible to perform the following sequence of transitions with no intervening transitions:
\begin{enumerate}
\item \nameref{omm:sat_from_mem}
\item \nameref{omm:complete_loads}
\item \nameref{omm:sail_interp} (zero or more times)
\item \nameref{omm:instantiate_store_value}
\item \nameref{omm:commit_stores}
\item \nameref{omm:prop_store}
\item \nameref{omm:complete_stores}
\end{enumerate}
并且额外地,完成指令的条件,除了不要求$i$在状态{\sc Plain}({\sc Done}之中,其余在这些过渡之后保持不变。行动:一个接一个地、不加干扰过渡地,执行上述序列的过渡(这不包括完成指令),
% and in addition, the condition of \nameref{omm:finish}, with the exception of not requiring $i$ to be in state {\sc Plain}({\sc Done}), holds after those transitions.
% Action: perform the above sequence of transitions (this does not include \nameref{omm:finish}), one after the other, with no intervening transitions.

\begin{commentary}
  注意,程序次序在先的存储不能被转发到一个AMO的加载。这纯粹是因为上面的过渡序列不包括转发过渡。但是即使它确实包含了转发过渡,当尝试执行传播存储操作过渡的时候,该序列也将失败,因为这个过渡需要所有的程序次序在先的针对重叠的内存足迹的存储指令都被传播,而传播需要存储操作是未被传播的。
% Notice that program-order-previous stores cannot be forwarded to the load of an AMO.
% This is simply because the sequence of transitions above does not include the forwarding transition.
% But even if it did include it, the sequence will fail when trying to do the \nameref{omm:prop_store} transition, as this transition requires all program-order-previous store operations to overlapping memory footprints to be propagated, and forwarding requires the store operation to be unpropagated.

此外,一个AMO的存储不能被转发到一个程序次序后继的加载。在采取上面的过渡之前,AMO的存储指令没有拥有它的值,并因此不能被转发;在采取上面的过渡之后,存储操作被传播,并因此不能被转发。
% In addition, the store of an AMO cannot be forwarded to a program-order-successor load.
% Before taking the transition above, the store operation of the AMO does not have its value and therefore cannot be forwarded; after taking the transition above the store operation is propagated and therefore cannot be forwarded.
\end{commentary}


\paragraph{提交屏障}\label{omm:commit_fence}
在状态{\sc Plain}({\sc Fence}({\it kind}, {\it next\_state}))中的一个屏障指令实例$i$可以被提交,如果:
% A fence instruction instance $i$ in state {\sc Plain}({\sc Fence}({\it kind}, {\it next\_state})) can be committed if:
\begin{enumerate}
\item 如果$i$是一个普通fence,并且它设置了{\tt .pr},那么所有的程序次序在先的加载指令都是完成的; % if $i$ is a normal fence and it has {\tt .pr} set, all program-order-previous load instructions are finished;
\item 如果$i$是一个普通fence,并且它设置了{\tt .pw},那么所有的程序次序在先的存储指令都是完成的;以及 % if $i$ is a normal fence and it has {\tt .pw} set, all program-order-previous store instructions are finished; and
\item 如果$i$是一个{\tt fence.tso},那么所有的程序次序在先的加载和存储指令都是完成的。 % if $i$ is a {\tt fence.tso}, all program-order-previous load and store instructions are finished.
% \item if $i$ is a {\tt fence.i} instruction, all program-order-previous memory access instructions have fully determined memory footprints.
\end{enumerate}
行动:
% Action:
\begin{enumerate}
\item 记录$i$被提交了;以及 % record that $i$ is committed; and
\item 把$i$的状态更新到{\sc Plain}({\it next\_state})。 % update the state of $i$ to {\sc Plain}({\it next\_state}).
\end{enumerate}


\paragraph{寄存器读}\label{omm:reg_read}
% \paragraph{Register read}\label{omm:reg_read}
状态{\sc Plain}({\sc Read\_reg}({\it reg\_name}中的一个指令实例$i$可以执行一次{\it reg\_name}的寄存器读,如果它需要读取自的每个指令实例都已经执行了所期待的{\it reg\_name}寄存器写。
% An instruction instance $i$ in state {\sc Plain}({\sc Read\_reg}({\it reg\_name}, {\it read\_cont})) can do a register read of {\it reg\_name} if every instruction instance that it needs to read from has already performed the expected {\it reg\_name} register write.

对于{\it reg\_name}的每一位,令{\it read\_sources}包括,可以写到该位的最近(以程序次序)的指令实例对于该位的写入(如果有的话)。如果没有这样的指令,来源就是来自{\it initial\_register\_state}的初始的寄存器值。令{\it reg\_value}为从{\it read\_sources}集合的值。行动:
% Let {\it read\_sources} include, for each bit of {\it reg\_name}, the write to
% that bit by the most recent (in program order) instruction instance that can write to that bit, if any. If there is no such instruction, the source is the initial register value from {\it initial\_register\_state}.
% Let  {\it reg\_value} be the value assembled from {\it read\_sources}.
% Action:
\begin{enumerate}
\item 把{\it reg\_name}添加到带有{\it read\_sources}和{\it reg\_value}的$i.\textit{reg\_reads}$;以及 % add {\it reg\_name} to $i.\textit{reg\_reads}$ with {\it read\_sources} and {\it reg\_value}; and
\item 把$i$的状态更新为{\sc Plain}({\it read\_cont(reg\_value)})。 % update the state of $i$ to {\sc Plain}({\it read\_cont(reg\_value)}).
\end{enumerate}


\paragraph{寄存器写}\label{omm:reg_write}
状态{\sc Plain}({\sc Write\_reg}({\it reg\_name}中的一个指令实例$i$总是可以执行一个{\it reg\_name}寄存器写。行动:
% An instruction instance $i$ in state {\sc Plain}({\sc Write\_reg}({\it reg\_name}, {\it reg\_value}, {\it next\_state})) can always do a {\it reg\_name} register write.
% Action:
\begin{enumerate}
\item 把{\it reg\_name}添加到带有$deps$和{\it reg\_value}的$i.\textit{reg\_writes}$;以及  % add {\it reg\_name} to $i.\textit{reg\_writes}$ with $deps$ and {\it reg\_value}; and
\item 把$i$的状态更新到{\sc Plain}({\it next\_state})。 % update the state of $i$ to {\sc Plain}({\it next\_state}).
\end{enumerate}
此处$deps$是所有来自$i.\textit{reg\_reads}$的读取来源的集合,和一个标志的配对,这个标志为真,当且仅当$i$是一个已经被完全地满足的加载指令实例。
% where $deps$ is a pair of the set of all {\it read\_sources} from $i.\textit{reg\_reads}$, and a flag that is true iff $i$ is a load instruction instance that has already been entirely satisfied.


\paragraph{伪代码内部步骤}\label{omm:sail_interp}
状态{\sc Plain}({\sc Internal}({\it next\_state}))中的一个指令实例$i$总是可以执行那个伪代码内部的步骤。
行动:把$i$的状态更新到{\sc Plain}({\it next\_state})。
% An instruction instance $i$ in state {\sc Plain}({\sc Internal}({\it next\_state})) can always do that pseudocode-internal step.
% Action: update the state of $i$ to {\sc Plain}({\it next\_state}).


\paragraph{完成指令}\label{omm:finish}
状态{\sc Plain}({\sc Done})中的一个未完成的指令实例$i$可以被完成,如果:
% A non-finished instruction instance $i$ in state {\sc Plain}({\sc Done}) can be finished if:
\begin{enumerate}
\item 如果$i$是一个加载指令: %if $i$ is a load instruction:
  \begin{enumerate}
  \item 所有的程序次序在先的load-acquire指令都被完成了; % all program-order-previous load-acquire instructions are finished;
  \item 所有的程序次序在先的设置了.sr的{\tt fence}指令都被完成了; % all program-order-previous {\tt fence} instructions with {\tt .sr} set are finished;
  \item 对于每个程序次序在先的没有完成的{\tt fence.tso}指令,$f$,所有的程序次序在$f$之前的加载指令都被完成了;并且  % for every program-order-previous {\tt fence.tso} instruction, $f$, that is not finished, all load instructions that are program-order-before $f$ are finished; and
  \item 保证$i$的内存加载指令所读取的值将不会引起对一致性的违背,即,对于任何程序次序在先的指令实例$i’$,令$\textit{cfp}$是来自程序次序在$i$和$i’$之间的存储指令的已传播的内存存储操作,和从程序次序在$i$和$i’$之间、包括$i’$的存储指令转发到$i$的{\em 固定的内存存储操作},这二者的组合,并令$\overline{\textit{cfp}}$是$i$的内存足迹中的$\textit{cfp}$的补。
  如果$\overline{\textit{cfp}}$是不空的: 
  % it is guaranteed that the values read by the memory load operations of $i$ will not cause coherence violations, i.e., for any program-order-previous instruction instance $i'$, let $\textit{cfp}$ be the combined footprint of propagated memory store operations from store instructions program-order-between $i$ and $i'$, and {\em fixed memory store operations} that were forwarded to $i$ from store instructions program-order-between $i$ and $i'$ including $i'$, and let $\overline{\textit{cfp}}$ be the complement of $\textit{cfp}$ in the memory footprint of $i$.
  % If $\overline{\textit{cfp}}$ is not empty:
    \begin{enumerate}
    \item $i’$具有一个完全决定的内存足迹; %  $i'$ has a fully determined memory footprint;
    \item $i’$没有与$\overline{\textit{cfp}}$重叠的未传播的内存存储操作;并且  % $i'$ has no unpropagated memory store operations that overlap with $\overline{\textit{cfp}}$; and
    \item 如果$i’$是一个带有与$\overline{\textit{cfp}}$重叠的内存足迹的加载,那么与$\overline{\textit{cfp}}$重叠的$i’$的所有的内存加载操作都被满足,并且$i’$是不可重启的(对于如何决定一个指令是否是不可重启的,见传播存储操作过渡)。  
    % if $i'$ is a load with a memory footprint that overlaps with $\overline{\textit{cfp}}$, then all the memory load operations of $i'$ that overlap with $\overline{\textit{cfp}}$ are satisfied and $i'$ is {\em non-restartable} (see the \nameref{omm:prop_store} transition for how to determined if an instruction is non-restartable).
    \end{enumerate}
  这里,一个内存存储操作被称为固定的,如果存储指令具有完全决定的数据。
  % Here, a memory store operation is called fixed if the store instruction has fully determined data.
  \end{enumerate}
\item $i$具有一个完全决定的数据;并且  % $i$ has a fully determined data; and
\item 如果$i$不是一个屏障,那么所有的程序次序在先的条件分支和间接跳转指令都被完成了。  % if $i$ is not a fence, all program-order-previous conditional branch and indirect jump instructions are finished.
\end{enumerate}
行动:
% Action:
\begin{enumerate}
\item 如果$i$是一个条件分支或者间接跳转指令,丢弃任何未采取的执行路径,即,移除{\it instruction\_tree}中的所有的不可被分支/跳转采用而达到的指令实例;并且  % if $i$ is a conditional branch or indirect jump instruction, discard any untaken paths of execution, i.e., remove all instruction instances that are not reachable by the branch/jump taken in {\it instruction\_tree}; and
\item 记录该指令为完成的,即,设置{\it finished}为真。  % record the instruction as finished, i.e., set {\it finished} to {\it true}.
\end{enumerate}


\subsection{局限性}\label{sec:omm:limitations}
\begin{itemize}
\item 模型覆盖用户级RV64I和RV64A。
特别地,它不支持未对齐的原子性扩展“Zam”或者总存储排序扩展“Ztso”。
使模型适应RV32I/A和G、Q还有C扩展应当是轻而易举的,但是我们从未尝试过它。这将主要涉及,为指令写Sail代码,同时对并发模型的改变(如果有的话)最小。
% The model covers user-level RV64I and RV64A.
% In particular, it does not support the misaligned atomics extension ``Zam'' or the total store ordering extension ``Ztso''.
% It should be trivial to adapt the model to RV32I/A and to the G, Q and C extensions, but we have never tried it. This will involve, mostly, writing Sail code for the instructions, with minimal, if any, changes to the concurrency model.
\item 模型只覆盖了一般的内存访问(它不处理I/O访问)。  % The model covers only normal memory accesses (it does not handle I/O accesses).
\item 模型没有覆盖TLB相关的效果。  % The model does not cover TLB-related effects.
\item 模型假设指令内存是固定的。特别地,Fetch指令过渡不会生成内存加载操作,而共享内存不会在过渡中被涉及。反而,模型依赖于一个在给定内存位置时提供操作码的外部指示。 % The model assumes the instruction memory is fixed.
% In particular, the \nameref{omm:fetch} transition does not generate memory load operations, and the shared memory is not involved in the transition.
% Instead, the model depends on an external oracle that provides an opcode when given a memory location.
\item 模型没有覆盖异常、陷入和中断。 % The model does not cover exceptions, traps and interrupts.
\end{itemize}




\bibliographystyle{plain}
\bibliography{riscv-spec}

\end{document}
