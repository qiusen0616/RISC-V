\chapter{历史
  %History
}

\section{ 加州大学伯克利分校的研究经费
  % Research Funding at UC Berkeley
}

RISC-V 架构和实现的开发部分由以下赞助商资助。
% Development of the RISC-V architecture and implementations has been
% partially funded by the following sponsors.

\begin{itemize}
\item {\bf Par实验室:}研究由微软(Award \#024263)和英特尔(Award \#024894)赞助,
并由U.C.Discovery(Award \#DIG07-10227)提供匹配资助。额外的支持来自于Par 实验室附属的诺基亚、英伟达、甲骨文和三星。
% {\bf Par Lab:} Research supported by Microsoft (Award \#024263)
%   and Intel (Award \#024894) funding and by matching funding by
%   U.C. Discovery (Award \#DIG07-10227). Additional support came from
%   Par Lab affiliates Nokia, NVIDIA, Oracle, and Samsung.

\item {\bf 项目Isis}:DoE Award DE-SC0003624。 
% DoE Award DE-SC0003624.

\item {\bf ASPIRE实验室}:DARPA PERFECT工程,Award HR0011-12-2-0016。DARPA POEM工程Award HR0011-11-C-0100。
未来架构研究中心(C-FAR),一个由半导体研究公司资助的STARnet中心。
额外的支持来自于ASPIRE工业赞助者,英特尔,和ASPIRE附属,谷歌,惠普企业,华为,诺基亚,英伟达,甲骨文,和三星。
% DARPA PERFECT program, Award HR0011-12-2-0016.
%   DARPA POEM program Award HR0011-11-C-0100.  The Center for Future
%   Architectures Research (C-FAR), a STARnet center funded by the
%   Semiconductor Research Corporation.  Additional support from ASPIRE
%   industrial sponsor, Intel, and ASPIRE affiliates, Google, Huawei,
%   Nokia, NVIDIA, Oracle, and Samsung.

\end{itemize}

本文的内容并不能必然地反映出美国政府的立场和政策,并且不应被推断出官方的认可。
% The content of this paper does not necessarily reflect the position or the
% policy of the US government and no official endorsement should be
% inferred. 


